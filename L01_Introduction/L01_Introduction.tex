\documentclass[8pt,aspectratio=169]{beamer}
\usetheme{Madrid}
\usepackage{graphicx}
\usepackage{booktabs}
\usepackage{adjustbox}
\usepackage{multicol}
\usepackage{amsmath}

% Color definitions
\definecolor{mlblue}{RGB}{0,102,204}
\definecolor{mlpurple}{RGB}{51,51,178}
\definecolor{mllavender}{RGB}{173,173,224}
\definecolor{mllavender2}{RGB}{193,193,232}
\definecolor{mllavender3}{RGB}{204,204,235}
\definecolor{mllavender4}{RGB}{214,214,239}
\definecolor{mlorange}{RGB}{255, 127, 14}
\definecolor{mlgreen}{RGB}{44, 160, 44}
\definecolor{mlred}{RGB}{214, 39, 40}
\definecolor{mlgray}{RGB}{127, 127, 127}

\definecolor{lightgray}{RGB}{240, 240, 240}
\definecolor{midgray}{RGB}{180, 180, 180}

% Apply custom colors to Madrid theme
\setbeamercolor{palette primary}{bg=mllavender3,fg=mlpurple}
\setbeamercolor{palette secondary}{bg=mllavender2,fg=mlpurple}
\setbeamercolor{palette tertiary}{bg=mllavender,fg=white}
\setbeamercolor{palette quaternary}{bg=mlpurple,fg=white}

\setbeamercolor{structure}{fg=mlpurple}
\setbeamercolor{section in toc}{fg=mlpurple}
\setbeamercolor{subsection in toc}{fg=mlblue}
\setbeamercolor{title}{fg=mlpurple}
\setbeamercolor{frametitle}{fg=mlpurple,bg=mllavender3}
\setbeamercolor{block title}{bg=mllavender2,fg=mlpurple}
\setbeamercolor{block body}{bg=mllavender4,fg=black}

\setbeamertemplate{navigation symbols}{}
\setbeamertemplate{itemize items}[circle]
\setbeamertemplate{enumerate items}[default]
\setbeamersize{text margin left=5mm,text margin right=5mm}

\newcommand{\bottomnote}[1]{%
\vfill
\vspace{-2mm}
\textcolor{mllavender2}{\rule{\textwidth}{0.4pt}}
\vspace{1mm}
\footnotesize
\textbf{#1}
}

\title{Introduction to the Economics of Digital Finance}
\subtitle{L01: Setting the Economic Framework}
\author{Economics of Digital Finance}
\institute{BSc Course}
\date{}

\begin{document}

% Title slide
\begin{frame}[plain]
\titlepage
\end{frame}

% Outline
\begin{frame}[t]{Lesson Overview}
\begin{columns}[T]
\column{0.48\textwidth}
\textbf{Today's Topics}
\begin{enumerate}
\item What is digital finance? (Economic definition)
\item Historical evolution of money and payments
\item The four economic lenses framework
\item Why economists should care
\end{enumerate}

\column{0.48\textwidth}
\textbf{Learning Objectives}
\begin{itemize}
\item Define digital finance from an economic perspective
\item Distinguish economic from technical questions
\item Apply multiple economic frameworks to digital finance
\end{itemize}
\end{columns}

\bottomnote{This course examines digital finance through economic theory, not technical implementation}
\end{frame}

% What is Digital Finance?
\begin{frame}[t]{What is Digital Finance?}
\begin{columns}[T]
\column{0.48\textwidth}
\textbf{Economic Definition}

Digital finance encompasses financial services and instruments that:
\begin{itemize}
\item Rely on digital infrastructure for value transfer
\item Create new forms of money and payment systems
\item Enable disintermediation or re-intermediation
\end{itemize}

\vspace{0.3em}
\textbf{Key Distinction}
\begin{itemize}
\item Technical: \textit{How does it work?}
\item Economic: \textit{What incentives drive adoption?}
\end{itemize}

\column{0.48\textwidth}
\textbf{Scope of Digital Finance}
\begin{itemize}
\item Cryptocurrencies (digital assets using cryptography) and stablecoins (cryptocurrencies pegged to stable assets like USD)
\item Central Bank Digital Currencies (CBDCs)---official digital money issued by central banks
\item Digital payment systems
\item Decentralized Finance (DeFi)---financial services on blockchain without banks
\item Tokenized assets (traditional assets represented as digital tokens on blockchain)
\end{itemize}

\vspace{0.3em}
\textbf{Economic Questions}
\begin{itemize}
\item Who captures value?
\item What are the welfare effects?
\item How does regulation affect outcomes?
\end{itemize}
\end{columns}

\bottomnote{Economics analyzes incentives, efficiency, and welfare---not code or protocols}
\end{frame}

% Historical Evolution
\begin{frame}[t]{Historical Evolution of Payment Methods}
\begin{center}
\includegraphics[width=0.65\textwidth]{01_payment_evolution/chart.pdf}
\end{center}

\bottomnote{Each transition was driven by economic forces: reducing transaction costs (time, fees, and friction in exchanges), enabling trade at scale}
\end{frame}

% Functions of Money
\begin{frame}[t]{The Economics of Money: Three Functions}
\begin{columns}[T]
\column{0.48\textwidth}
\textbf{Classical Functions of Money}

\vspace{0.3em}
\textbf{1. Medium of Exchange}
\begin{itemize}
\item Solves double coincidence of wants (without money, you'd need to find someone who wants exactly what you have and has exactly what you want---highly inefficient)
\item Reduces transaction costs
\item Requires acceptability
\end{itemize}

\vspace{0.3em}
\textbf{2. Unit of Account}
\begin{itemize}
\item Simplifies price comparisons (imagine comparing prices when bread costs 3 chickens, milk costs 2 eggs, and eggs cost 0.5 chickens---money provides one common yardstick)
\item Enables economic calculation
\item Reduces cognitive costs
\end{itemize}

\column{0.48\textwidth}
\textbf{3. Store of Value}
\begin{itemize}
\item Preserves purchasing power
\item Enables intertemporal trade
\item Requires stability
\end{itemize}

\vspace{0.5em}
\textbf{Digital Finance Challenge}

Do cryptocurrencies fulfill these functions?
\begin{itemize}
\item Bitcoin: Limited as medium (volatility---rapid, unpredictable price swings)
\item Stablecoins: Better but trust issues
\item CBDCs: Designed to fulfill all three
\end{itemize}
\end{columns}

\bottomnote{Jevons (1875), a foundational economist: ``Money is what money does''---evaluate money by how well it performs these three functions}
\end{frame}

% Four Economic Lenses
\begin{frame}[t]{Four Economic Lenses for Digital Finance}
\begin{center}
\includegraphics[width=0.55\textwidth]{02_digital_finance_taxonomy/chart.pdf}
\end{center}

\bottomnote{This course applies all four lenses to understand digital finance comprehensively}
\end{frame}

% Lens 1: Monetary Economics
\begin{frame}[t]{Lens 1: Monetary Economics}
\begin{columns}[T]
\column{0.48\textwidth}
\textbf{Key Questions}
\begin{itemize}
\item How do digital currencies affect money supply?
\item What happens to monetary policy transmission?
\item Can cryptocurrencies replace fiat money (government-issued currency like dollars)?
\end{itemize}

\vspace{0.3em}
\textbf{Theoretical Tools}
\begin{itemize}
\item Quantity theory of money (relationship between money supply, prices, and output)
\item Money demand functions (models of how much money people want to hold)
\item Currency substitution models (when people switch from one currency to another)
\end{itemize}

\column{0.48\textwidth}
\textbf{Key Concepts}
\begin{itemize}
\item Seigniorage (profit from issuing currency) and its distribution
\item Velocity of money (how fast money circulates in the economy) in digital systems
\item Gresham's Law (bad money drives out good)
\end{itemize}

\vspace{0.3em}
\textbf{Applications}
\begin{itemize}
\item CBDC design trade-offs
\item Stablecoin stability mechanisms
\item Dollarization vs. crypto-ization
\end{itemize}
\end{columns}

\bottomnote{Lessons 2-3 focus on monetary economics of digital currencies and CBDCs}
\end{frame}

% Lens 2: Platform Economics
\begin{frame}[t]{Lens 2: Platform Economics}
\begin{columns}[T]
\column{0.48\textwidth}
\textbf{Key Questions}
\begin{itemize}
\item Why do some cryptocurrencies dominate?
\item How do network effects shape adoption?
\item What determines token value?
\end{itemize}

\vspace{0.3em}
\textbf{Theoretical Tools}
\begin{itemize}
\item Network effects models
\item Two-sided market theory
\item Mechanism design (designing rules/incentives to achieve desired outcomes)
\end{itemize}

\column{0.48\textwidth}
\textbf{Key Concepts}
\begin{itemize}
\item Critical mass (minimum users needed for viability) and tipping points (moments when adoption accelerates rapidly)
\item Winner-take-all dynamics
\item Platform governance
\end{itemize}

\vspace{0.3em}
\textbf{Applications}
\begin{itemize}
\item Token economics (design of digital token value and incentives) design
\item Blockchain adoption dynamics
\item DeFi protocol competition
\end{itemize}
\end{columns}

\bottomnote{Lessons 4-5 apply platform economics to payments and token systems}
\end{frame}

% Lens 3: Market Microstructure
\begin{frame}[t]{Lens 3: Market Microstructure}
\begin{columns}[T]
\column{0.48\textwidth}
\textbf{Key Questions}
\begin{itemize}
\item How do crypto markets discover prices?
\item Why are spreads wider in crypto?
\item How do Automated Market Makers (AMMs) differ from order books?
\end{itemize}

\vspace{0.3em}
\textbf{Theoretical Tools}
\begin{itemize}
\item Bid-ask spread (difference between buy and sell prices) models
\item Liquidity provision theory (how market makers supply tradability)
\item Information asymmetry (when one party knows more than another) models
\end{itemize}

\column{0.48\textwidth}
\textbf{Key Concepts}
\begin{itemize}
\item Market making (providing buy/sell offers) and inventory risk (risk from holding assets)
\item Price impact (how trades move prices) and slippage (difference between expected and actual price)
\item Impermanent loss (temporary value loss from providing liquidity) in AMMs
\end{itemize}

\vspace{0.3em}
\textbf{Applications}
\begin{itemize}
\item Decentralized Exchange (DEX---blockchain-based) vs. Centralized Exchange (CEX---company-run) efficiency
\item MEV (Maximal Extractable Value---profit from reordering transactions)
\item Market manipulation detection
\end{itemize}
\end{columns}

\bottomnote{Lesson 6 provides deep dive into market microstructure of digital finance}
\end{frame}

% Lens 4: Regulatory Economics
\begin{frame}[t]{Lens 4: Regulatory Economics}
\begin{columns}[T]
\column{0.48\textwidth}
\textbf{Key Questions}
\begin{itemize}
\item What market failures justify regulation?
\item How should crypto be classified legally?
\item What are costs of regulatory arbitrage?
\end{itemize}

\vspace{0.3em}
\textbf{Theoretical Tools}
\begin{itemize}
\item Market failure analysis
\item Public interest (regulation benefits society) vs. capture theory (special interests control regulation)
\item Cost-benefit analysis
\end{itemize}

\column{0.48\textwidth}
\textbf{Key Concepts}
\begin{itemize}
\item Asymmetric information
\item Systemic risk (risk of entire system failing) externalities (spillover costs to third parties)
\item Consumer protection rationale
\end{itemize}

\vspace{0.3em}
\textbf{Applications}
\begin{itemize}
\item Principles vs. rules-based regulation
\item Regulatory sandbox (safe testing environment for new products) design
\item Regulatory arbitrage (exploiting differences across jurisdictions)
\end{itemize}
\end{columns}

\bottomnote{Lesson 7 applies regulatory economics; Lesson 8 synthesizes all four lenses}
\end{frame}

% Why Economists Should Care
\begin{frame}[t]{Why Economists Should Study Digital Finance}
\begin{columns}[T]
\column{0.48\textwidth}
\textbf{Disruption Potential}
\begin{itemize}
\item \$15+ trillion digital payments by 2027
\item 130+ countries exploring CBDCs
\item DeFi challenging traditional finance
\end{itemize}

\vspace{0.3em}
\textbf{Policy Relevance}
\begin{itemize}
\item Central banks need economic analysis
\item Regulators need welfare frameworks
\item Governments need tax policy guidance
\end{itemize}

\column{0.48\textwidth}
\textbf{Theoretical Innovation}
\begin{itemize}
\item New forms of money creation
\item Novel market mechanisms (AMMs)
\item Programmable financial contracts
\end{itemize}

\vspace{0.3em}
\textbf{Research Opportunities}
\begin{itemize}
\item High-frequency blockchain data
\item Natural experiments in adoption
\item Cross-country regulatory variation
\end{itemize}
\end{columns}

\bottomnote{Digital finance is a laboratory for testing economic theories with real-world data}
\end{frame}

% Global Digital Payments
\begin{frame}[t]{The Scale of Digital Finance Transformation}
\begin{center}
\includegraphics[width=0.62\textwidth]{03_global_digital_payments/chart.pdf}
\end{center}

\bottomnote{COVID-19 accelerated digital payment adoption; economists must understand these trends}
\end{frame}

% Economic vs Technical Questions
\begin{frame}[t]{Economic Questions vs. Technical Questions}
\begin{columns}[T]
\column{0.48\textwidth}
\textbf{Technical Questions}
\begin{itemize}
\item How does proof-of-work function?
\item What is a smart contract?
\item How do hash functions secure data?
\item What programming languages are used?
\end{itemize}

\vspace{0.3em}
\textit{Focus: Mechanisms and implementation}

\column{0.48\textwidth}
\textbf{Economic Questions}
\begin{itemize}
\item Why do miners invest in PoW systems?
\item How do smart contracts reduce costs?
\item What incentives secure the network?
\item Who benefits from decentralization (distributing control away from central authorities)?
\end{itemize}

\vspace{0.3em}
\textit{Focus: Incentives and welfare}
\end{columns}

\vspace{0.5em}
\begin{center}
\textbf{This course focuses on economic analysis, not technical implementation}
\end{center}

\bottomnote{You don't need to understand HOW proof-of-work or hash functions work---just know they exist so you can see what economists focus on instead}
\end{frame}

% Key Takeaways
\begin{frame}[t]{Key Takeaways}
\begin{columns}[T]
\column{0.48\textwidth}
\textbf{What We Covered}
\begin{enumerate}
\item Digital finance defined economically
\item Historical context of money evolution
\item Four economic lenses framework
\item Why economic analysis matters
\end{enumerate}

\column{0.48\textwidth}
\textbf{Looking Ahead}
\begin{itemize}
\item L02: Monetary economics of crypto
\item L03: CBDCs and monetary policy
\item L04: Payment systems economics
\item L05-L08: Further applications
\end{itemize}
\end{columns}

\vspace{0.5em}
\textbf{Core Message}

Digital finance raises fundamental economic questions about money, markets, platforms, and regulation. This course provides the analytical tools to address them.

\bottomnote{Next lesson: Monetary Economics of Digital Currencies}
\end{frame}

% Key Terms
\begin{frame}[t]{Key Terms}
\begin{columns}[T]
\column{0.48\textwidth}
\textbf{Blockchain}
Distributed digital ledger recording transactions across many computers.

\vspace{0.3em}
\textbf{CBDC}
Central Bank Digital Currency; digital form of official currency.

\vspace{0.3em}
\textbf{Cryptocurrency}
Digital asset using cryptography, not issued by government.

\vspace{0.3em}
\textbf{Decentralization}
Distribution of power away from single authority to many participants.

\vspace{0.3em}
\textbf{DeFi}
Decentralized Finance; blockchain financial services without intermediaries.

\vspace{0.3em}
\textbf{Digital Finance}
Financial services relying on digital infrastructure for value transfer.

\vspace{0.3em}
\textbf{Disintermediation}
Removal of intermediaries like banks from transactions.

\vspace{0.3em}
\textbf{Externality}
Cost or benefit affecting parties outside a transaction.

\vspace{0.3em}
\textbf{Fiat Money}
Government-issued currency not backed by physical commodity.

\vspace{0.3em}
\textbf{Liquidity}
How easily an asset trades without affecting its price.

\column{0.48\textwidth}
\textbf{Market Failure}
When free markets fail to allocate resources efficiently.

\vspace{0.3em}
\textbf{Market Microstructure}
How trading mechanisms affect price formation and efficiency.

\vspace{0.3em}
\textbf{Network Effects}
Value of a service increases as more users join.

\vspace{0.3em}
\textbf{Seigniorage}
Profit from issuing currency above production cost.

\vspace{0.3em}
\textbf{Stablecoin}
Cryptocurrency designed to maintain stable value, pegged to fiat.

\vspace{0.3em}
\textbf{Token}
Digital unit of value representing assets, rights, or access.

\vspace{0.3em}
\textbf{Transaction Costs}
All exchange costs: time, fees, search, and friction.

\vspace{0.3em}
\textbf{Two-Sided Market}
Platform connecting distinct groups providing mutual network benefits.

\vspace{0.3em}
\textbf{Volatility}
Degree of price fluctuation; high volatility means rapid changes.

\vspace{0.3em}
\textbf{Welfare}
Total societal well-being; measures efficiency plus fairness.
\end{columns}

\bottomnote{Master these terms before proceeding to subsequent lessons}
\end{frame}

% References
\begin{frame}[t]{Further Reading}
\begin{columns}[T]
\column{0.48\textwidth}
\textbf{Foundational Papers}
\begin{itemize}
\item Brunnermeier \& Niepelt (2019): ``On the Equivalence of Private and Public Money''
\item Catalini \& Gans (2020): ``Some Simple Economics of the Blockchain''
\end{itemize}

\column{0.48\textwidth}
\textbf{Policy Reports}
\begin{itemize}
\item BIS (Bank for International Settlements) Annual Economic Report (2022), Chapter III
\item IMF (International Monetary Fund) Global Financial Stability Report (2023)
\item FSB (Financial Stability Board) Crypto-asset Reports (2022-2023)
\end{itemize}
\end{columns}

\bottomnote{All readings available on course platform}
\end{frame}

\end{document}
