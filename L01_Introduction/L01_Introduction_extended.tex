\documentclass[8pt,aspectratio=169]{beamer}
\usetheme{Madrid}
\usepackage{graphicx}
\usepackage{booktabs}
\usepackage{adjustbox}
\usepackage{multicol}
\usepackage{amsmath}

% Color definitions
\definecolor{mlblue}{RGB}{0,102,204}
\definecolor{mlpurple}{RGB}{51,51,178}
\definecolor{mllavender}{RGB}{173,173,224}
\definecolor{mllavender2}{RGB}{193,193,232}
\definecolor{mllavender3}{RGB}{204,204,235}
\definecolor{mllavender4}{RGB}{214,214,239}
\definecolor{mlorange}{RGB}{255, 127, 14}
\definecolor{mlgreen}{RGB}{44, 160, 44}
\definecolor{mlred}{RGB}{214, 39, 40}
\definecolor{mlgray}{RGB}{127, 127, 127}

\definecolor{lightgray}{RGB}{240, 240, 240}
\definecolor{midgray}{RGB}{180, 180, 180}

\setbeamercolor{palette primary}{bg=mllavender3,fg=mlpurple}
\setbeamercolor{palette secondary}{bg=mllavender2,fg=mlpurple}
\setbeamercolor{palette tertiary}{bg=mllavender,fg=white}
\setbeamercolor{palette quaternary}{bg=mlpurple,fg=white}

\setbeamercolor{structure}{fg=mlpurple}
\setbeamercolor{section in toc}{fg=mlpurple}
\setbeamercolor{subsection in toc}{fg=mlblue}
\setbeamercolor{title}{fg=mlpurple}
\setbeamercolor{frametitle}{fg=mlpurple,bg=mllavender3}
\setbeamercolor{block title}{bg=mllavender2,fg=mlpurple}
\setbeamercolor{block body}{bg=mllavender4,fg=black}

\setbeamertemplate{navigation symbols}{}
\setbeamertemplate{itemize items}[circle]
\setbeamertemplate{enumerate items}[default]
\setbeamersize{text margin left=5mm,text margin right=5mm}

\newcommand{\bottomnote}[1]{%
\vfill
\vspace{-2mm}
\textcolor{mllavender2}{\rule{\textwidth}{0.4pt}}
\vspace{1mm}
\footnotesize
\textbf{#1}
}

\title{Introduction to Digital Finance: Mathematical Models and Technology Adoption}
\subtitle{L01 Extended: Formalizing the S-Curve and Economics of Money\\[0.3em]\normalsize From barter inefficiency to Metcalfe's Law and diffusion dynamics}
\author{Economics of Digital Finance}
\institute{BSc Course}
\date{}

\begin{document}

%% ============================================================
%% SECTION 1: Bridge from Basic Lecture
%% ============================================================
\section{Bridge from Basic Lecture}

% Frame 0 (title)
\begin{frame}[plain]
\titlepage
\end{frame}

% Frame 1
\begin{frame}[t]{Welcome Back: From Concepts to Math}
\begin{center}
\vspace{1em}
{\Large\textit{``In God we trust; all others must bring data.''}}\\[0.5em]
{\normalsize --- W.\ Edwards Deming}

\vspace{1.5em}
{\large In L01, we introduced \textbf{four economic lenses} and the \textbf{S-curve} of technology adoption.}

\vspace{1em}
{\large Now we formalize these ideas with \textbf{mathematical models} that let us:}

\vspace{0.5em}
\begin{itemize}
\item \textbf{Predict} --- When will a payment technology reach critical mass?
\item \textbf{Quantify} --- How much does money reduce transaction costs versus barter?
\item \textbf{Compare} --- Is a network worth $n^2$ or $n \ln(n)$?
\item \textbf{Test} --- Do the models fit real-world data?
\end{itemize}

\vspace{1em}
{\footnotesize (Relevant XKCD: \#435 --- ``Purity'' --- mathematics as the foundation of all sciences)}
\end{center}

\bottomnote{This extended lecture adds formal models to the conceptual framework from L01}
\end{frame}

% Frame 2
\begin{frame}[t]{From Concepts to Models}
\begin{columns}[T]
\column{0.48\textwidth}
\textbf{What We Know (from L01)}
\begin{enumerate}
\item S-curve adoption patterns in payment technology
\item Four economic lenses (monetary, platform, microstructure, regulatory)
\item Historical evolution from barter to digital payments
\item Digital finance as economic disruption
\end{enumerate}

\vspace{0.5em}
\textit{These are qualitative insights --- useful for intuition but insufficient for precise analysis.}

\column{0.48\textwidth}
\textbf{What We'll Formalize (today)}
\begin{enumerate}
\item \textbf{Kiyotaki-Wright barter model}: Why money emerges as an equilibrium outcome
\item \textbf{Bass diffusion model}: S-curve dynamics with calibrated parameters
\item \textbf{Metcalfe/Odlyzko valuation}: Network value scaling laws
\item \textbf{Coase transaction costs}: Digital finance cost decomposition
\end{enumerate}

\vspace{0.5em}
\textit{Formal models let us make testable predictions and evaluate policy.}
\end{columns}

\bottomnote{Models are simplified representations of reality --- useful precisely because they omit irrelevant details}
\end{frame}

% Frame 3 --- NO GREEK LETTERS
\begin{frame}[t]{Mathematical Toolkit}
\begin{columns}[T]
\column{0.48\textwidth}
\textbf{Key Concepts (in plain English)}
\begin{itemize}
\item \textbf{Optimization}: Finding the best choice given constraints (e.g., minimizing cost)
\item \textbf{Equilibrium}: A stable state where no participant wants to change behavior
\item \textbf{Differential equations}: Rules describing how quantities change over time (e.g., adoption rate depends on current adoption level)
\item \textbf{Comparative statics}: Asking ``what happens to the answer if we change one input?''
\end{itemize}

\column{0.48\textwidth}
\textbf{Notation Guide}

\vspace{0.3em}
\begin{tabular}{@{}ll@{}}
\toprule
\textbf{Symbol} & \textbf{Meaning} \\
\midrule
$n$ & Number of goods or network users \\
$c_b$ & Cost per barter trade (\$/trade) \\
$c_m$ & Cost per money trade (\$/trade) \\
$V$ & Network value (\$) \\
$t$ & Time (years) \\
$p$ & Innovation coefficient (Bass) \\
$q$ & Imitation coefficient (Bass) \\
$F$ & Cumulative adoption fraction \\
$TC$ & Total transaction cost (\$) \\
\bottomrule
\end{tabular}

\vspace{0.3em}
\footnotesize Greek letters (introduced in later slides) will always be defined at first use.
\end{columns}

\bottomnote{No prior math beyond basic algebra is assumed --- every formula is derived step by step}
\end{frame}

%% ============================================================
%% SECTION 2: Barter Inefficiency and the Emergence of Money
%% ============================================================
\section{Barter Inefficiency and the Emergence of Money}

% Frame 4
\begin{frame}[t]{The Barter Problem --- Formal Setup}
\begin{columns}[T]
\column{0.48\textwidth}
\textbf{The Double Coincidence Problem}

In an economy with $n$ goods, barter requires:
\begin{itemize}
\item Each pair of goods needs its own exchange rate
\item Number of exchange rates: $P_{barter}(n) = \frac{n(n-1)}{2}$
\item With money: only $P_{money}(n) = n - 1$ prices needed
\end{itemize}

\vspace{0.3em}
\textbf{Introducing} $\alpha$ (alpha):
\begin{itemize}
\item $\alpha$ = probability that two random traders each want what the other has (``double coincidence of wants'')
\item With $n$ goods: $\alpha \approx 1/n^2$ (very small for large $n$)
\item Money eliminates the need for double coincidence
\end{itemize}

\column{0.48\textwidth}
\textbf{Worked Example: $n = 10$ goods}

\vspace{0.3em}
\textit{Under barter:}
\begin{itemize}
\item Exchange rates: $\frac{10 \times 9}{2} = 45$
\item You must memorize or look up 45 relative prices
\item Probability of match: $\alpha \approx 1/100 = 1\%$
\end{itemize}

\vspace{0.3em}
\textit{With money (one medium of exchange):}
\begin{itemize}
\item Prices: $10 - 1 = 9$
\item Reduction: $45 - 9 = 36$ fewer prices (80\% reduction)
\item Any trade is possible through money
\end{itemize}

\vspace{0.3em}
\textit{At $n = 100$ goods:}
\begin{itemize}
\item Barter: 4,950 exchange rates
\item Money: 99 prices (98\% reduction)
\end{itemize}
\end{columns}

\bottomnote{Jevons (1875) first formalized why barter is inefficient; Kiyotaki \& Wright (1989) proved money emerges as equilibrium}
\end{frame}

% Frame 5
\begin{frame}[t]{Money as Information Technology}
\begin{columns}[T]
\column{0.48\textwidth}
\textbf{Kiyotaki-Wright (1989) Search Model}

\vspace{0.3em}
Key setup:
\begin{itemize}
\item Agents are randomly matched each period
\item Each agent produces one good, consumes another
\item Without money: agents search for direct swap partners
\end{itemize}

\vspace{0.3em}
\textbf{Introducing} $\pi$ (pi):
\begin{itemize}
\item $\pi$ = search cost per period (time, effort, travel)
\item Expected search time under barter: $1/\alpha$ periods
\item Expected total barter cost: $\pi / \alpha$
\item With money: search cost drops to $\pi$ (one search)
\end{itemize}

\vspace{0.3em}
\textbf{Why does money emerge?}\\
When $\pi / \alpha > \pi + c_m$ (barter search cost exceeds money search cost plus money holding cost $c_m$), all agents prefer to accept money.

\column{0.48\textwidth}
\textbf{Worked Example}

\vspace{0.3em}
Parameters:
\begin{itemize}
\item $n = 20$ goods, so $\alpha = 1/400$
\item $\pi = \$2$ per search period
\item $c_m = \$0.50$ (cost of carrying money)
\end{itemize}

\vspace{0.3em}
\textit{Barter cost per trade:}
$$\frac{\pi}{\alpha} = \frac{2}{1/400} = \$800$$

\vspace{0.3em}
\textit{Money cost per trade:}
$$\pi + c_m = 2 + 0.50 = \$2.50$$

\vspace{0.3em}
\textit{Savings:}
$$\$800 - \$2.50 = \$797.50 \text{ per trade}$$

\vspace{0.3em}
Money reduces transaction costs by \textbf{99.7\%} in this example.
\end{columns}

\bottomnote{Kiyotaki \& Wright (1989): money is an equilibrium institution, not an arbitrary convention}
\end{frame}

% Frame 6
\begin{frame}[t]{Barter Cost Reduction}
\begin{center}
\includegraphics[width=0.55\textwidth]{06_barter_cost_reduction/chart.pdf}
\end{center}

\begin{itemize}
\item Panel (a): Barter exchange rates grow quadratically ($n^2/2$) while money prices grow linearly ($n-1$) --- the gap widens dramatically
\item Panel (b): Transaction cost savings from money (green area) increase rapidly; at $n{=}50$, barter costs \$15,313 vs money costs \$24.50 per round
\item Parameters: $c_b = \$5.00$/trade (barter), $c_m = \$0.50$/trade (money) --- calibrated from Kiyotaki \& Wright (1989)
\end{itemize}

\bottomnote{Money is the oldest and most successful financial technology --- digital money extends the same logic}
\end{frame}

% Frame 7
\begin{frame}[t]{Search Equilibrium Results}
\begin{columns}[T]
\column{0.48\textwidth}
\textbf{Equilibrium Conditions}

\vspace{0.3em}
Money emerges when:
$$\frac{\pi}{\alpha} > \pi + c_m$$

Rearranging:
$$\alpha < \frac{\pi}{\pi + c_m}$$

This means: money is adopted when the probability of a double coincidence ($\alpha$) is sufficiently low --- i.e., when the economy is sufficiently complex.

\vspace{0.3em}
\textbf{Digital extension:}
\begin{itemize}
\item Digital money reduces $c_m$ further (near zero storage/transfer cost)
\item This makes the money equilibrium even more dominant
\end{itemize}

\column{0.48\textwidth}
\textbf{Numerical Results}

\vspace{0.3em}
\begin{adjustbox}{max width=\textwidth}
\begin{tabular}{@{}lccc@{}}
\toprule
\textbf{Goods} & $\alpha$ & \textbf{Barter} & \textbf{Money} \\
$n$ & $(1/n^2)$ & \textbf{Cost} & \textbf{Cost} \\
\midrule
5 & 0.040 & \$50 & \$2.50 \\
10 & 0.010 & \$200 & \$2.50 \\
20 & 0.003 & \$800 & \$2.50 \\
50 & 0.000 & \$5,000 & \$2.50 \\
100 & 0.000 & \$20,000 & \$2.50 \\
\bottomrule
\end{tabular}
\end{adjustbox}

\vspace{0.5em}
\footnotesize
$\pi = \$2.00$, $c_m = \$0.50$.\\
Barter cost = $\pi/\alpha$; Money cost = $\pi + c_m$.

\vspace{0.3em}
\normalsize
\textbf{Key insight}: Money's advantage grows quadratically with economic complexity. Digital money amplifies this by reducing $c_m$ to near zero.
\end{columns}

\bottomnote{Even with just 10 goods, money reduces search costs by 98.75\% --- the case for digital money is even stronger}
\end{frame}

%% ============================================================
%% SECTION 3: Technology Diffusion: The Bass Model
%% ============================================================
\section{Technology Diffusion: The Bass Model}

% Frame 8
\begin{frame}[t]{The Bass Diffusion Model --- Setup}
\begin{columns}[T]
\column{0.48\textwidth}
\textbf{Bass (1969) Model}

\vspace{0.3em}
The adoption rate depends on two forces:

$$\frac{dF}{dt} = (p + qF)(1 - F)$$

where:
\begin{itemize}
\item $F(t)$ = fraction of population that has adopted (0 to 1)
\item $p$ = innovation coefficient (external influence: advertising, media)
\item $q$ = imitation coefficient (internal influence: word-of-mouth, network effects)
\item $(1-F)$ = remaining potential adopters
\end{itemize}

\vspace{0.3em}
\textbf{Introducing} $\tau$ (tau):
\begin{itemize}
\item $\tau$ = time to inflection (steepest adoption)
\item Inflection when $F = (1 - p/q)/2$ approximately
\end{itemize}

\column{0.48\textwidth}
\textbf{Intuition}

\vspace{0.3em}
The $(p + qF)$ term captures:
\begin{itemize}
\item $p$: ``innovators'' who adopt independently
\item $qF$: ``imitators'' who adopt because others did (grows as $F$ increases)
\end{itemize}

\vspace{0.3em}
\textbf{Parameter meaning:}

\vspace{0.3em}
\begin{tabular}{@{}lll@{}}
\toprule
& \textbf{Low} & \textbf{High} \\
\midrule
$p$ & Slow start & Fast start \\
$q$ & Weak network & Strong network \\
$q/p$ & Tech-driven & Social-driven \\
\bottomrule
\end{tabular}

\vspace{0.5em}
Digital payment technologies have \textit{higher} $p$ (better marketing/UX) and \textit{higher} $q$ (stronger network effects) than earlier technologies.
\end{columns}

\bottomnote{Bass (1969): the most widely used model for forecasting technology adoption timing}
\end{frame}

% Frame 9
\begin{frame}[t]{Fitting the S-Curve}
\begin{columns}[T]
\column{0.48\textwidth}
\textbf{Closed-Form Solution}

\vspace{0.3em}
The differential equation has an exact solution:

$$F(t) = \frac{1 - e^{-(p+q)t}}{1 + \frac{q}{p} e^{-(p+q)t}}$$

\vspace{0.3em}
\textbf{Peak adoption time:}
$$t^* = \frac{\ln(q/p)}{p + q}$$

At $t^*$, the adoption rate $dF/dt$ reaches its maximum --- this is the ``tipping point.''

\vspace{0.3em}
\textbf{Peak adoption rate:}
$$\left.\frac{dF}{dt}\right|_{t^*} = \frac{(p+q)^2}{4q}$$

\column{0.48\textwidth}
\textbf{Worked Example: Digital Payments}

\vspace{0.3em}
Parameters: $p = 0.03$, $q = 0.50$

\vspace{0.3em}
\textit{Peak time:}
$$t^* = \frac{\ln(0.50/0.03)}{0.03 + 0.50} = \frac{2.81}{0.53} = 5.3 \text{ years}$$

\vspace{0.3em}
\textit{Peak rate:}
$$\frac{(0.03 + 0.50)^2}{4 \times 0.50} = \frac{0.281}{2.0} = 14.0\% \text{/year}$$

\vspace{0.3em}
\textit{Compare with Cash} ($p=0.01$, $q=0.20$):
$$t^*_{cash} = \frac{\ln(0.20/0.01)}{0.21} = \frac{3.00}{0.21} = 14.3 \text{ years}$$

\vspace{0.3em}
Digital payments reach peak adoption \textbf{2.7$\times$ faster} than cash equivalents.
\end{columns}

\bottomnote{The ratio $q/p$ determines whether adoption is innovation-driven ($q/p < 1$) or imitation-driven ($q/p > 1$)}
\end{frame}

% Frame 10
\begin{frame}[t]{Diffusion Dynamics Visualization}
\begin{center}
\includegraphics[width=0.55\textwidth]{07_bass_diffusion_payment/chart.pdf}
\end{center}

\begin{itemize}
\item Panel (a): Cumulative S-curves show digital payments reaching 50\% adoption in 5.3 years vs 14.3 years for cash --- network effects ($q$) are the accelerator
\item Panel (b): Adoption rate curves peak earlier and higher for digital ($dF/dt$ max = 14\%/year) vs cash (max = 5.3\%/year)
\item Dots mark inflection points $t^*$ --- the moment adoption accelerates most rapidly
\end{itemize}

\bottomnote{Bass parameters: Cash ($p{=}0.01$, $q{=}0.20$), Cards ($p{=}0.02$, $q{=}0.35$), Digital ($p{=}0.03$, $q{=}0.50$)}
\end{frame}

% Frame 11
\begin{frame}[t]{Speed of Adoption: Why Digital Is Faster}
\begin{columns}[T]
\column{0.48\textwidth}
\textbf{Comparative Diffusion Parameters}

\vspace{0.3em}
\begin{adjustbox}{max width=\textwidth}
\begin{tabular}{@{}lcccc@{}}
\toprule
\textbf{Technology} & $p$ & $q$ & $q/p$ & $t^*$ (yr) \\
\midrule
Cash & 0.01 & 0.20 & 20 & 14.3 \\
Cards & 0.02 & 0.35 & 17.5 & 7.6 \\
Digital & 0.03 & 0.50 & 16.7 & 5.3 \\
\bottomrule
\end{tabular}
\end{adjustbox}

\vspace{0.5em}
\textbf{Why is digital faster?}
\begin{itemize}
\item Higher $p$: better initial UX, lower friction, smartphone penetration
\item Higher $q$: stronger network effects (instant, global, interoperable)
\item Lower $q/p$: more balanced adoption (both innovators and imitators contribute)
\end{itemize}

\column{0.48\textwidth}
\textbf{Implications for Policy}

\vspace{0.3em}
\begin{itemize}
\item Fast adoption creates \textbf{regulatory lag} --- rules cannot keep up with technology
\item Network effects create \textbf{winner-take-all} dynamics --- early movers have enormous advantages
\item High $q$ means \textbf{tipping points are abrupt} --- small changes in conditions can trigger rapid shifts
\end{itemize}

\vspace{0.5em}
\textbf{Historical comparison:}
\begin{itemize}
\item Telephone: $\sim$75 years to 50\% adoption
\item Internet: $\sim$15 years
\item Smartphones: $\sim$10 years
\item Mobile payments: $\sim$5--7 years
\end{itemize}

\vspace{0.3em}
Each generation adopts faster because $q$ increases with digital connectivity.
\end{columns}

\bottomnote{Rogers (1962): adoption speed depends on relative advantage, compatibility, complexity, trialability, and observability}
\end{frame}

%% ============================================================
%% SECTION 4: Network Value Estimation: Metcalfe and Beyond
%% ============================================================
\section{Network Value Estimation: Metcalfe and Beyond}

% Frame 12
\begin{frame}[t]{Metcalfe's Law --- Formal Derivation}
\begin{columns}[T]
\column{0.48\textwidth}
\textbf{The Argument}

\vspace{0.3em}
In a network with $n$ users:
\begin{itemize}
\item Each user can connect with $n - 1$ others
\item Total possible connections: $\frac{n(n-1)}{2}$
\item If each connection has value $v$, total value:
\end{itemize}
$$V = \frac{n(n-1)}{2} \cdot v \approx \alpha n^2$$

where $\alpha = v/2$ (for large $n$).

\vspace{0.3em}
\textbf{Calibration:}
\begin{itemize}
\item $\alpha = 0.001$ (fitted to crypto network data)
\item At $n = 1{,}000$: $V = 0.001 \times 10^6 = 1{,}000$
\item At $n = 10{,}000$: $V = 100{,}000$ (100$\times$ more)
\end{itemize}

\column{0.48\textwidth}
\textbf{Worked Example: Bitcoin Network}

\vspace{0.3em}
\textit{2013}: $n \approx 100{,}000$ active addresses
$$V_{Metcalfe} = 0.001 \times (10^5)^2 = \$10 \text{ billion}$$

\vspace{0.3em}
\textit{2021}: $n \approx 1{,}000{,}000$ active addresses
$$V_{Metcalfe} = 0.001 \times (10^6)^2 = \$1 \text{ trillion}$$

\vspace{0.3em}
\textit{Actual market cap (peak 2021):} $\sim$\$1.2 trillion

\vspace{0.5em}
\textbf{Key insight:}\\
Metcalfe's Law predicts that a 10$\times$ increase in users leads to a 100$\times$ increase in value. This explains why crypto projects obsess over user growth.
\end{columns}

\bottomnote{Metcalfe (2013): validated $V \propto n^2$ for Facebook data; crypto markets show similar scaling}
\end{frame}

% Frame 13
\begin{frame}[t]{Beyond Metcalfe: Odlyzko-Tilly and Zipf}
\begin{columns}[T]
\column{0.48\textwidth}
\textbf{The Critique of $n^2$}

\vspace{0.3em}
Odlyzko \& Tilly (2005) argued:
\begin{itemize}
\item Not all connections are equally valuable
\item Value follows Zipf's Law: the $k$-th most valuable connection is worth $\sim v/k$
\item Total value sums a harmonic series:
\end{itemize}
$$V_{Odlyzko} = \beta \cdot n \cdot \ln(n)$$

where $\beta = 0.05$.

\vspace{0.3em}
\textbf{Comparison at $n = 1{,}000{,}000$:}
\begin{itemize}
\item Metcalfe: $V = 0.001 \times 10^{12} = 10^9$
\item Odlyzko: $V = 0.05 \times 10^6 \times 13.8 = 6.9 \times 10^5$
\item Ratio: Metcalfe is $\sim$1,450$\times$ higher
\end{itemize}

\column{0.48\textwidth}
\textbf{When Each Model Applies}

\vspace{0.3em}
\begin{adjustbox}{max width=\textwidth}
\begin{tabular}{@{}lll@{}}
\toprule
\textbf{Model} & \textbf{Scaling} & \textbf{Best for} \\
\midrule
Metcalfe & $n^2$ & Small networks \\
 & & (all connections used) \\
Odlyzko & $n \ln(n)$ & Large networks \\
 & & (diminishing returns) \\
Linear & $n$ & Utility networks \\
 & & (value per user fixed) \\
\bottomrule
\end{tabular}
\end{adjustbox}

\vspace{0.5em}
\textbf{Practical rule of thumb:}
\begin{itemize}
\item $n < 10{,}000$: Metcalfe is reasonable
\item $10{,}000 < n < 1{,}000{,}000$: Odlyzko is more realistic
\item $n > 1{,}000{,}000$: consider saturation effects
\end{itemize}

\vspace{0.3em}
For crypto valuation, Odlyzko typically fits better because not all wallet addresses actively transact.
\end{columns}

\bottomnote{Briscoe et al.\ (2006): the true scaling law depends on the network's communication structure}
\end{frame}

% Frame 14
\begin{frame}[t]{Network Valuation Models}
\begin{center}
\includegraphics[width=0.55\textwidth]{08_network_valuation_models/chart.pdf}
\end{center}

\begin{itemize}
\item Panel (a): On log-log axes, Metcalfe ($n^2$) grows fastest; the shaded area shows the ``Metcalfe premium'' over Odlyzko ($n \ln n$)
\item Panel (b): The ratio $V_{Metcalfe}/V_{Odlyzko}$ diverges as $n$ grows --- Metcalfe increasingly overestimates for large networks
\item Panel (c): Ethereum data (illustrative) shows Odlyzko fits better at scale as saturation effects dominate
\end{itemize}

\bottomnote{Parameters: $\alpha{=}0.001$, $\beta{=}0.05$, $\gamma{=}1.0$ --- Metcalfe (2013), Odlyzko \& Tilly (2005)}
\end{frame}

% Frame 15
\begin{frame}[t]{Empirical Network Value}
\begin{columns}[T]
\column{0.48\textwidth}
\textbf{Regression Approach}

\vspace{0.3em}
To test which model fits best, estimate:
$$\ln(V) = a + b \cdot \ln(n) + \varepsilon$$

\vspace{0.3em}
\begin{itemize}
\item If $b \approx 2$: Metcalfe ($V \propto n^2$)
\item If $b \approx 1$: linear ($V \propto n$)
\item If $b$ between 1 and 2: sub-quadratic (Odlyzko region)
\end{itemize}

\vspace{0.3em}
\textbf{Introducing} $\varepsilon$ (epsilon):
\begin{itemize}
\item $\varepsilon$ = error term capturing factors not in the model (speculation, regulation, sentiment)
\end{itemize}

\column{0.48\textwidth}
\textbf{Empirical Findings}

\vspace{0.3em}
\begin{adjustbox}{max width=\textwidth}
\begin{tabular}{@{}lccc@{}}
\toprule
\textbf{Network} & $b$ & $R^2$ & \textbf{Best fit} \\
\midrule
Bitcoin & 1.69 & 0.78 & Sub-quadratic \\
Ethereum & 1.43 & 0.82 & Odlyzko \\
Facebook & 1.95 & 0.91 & Metcalfe \\
Tencent & 1.88 & 0.89 & Near-Metcalfe \\
\bottomrule
\end{tabular}
\end{adjustbox}

\vspace{0.5em}
\textbf{Interpretation:}
\begin{itemize}
\item Social networks (Facebook, Tencent) are closer to $n^2$ --- users actively connect
\item Crypto networks are closer to $n \ln(n)$ --- many addresses are inactive or single-use
\item $R^2$ indicates how much of value variation is explained by user count alone
\end{itemize}
\end{columns}

\bottomnote{$R^2$ (R-squared) measures goodness of fit: $R^2 = 0.82$ means 82\% of value variation is explained by network size}
\end{frame}

% Frame 16
\begin{frame}[t]{When Metcalfe Breaks}
\begin{columns}[T]
\column{0.48\textwidth}
\textbf{Saturation and Critique}

\vspace{0.3em}
Metcalfe's Law breaks down when:
\begin{itemize}
\item \textbf{Congestion}: Adding users degrades experience (e.g., Ethereum gas fees rising with usage)
\item \textbf{Sybil attacks}: Fake accounts inflate $n$ without adding real value
\item \textbf{Inactive users}: Many accounts never transact --- $n_{active} \ll n_{total}$
\item \textbf{Diminishing returns}: The 1,000th connection is worth less than the 10th
\end{itemize}

\vspace{0.3em}
\textbf{Modified Metcalfe:}
$$V = \alpha \cdot n^b, \quad 1 < b < 2$$

The exponent $b$ captures the degree of network effect saturation.

\column{0.48\textwidth}
\textbf{Implications for Crypto Valuation}

\vspace{0.3em}
\begin{itemize}
\item Using $n^2$ overestimates value by 10--1,000$\times$ for large networks
\item Better to use $n \ln(n)$ or estimate $b$ empirically
\item User quality matters more than quantity
\end{itemize}

\vspace{0.5em}
\textbf{Real-world examples:}
\begin{itemize}
\item \textbf{MySpace} ($n{=}100$M): Metcalfe predicted enormous value, but the network collapsed when users migrated
\item \textbf{Bitcoin} ($n{=}1$M active): Metcalfe predicts \$1T; actual is volatile between \$200B and \$1.2T
\end{itemize}

\vspace{0.5em}
\textit{Cross-reference: For equilibrium analysis of network adoption under fulfilled expectations, see L05 Extended.}
\end{columns}

\bottomnote{Metcalfe gives an upper bound; Odlyzko gives a more realistic estimate for mature networks}
\end{frame}

%% ============================================================
%% SECTION 5: Transaction Cost Economics of Digital Finance
%% ============================================================
\section{Transaction Cost Economics of Digital Finance}

% Frame 17
\begin{frame}[t]{Coase and Transaction Costs --- Setup}
\begin{columns}[T]
\column{0.48\textwidth}
\textbf{Coase (1937): The Nature of the Firm}

\vspace{0.3em}
Why do firms exist? Because markets have \textbf{transaction costs}:
\begin{itemize}
\item Finding a trading partner (search costs)
\item Verifying quality and identity (verification costs)
\item Executing and recording the exchange (settlement costs)
\item Following rules and reporting (compliance costs)
\end{itemize}

$$TC_{total} = TC_{search} + TC_{verify} + TC_{settle} + TC_{comply}$$

\vspace{0.3em}
\textbf{Introducing} $\gamma$ (gamma):
\begin{itemize}
\item $\gamma$ = technology efficiency parameter
\item Digital reduces each component: $TC_i^{digital} = \gamma_i \cdot TC_i^{traditional}$
\item Where $0 < \gamma_i < 1$ (lower is better)
\end{itemize}

\column{0.48\textwidth}
\textbf{Williamson (1985) Categories}

\vspace{0.3em}
\begin{adjustbox}{max width=\textwidth}
\begin{tabular}{@{}lp{4.5cm}@{}}
\toprule
\textbf{Category} & \textbf{Digital Impact} \\
\midrule
Search & Algorithms replace brokers; near-zero marginal cost \\
\addlinespace
Verification & Cryptographic proof replaces trust; automated KYC \\
\addlinespace
Settlement & Instant (blockchain) vs 2--3 day (bank wire) \\
\addlinespace
Compliance & Smart contracts automate; but regulatory burden grows \\
\bottomrule
\end{tabular}
\end{adjustbox}

\vspace{0.5em}
\textbf{Key insight:}\\
Digital finance reduces search, verification, and settlement costs dramatically, but may \textit{increase} compliance costs (new regulations, AML requirements).
\end{columns}

\bottomnote{Coase (1937): transaction costs explain why firms exist; Williamson (1985) classified the types of costs}
\end{frame}

% Frame 18
\begin{frame}[t]{Digital Finance Transaction Cost Reduction}
\begin{columns}[T]
\column{0.48\textwidth}
\textbf{Cost Decomposition per \$100 Transaction}

\vspace{0.3em}
\begin{adjustbox}{max width=\textwidth}
\begin{tabular}{@{}lcccc@{}}
\toprule
\textbf{Component} & \textbf{Cash} & \textbf{Card} & \textbf{Crypto} & \textbf{CBDC} \\
\midrule
Search & \$0.50 & \$0.10 & \$0.05 & \$0.05 \\
Verification & \$0.20 & \$0.15 & \$0.10 & \$0.05 \\
Settlement & \$0.30 & \$0.25 & \$0.05 & \$0.05 \\
Compliance & \$0.00 & \$0.50 & \$0.30 & \$0.15 \\
\midrule
\textbf{Total} & \textbf{\$1.00} & \textbf{\$1.00} & \textbf{\$0.50} & \textbf{\$0.30} \\
\bottomrule
\end{tabular}
\end{adjustbox}

\vspace{0.5em}
\textbf{Key observations:}
\begin{itemize}
\item Cash has zero compliance cost (anonymous) but high search/settlement
\item Cards shift costs to compliance (fraud monitoring, chargebacks)
\item Crypto reduces settlement but adds compliance burden
\end{itemize}

\column{0.48\textwidth}
\textbf{Worked Example: \$200 Remittance}

\vspace{0.3em}
\textit{Traditional (Western Union):}
\begin{itemize}
\item Fee: \$14 (7\% of \$200)
\item Time: 1--3 business days
\item Exchange rate markup: 2--4\%
\item Total effective cost: $\sim$\$22
\end{itemize}

\vspace{0.3em}
\textit{Crypto (stablecoin transfer):}
\begin{itemize}
\item Network fee: \$0.50--\$2.00
\item Time: 2--30 minutes
\item Exchange rate: market rate
\item Total effective cost: $\sim$\$2
\end{itemize}

\vspace{0.3em}
\textit{Cost reduction: 91\%}

\vspace{0.3em}
This explains why remittance corridors are among the first use cases for digital finance (see L04).
\end{columns}

\bottomnote{World Bank (2023): global average remittance cost is 6.2\% --- crypto can reduce this to under 1\%}
\end{frame}

% Frame 19
\begin{frame}[t]{Transaction Cost Waterfall}
\begin{center}
\includegraphics[width=0.55\textwidth]{09_transaction_cost_waterfall/chart.pdf}
\end{center}

\begin{itemize}
\item Panel (a): Stacked bars decompose total cost into four Williamson categories --- CBDC achieves the lowest total at \$0.30 per \$100
\item Panel (b): Crypto and CBDC achieve 50\% and 70\% cost reduction versus cash, respectively
\item Note: Cash and Card have equal totals (\$1.00) but different cost \textit{structures} --- cards shift costs from search/settlement to compliance
\end{itemize}

\bottomnote{Transaction costs per \$100 transaction --- Coase (1937), Williamson (1985)}
\end{frame}

% Frame 20
\begin{frame}[t]{Disintermediation Economics}
\begin{columns}[T]
\column{0.48\textwidth}
\textbf{Williamson's Asset Specificity}

\vspace{0.3em}
Intermediaries (banks, exchanges) exist because of:
\begin{itemize}
\item \textbf{Asset specificity}: specialized infrastructure is expensive to build but cheap to share
\item \textbf{Uncertainty}: intermediaries absorb risk (fraud, default, settlement failure)
\item \textbf{Frequency}: high-volume transactions justify fixed costs of infrastructure
\end{itemize}

\vspace{0.3em}
\textbf{Digital finance disrupts this:}
\begin{itemize}
\item Blockchain reduces asset specificity (open infrastructure)
\item Smart contracts reduce uncertainty (automated execution)
\item APIs reduce frequency thresholds (even small transactions are economical)
\end{itemize}

\column{0.48\textwidth}
\textbf{Disintermediation vs Re-intermediation}

\vspace{0.3em}
\begin{itemize}
\item \textbf{Disintermediation}: removing banks from payments (e.g., Bitcoin peer-to-peer)
\item \textbf{Re-intermediation}: replacing old intermediaries with new ones (e.g., Coinbase replaces a traditional broker)
\end{itemize}

\vspace{0.5em}
\textbf{The paradox:}
\begin{itemize}
\item DeFi promises disintermediation
\item But users still rely on: wallets, bridges, DEX front-ends, oracles (data feeds connecting blockchain to real-world information)
\item Transaction costs shift from one intermediary to another, not to zero
\end{itemize}

\vspace{0.3em}
\textbf{Coase's insight remains:} as long as transaction costs exist, some form of intermediation is efficient.
\end{columns}

\bottomnote{Complete disintermediation is a myth --- digital finance \textit{restructures} intermediation, it does not eliminate it}
\end{frame}

%% ============================================================
%% SECTION 6: Adoption Threshold Analysis
%% ============================================================
\section{Adoption Threshold Analysis}

% Frame 21
\begin{frame}[t]{The Standards Problem in Digital Finance}
\begin{center}
\vspace{1em}
{\Large\textbf{The Competing Standards Dilemma}}

\vspace{1em}
{\large\textit{``The nice thing about standards is that there are so many to choose from.''}}\\[0.3em]
{\normalsize --- Andrew S.\ Tanenbaum (paraphrased in XKCD \#927)}

\vspace{1em}
\textbf{Situation}: 14 competing digital payment standards exist.\\
\textbf{``Solution''}: Create a universal standard that covers all use cases.\\
\textbf{Result}: Now there are 15 competing standards.

\vspace{1em}
\begin{itemize}
\item This is exactly the adoption threshold problem: each standard needs a \textbf{critical mass} of users to be viable
\item With 15 standards, users are split across too many networks --- none reaches critical mass
\item \textbf{Network effects} create winner-take-all dynamics: only 1--2 standards can survive
\item The Katz-Shapiro (1985) model formalizes when adoption succeeds vs.\ fails
\end{itemize}
\end{center}

\bottomnote{XKCD \#927 ``Standards'' --- the proliferation problem applies directly to crypto tokens, L2 chains, and payment protocols}
\end{frame}

% Frame 22
\begin{frame}[t]{Growth Accounting Framework}
\begin{columns}[T]
\column{0.48\textwidth}
\textbf{Solow-Style Decomposition}

\vspace{0.3em}
Adapting Solow's (1957) growth accounting to digital finance:

$$g_{total} = g_{tech} + g_{adopt} + g_{network} + g_{reg}$$

where each term is the contribution (in percentage points) of:
\begin{itemize}
\item $g_{tech}$: Technology infrastructure (cloud, APIs, mobile)
\item $g_{adopt}$: User adoption and penetration
\item $g_{network}$: Network effect value creation
\item $g_{reg}$: Regulatory environment (enabling or constraining)
\end{itemize}

\vspace{0.3em}
\textbf{Key difference from Solow:}\\
In traditional growth accounting, there is a residual (TFP). Here, we attribute all growth to observable factors.

\column{0.48\textwidth}
\textbf{Era-by-Era Decomposition}

\vspace{0.3em}
\begin{adjustbox}{max width=\textwidth}
\begin{tabular}{@{}lcccc@{}}
\toprule
\textbf{Era} & \textbf{Tech} & \textbf{Adopt} & \textbf{Net} & \textbf{Reg} \\
\midrule
2010--14 & 60\% & 20\% & 10\% & 10\% \\
2015--19 & 35\% & 35\% & 18\% & 12\% \\
2020--24 & 25\% & 30\% & 28\% & 17\% \\
2025--30 & 20\% & 25\% & 30\% & 25\% \\
\bottomrule
\end{tabular}
\end{adjustbox}

\vspace{0.5em}
\textbf{Interpretation:}
\begin{itemize}
\item Early phase: technology-driven (building infrastructure)
\item Middle phase: adoption-driven (user growth)
\item Late phase: network + regulation dominate (maturity)
\end{itemize}

\vspace{0.3em}
The dominant driver shifts over the lifecycle --- policy must adapt accordingly.
\end{columns}

\bottomnote{Solow (1957): growth accounting decomposes output growth into input contributions --- we apply the same logic to fintech}
\end{frame}

% Frame 23
\begin{frame}[t]{Digital Finance Growth Decomposition}
\begin{center}
\includegraphics[width=0.55\textwidth]{10_growth_accounting_decomposition/chart.pdf}
\end{center}

\begin{itemize}
\item Panel (a): Stacked area shows how the dominant growth driver shifts from technology (blue, 2010--14) to adoption (green, 2015--19) to network effects (orange, 2020+)
\item Panel (b): Bar chart confirms the transition --- regulation's share increases steadily as the sector matures
\item Policy implication: regulators who focus only on technology risk missing the network-effects phase where systemic risk emerges
\end{itemize}

\bottomnote{Growth accounting: $g_{total} = g_{tech} + g_{adopt} + g_{network} + g_{reg}$ --- Solow (1957)}
\end{frame}

% Frame 24
\begin{frame}[t]{Adoption Threshold Phase Diagram}
\begin{center}
\includegraphics[width=0.55\textwidth]{11_adoption_threshold_phase/chart.pdf}
\end{center}

\begin{itemize}
\item Panel (a): Phase diagram maps ($\sigma$, $c$) space --- above the curve adoption fails; below it succeeds. BTC and ETH are in the viable zone; a failed token is not
\item Panel (b): Cross-section at $\sigma{=}1.5$ shows benefit curve $b(n) = \sigma n / (1 + \sigma n)$ intersecting cost lines --- lower cost or stronger network effects shifts equilibrium toward adoption
\item Parameters: BTC ($\sigma{=}2.5$, $c{=}0.15$), ETH ($\sigma{=}2.0$, $c{=}0.20$), Failed ($\sigma{=}0.5$, $c{=}0.35$)
\end{itemize}

\bottomnote{Katz \& Shapiro (1985): fulfilled expectations equilibrium determines which technologies survive}
\end{frame}

%% ============================================================
%% SECTION 7: Synthesis and Policy Implications
%% ============================================================
\section{Synthesis and Policy Implications}

% Frame 25
\begin{frame}[t]{Model Synthesis}
\begin{columns}[T]
\column{0.48\textwidth}
\textbf{How the Five Models Connect}

\vspace{0.3em}
\begin{enumerate}
\item \textbf{Barter model} (Kiyotaki-Wright): Money reduces transaction costs by eliminating double coincidence
\item \textbf{Bass diffusion}: Adoption follows S-curves driven by innovation ($p$) and imitation ($q$)
\item \textbf{Metcalfe/Odlyzko}: Network value scales with user count (quadratic or sub-quadratic)
\item \textbf{Coase-Williamson}: Digital finance restructures (not eliminates) transaction costs
\item \textbf{Katz-Shapiro}: Adoption requires sufficient network effects relative to cost
\end{enumerate}

\vspace{0.3em}
\textit{Each model illuminates a different dimension of the same phenomenon: why digital money is emerging as a dominant form.}

\column{0.48\textwidth}
\textbf{Policy Implications Table}

\vspace{0.3em}
\begin{adjustbox}{max width=\textwidth}
\begin{tabular}{@{}lp{4.5cm}@{}}
\toprule
\textbf{Model} & \textbf{Policy Implication} \\
\midrule
Barter & Money is a public good; CBDCs extend this logic digitally \\
\addlinespace
Bass & Regulate during early phase (pre-tipping point) or lose the window \\
\addlinespace
Metcalfe & Antitrust matters: $n^2$ creates natural monopolies \\
\addlinespace
Coase & Focus regulation on the cost component that is \textit{increasing} (compliance) \\
\addlinespace
Katz-Shapiro & Interoperability standards reduce critical mass threshold \\
\bottomrule
\end{tabular}
\end{adjustbox}
\end{columns}

\bottomnote{These five models form the analytical foundation for all subsequent lectures in this course}
\end{frame}

% Frame 26
\begin{frame}[t]{Open Questions}
\begin{columns}[T]
\column{0.48\textwidth}
\textbf{Unresolved Theoretical Questions}

\vspace{0.3em}
\begin{enumerate}
\item \textbf{Is digital money a substitute or complement to cash?}\\
Bass model assumes independent adoption, but cash and digital may cannibalize each other
\item \textbf{Does Metcalfe hold for multi-chain ecosystems?}\\
Ethereum L2s fragment the network --- does $V \propto n^2$ apply to the sum or individual chains?
\item \textbf{Can transaction costs reach zero?}\\
Coase says no --- but how close can blockchain get?
\end{enumerate}

\column{0.48\textwidth}
\textbf{Unresolved Empirical Questions}

\vspace{0.3em}
\begin{enumerate}
\item \textbf{What are the true Bass parameters for CBDCs?}\\
Only 3 countries have launched retail CBDCs (Nigeria, Bahamas, Jamaica) --- too little data
\item \textbf{How do network effects interact across payment systems?}\\
Cross-network effects (e.g., Apple Pay users boosting Visa) are poorly measured
\item \textbf{What is the optimal regulatory intensity?}\\
Growth accounting suggests regulation contributes 10--25\% of growth, but causality is unclear
\end{enumerate}
\end{columns}

\bottomnote{These questions motivate the detailed analyses in L02--L08 of this course}
\end{frame}

% Frame 27
\begin{frame}[t]{References}

\textbf{Core References}
\begin{itemize}
\item Bass, F.\ M.\ (1969). A new product growth for model consumer durables. \textit{Management Science}, 15(5), 215--227.
\item Briscoe, B., Odlyzko, A., \& Tilly, B.\ (2006). Metcalfe's Law is wrong. \textit{IEEE Spectrum}, 43(7), 34--39.
\item Coase, R.\ H.\ (1937). The nature of the firm. \textit{Economica}, 4(16), 386--405.
\item Jevons, W.\ S.\ (1875). \textit{Money and the Mechanism of Exchange}. London: Henry S.\ King.
\item Katz, M.\ L., \& Shapiro, C.\ (1985). Network externalities, competition, and compatibility. \textit{American Economic Review}, 75(3), 424--440.
\item Kiyotaki, N., \& Wright, R.\ (1989). On money as a medium of exchange. \textit{Journal of Political Economy}, 97(4), 927--954.
\item Metcalfe, B.\ (2013). Metcalfe's Law after 40 years of Ethernet. \textit{Computer}, 46(12), 26--31.
\item Odlyzko, A., \& Tilly, B.\ (2005). A refutation of Metcalfe's Law. Working Paper, University of Minnesota.
\item Rogers, E.\ M.\ (1962). \textit{Diffusion of Innovations}. New York: Free Press.
\item Solow, R.\ M.\ (1957). Technical change and the aggregate production function. \textit{Review of Economics and Statistics}, 39(3), 312--320.
\item Williamson, O.\ E.\ (1985). \textit{The Economic Institutions of Capitalism}. New York: Free Press.
\end{itemize}

\bottomnote{All references are foundational --- students are encouraged to read the original papers}
\end{frame}

%% ============================================================
%% APPENDIX
%% ============================================================
\appendix
\section{Appendix}

% Frame 28
\begin{frame}[t]{Appendix A1: Complete Notation Table}

\begin{adjustbox}{max width=\textwidth}
\begin{tabular}{@{}llll@{}}
\toprule
\textbf{Symbol} & \textbf{Name} & \textbf{Definition} & \textbf{First used} \\
\midrule
$n$ & Network size / goods & Number of goods (barter) or users (network) & Frame 3 \\
$c_b$ & Barter cost & Cost per barter trade (\$/trade); calibrated at 5.0 & Frame 4 \\
$c_m$ & Money cost & Cost per money trade (\$/trade); calibrated at 0.50 & Frame 4 \\
$\alpha$ & Match probability & Probability of double coincidence of wants & Frame 4 \\
$\pi$ & Search cost & Cost per search period (\$/period) & Frame 5 \\
$F(t)$ & Cumulative adoption & Fraction of population adopted at time $t$ & Frame 8 \\
$p$ & Innovation coefficient & Bass external influence parameter & Frame 8 \\
$q$ & Imitation coefficient & Bass internal influence (network effects) parameter & Frame 8 \\
$\tau$ & Inflection time & Time to maximum adoption rate & Frame 8 \\
$t^*$ & Peak adoption time & $t^* = \ln(q/p)/(p+q)$ & Frame 9 \\
$V$ & Network value & Total value of network (\$) & Frame 12 \\
$\alpha$ (network) & Metcalfe coefficient & Scaling parameter in $V = \alpha n^2$ & Frame 12 \\
$\beta$ & Odlyzko coefficient & Scaling parameter in $V = \beta n \ln(n)$ & Frame 13 \\
$\gamma$ & Technology efficiency & Reduction factor for digital transaction costs & Frame 17 \\
$\varepsilon$ & Error term & Regression residual & Frame 15 \\
$TC$ & Transaction cost & Total cost of executing a transaction (\$) & Frame 17 \\
$g$ & Growth rate & Percentage point contribution to total growth & Frame 22 \\
$\sigma$ & Network strength & Network effect parameter (Katz-Shapiro) & Frame 24 \\
$c$ (KS) & Adoption cost & Cost of joining the network (Katz-Shapiro) & Frame 24 \\
$b(n)$ & Network benefit & Benefit from joining network of size $n$ & Frame 24 \\
\bottomrule
\end{tabular}
\end{adjustbox}

\bottomnote{Context-dependent: $\alpha$ means match probability in Section 2 and Metcalfe coefficient in Section 4}
\end{frame}

% Frame 29
\begin{frame}[t]{Appendix A2: Bass Model Full Derivation}

\textbf{Step 1: The differential equation}
$$\frac{dF}{dt} = (p + qF)(1 - F) = p + (q - p)F - qF^2$$

This is a Riccati equation (a type of nonlinear ODE) with known solution technique.

\vspace{0.3em}
\textbf{Step 2: Separation of variables}
$$\frac{dF}{(p + qF)(1 - F)} = dt$$

Using partial fractions: $\frac{1}{(p + qF)(1-F)} = \frac{1}{p+q}\left(\frac{1}{1-F} + \frac{q}{p+qF}\right)$

\vspace{0.3em}
\textbf{Step 3: Integration}
$$\frac{1}{p+q}\left[-\ln(1-F) + \ln(p + qF)\right] = t + C$$

With initial condition $F(0) = 0$: $C = \frac{1}{p+q}\ln(p)$

\vspace{0.3em}
\textbf{Step 4: Solve for $F(t)$}
$$\ln\!\left(\frac{p + qF}{p(1-F)}\right) = (p+q)t \implies F(t) = \frac{1 - e^{-(p+q)t}}{1 + \frac{q}{p}e^{-(p+q)t}}$$

\vspace{0.3em}
\textbf{Step 5: Inflection point} --- Set $d^2F/dt^2 = 0$: occurs at $F^* = \frac{1}{2}\left(1 - \frac{p}{q}\right)$, giving $t^* = \frac{\ln(q/p)}{p+q}$

\bottomnote{The Bass model is one of the few nonlinear ODEs with a clean closed-form solution}
\end{frame}

% Frame 30
\begin{frame}[t]{Appendix A3: Metcalfe's Law Conditions and Limitations}
\begin{columns}[T]
\column{0.48\textwidth}
\textbf{When $V \sim n^2$ Holds}

\vspace{0.3em}
Metcalfe's Law requires:
\begin{enumerate}
\item \textbf{Homogeneity}: All users are equally valuable connections (e.g., a telephone network where every call is equally important)
\item \textbf{Full connectivity}: Each user communicates with all others (or a constant fraction)
\item \textbf{No congestion}: Adding users does not degrade experience
\item \textbf{Positive externality}: Each new user benefits all existing users
\end{enumerate}

\vspace{0.3em}
\textbf{Networks where $n^2$ works:}
\begin{itemize}
\item Early-stage social networks
\item Small cryptocurrency communities
\item Interbank payment systems (e.g., SWIFT with $\sim$11,000 members)
\end{itemize}

\column{0.48\textwidth}
\textbf{When $V \sim n^2$ Breaks}

\vspace{0.3em}
\begin{enumerate}
\item \textbf{Heterogeneity}: Some connections are worth much more than others (Zipf distribution) $\Rightarrow$ $V \sim n \ln(n)$
\item \textbf{Clustering}: Users form subgroups that don't interact $\Rightarrow$ $V \sim k \cdot n_i^2$ (sum of cluster values)
\item \textbf{Congestion}: More users = slower/more expensive (Ethereum gas) $\Rightarrow$ $V$ can decrease with $n$
\item \textbf{Negative externalities}: Spam, fraud, misinformation reduce per-connection value
\end{enumerate}

\vspace{0.3em}
\textbf{Summary formula:}
$$V(n) = \alpha \cdot n^b \cdot e^{-\delta n}$$
where $b \in [1,2]$ captures network effect strength and $\delta$ captures congestion/saturation.
\end{columns}

\bottomnote{The modified formula nests all three models: $\delta{=}0, b{=}2$ (Metcalfe); $\delta{=}0, b{\approx}1.3$ (Odlyzko); $\delta{>}0$ (saturation)}
\end{frame}

\end{document}
