\documentclass[8pt,aspectratio=169]{beamer}
\usetheme{Madrid}
\usepackage{graphicx}
\usepackage{booktabs}
\usepackage{adjustbox}
\usepackage{multicol}
\usepackage{amsmath}

% Color definitions
\definecolor{mlblue}{RGB}{0,102,204}
\definecolor{mlpurple}{RGB}{51,51,178}
\definecolor{mllavender}{RGB}{173,173,224}
\definecolor{mllavender2}{RGB}{193,193,232}
\definecolor{mllavender3}{RGB}{204,204,235}
\definecolor{mllavender4}{RGB}{214,214,239}
\definecolor{mlorange}{RGB}{255, 127, 14}
\definecolor{mlgreen}{RGB}{44, 160, 44}
\definecolor{mlred}{RGB}{214, 39, 40}
\definecolor{mlgray}{RGB}{127, 127, 127}

\definecolor{lightgray}{RGB}{240, 240, 240}
\definecolor{midgray}{RGB}{180, 180, 180}

% Apply custom colors to Madrid theme
\setbeamercolor{palette primary}{bg=mllavender3,fg=mlpurple}
\setbeamercolor{palette secondary}{bg=mllavender2,fg=mlpurple}
\setbeamercolor{palette tertiary}{bg=mllavender,fg=white}
\setbeamercolor{palette quaternary}{bg=mlpurple,fg=white}

\setbeamercolor{structure}{fg=mlpurple}
\setbeamercolor{section in toc}{fg=mlpurple}
\setbeamercolor{subsection in toc}{fg=mlblue}
\setbeamercolor{title}{fg=mlpurple}
\setbeamercolor{frametitle}{fg=mlpurple,bg=mllavender3}
\setbeamercolor{block title}{bg=mllavender2,fg=mlpurple}
\setbeamercolor{block body}{bg=mllavender4,fg=black}

\setbeamertemplate{navigation symbols}{}
\setbeamertemplate{itemize items}[circle]
\setbeamertemplate{enumerate items}[default]
\setbeamersize{text margin left=5mm,text margin right=5mm}

\newcommand{\bottomnote}[1]{%
\vfill
\vspace{-2mm}
\textcolor{mllavender2}{\rule{\textwidth}{0.4pt}}
\vspace{1mm}
\footnotesize
\textbf{#1}
}

\title{Synthesis and Future Directions}
\subtitle{L08: Integrating the Four Economic Lenses}
\author{Economics of Digital Finance}
\institute{BSc Course}
\date{}

\begin{document}

% Title slide
\begin{frame}[plain]
\titlepage
\end{frame}

% Outline
\begin{frame}[t]{Lesson Overview}
\begin{columns}[T]
\column{0.48\textwidth}
\textbf{Today's Topics}
\begin{enumerate}
\item Reviewing the four economic lenses
\item Systemic risk and contagion in digital finance
\item TradFi-DeFi convergence
\item Integrating the lenses for holistic analysis
\item The digital finance trilemma
\item Policy effectiveness across domains
\item Future research directions
\end{enumerate}

\column{0.48\textwidth}
\textbf{Learning Objectives}
\begin{itemize}
\item Synthesize insights across monetary, platform, microstructure, and regulatory economics
\item Analyze systemic risk linkages
\item Evaluate policy trade-offs
\item Identify research frontiers
\end{itemize}
\end{columns}

\bottomnote{This lesson integrates all four lenses to provide a comprehensive view of digital finance}
\end{frame}

% Reviewing the Four Lenses
\begin{frame}[t]{Reviewing the Four Economic Lenses}
\begin{columns}[T]
\column{0.48\textwidth}
\textbf{Lens 1: Monetary Economics}
\begin{itemize}
\item Money creation and seigniorage
\item Monetary policy transmission
\item CBDCs (Central Bank Digital Currencies) and central bank balance sheets
\item Dollarization and currency substitution
\end{itemize}

\vspace{0.3em}
\textbf{Lens 2: Platform Economics}
\begin{itemize}
\item Network effects and adoption dynamics
\item Token economics and mechanism design
\item Two-sided markets and intermediaries
\item Platform governance
\end{itemize}

\column{0.48\textwidth}
\textbf{Lens 3: Market Microstructure}
\begin{itemize}
\item Price discovery and liquidity
\item Order book vs. AMM (Automated Market Maker) mechanisms
\item MEV (Maximal Extractable Value) and information asymmetry (when one party has more information)
\item Market design and efficiency
\end{itemize}

\vspace{0.3em}
\textbf{Lens 4: Regulatory Economics}
\begin{itemize}
\item Market failures and intervention
\item Regulatory capture and arbitrage
\item Consumer protection and systemic risk
\item International coordination
\end{itemize}
\end{columns}

\bottomnote{Each lens provides unique insights, but synthesis reveals deeper patterns}
\end{frame}

% Systemic Risk and Contagion
\begin{frame}[t]{Systemic Risk and Contagion in Digital Finance}
\begin{center}
\includegraphics[width=0.65\textwidth]{01_systemic_risk_contagion/chart.pdf}
\end{center}

\bottomnote{Allen \& Gale (2000) contagion model: shows how financial distress spreads through interbank lending networks depending on network structure; digital finance adds instant velocity}
\end{frame}

% Understanding Systemic Risk
\begin{frame}[t]{Understanding Systemic Risk Channels}
\begin{columns}[T]
\column{0.48\textwidth}
\textbf{Traditional Finance Channels}
\begin{itemize}
\item Interbank lending networks
\item Common asset exposures
\item Fire sales and liquidity spirals
\item Bank runs (Diamond-Dybvig model: bank runs occur when depositors panic and withdraw simultaneously)
\end{itemize}

\vspace{0.3em}
\textbf{Allen \& Gale Framework}
\begin{itemize}
\item Complete vs. incomplete networks (how densely connected institutions are)
\item Contagion (financial problems spreading) depends on structure
\item Diversification can increase risk (paradox: more connections can spread problems faster)
\end{itemize}

\column{0.48\textwidth}
\textbf{Digital Finance Amplifications}
\begin{itemize}
\item Instantaneous settlement
\item Smart contract interconnections
\item Stablecoin redemption spirals
\item Cross-platform contagion (CEX (Centralized Exchange) and DEX (Decentralized Exchange))
\end{itemize}

\vspace{0.3em}
\textbf{New Systemic Risk Sources}
\begin{itemize}
\item Oracle failures
\item Protocol exploit contagion
\item Collateral liquidation cascades
\item Wrapped asset depegging
\end{itemize}
\end{columns}

\bottomnote{Digital finance requires updating classic contagion models for instantaneous settlement and algorithmic linkages}
\end{frame}

% TradFi-DeFi Convergence
\begin{frame}[t]{TradFi-DeFi Convergence}
\begin{center}
\includegraphics[width=0.65\textwidth]{02_tradfi_defi_convergence/chart.pdf}
\end{center}

\bottomnote{TradFi (Traditional Finance) and DeFi (Decentralized Finance) are converging through tokenization, institutional adoption, and regulatory clarity}
\end{frame}

% Convergence Mechanisms
\begin{frame}[t]{Mechanisms of TradFi-DeFi Convergence}
\begin{columns}[T]
\column{0.48\textwidth}
\textbf{Institutional Adoption}
\begin{itemize}
\item Banks offering crypto custody
\item Asset managers tokenizing funds
\item Payment rails integration
\item Regulated stablecoin issuance
\end{itemize}

\vspace{0.3em}
\textbf{Technology Lifecycle Theory}
\begin{itemize}
\item Innovators: Early crypto natives
\item Early adopters: Fintech firms
\item Early majority: Incumbent banks (now)
\item Mainstream adoption ahead
\end{itemize}

\column{0.48\textwidth}
\textbf{Regulatory Clarity}
\begin{itemize}
\item MiCA (Markets in Crypto-Assets---EU regulation) in Europe
\item US framework emerging
\item Licensing regimes for exchanges
\item Stablecoin regulation
\end{itemize}

\vspace{0.3em}
\textbf{Hybrid Models Emerging}
\begin{itemize}
\item Permissioned DeFi
\item Tokenized deposits
\item Programmable money with compliance
\item Central bank-DeFi interoperability
\end{itemize}
\end{columns}

\bottomnote{Rogers (1962) diffusion of innovation theory: new technologies spread through population in stages (innovators → early adopters → early majority → late majority → laggards); DeFi entering early majority phase}
\end{frame}

% Integrating the Four Lenses
\begin{frame}[t]{Integrating the Four Economic Lenses}
\begin{center}
\includegraphics[width=0.65\textwidth]{03_four_lenses_integration/chart.pdf}
\end{center}

\bottomnote{Real-world digital finance phenomena require simultaneous application of all four lenses}
\end{frame}

% Integration Examples
\begin{frame}[t]{Case Studies: Integrating Multiple Lenses}
\begin{columns}[T]
\column{0.48\textwidth}
\textbf{Case 1: Stablecoin Runs}
\begin{itemize}
\item \textbf{Monetary}: Money demand shock
\item \textbf{Platform}: Network effects in redemptions
\item \textbf{Microstructure}: Liquidity dry-up, price impact
\item \textbf{Regulatory}: Reserve requirements, insurance
\end{itemize}

\textit{Example: Terra/Luna collapse (2022)---algorithmic stablecoin death spiral triggered by loss of confidence.}

\vspace{0.3em}
\textbf{Case 2: CBDC Adoption}
\begin{itemize}
\item \textbf{Monetary}: Bank disintermediation risk
\item \textbf{Platform}: Adoption critical mass
\item \textbf{Microstructure}: Payment vs. settlement design
\item \textbf{Regulatory}: Privacy-AML (Anti-Money Laundering) trade-offs
\end{itemize}

\textit{Example: USDC depeg (March 2023)---exposure to Silicon Valley Bank failure showed TradFi-DeFi interconnection.}

\column{0.48\textwidth}
\textbf{Case 3: DEX Competition}
\begin{itemize}
\item \textbf{Monetary}: Medium of exchange function
\item \textbf{Platform}: Liquidity network effects
\item \textbf{Microstructure}: AMM vs. order book efficiency
\item \textbf{Regulatory}: Decentralization and liability
\end{itemize}

\textit{Example: JPMorgan Onyx---institutional blockchain platform for tokenized deposits and programmable payments.}

\vspace{0.3em}
\textbf{Key Insight}

Single-lens analysis misses interactions:
\begin{itemize}
\item Regulation affects platform dynamics
\item Microstructure influences monetary policy
\item Platform effects amplify contagion
\end{itemize}
\end{columns}

\bottomnote{Holistic analysis requires tracing effects across all four economic domains}
\end{frame}

% The Digital Finance Trilemma
\begin{frame}[t]{The Digital Finance Trilemma}
\begin{center}
\includegraphics[width=0.65\textwidth]{04_digital_finance_trilemma/chart.pdf}
\end{center}

\bottomnote{Similar to Mundell's impossible trinity (can't have fixed exchange rate, free capital flow, AND independent monetary policy), digital finance faces fundamental trade-offs between decentralization, efficiency, and security/compliance}
\end{frame}

% Understanding the Trilemma
\begin{frame}[t]{The Digital Finance Trilemma Explained}
\begin{columns}[T]
\column{0.48\textwidth}
\textbf{Three Desirable Properties}

\vspace{0.3em}
\textbf{1. Decentralization}
\begin{itemize}
\item No single point of control
\item Censorship resistance
\item Open participation
\end{itemize}

\vspace{0.3em}
\textbf{2. Efficiency}
\begin{itemize}
\item High throughput, low latency
\item Low transaction costs
\item Scalability
\end{itemize}

\vspace{0.3em}
\textbf{3. Security/Compliance}
\begin{itemize}
\item AML (Anti-Money Laundering)/CFT (Combating Financing of Terrorism) enforcement
\item Consumer protection
\item Systemic risk (risk of cascading failures) management
\end{itemize}

\column{0.48\textwidth}
\textbf{Pick Any Two}

\vspace{0.3em}
\textbf{Decentralized + Efficient}
\begin{itemize}
\item Example: Early Bitcoin, DeFi protocols
\item Problem: Regulatory arbitrage, illicit use
\end{itemize}

\vspace{0.3em}
\textbf{Decentralized + Secure/Compliant}
\begin{itemize}
\item Example: Proof-of-work with whitelisting
\item Problem: Slow, expensive (efficiency loss)
\end{itemize}

\vspace{0.3em}
\textbf{Efficient + Secure/Compliant}
\begin{itemize}
\item Example: CBDCs, permissioned blockchains
\item Problem: Centralized control (decentralization loss)
\end{itemize}
\end{columns}

\bottomnote{Fundamental trade-offs constrain design space; technological advances may shift boundaries but not eliminate trilemma}
\end{frame}

% Policy Effectiveness Heatmap
\begin{frame}[t]{Policy Effectiveness Across Digital Finance Domains}
\begin{center}
\includegraphics[width=0.65\textwidth]{05_policy_effectiveness_heatmap/chart.pdf}
\end{center}

\bottomnote{Policy effectiveness varies by domain; no one-size-fits-all regulatory approach}
\end{frame}

% Policy Analysis Framework
\begin{frame}[t]{Evaluating Policy Effectiveness}
\begin{columns}[T]
\column{0.48\textwidth}
\textbf{High Effectiveness Domains}
\begin{itemize}
\item CBDCs: Direct central bank control
\item Licensed stablecoins: Clear authority
\item Custodial exchanges: Traditional oversight
\end{itemize}

\vspace{0.3em}
\textbf{Medium Effectiveness}
\begin{itemize}
\item Payment systems: Competing jurisdictions
\item Tokenized assets: Classification issues
\item Lending platforms: DeFi vs. CeFi ambiguity
\end{itemize}

\column{0.48\textwidth}
\textbf{Low Effectiveness Domains}
\begin{itemize}
\item Fully decentralized protocols
\item Cross-border crypto flows
\item Privacy-preserving systems
\end{itemize}

\vspace{0.3em}
\textbf{Multi-Criteria Welfare Analysis}

Policy evaluation must consider:
\begin{itemize}
\item Consumer protection
\item Financial stability
\item Innovation incentives
\item Market efficiency
\item Distributional effects
\end{itemize}
\end{columns}

\bottomnote{Effective policy requires matching regulatory tools to degree of centralization and jurisdictional reach}
\end{frame}

% Future Research Directions
\begin{frame}[t]{Future Research Directions}
\begin{columns}[T]
\column{0.48\textwidth}
\textbf{Monetary Economics}
\begin{itemize}
\item CBDC impact on bank disintermediation
\item Stablecoin systemic importance
\item Cross-border CBDC arrangements
\item Digital dollarization in emerging markets
\end{itemize}

\vspace{0.3em}
\textbf{Platform Economics}
\begin{itemize}
\item Optimal token design for governance
\item Multi-chain network effects
\item Platform competition with composability (systems working together)
\item DAO (Decentralized Autonomous Organization) organizational economics
\end{itemize}

\column{0.48\textwidth}
\textbf{Market Microstructure}
\begin{itemize}
\item MEV mitigation mechanisms
\item Cross-chain liquidity fragmentation
\item Algorithmic stablecoin stability
\item Oracle design and manipulation
\end{itemize}

\vspace{0.3em}
\textbf{Regulatory Economics}
\begin{itemize}
\item Optimal regulatory perimeter
\item International coordination mechanisms
\item Regulating code as law
\item Balancing innovation and stability
\end{itemize}
\end{columns}

\bottomnote{Digital finance opens rich research agendas across all economic sub-fields}
\end{frame}

% Methodological Frontiers
\begin{frame}[t]{Methodological Frontiers}
\begin{columns}[T]
\column{0.48\textwidth}
\textbf{Data Opportunities}
\begin{itemize}
\item High-frequency blockchain data
\item Complete transaction histories
\item Natural experiments in adoption
\item Cross-country regulatory variation
\end{itemize}

\vspace{0.3em}
\textbf{Theoretical Challenges}
\begin{itemize}
\item Modeling algorithmic mechanisms
\item Game theory with smart contracts
\item Network contagion with instant settlement
\item Governance without hierarchy
\end{itemize}

\column{0.48\textwidth}
\textbf{Empirical Methods}
\begin{itemize}
\item Agent-based simulation
\item Mechanism design experiments
\item Event studies with second-level data
\item Machine learning for pattern detection
\end{itemize}

\vspace{0.3em}
\textbf{Interdisciplinary Approaches}
\begin{itemize}
\item Economics + computer science
\item Finance + cryptography
\item Regulation + protocol design
\item Behavioral economics + UI/UX
\end{itemize}
\end{columns}

\bottomnote{Digital finance requires economists to expand methodological toolkit while maintaining theoretical rigor}
\end{frame}

% Course Conclusions
\begin{frame}[t]{Course Conclusions: What We've Learned}
\begin{columns}[T]
\column{0.48\textwidth}
\textbf{Core Insights}
\begin{enumerate}
\item Digital finance is fundamentally economic, not just technical
\item Multiple economic frameworks needed for complete analysis
\item Traditional theories remain relevant but require adaptation
\item Policy must balance competing objectives
\end{enumerate}

\vspace{0.3em}
\textbf{Key Takeaway}

Digital finance is not replacing traditional finance---it's transforming how we think about money, markets, and intermediation.

\column{0.48\textwidth}
\textbf{Skills Developed}
\begin{itemize}
\item Apply monetary economics to cryptocurrencies
\item Analyze network effects in payment systems
\item Evaluate market microstructure innovations
\item Assess regulatory trade-offs
\item Integrate multiple economic perspectives
\end{itemize}

\vspace{0.3em}
\textbf{Looking Forward}

Digital finance will continue evolving. The economic frameworks learned here provide foundations for analyzing future innovations.
\end{columns}

\bottomnote{Economics provides the analytical lens to understand digital finance beyond the hype cycle}
\end{frame}

% Summary and Key Takeaways
\begin{frame}[t]{Summary and Key Takeaways}
\begin{columns}[T]
\column{0.48\textwidth}
\textbf{What We Covered}
\begin{enumerate}
\item Reviewed four economic lenses
\item Analyzed systemic risk and contagion
\item Examined TradFi-DeFi convergence
\item Integrated multiple frameworks
\item Explored the digital finance trilemma
\item Evaluated policy effectiveness
\item Identified research frontiers
\end{enumerate}

\column{0.48\textwidth}
\textbf{Final Messages}
\begin{itemize}
\item No single economic lens suffices
\item Systemic linkages amplify risks
\item Fundamental trade-offs constrain design
\item Policy effectiveness varies by domain
\item Rich research opportunities ahead
\end{itemize}

\vspace{0.5em}
\textbf{Course Philosophy}

Economics is essential for understanding digital finance. Technical knowledge helps, but economic incentives ultimately determine adoption, stability, and welfare outcomes.
\end{columns}

\bottomnote{Thank you for engaging with the economics of digital finance}
\end{frame}

% Key Terms (1/2)
\begin{frame}[t]{Key Terms (1/2)}
\begin{columns}[T]
\column{0.48\textwidth}
\textbf{Four Economic Lenses}
The four analytical frameworks used throughout the course: monetary economics, platform economics, market microstructure, and regulatory economics.

\vspace{0.3em}
\textbf{Monetary Economics}
Study of money creation, central banking, and monetary policy transmission.

\vspace{0.3em}
\textbf{Platform Economics}
Analysis of network effects, two-sided markets, and ecosystem dynamics in digital systems.

\vspace{0.3em}
\textbf{Market Microstructure}
Study of price discovery, liquidity provision, and trading mechanisms.

\vspace{0.3em}
\textbf{Regulatory Economics}
Analysis of market failures, regulatory interventions, and policy effectiveness.

\vspace{0.3em}
\textbf{Systemic Risk}
Risk that failure of one entity triggers cascading failures throughout the financial system.

\column{0.48\textwidth}
\textbf{Contagion}
Spread of financial distress from one institution or market to others through direct links or sentiment.

\vspace{0.3em}
\textbf{Digital Finance Trilemma}
Trade-off between decentralization, efficiency, and security/compliance---difficult to maximize all three simultaneously.

\vspace{0.3em}
\textbf{TradFi-DeFi Convergence}
Trend of traditional finance adopting DeFi innovations while DeFi increasingly seeks regulatory compliance.

\vspace{0.3em}
\textbf{Interoperability}
Ability of different financial systems and protocols to work together seamlessly.

\vspace{0.3em}
\textbf{Run Risk}
Risk that depositors or token holders simultaneously attempt to withdraw, causing liquidity crisis.
\end{columns}

\bottomnote{Synthesis requires integrating all four economic lenses to understand digital finance holistically}
\end{frame}

% Key Terms (2/2)
\begin{frame}[t]{Key Terms (2/2)}
\begin{columns}[T]
\column{0.48\textwidth}
\textbf{Liquidity Risk}
Risk that an asset cannot be sold quickly without significant price impact.

\vspace{0.3em}
\textbf{Regulatory Arbitrage}
Practice of exploiting differences between regulatory regimes to minimize compliance costs.

\vspace{0.3em}
\textbf{Sandbox}
Controlled regulatory environment where firms can test innovations with reduced requirements.

\vspace{0.3em}
\textbf{Information Asymmetry}
Situation where one party in a transaction has more or better information than the other.

\vspace{0.3em}
\textbf{Fire Sales}
Forced selling of assets at distressed prices, which can trigger further price declines.

\vspace{0.3em}
\textbf{Disintermediation}
Reduction or elimination of intermediaries (like banks) in financial transactions.

\column{0.48\textwidth}
\textbf{Composability}
Ability of different DeFi protocols to interact and build on each other, creating new functionality.

\vspace{0.3em}
\textbf{Oracle}
System that provides external real-world data to blockchain smart contracts.

\vspace{0.3em}
\textbf{Wrapped Assets}
Tokenized versions of assets from one blockchain that can be used on another blockchain.

\vspace{0.3em}
\textbf{Collateral Liquidation}
Automatic sale of collateral when borrowers fail to maintain required collateralization ratios.

\vspace{0.3em}
\textbf{Network Effects}
Phenomenon where a product or service becomes more valuable as more people use it.

\vspace{0.3em}
\textbf{Censorship Resistance}
Property of a system that prevents any single entity from blocking or reversing transactions.
\end{columns}

\bottomnote{Understanding these terms is essential for analyzing digital finance comprehensively}
\end{frame}

% Further Reading
\begin{frame}[t]{Further Reading and Research Agenda}
\begin{columns}[T]
\column{0.48\textwidth}
\textbf{Synthesis Papers}
\begin{itemize}
\item Auer, Cornelli \& Frost (2020): ``Rise of the Central Bank Digital Currencies''
\item Schilling \& Uhlig (2019): ``Some Simple Bitcoin Economics''
\item Cong \& He (2019): ``Blockchain Disruption and Smart Contracts''
\end{itemize}

\vspace{0.3em}
\textbf{Systemic Risk}
\begin{itemize}
\item Allen \& Gale (2000): ``Financial Contagion''
\item Makarov \& Schoar (2023): ``Cryptocurrencies and Decentralized Finance''
\end{itemize}

\column{0.48\textwidth}
\textbf{Policy Analysis}
\begin{itemize}
\item BIS (Bank for International Settlements) (2023): ``Blueprint for the Future Monetary System''
\item IMF (2023): ``Elements of Effective Policies for Crypto Assets''
\item FSB (Financial Stability Board) (2023): ``High-Level Recommendations for Regulation of Crypto Assets''
\end{itemize}

\vspace{0.3em}
\textbf{Future Directions}
\begin{itemize}
\item Duffie (2024): ``Digital Currencies: Principles and Practicalities''
\item Saleh (2024): ``Blockchain Economics and Governance''
\end{itemize}
\end{columns}

\bottomnote{All readings and course materials available on platform}
\end{frame}

\end{document}
