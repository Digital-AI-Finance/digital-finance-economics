\documentclass[8pt,aspectratio=169]{beamer}
\usetheme{Madrid}
\usepackage{graphicx}
\usepackage{booktabs}
\usepackage{adjustbox}
\usepackage{multicol}
\usepackage{amsmath}

% Color definitions
\definecolor{mlblue}{RGB}{0,102,204}
\definecolor{mlpurple}{RGB}{51,51,178}
\definecolor{mllavender}{RGB}{173,173,224}
\definecolor{mllavender2}{RGB}{193,193,232}
\definecolor{mllavender3}{RGB}{204,204,235}
\definecolor{mllavender4}{RGB}{214,214,239}
\definecolor{mlorange}{RGB}{255, 127, 14}
\definecolor{mlgreen}{RGB}{44, 160, 44}
\definecolor{mlred}{RGB}{214, 39, 40}
\definecolor{mlgray}{RGB}{127, 127, 127}

\definecolor{lightgray}{RGB}{240, 240, 240}
\definecolor{midgray}{RGB}{180, 180, 180}

\setbeamercolor{palette primary}{bg=mllavender3,fg=mlpurple}
\setbeamercolor{palette secondary}{bg=mllavender2,fg=mlpurple}
\setbeamercolor{palette tertiary}{bg=mllavender,fg=white}
\setbeamercolor{palette quaternary}{bg=mlpurple,fg=white}

\setbeamercolor{structure}{fg=mlpurple}
\setbeamercolor{section in toc}{fg=mlpurple}
\setbeamercolor{subsection in toc}{fg=mlblue}
\setbeamercolor{title}{fg=mlpurple}
\setbeamercolor{frametitle}{fg=mlpurple,bg=mllavender3}
\setbeamercolor{block title}{bg=mllavender2,fg=mlpurple}
\setbeamercolor{block body}{bg=mllavender4,fg=black}

\setbeamertemplate{navigation symbols}{}
\setbeamertemplate{itemize items}[circle]
\setbeamertemplate{enumerate items}[default]
\setbeamersize{text margin left=5mm,text margin right=5mm}

\newcommand{\bottomnote}[1]{%
\vfill
\vspace{-2mm}
\textcolor{mllavender2}{\rule{\textwidth}{0.4pt}}
\vspace{1mm}
\footnotesize
\textbf{#1}
}

\title{Central Bank Digital Currencies (CBDCs)}
\subtitle{L03: The Economics of Public Digital Money\\[0.3em]\normalsize Why 130+ countries are racing to digitize money---and what it means for you}
\author{Economics of Digital Finance}
\institute{BSc Course}
\date{}

\begin{document}

% Title slide
\begin{frame}[plain]
\titlepage
\end{frame}

% Outline
\begin{frame}[t]{Lesson Overview}
\begin{columns}[T]
\column{0.48\textwidth}
\textbf{Today's Topics}
\begin{enumerate}
\item CBDC design choices and trade-offs
\item Monetary policy transmission
\item Bank disintermediation risk
\item Financial inclusion economics
\item International currency competition
\end{enumerate}

\column{0.48\textwidth}
\textbf{Learning Objectives}
\begin{itemize}
\item Analyze CBDC design trade-offs
\item Assess how CBDCs could change how central banks control the economy
\item Evaluate disintermediation risks (the danger that people move money out of banks into CBDC)
\item Understand global CBDC landscape
\end{itemize}
\end{columns}

\bottomnote{CBDCs represent central banks' response to private digital currencies}
\end{frame}

% Background Frame A
\begin{frame}[t]{Background: The Banking System}
\begin{columns}[T]
\column{0.48\textwidth}
\textbf{Central Bank vs. Commercial Bank}

\textbf{Central bank} (e.g., ECB, Federal Reserve):
\begin{itemize}
\item The government's bank
\item Creates the national currency
\item Sets interest rates for the economy
\item Lender of last resort (providing emergency loans when no one else will) in crises
\end{itemize}

\column{0.48\textwidth}
\textbf{Commercial bank} (e.g., Deutsche Bank, Chase):
\begin{itemize}
\item Private companies where you have accounts
\item Accept deposits, make loans
\item Can fail (unlike central banks)
\item Must hold reserves (money banks must keep at the central bank) at central bank
\end{itemize}

\vspace{0.3em}
\textbf{Key Point}: Your savings account is at a commercial bank. The central bank is the bank for banks.
\end{columns}

\bottomnote{Understanding this distinction is essential for CBDC analysis}
\end{frame}

% Background Frame B
\begin{frame}[t]{Background: How Banks Work}
\begin{columns}[T]
\column{0.48\textwidth}
\textbf{Money Creation Through Lending}

Banks don't lend from a vault of cash:
\begin{itemize}
\item When a bank lends you 100 EUR, it \textit{creates} 100 EUR in your account
\item Most money in modern economies is created this way
\item This is called \textbf{fractional reserve banking}
\end{itemize}

\vspace{0.3em}
\textbf{Why Deposits Matter to Banks}
\begin{itemize}
\item Banks borrow from depositors at low rates (e.g., 1\%)
\item Banks lend at higher rates (e.g., 5\%)
\item The difference is their profit
\end{itemize}

\column{0.48\textwidth}
\textbf{Bank Runs}

A \textbf{bank run} occurs when:
\begin{itemize}
\item Many depositors withdraw simultaneously
\item They fear the bank will fail
\item Banks don't keep all deposits as cash
\item Mass withdrawal causes collapse
\end{itemize}

\vspace{0.3em}
\textbf{Why Interest Rates Affect Spending}
\begin{itemize}
\item Lower rates = cheaper loans
\item Mortgages at 3\% vs 6\% = more homebuyers
\item Cheap business loans = more investment
\end{itemize}
\end{columns}

\bottomnote{These mechanics explain why CBDCs could disrupt traditional banking}
\end{frame}

% What is a CBDC?
\begin{frame}[t]{What is a Central Bank Digital Currency?}
\begin{columns}[T]
\column{0.48\textwidth}
\textbf{Definition}

A CBDC is a digital form of central bank money:
\begin{itemize}
\item Direct liability of central bank (a legal obligation---the central bank owes you that value)
\item Digital (not physical)
\item Widely accessible to the public (retail) or restricted to banks and financial institutions (wholesale)
\end{itemize}

\vspace{0.3em}
\textbf{Not a CBDC}
\begin{itemize}
\item Bank reserves (money banks hold at the central bank---like a bank's own bank account) are already digital
\item Commercial bank money \textit{(Your bank deposit is a promise from your bank, which can fail. CBDC is a promise from the central bank, backed by the state---it cannot fail.)}
\item Stablecoins (private liability---can collapse, as Terra/UST did in 2022 losing \$40B+)
\end{itemize}

\column{0.48\textwidth}
\textbf{Motivations}

Central banks cite multiple goals:
\begin{itemize}
\item Maintain monetary sovereignty---without it, a country cannot fight recessions independently (like Greece in 2010, which couldn't devalue its currency)
\item Improve payment efficiency
\item Promote financial inclusion
\item Counter private digital currencies---if citizens abandon national currency for Bitcoin or stablecoins, the central bank loses its ability to stabilize the economy
\end{itemize}

\vspace{0.3em}
\textbf{Key Economic Question}

Does public benefit exceed costs and risks?
\end{columns}

\bottomnote{CBDC = digital cash issued by central bank; distinct from existing digital money}
\end{frame}

% Global CBDC Status
\begin{frame}[t]{Global CBDC Development Status}
\begin{center}
\includegraphics[width=0.55\textwidth]{01_cbdc_world_map/chart.pdf}
\end{center}

\begin{itemize}
\item As of 2024, 147 countries (representing 98\% of global GDP) are exploring or have launched CBDCs
\item Colors indicate development stage: from early research to live deployment
\item China's e-CNY (digital yuan) is the most advanced large-economy pilot; the Bahamas and Nigeria have fully launched
\end{itemize}

\bottomnote{Source: Atlantic Council CBDC Tracker. The pace of exploration has accelerated sharply since 2020}
\end{frame}

% CBDC Design Space
\begin{frame}[t]{CBDC Design Space: Key Choices}
\begin{center}
\includegraphics[width=0.48\textwidth]{02_cbdc_design_space/chart.pdf}
\end{center}

\begin{itemize}
\item Each axis represents a design goal (e.g., privacy, efficiency, control). No single design can maximize all goals simultaneously
\item The Pareto frontier (curved surface) shows the \textit{best achievable} trade-offs---improving one dimension requires sacrificing another
\item Dominated designs (dots below the frontier) are inferior: another design is better in \textit{every} dimension
\end{itemize}

\bottomnote{Pareto frontier: named after economist Vilfredo Pareto. Think of it as the ``best you can do'' boundary---you cannot move closer to one goal without moving away from another}
\end{frame}

% Retail vs Wholesale
\begin{frame}[t]{Retail vs. Wholesale CBDCs}
\begin{columns}[T]
\column{0.48\textwidth}
\textbf{Retail CBDC}

For general public use:
\begin{itemize}
\item Replaces/complements cash
\item Consumer payment instrument
\item Requires distribution network \textit{(Unlike wholesale CBDC which uses existing bank infrastructure, retail CBDC needs new channels to reach every citizen.)}
\end{itemize}

\vspace{0.3em}
Economic considerations:
\begin{itemize}
\item High operational costs
\item Privacy vs. AML (Anti-Money Laundering---criminals using untraceable money enables drug trafficking, terrorism financing, and tax evasion) trade-off
\item Competition with banks
\end{itemize}

\column{0.48\textwidth}
\textbf{Wholesale CBDC}

For financial institutions:
\begin{itemize}
\item Interbank settlement (transferring money between banks)
\item Securities transactions (buying/selling stocks and bonds between institutions)
\item Cross-border payments
\end{itemize}

\vspace{0.3em}
Economic considerations:
\begin{itemize}
\item Lower operational burden
\item Efficiency gains clearer
\item Less disruptive to banking
\end{itemize}
\end{columns}

\bottomnote{Most advanced economies focus on retail; wholesale offers clearer near-term benefits}
\end{frame}

% Token vs Account
\begin{frame}[t]{Token-Based vs. Account-Based CBDCs}
\begin{columns}[T]
\column{0.48\textwidth}
\textbf{Token-Based}

Like digital cash:
\begin{itemize}
\item Verify the instrument, not holder
\item Can enable anonymity
\item Offline transactions possible \textit{(This matters because rural areas and developing countries may lack reliable internet---offline capability ensures universal access.)}
\end{itemize}

\vspace{0.3em}
Economic implications:
\begin{itemize}
\item Lower transaction costs
\item Privacy preserving
\item Harder to implement AML
\end{itemize}

\column{0.48\textwidth}
\textbf{Account-Based}

Like bank accounts:
\begin{itemize}
\item Verify the identity of holder
\item Full transaction records
\item Programmable features possible---e.g., stimulus payments that can only be spent at small businesses, or rent subsidies that can only pay landlords
\end{itemize}

\vspace{0.3em}
Economic implications:
\begin{itemize}
\item Interest-bearing feasible \textit{(Because account-based systems track balances, they can calculate and pay interest---token-based systems cannot easily distinguish one holder's balance from another.)}
\item Targeted policies possible
\item Privacy concerns
\end{itemize}
\end{columns}

\bottomnote{Most designs are hybrid: token-like for small values, account-like for large}
\end{frame}

% Monetary Policy Transmission (1/2)
\begin{frame}[t]{Monetary Policy Transmission with CBDCs}
\begin{columns}[T]
\column{0.48\textwidth}
\textbf{Traditional Channels}

Interest rate channel:
$$i_{\text{policy}} \rightarrow i_{\text{deposit}} \rightarrow C, I$$
\textit{(When the central bank changes its policy rate ($i_{\text{policy}}$), banks adjust deposit rates ($i_{\text{deposit}}$), affecting consumption (C) and investment (I).)}
\begin{itemize}
\item Works through bank intermediation
\item Banks pass rate changes to customers
\item Time lags in transmission (months can pass before rate changes affect consumer behavior)
\end{itemize}

\column{0.48\textwidth}
\textbf{CBDC Impact}

If CBDC is interest-bearing:
$$i_{\text{CBDC}} \rightarrow i_{\text{deposit}}$$
\textit{(If CBDC pays 2\% interest, banks must match or exceed this to keep deposits---CBDC rate sets a floor.)}
\begin{itemize}
\item Direct transmission to public
\item Floor on deposit rates
\end{itemize}

\smallskip\textbf{Example:} Suppose CBDC pays 2\% and your bank pays 0.5\% on savings. You would move money to CBDC. To keep your deposits, the bank must raise its rate to at least 2\%. This is the ``floor'' effect.
\end{columns}

\bottomnote{CBDC could strengthen monetary policy by providing a direct channel to households}
\end{frame}

% Monetary Policy Transmission (2/2)
\begin{frame}[t]{Enhanced Policy Options with CBDCs}
\begin{columns}[T]
\column{0.48\textwidth}
\textbf{New Policy Tools}

Interest-bearing CBDC enables:
\begin{itemize}
\item Negative interest rates (charging people to hold money, to encourage spending)
\item Helicopter money (direct cash from central bank to citizens---named for the image of dropping money from helicopters)
\item Time-limited money---e.g., China's Chengdu pilot gave citizens digital yuan that expired in 3 months, forcing spending to boost local businesses
\end{itemize}

\column{0.48\textwidth}
\textbf{Concerns}
\begin{itemize}
\item Political resistance to negative rates
\textit{(Why accept losing money? Without physical cash, people cannot escape---some see this as coercive.)}
\item Privacy implications of targeting
\item Complexity of implementation
\end{itemize}
\end{columns}

\bottomnote{These tools are powerful but raise fundamental questions about government control over spending}
\end{frame}

% Monetary Policy Transmission Chart
\begin{frame}[t]{How Fast Does CBDC Transmit Policy?}
\begin{center}
\includegraphics[width=0.55\textwidth]{04_monetary_policy_transmission/chart.pdf}
\end{center}

\begin{itemize}
\item The chart compares how quickly a central bank rate change reaches consumers through traditional banks vs.\ through CBDC
\item Traditional banking passes rate changes slowly (months of delay); CBDC transmits almost instantly
\item Faster transmission means monetary policy becomes more powerful---but also more disruptive
\end{itemize}

\bottomnote{Impulse response model based on Bindseil (2020). CBDC pass-through rate exceeds traditional banking channel}
\end{frame}

% Bank Disintermediation
\begin{frame}[t]{Bank Disintermediation Risk}
\begin{center}
\includegraphics[width=0.55\textwidth]{03_bank_disintermediation/chart.pdf}
\end{center}

\begin{itemize}
\item Each scenario shows how bank deposits evolve over time as CBDC becomes available
\item Steeper decline = more deposits moving to CBDC = greater stress on banks
\item The key variable is CBDC attractiveness: higher interest or better features accelerate deposit flight
\end{itemize}

\bottomnote{Deposit flight to CBDC could force banks to rely on costlier wholesale funding (borrowing from other banks instead of using deposits)}
\end{frame}

% Disintermediation Economics
\begin{frame}[t]{Economics of Disintermediation}
\begin{columns}[T]
\column{0.48\textwidth}
\textbf{The Concern}

If CBDC is attractive:
\begin{itemize}
\item Deposits migrate to CBDC
\item Banks lose cheap funding
\item Credit supply (total loans available) may contract
\end{itemize}

\textit{Example: If 20\% of deposits move to CBDC, banks must borrow elsewhere at higher rates, meaning fewer or costlier loans.}

\vspace{0.3em}
\textbf{Andolfatto (2021), an economist at the Federal Reserve Bank of St. Louis, modeled this:}
\begin{itemize}
\item CBDC gives depositors a credible alternative: if your bank offers poor rates, you can switch to CBDC, forcing banks to compete
\item Forces competitive deposit rates
\item Net welfare effect ambiguous (welfare = total well-being of society---economists cannot agree if everyone is better or worse off overall)
\end{itemize}

\column{0.48\textwidth}
\textbf{Mitigation Strategies}

Design features to limit migration:
\begin{itemize}
\item Holding limits (e.g., 3000 EUR)
\item Tiered remuneration (different rates for different amounts---e.g., 0\% up to 3000 EUR, negative above)
\item No interest on CBDC
\end{itemize}

\vspace{0.3em}
\textbf{Financial Stability}
\begin{itemize}
\item Digital bank runs faster---in 2023, Silicon Valley Bank lost \$42 billion in 24 hours via mobile apps; traditional runs took days as people queued at branches
\item Flight to safety amplified (the rush to safe assets is faster when transfers are instant)
\item Requires careful design
\end{itemize}
\end{columns}

\bottomnote{Design constraints trade off CBDC usefulness against banking system stability}
\end{frame}

% Financial Inclusion
\begin{frame}[t]{Financial Inclusion Economics}
\begin{columns}[T]
\column{0.48\textwidth}
\textbf{The Unbanked Problem}

Globally 1.4 billion unbanked adults:
\begin{itemize}
\item Lack documentation for accounts
\item Live far from bank branches
\item Cannot afford minimum balances
\end{itemize}

\vspace{0.3em}
\textbf{CBDC Potential}
\begin{itemize}
\item Lower KYC (Know Your Customer) requirements for small values (e.g., open account with just a phone number for transactions under 500 EUR)
\item Mobile-based access
\item No minimum balance required
\end{itemize}

\column{0.48\textwidth}
\textbf{Economic Analysis}

Benefits:
\begin{itemize}
\item Lower transaction costs
\item Entry to formal finance
\item Government transfer efficiency \textit{(Example: During COVID, US paper stimulus checks took weeks. Direct CBDC transfers could reach citizens in seconds at near-zero cost.)}
\end{itemize}

\vspace{0.3em}
Challenges:
\begin{itemize}
\item Digital divide (gap between those with and without internet access) persists \textit{(Example: In India, 300 million people lack smartphones---CBDC only on phones would exclude them, worsening inequality.)}
\item Infrastructure requirements
\item Financial literacy needs
\end{itemize}
\end{columns}

\bottomnote{Inclusion requires complementary policies; technology alone is insufficient}
\end{frame}

% Financial Inclusion Frontier Chart
\begin{frame}[t]{Financial Inclusion: The Technology Shift}
\begin{center}
\includegraphics[width=0.55\textwidth]{05_financial_inclusion_frontier/chart.pdf}
\end{center}

\begin{itemize}
\item The production possibility frontier (PPF) shows the best achievable trade-off between cost and inclusion
\item CBDC technology shifts the frontier outward: more inclusion is achievable at the same cost
\item The gap between the two curves represents the efficiency gain from digital infrastructure
\end{itemize}

\bottomnote{PPF model: traditional banking (inner curve) vs.\ CBDC-enabled inclusion (outer curve). Based on World Bank Findex data framework}
\end{frame}

% International Dimension
\begin{frame}[t]{International Currency Competition}
\begin{columns}[T]
\column{0.48\textwidth}
\textbf{Currency Competition}

CBDCs could intensify:
\begin{itemize}
\item Cross-border CBDC use
\item Challenge to dollar dominance \textit{(The US gains immense power from dollar dominance: ability to sanction enemies, borrow cheaply, and export inflation. Challengers want these privileges.)}
\item Regional currency blocs (groups of countries sharing or linking their currencies)
\end{itemize}

\vspace{0.3em}
\textbf{China's Strategy}
\begin{itemize}
\item e-CNY for domestic use
\item mBridge (a multi-CBDC platform connecting central banks) for wholesale cross-border
\item Reduce dependence on SWIFT (the global interbank messaging system)
\end{itemize}

\column{0.48\textwidth}
\textbf{US Response Dilemma}
\begin{itemize}
\item Digital dollar slower to develop
\item Privacy concerns prominent
\item Risk of losing first-mover advantage (the benefit of being first to market)
\end{itemize}

\vspace{0.3em}
\textbf{Economic Implications}
\begin{itemize}
\item Seigniorage redistribution \textit{(If people worldwide use digital yuan instead of dollars, China earns the profit from money creation that previously went to the US.)}
\item Sanctions effectiveness (the ability to economically punish other nations)
\item Monetary policy spillovers (when one country's policy unintentionally affects others---e.g., US rate hikes cause capital flight from emerging markets)
\end{itemize}
\end{columns}

\bottomnote{CBDCs add new dimension to international monetary system competition}
\end{frame}

% Cross-Border Payments
\begin{frame}[t]{Cross-Border CBDC Applications}
\begin{columns}[T]
\column{0.48\textwidth}
\textbf{Current Pain Points}
\begin{itemize}
\item High costs (average 6\%)
\item Slow settlement (2-5 days)
\item Limited transparency
\item Correspondent banking (international payments through intermediary banks)
\end{itemize}

\textit{Example: A worker sending \$200 home may lose \$12 in fees and wait 3 days.}

\vspace{0.3em}
\textbf{Wholesale CBDC Solution}
\begin{itemize}
\item Direct central bank settlement
\item Atomic swap (both sides of exchange happen together, or neither happens)
\item 24/7 operation possible
\end{itemize}

\column{0.48\textwidth}
\textbf{Multi-CBDC Projects}
\begin{itemize}
\item mBridge (China, UAE, HK, Thailand)---Asia-Middle East corridor, largest pilot
\item Project Dunbar (Singapore, Australia)---ASEAN focus, tests multi-currency settlement
\item Project Icebreaker (Nordic countries)---Nordic corridor, explores retail cross-border
\end{itemize}

\vspace{0.3em}
\textbf{Economic Benefits}
\begin{itemize}
\item Reduced Foreign Exchange (FX) settlement risk
\item Lower remittance costs
\item Faster trade finance (loans and guarantees for international commerce)
\end{itemize}
\end{columns}

\bottomnote{Wholesale CBDCs show clearer efficiency gains for cross-border payments}
\end{frame}

% Privacy vs Control
\begin{frame}[t]{Privacy vs. Policy Control Trade-off}
\begin{columns}[T]
\column{0.48\textwidth}
\textbf{Privacy Concerns}
\begin{itemize}
\item Government surveillance potential
\item Transaction tracking
\item Political control over spending
\end{itemize}

\vspace{0.3em}
\textbf{Design Options}
\begin{itemize}
\item Tiered privacy (small = anonymous)
\item Zero-knowledge proofs (proving you meet a requirement without revealing your data)
\item Third-party anonymity services (companies that act as privacy shields, similar to how a VPN hides internet activity)
\end{itemize}

\column{0.48\textwidth}
\textbf{Policy Control Benefits}
\begin{itemize}
\item AML/CFT (Combating the Financing of Terrorism) compliance
\item Tax enforcement
\item Targeted stimulus \textit{(Example: Instead of giving everyone \$1000, programmable CBDC could give \$2000 only to unemployed workers, or restrict spending to domestic goods.)}
\end{itemize}

\vspace{0.3em}
\textbf{Economic Framework}

The core trade-off in practice:
\begin{itemize}
\item Full anonymity: 0\% tax enforcement, easy money laundering
\item Full transparency: 100\% tax compliance, but citizens lose all financial privacy
\item Most designs aim for a middle ground (e.g., anonymous below 500 EUR, identified above)
\item Social preferences vary by country---no one-size-fits-all design
\end{itemize}
\end{columns}

\bottomnote{Privacy preferences differ: Germany (with Stasi memories) values privacy; China's social credit system reflects different norms}
\end{frame}

% Digital Euro Case
\begin{frame}[t]{Case Study: The Digital Euro}
\begin{columns}[T]
\column{0.48\textwidth}
\textbf{ECB (European Central Bank) Design Principles}
\begin{itemize}
\item Complement to cash, not replacement
\item Privacy by design (building privacy protections into the system from the start) (small payments)
\item Holding limits (\textasciitilde3000 EUR proposed)
\item No interest initially
\end{itemize}

\vspace{0.3em}
\textbf{Timeline}
\begin{itemize}
\item Investigation phase: 2021--2023
\item In October 2023, the ECB decided to move to the preparation phase
\item Preparation phase: 2023--2025
\item Potential launch: 2027--2028
\end{itemize}

\column{0.48\textwidth}
\textbf{Economic Rationale}
\begin{itemize}
\item Strategic autonomy (not dependent on Visa, Mastercard, or US tech giants)
\item Payment system resilience (if Visa goes down or a foreign company exits, CBDC ensures people can still pay)
\item Declining cash usage---if cash disappears, citizens lose their only form of public money; all transactions would go through private banks or card companies who charge fees and track purchases
\end{itemize}

\vspace{0.3em}
\textbf{Criticisms}
\begin{itemize}
\item Banks lobby (companies trying to influence politicians) against disintermediation
\item Privacy advocates concerned
\item Unclear consumer demand
\end{itemize}
\end{columns}

\bottomnote{Digital Euro reflects European values: privacy, strategic autonomy, bank coexistence}
\end{frame}

% Key Takeaways
\begin{frame}[t]{Key Takeaways}
\begin{columns}[T]
\column{0.48\textwidth}
\textbf{Main Conclusions}
\begin{enumerate}
\item CBDC design involves fundamental trade-offs
\item Disintermediation risk requires mitigation
\item Monetary policy transmission could improve
\item International competition is intensifying
\end{enumerate}

\column{0.48\textwidth}
\textbf{Economic Framework}
\begin{itemize}
\item Retail vs. wholesale scope
\item Token vs. account architecture
\item Privacy vs. policy control
\item Inclusion vs. stability
\end{itemize}
\end{columns}

\vspace{0.5em}
\textbf{Core Insight}

CBDCs are not simply ``digital cash''---they require careful economic analysis of trade-offs between competing objectives. No design satisfies all goals simultaneously. Example: the Digital Euro proposes a \textasciitilde3000 EUR holding limit to balance usefulness against bank disintermediation risk.

\bottomnote{Next lesson: Payment Systems Economics}
\end{frame}

% Key Terms Part 1
\begin{frame}[t]{Key Terms (1/2)}
\begin{columns}[T]
\column{0.48\textwidth}
\textbf{CBDC (Central Bank Digital Currency)}
Digital cash issued by the central bank. If you hold 100 in CBDC, the central bank guarantees you that value---unlike bank deposits which depend on your bank staying solvent.

\vspace{0.3em}
\textbf{Retail CBDC}
CBDC available to the general public for everyday transactions.

\vspace{0.3em}
\textbf{Wholesale CBDC}
CBDC restricted to financial institutions for interbank settlements.

\vspace{0.3em}
\textbf{Token-Based CBDC}
CBDC where validity is verified by the instrument itself (like cash), enabling offline transactions.

\vspace{0.3em}
\textbf{Account-Based CBDC}
CBDC where validity requires verification of the holder's identity against an account.

\column{0.48\textwidth}
\textbf{Bank Disintermediation}
Risk that CBDC adoption draws deposits away from commercial banks, reducing their lending capacity.

\vspace{0.3em}
\textbf{Monetary Sovereignty}
A nation's ability to control its own money supply and monetary policy independently.

\vspace{0.3em}
\textbf{Seigniorage}
The profit a government earns from issuing currency---the difference between the face value of money and its production cost.

\vspace{0.3em}
\textbf{AML (Anti-Money Laundering)}
Laws and regulations designed to prevent criminals from disguising illegally obtained money as legitimate income.

\vspace{0.3em}
\textbf{KYC (Know Your Customer)}
The process of verifying the identity of customers, required by financial regulations.
\end{columns}

\bottomnote{Terms continued on next slide}
\end{frame}

% Key Terms Part 2
\begin{frame}[t]{Key Terms (2/2)}
\begin{columns}[T]
\column{0.48\textwidth}
\textbf{Correspondent Banking}
An arrangement where one bank provides services on behalf of another, commonly used for international payments.

\vspace{0.3em}
\textbf{Atomic Swap}
A technology enabling exchange of different currencies simultaneously---both transfers complete or neither does.

\vspace{0.3em}
\textbf{Wholesale Funding}
Money banks borrow from other financial institutions (rather than customer deposits) to fund operations.

\vspace{0.3em}
\textbf{Tiered Remuneration}
Different interest rates for different amounts held---e.g., 0\% on first 3000 EUR, negative rates above.

\vspace{0.3em}
\textbf{Helicopter Money}
Direct cash transfers from central bank to citizens, bypassing banks---named for the image of dropping money from helicopters.

\column{0.48\textwidth}
\textbf{Negative Interest Rates}
A policy where depositors pay to keep money in accounts rather than earning interest---used to encourage spending.

\vspace{0.3em}
\textbf{Interbank Settlement}
The process by which banks transfer money between themselves to complete transactions.

\vspace{0.3em}
\textbf{Flight to Safety}
When investors move money from risky assets to safe ones during uncertainty---with CBDC, could mean moving from bank deposits to CBDC.

\vspace{0.3em}
\textbf{Financial Inclusion}
Ensuring all people have access to useful and affordable financial services.

\vspace{0.3em}
\textbf{Zero-Knowledge Proofs}
Cryptography that proves you meet a requirement without revealing your data---like proving you're over 18 without showing your birthdate.
\end{columns}

\bottomnote{CBDC design choices have profound implications for monetary policy and financial stability}
\end{frame}

% Key Terms Part 3
\begin{frame}[t]{Key Terms (3/3)}
\begin{columns}[T]
\column{0.48\textwidth}
\textbf{Liability (Finance)}
A legal obligation to pay---if you hold CBDC, the central bank owes you that value. Unlike an asset (what you own), a liability is what you owe.

\vspace{0.3em}
\textbf{Monetary Policy}
The central bank's decisions about interest rates and money supply to control inflation and support the economy. Examples: raising rates to fight inflation, cutting rates to boost growth.

\vspace{0.3em}
\textbf{Bank Reserves}
Money that commercial banks hold at the central bank---like a bank's own bank account. Required by regulation to ensure banks can meet withdrawals.

\column{0.48\textwidth}
\textbf{Credit Supply}
The total amount of loans banks can offer to businesses and people. When credit supply contracts, fewer loans are available, slowing economic activity.

\vspace{0.3em}
\textbf{Spillovers}
When one country's policy unintentionally affects other countries. Example: US interest rate hikes cause capital to leave emerging markets, weakening their currencies.

\vspace{0.3em}
\textbf{Trade Finance}
Loans and guarantees that help companies buy and sell goods internationally. Without trade finance, global commerce would be much slower and riskier.

\vspace{0.3em}
\textbf{CFT (Combating the Financing of Terrorism)}
Laws preventing terrorist groups from receiving money---often paired with AML (Anti-Money Laundering).

\vspace{0.3em}
\textbf{FX (Foreign Exchange)}
The exchange of one currency for another---e.g., converting dollars to euros.

\vspace{0.3em}
\textbf{First-Mover Advantage}
The benefit of being first to market---early CBDC adopters may set standards others must follow.

\vspace{0.3em}
\textbf{Sanctions}
Economic penalties imposed by governments---restricting trade or freezing assets of target countries.

\vspace{0.3em}
\textbf{Digital Bank Run}
A bank run accelerated by digital technology---deposits can flee via mobile apps in hours, not days.
\end{columns}

\bottomnote{These Key Terms slides serve as a reference glossary---focus on the 5--8 bolded terms most relevant to your assignments; return here as needed}
\end{frame}

% References
\begin{frame}[t]{Further Reading}
\begin{columns}[T]
\column{0.48\textwidth}
\textbf{Academic Papers}
\begin{itemize}
\item Andolfatto (2021): ``Assessing the Impact of CBDC on Private Banks''
\item Brunnermeier \& Landau (2022): ``The Digital Euro''
\item Auer et al. (2022): ``CBDCs Beyond Borders''
\end{itemize}

\column{0.48\textwidth}
\textbf{Central Bank Publications}
\begin{itemize}
\item ECB (2023): ``A Stocktake on the Digital Euro''
\item BIS -- Bank for International Settlements (2021): ``CBDCs: An Opportunity for the Monetary System''
\item Fed -- Federal Reserve (2022): ``Money and Payments''
\end{itemize}
\end{columns}

\bottomnote{All readings available on course platform}
\end{frame}

\end{document}
