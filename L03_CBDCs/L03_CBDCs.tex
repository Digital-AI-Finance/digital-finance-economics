\documentclass[8pt,aspectratio=169]{beamer}
\usetheme{Madrid}
\usepackage{graphicx}
\usepackage{booktabs}
\usepackage{adjustbox}
\usepackage{multicol}
\usepackage{amsmath}

% Color definitions
\definecolor{mlblue}{RGB}{0,102,204}
\definecolor{mlpurple}{RGB}{51,51,178}
\definecolor{mllavender}{RGB}{173,173,224}
\definecolor{mllavender2}{RGB}{193,193,232}
\definecolor{mllavender3}{RGB}{204,204,235}
\definecolor{mllavender4}{RGB}{214,214,239}
\definecolor{mlorange}{RGB}{255, 127, 14}
\definecolor{mlgreen}{RGB}{44, 160, 44}
\definecolor{mlred}{RGB}{214, 39, 40}
\definecolor{mlgray}{RGB}{127, 127, 127}

\definecolor{lightgray}{RGB}{240, 240, 240}
\definecolor{midgray}{RGB}{180, 180, 180}

\setbeamercolor{palette primary}{bg=mllavender3,fg=mlpurple}
\setbeamercolor{palette secondary}{bg=mllavender2,fg=mlpurple}
\setbeamercolor{palette tertiary}{bg=mllavender,fg=white}
\setbeamercolor{palette quaternary}{bg=mlpurple,fg=white}

\setbeamercolor{structure}{fg=mlpurple}
\setbeamercolor{section in toc}{fg=mlpurple}
\setbeamercolor{subsection in toc}{fg=mlblue}
\setbeamercolor{title}{fg=mlpurple}
\setbeamercolor{frametitle}{fg=mlpurple,bg=mllavender3}
\setbeamercolor{block title}{bg=mllavender2,fg=mlpurple}
\setbeamercolor{block body}{bg=mllavender4,fg=black}

\setbeamertemplate{navigation symbols}{}
\setbeamertemplate{itemize items}[circle]
\setbeamertemplate{enumerate items}[default]
\setbeamersize{text margin left=5mm,text margin right=5mm}

\newcommand{\bottomnote}[1]{%
\vfill
\vspace{-2mm}
\textcolor{mllavender2}{\rule{\textwidth}{0.4pt}}
\vspace{1mm}
\footnotesize
\textbf{#1}
}

\title{Central Bank Digital Currencies (CBDCs)}
\subtitle{L03: The Economics of Public Digital Money}
\author{Economics of Digital Finance}
\institute{BSc Course}
\date{}

\begin{document}

% Title slide
\begin{frame}[plain]
\titlepage
\end{frame}

% Outline
\begin{frame}[t]{Lesson Overview}
\begin{columns}[T]
\column{0.48\textwidth}
\textbf{Today's Topics}
\begin{enumerate}
\item CBDC design choices and trade-offs
\item Monetary policy transmission
\item Bank disintermediation risk
\item Financial inclusion economics
\item International currency competition
\end{enumerate}

\column{0.48\textwidth}
\textbf{Learning Objectives}
\begin{itemize}
\item Analyze CBDC design trade-offs
\item Assess monetary policy implications
\item Evaluate disintermediation risks
\item Understand global CBDC landscape
\end{itemize}
\end{columns}

\bottomnote{CBDCs represent central banks' response to private digital currencies}
\end{frame}

% What is a CBDC?
\begin{frame}[t]{What is a Central Bank Digital Currency?}
\begin{columns}[T]
\column{0.48\textwidth}
\textbf{Definition}

A CBDC is a digital form of central bank money:
\begin{itemize}
\item Direct liability of central bank
\item Digital (not physical)
\item Widely accessible (retail) or restricted (wholesale)
\end{itemize}

\vspace{0.3em}
\textbf{Not a CBDC}
\begin{itemize}
\item Bank reserves (already digital)
\item Commercial bank money
\item Stablecoins (private liability---can collapse, as Terra/UST did in 2022 losing \$40B+)
\end{itemize}

\column{0.48\textwidth}
\textbf{Motivations}

Central banks cite multiple goals:
\begin{itemize}
\item Maintain monetary sovereignty (control over the nations money supply)
\item Improve payment efficiency
\item Promote financial inclusion
\item Counter private digital currencies
\end{itemize}

\vspace{0.3em}
\textbf{Key Economic Question}

Does public benefit exceed costs and risks?
\end{columns}

\bottomnote{CBDC = digital cash issued by central bank; distinct from existing digital money}
\end{frame}

% Global CBDC Status
\begin{frame}[t]{Global CBDC Development Status}
\begin{center}
\includegraphics[width=0.65\textwidth]{01_cbdc_world_map/chart.pdf}
\end{center}

\bottomnote{130+ countries exploring CBDCs; China's e-CNY (digital yuan) is most advanced large-economy pilot}
\end{frame}

% CBDC Design Space
\begin{frame}[t]{CBDC Design Space: Key Choices}
\begin{center}
\includegraphics[width=0.58\textwidth]{02_cbdc_design_space/chart.pdf}
\end{center}

\bottomnote{Design choices involve trade-offs between privacy, efficiency, and policy goals}
\end{frame}

% Retail vs Wholesale
\begin{frame}[t]{Retail vs. Wholesale CBDCs}
\begin{columns}[T]
\column{0.48\textwidth}
\textbf{Retail CBDC}

For general public use:
\begin{itemize}
\item Replaces/complements cash
\item Consumer payment instrument
\item Requires distribution network
\end{itemize}

\vspace{0.3em}
Economic considerations:
\begin{itemize}
\item High operational costs
\item Privacy vs. AML (Anti-Money Laundering) trade-off
\item Competition with banks
\end{itemize}

\column{0.48\textwidth}
\textbf{Wholesale CBDC}

For financial institutions:
\begin{itemize}
\item Interbank settlement (transferring money between banks)
\item Securities transactions
\item Cross-border payments
\end{itemize}

\vspace{0.3em}
Economic considerations:
\begin{itemize}
\item Lower operational burden
\item Efficiency gains clearer
\item Less disruptive to banking
\end{itemize}
\end{columns}

\bottomnote{Most advanced economies focus on retail; wholesale offers clearer near-term benefits}
\end{frame}

% Token vs Account
\begin{frame}[t]{Token-Based vs. Account-Based CBDCs}
\begin{columns}[T]
\column{0.48\textwidth}
\textbf{Token-Based}

Like digital cash:
\begin{itemize}
\item Verify the instrument, not holder
\item Can enable anonymity
\item Offline transactions possible
\end{itemize}

\vspace{0.3em}
Economic implications:
\begin{itemize}
\item Lower transaction costs
\item Privacy preserving
\item Harder to implement AML
\end{itemize}

\column{0.48\textwidth}
\textbf{Account-Based}

Like bank accounts:
\begin{itemize}
\item Verify the identity of holder
\item Full transaction records
\item Programmable features possible (money with built-in rules, like expiration dates)
\end{itemize}

\vspace{0.3em}
Economic implications:
\begin{itemize}
\item Interest-bearing feasible
\item Targeted policies possible
\item Privacy concerns
\end{itemize}
\end{columns}

\bottomnote{Most designs are hybrid: token-like for small values, account-like for large}
\end{frame}

% Monetary Policy Transmission
\begin{frame}[t]{Monetary Policy Transmission with CBDCs}
\begin{columns}[T]
\column{0.48\textwidth}
\textbf{Traditional Channels}

Interest rate channel:
$$i_{\text{policy}} \rightarrow i_{\text{deposit}} \rightarrow C, I$$
\textit{(When the central bank changes its policy rate ($i_{\text{policy}}$), banks adjust deposit rates ($i_{\text{deposit}}$), affecting consumption (C) and investment (I).)}
\begin{itemize}
\item Works through bank intermediation
\item Banks pass rate changes to customers
\item Time lags in transmission
\end{itemize}

\vspace{0.3em}
\textbf{CBDC Impact}

If CBDC is interest-bearing:
$$i_{\text{CBDC}} \rightarrow i_{\text{deposit}}$$
\textit{(If CBDC pays 2\% interest, banks must match or exceed this to keep deposits---CBDC rate sets a floor.)}
\begin{itemize}
\item Direct transmission to public
\item Floor on deposit rates
\end{itemize}

\column{0.48\textwidth}
\textbf{Enhanced Policy Options}

Interest-bearing CBDC enables:
\begin{itemize}
\item Negative interest rates (charging people to hold money, to encourage spending)
\item Helicopter money (direct cash from central bank to citizens---named for the image of dropping money from helicopters)
\item Time-limited money (expiring)
\end{itemize}

\vspace{0.3em}
\textbf{Concerns}
\begin{itemize}
\item Political resistance to negative rates
\textit{(Why accept losing money? Without physical cash, people cannot escape---some see this as coercive.)}
\item Privacy implications of targeting
\item Complexity of implementation
\end{itemize}
\end{columns}

\bottomnote{CBDC could strengthen monetary policy but raises political economy concerns}
\end{frame}

% Bank Disintermediation
\begin{frame}[t]{Bank Disintermediation Risk}
\begin{center}
\includegraphics[width=0.70\textwidth]{03_bank_disintermediation/chart.pdf}
\end{center}

\bottomnote{Deposit flight to CBDC could force banks to rely on costlier wholesale funding (borrowing from other banks instead of using deposits)}
\end{frame}

% Disintermediation Economics
\begin{frame}[t]{Economics of Disintermediation}
\begin{columns}[T]
\column{0.48\textwidth}
\textbf{The Concern}

If CBDC is attractive:
\begin{itemize}
\item Deposits migrate to CBDC
\item Banks lose cheap funding
\item Credit supply may contract
\end{itemize}

\textit{Example: If 20\% of deposits move to CBDC, banks must borrow elsewhere at higher rates, meaning fewer or costlier loans.}

\vspace{0.3em}
\textbf{Andolfatto (2021), a Federal Reserve economist, modeled this:}
\begin{itemize}
\item CBDC as outside option
\item Forces competitive deposit rates
\item Net welfare effect ambiguous
\end{itemize}

\column{0.48\textwidth}
\textbf{Mitigation Strategies}

Design features to limit migration:
\begin{itemize}
\item Holding limits (e.g., 3000 EUR)
\item Tiered remuneration (different rates for different amounts---e.g., 0\% up to 3000 EUR, negative above)
\item No interest on CBDC
\end{itemize}

\vspace{0.3em}
\textbf{Financial Stability}
\begin{itemize}
\item Digital bank runs faster (instant CBDC transfers vs. slow cash withdrawals)
\item Flight to safety amplified (the rush to safe assets is faster when transfers are instant)
\item Requires careful design
\end{itemize}
\end{columns}

\bottomnote{Design constraints trade off CBDC usefulness against banking system stability}
\end{frame}

% Financial Inclusion
\begin{frame}[t]{Financial Inclusion Economics}
\begin{columns}[T]
\column{0.48\textwidth}
\textbf{The Unbanked Problem}

Globally 1.4 billion unbanked adults:
\begin{itemize}
\item Lack documentation for accounts
\item Live far from bank branches
\item Cannot afford minimum balances
\end{itemize}

\vspace{0.3em}
\textbf{CBDC Potential}
\begin{itemize}
\item Lower KYC (Know Your Customer) requirements for small values
\item Mobile-based access
\item No minimum balance required
\end{itemize}

\column{0.48\textwidth}
\textbf{Economic Analysis}

Benefits:
\begin{itemize}
\item Lower transaction costs
\item Entry to formal finance
\item Government transfer efficiency
\end{itemize}

\vspace{0.3em}
Challenges:
\begin{itemize}
\item Digital divide persists
\item Infrastructure requirements
\item Financial literacy needs
\end{itemize}
\end{columns}

\bottomnote{Inclusion requires complementary policies; technology alone is insufficient}
\end{frame}

% International Dimension
\begin{frame}[t]{International Currency Competition}
\begin{columns}[T]
\column{0.48\textwidth}
\textbf{Currency Competition}

CBDCs could intensify:
\begin{itemize}
\item Cross-border CBDC use
\item Challenge to dollar dominance
\item Regional currency blocs
\end{itemize}

\vspace{0.3em}
\textbf{China's Strategy}
\begin{itemize}
\item e-CNY for domestic use
\item mBridge (a multi-CBDC platform connecting central banks) for wholesale cross-border
\item Reduce dependence on SWIFT (the global interbank messaging system)
\end{itemize}

\column{0.48\textwidth}
\textbf{US Response Dilemma}
\begin{itemize}
\item Digital dollar slower to develop
\item Privacy concerns prominent
\item Risk of losing first-mover advantage
\end{itemize}

\vspace{0.3em}
\textbf{Economic Implications}
\begin{itemize}
\item Seigniorage redistribution
\item Sanctions effectiveness
\item Monetary policy spillovers
\end{itemize}
\end{columns}

\bottomnote{CBDCs add new dimension to international monetary system competition}
\end{frame}

% Cross-Border Payments
\begin{frame}[t]{Cross-Border CBDC Applications}
\begin{columns}[T]
\column{0.48\textwidth}
\textbf{Current Pain Points}
\begin{itemize}
\item High costs (average 6\%)
\item Slow settlement (2-5 days)
\item Limited transparency
\item Correspondent banking (international payments through intermediary banks)
\end{itemize}

\textit{Example: A worker sending \$200 home may lose \$12 in fees and wait 3 days.}

\vspace{0.3em}
\textbf{Wholesale CBDC Solution}
\begin{itemize}
\item Direct central bank settlement
\item Atomic swap (both sides of exchange happen together, or neither happens)
\item 24/7 operation possible
\end{itemize}

\column{0.48\textwidth}
\textbf{Multi-CBDC Projects}
\begin{itemize}
\item mBridge (China, UAE, HK, Thailand)
\item Project Dunbar (Singapore, Australia)
\item Project Icebreaker (Nordic countries)
\end{itemize}

\vspace{0.3em}
\textbf{Economic Benefits}
\begin{itemize}
\item Reduced FX (foreign exchange) settlement risk
\item Lower remittance costs
\item Faster trade finance
\end{itemize}
\end{columns}

\bottomnote{Wholesale CBDCs show clearer efficiency gains for cross-border payments}
\end{frame}

% Privacy vs Control
\begin{frame}[t]{Privacy vs. Policy Control Trade-off}
\begin{columns}[T]
\column{0.48\textwidth}
\textbf{Privacy Concerns}
\begin{itemize}
\item Government surveillance potential
\item Transaction tracking
\item Political control over spending
\end{itemize}

\vspace{0.3em}
\textbf{Design Options}
\begin{itemize}
\item Tiered privacy (small = anonymous)
\item Zero-knowledge proofs (proving you meet a requirement without revealing your data)
\item Third-party anonymity services
\end{itemize}

\column{0.48\textwidth}
\textbf{Policy Control Benefits}
\begin{itemize}
\item AML/CFT (Combating the Financing of Terrorism) compliance
\item Tax enforcement
\item Targeted stimulus
\end{itemize}

\vspace{0.3em}
\textbf{Economic Framework}

Trade-off function:
$$U = f(\text{Privacy}, \text{Policy Effectiveness})$$
\textit{(Usefulness (U) depends on both privacy and policy control---more of one often means less of the other.)}
\begin{itemize}
\item Social preferences vary by country
\item No one-size-fits-all design
\end{itemize}
\end{columns}

\bottomnote{Privacy preferences differ: Germany (with Stasi memories) values privacy; China's social credit system reflects different norms}
\end{frame}

% Digital Euro Case
\begin{frame}[t]{Case Study: The Digital Euro}
\begin{columns}[T]
\column{0.48\textwidth}
\textbf{ECB (European Central Bank) Design Principles}
\begin{itemize}
\item Complement to cash, not replacement
\item Privacy by design (small payments)
\item Holding limits (\textasciitilde3000 EUR proposed)
\item No interest initially
\end{itemize}

\vspace{0.3em}
\textbf{Timeline}
\begin{itemize}
\item Investigation phase: 2021-2023
\item Preparation phase: 2023-2025
\item Potential launch: 2027-2028
\end{itemize}

\column{0.48\textwidth}
\textbf{Economic Rationale}
\begin{itemize}
\item Strategic autonomy (not dependent on Visa, Mastercard, or US tech giants)
\item Payment system resilience
\item Declining cash usage
\end{itemize}

\vspace{0.3em}
\textbf{Criticisms}
\begin{itemize}
\item Banks lobby against disintermediation
\item Privacy advocates concerned
\item Unclear consumer demand
\end{itemize}
\end{columns}

\bottomnote{Digital Euro reflects European values: privacy, strategic autonomy, bank coexistence}
\end{frame}

% Key Takeaways
\begin{frame}[t]{Key Takeaways}
\begin{columns}[T]
\column{0.48\textwidth}
\textbf{Main Conclusions}
\begin{enumerate}
\item CBDC design involves fundamental trade-offs
\item Disintermediation risk requires mitigation
\item Monetary policy transmission could improve
\item International competition is intensifying
\end{enumerate}

\column{0.48\textwidth}
\textbf{Economic Framework}
\begin{itemize}
\item Retail vs. wholesale scope
\item Token vs. account architecture
\item Privacy vs. policy control
\item Inclusion vs. stability
\end{itemize}
\end{columns}

\vspace{0.5em}
\textbf{Core Insight}

CBDCs are not simply ``digital cash''---they require careful economic analysis of trade-offs between competing objectives. No design satisfies all goals simultaneously.

\bottomnote{Next lesson: Payment Systems Economics}
\end{frame}

% Key Terms Part 1
\begin{frame}[t]{Key Terms (1/2)}
\begin{columns}[T]
\column{0.48\textwidth}
\textbf{CBDC (Central Bank Digital Currency)}
Digital form of central bank money, a liability of the central bank denominated in the national unit of account.

\vspace{0.3em}
\textbf{Retail CBDC}
CBDC available to the general public for everyday transactions.

\vspace{0.3em}
\textbf{Wholesale CBDC}
CBDC restricted to financial institutions for interbank settlements.

\vspace{0.3em}
\textbf{Token-Based CBDC}
CBDC where validity is verified by the instrument itself (like cash), enabling offline transactions.

\vspace{0.3em}
\textbf{Account-Based CBDC}
CBDC where validity requires verification of the holder's identity against an account.

\column{0.48\textwidth}
\textbf{Bank Disintermediation}
Risk that CBDC adoption draws deposits away from commercial banks, reducing their lending capacity.

\vspace{0.3em}
\textbf{Monetary Sovereignty}
A nation's ability to control its own money supply and monetary policy independently.

\vspace{0.3em}
\textbf{Seigniorage}
The profit a government earns from issuing currency---the difference between the face value of money and its production cost.

\vspace{0.3em}
\textbf{AML (Anti-Money Laundering)}
Laws and regulations designed to prevent criminals from disguising illegally obtained money as legitimate income.

\vspace{0.3em}
\textbf{KYC (Know Your Customer)}
The process of verifying the identity of customers, required by financial regulations.
\end{columns}

\bottomnote{Terms continued on next slide}
\end{frame}

% Key Terms Part 2
\begin{frame}[t]{Key Terms (2/2)}
\begin{columns}[T]
\column{0.48\textwidth}
\textbf{Correspondent Banking}
An arrangement where one bank provides services on behalf of another, commonly used for international payments.

\vspace{0.3em}
\textbf{Atomic Swap}
A technology enabling exchange of different currencies simultaneously---both transfers complete or neither does.

\vspace{0.3em}
\textbf{Wholesale Funding}
Money banks borrow from other financial institutions (rather than customer deposits) to fund operations.

\vspace{0.3em}
\textbf{Tiered Remuneration}
Different interest rates for different amounts held---e.g., 0\% on first 3000 EUR, negative rates above.

\vspace{0.3em}
\textbf{Helicopter Money}
Direct cash transfers from central bank to citizens, bypassing banks---named for the image of dropping money from helicopters.

\column{0.48\textwidth}
\textbf{Negative Interest Rates}
A policy where depositors pay to keep money in accounts rather than earning interest---used to encourage spending.

\vspace{0.3em}
\textbf{Interbank Settlement}
The process by which banks transfer money between themselves to complete transactions.

\vspace{0.3em}
\textbf{Flight to Safety}
When investors move money from risky assets to safe ones during uncertainty---with CBDC, could mean moving from bank deposits to CBDC.

\vspace{0.3em}
\textbf{Financial Inclusion}
Ensuring all people have access to useful and affordable financial services.

\vspace{0.3em}
\textbf{Zero-Knowledge Proofs}
Cryptography that proves you meet a requirement without revealing your data---like proving you're over 18 without showing your birthdate.
\end{columns}

\bottomnote{CBDC design choices have profound implications for monetary policy and financial stability}
\end{frame}

% References
\begin{frame}[t]{Further Reading}
\begin{columns}[T]
\column{0.48\textwidth}
\textbf{Academic Papers}
\begin{itemize}
\item Andolfatto (2021): ``Assessing the Impact of CBDC on Private Banks''
\item Brunnermeier \& Landau (2022): ``The Digital Euro''
\item Auer et al. (2022): ``CBDCs Beyond Borders''
\end{itemize}

\column{0.48\textwidth}
\textbf{Central Bank Publications}
\begin{itemize}
\item ECB (2023): ``A Stocktake on the Digital Euro''
\item BIS -- Bank for International Settlements (2021): ``CBDCs: An Opportunity for the Monetary System''
\item Fed -- Federal Reserve (2022): ``Money and Payments''
\end{itemize}
\end{columns}

\bottomnote{All readings available on course platform}
\end{frame}

\end{document}
