\documentclass[8pt,aspectratio=169]{beamer}
\usetheme{Madrid}
\usepackage{graphicx}
\usepackage{booktabs}
\usepackage{adjustbox}
\usepackage{multicol}
\usepackage{amsmath}

% Color definitions
\definecolor{mlblue}{RGB}{0,102,204}
\definecolor{mlpurple}{RGB}{51,51,178}
\definecolor{mllavender}{RGB}{173,173,224}
\definecolor{mllavender2}{RGB}{193,193,232}
\definecolor{mllavender3}{RGB}{204,204,235}
\definecolor{mllavender4}{RGB}{214,214,239}
\definecolor{mlorange}{RGB}{255, 127, 14}
\definecolor{mlgreen}{RGB}{44, 160, 44}
\definecolor{mlred}{RGB}{214, 39, 40}
\definecolor{mlgray}{RGB}{127, 127, 127}

\definecolor{lightgray}{RGB}{240, 240, 240}
\definecolor{midgray}{RGB}{180, 180, 180}

\setbeamercolor{palette primary}{bg=mllavender3,fg=mlpurple}
\setbeamercolor{palette secondary}{bg=mllavender2,fg=mlpurple}
\setbeamercolor{palette tertiary}{bg=mllavender,fg=white}
\setbeamercolor{palette quaternary}{bg=mlpurple,fg=white}

\setbeamercolor{structure}{fg=mlpurple}
\setbeamercolor{section in toc}{fg=mlpurple}
\setbeamercolor{subsection in toc}{fg=mlblue}
\setbeamercolor{title}{fg=mlpurple}
\setbeamercolor{frametitle}{fg=mlpurple,bg=mllavender3}
\setbeamercolor{block title}{bg=mllavender2,fg=mlpurple}
\setbeamercolor{block body}{bg=mllavender4,fg=black}

\setbeamertemplate{navigation symbols}{}
\setbeamertemplate{itemize items}[circle]
\setbeamertemplate{enumerate items}[default]
\setbeamersize{text margin left=5mm,text margin right=5mm}

\newcommand{\bottomnote}[1]{%
\vfill
\vspace{-2mm}
\textcolor{mllavender2}{\rule{\textwidth}{0.4pt}}
\vspace{1mm}
\footnotesize
\textbf{#1}
}

\title{CBDCs: Mathematical Models and Welfare Analysis}
\subtitle{L03 Extended: Formalizing the Economics of Public Digital Money\\[0.3em]\normalsize From deposit market competition to international game theory}
\author{Economics of Digital Finance}
\institute{BSc Course}
\date{}

\begin{document}

% Title slide
\begin{frame}[plain]
\titlepage
\end{frame}

%% ============================================================
%% SECTION 1: Bridge from Basic Lecture (3 slides)
%% ============================================================
\section{Bridge from Basic Lecture}

% --- Frame 1: Opening Comic ---
\begin{frame}[t]{Welcome Back: From Concepts to Math}
\begin{center}
\textit{[XKCD \#2030: ``Blockchain'']}\\[0.5em]
\small Source: xkcd.com/2030 by Randall Munroe, CC BY-NC 2.5\\[1em]
\normalsize ``Today we formalize the economic models behind the concepts introduced in the basic lecture.''
\end{center}
\bottomnote{Comic sets the stage for our mathematical deep dive into CBDC economics.}
\end{frame}

% --- Frame 2: From Concepts to Models ---
\begin{frame}[t]{From Concepts to Models}
\begin{columns}[T]
\column{0.48\textwidth}
\textbf{What We Know}
\begin{enumerate}
\item CBDC is a direct liability of the central bank
\item Design trade-offs exist (privacy vs compliance, access vs control)
\item Bank disintermediation (loss of deposits to CBDC) is a key risk
\item CBDC can improve monetary policy transmission (how central bank rate changes flow through to the real economy)
\end{enumerate}

\column{0.48\textwidth}
\textbf{What We'll Formalize}
\begin{enumerate}
\item Andolfatto (2021): bank competition model with CBDC entry
\item Brunnermeier--Niepelt (2019): equivalence theorem for public vs private money
\item Barrdear--Kumhof (2022): optimal CBDC interest rate
\item Benigno et al.\ (2022): international CBDC game theory
\end{enumerate}
\end{columns}

\bottomnote{This lecture builds mathematical foundations for the concepts introduced in the basic CBDC lecture}
\end{frame}

% --- Frame 3: Mathematical Toolkit ---
\begin{frame}[t]{Mathematical Toolkit}
\begin{columns}[T]
\column{0.48\textwidth}
\textbf{Key Mathematical Concepts}
\begin{itemize}
\item Utility maximization (choosing the best outcome given constraints)
\item First-order conditions (FOC = setting the derivative equal to zero to find the optimum)
\item Nash equilibrium (a situation where no player benefits from changing strategy unilaterally)
\end{itemize}

\column{0.48\textwidth}
\textbf{Notation Preview}

\vspace{0.3em}
\begin{adjustbox}{max width=\textwidth}
\begin{tabular}{ll}
\toprule
\textbf{Symbol} & \textbf{Meaning} \\
\midrule
$a$ & Inelastic deposit base (EUR trillions) \\
$b$ & Deposit rate sensitivity \\
$r_D$ & Bank deposit interest rate \\
$r_L$ & Bank lending rate \\
$r_{CBDC}$ & CBDC interest rate \\
$D$ & Bank deposits (EUR trillions) \\
$W$ & Aggregate welfare \\
$V$ & Country payoff in game \\
\bottomrule
\end{tabular}
\end{adjustbox}
\end{columns}

\bottomnote{All derivations use BSc-level calculus. Full notation table in Appendix A1}
\end{frame}

%% ============================================================
%% SECTION 2: Andolfatto Bank Competition Model (5 slides)
%% ============================================================
\section{Andolfatto Bank Competition Model}

% --- Frame 4: The Bank's Problem -- Setup ---
\begin{frame}[t]{The Bank's Problem -- Setup}
\begin{columns}[T]
\column{0.48\textwidth}
\textbf{Model Setup}
\begin{itemize}
\item Monopoly bank chooses deposit rate $r_D$ to maximize profit
\item Deposit supply: $D(r_D) = a + b \cdot r_D$ (depositors supply more when rate is higher)
\item Parameters: $a = 5.4$ (EUR trillions, the inelastic deposit base---deposits that stay regardless of rate), $b = 200$ (EUR trillions per unit rate, deposit sensitivity to rate changes)
\end{itemize}

\vspace{0.3em}
\textbf{Worked example:} if $r_D = 0.4\%$ then $D = 5.4 + 200 \times 0.004 = 6.2$ trillion EUR

\column{0.48\textwidth}
\textbf{Bank Profit Function}
\begin{itemize}
\item $\Pi_B = (r_L - r_D) \cdot D(r_D) - FC$
\item Spread $= r_L - r_D$ (the bank's margin on each euro of deposits)
\item Lending rate $r_L = 3.5\%$, fixed costs $FC = 0$ (simplified)
\end{itemize}

\vspace{0.3em}
\textbf{Worked example:}
$$\Pi_B = (0.035 - 0.004) \times 6{,}200\text{B} = 0.031 \times 6{,}200\text{B} = 192.2\text{B EUR}$$
\end{columns}

\bottomnote{Based on Andolfatto (2021), combining Klein--Monti monopoly bank with Diamond (1965) government debt}
\end{frame}

% --- Frame 5: The Bank's Optimal Choice (Pre-CBDC) ---
\begin{frame}[t]{The Bank's Optimal Choice (Pre-CBDC)}
\begin{columns}[T]
\column{0.48\textwidth}
\textbf{Profit Maximization}
\begin{itemize}
\item FOC: $\dfrac{d\Pi_B}{dr_D} = 0$
\item With linear deposit supply:
$$r_D^* = \frac{r_L - a/b}{2}$$
\item \textbf{Worked example:} $a/b = 5.4/200 = 0.027$, so
$$r_D^* = \frac{0.035 - 0.027}{2} = 0.004 = 0.4\%$$
\item At this rate: $D^* = 6.2$T EUR, $\Pi_B^* = 192.2$B EUR
\end{itemize}

\column{0.48\textwidth}
\textbf{Monopoly Distortion}
\begin{itemize}
\item Monopoly bank pays LESS than competitive rate
\item Competitive rate would approach $r_L = 3.5\%$ (zero profit)
\item Gap $= r_L - r_D^* = 3.5\% - 0.4\% = 3.1$ percentage points measures monopoly power
\item Depositors lose surplus; bank captures it as profit
\end{itemize}
\end{columns}

\bottomnote{The monopoly bank restricts deposit rates below the competitive level, just as a monopolist restricts output}
\end{frame}

% --- Frame 6: Enter CBDC -- The Outside Option ---
\begin{frame}[t]{Enter CBDC -- The Outside Option}
\begin{columns}[T]
\column{0.48\textwidth}
\textbf{CBDC as Competitive Constraint}
\begin{itemize}
\item Central bank introduces CBDC paying $r_{CBDC} = 1.0\%$
\item Depositors now have an outside option: if $r_D < r_{CBDC}$, they switch
\item CBDC acts as a floor on deposit rates
\end{itemize}

\column{0.48\textwidth}
\textbf{Bank's New Problem}
\begin{itemize}
\item Bank must set $r_D \geq r_{CBDC}$ or lose all deposits
\item Monopoly optimum was $r_D^* = 0.4\%$, but $r_{CBDC} = 1.0\% > 0.4\%$, so CBDC floor \textbf{binds}
\item Bank forced to raise rate to $r_D = 1.0\%$
\end{itemize}

\vspace{0.3em}
\textbf{Worked example:}
\begin{itemize}
\item $D(0.01) = 5.4 + 200 \times 0.01 = 7.4$T EUR (up from 6.2T)
\item $\Pi_{post} = (0.035 - 0.01) \times 7{,}400\text{B} = 185.0$B EUR (down from 192.2B)
\item Higher $r_D$ $\Rightarrow$ higher $D$ $\Rightarrow$ more lending $\Rightarrow$ more inclusion
\end{itemize}
\end{columns}

\bottomnote{CBDC disciplines the bank like a new competitor entering the market---depositors benefit from the threat}
\end{frame}

% --- Frame 7: Andolfatto Equilibrium Comparison ---
\begin{frame}[t]{Andolfatto Equilibrium Comparison}
\begin{center}
\includegraphics[width=0.55\textwidth]{06_andolfatto_bank_competition/chart.pdf}
\end{center}

\begin{itemize}
\item Panel~(a): CBDC floor at 1.0\% forces the bank above its monopoly optimum of 0.4\%, shifting equilibrium from 6.2T to 7.4T deposits
\item Panel~(b): Bank profit falls from 192.2B to 185.0B~EUR as the monopoly margin is compressed, but total deposits increase by 1.2~trillion
\item Panel~(c): Consumer surplus rises significantly; total welfare increases because the deposit market moves closer to the competitive outcome
\end{itemize}

\bottomnote{Andolfatto (2021) shows CBDC may increase total deposits despite reducing bank profit}
\end{frame}

% --- Frame 8: Andolfatto Key Results ---
\begin{frame}[t]{Andolfatto Key Results}
\begin{columns}[T]
\column{0.48\textwidth}
\textbf{Surprising Results}
\begin{itemize}
\item CBDC does NOT necessarily reduce lending
\item Higher deposit rates $\Rightarrow$ more deposits $\Rightarrow$ more funds to lend
\item Bank profit falls but depositor welfare rises
\item Net welfare effect depends on parameters
\end{itemize}

\column{0.48\textwidth}
\textbf{Numerical Summary}

\vspace{0.3em}
\begin{adjustbox}{max width=\textwidth}
\begin{tabular}{lrr}
\toprule
\textbf{Variable} & \textbf{Pre-CBDC} & \textbf{Post-CBDC} \\
\midrule
Deposit rate & 0.4\% & 1.0\% \\
Deposits & 6.2T EUR & 7.4T EUR \\
Bank profit & 192.2B & 185.0B \\
Lending & 6.2T & 7.4T \\
Profit change & -- & $-7.2$B ($-3.7\%$) \\
\bottomrule
\end{tabular}
\end{adjustbox}
\end{columns}

\bottomnote{Key insight: CBDC can be pro-competitive, increasing deposits and lending while reducing monopoly rents}
\end{frame}

%% ============================================================
%% SECTION 3: Brunnermeier-Niepelt Equivalence (4 slides)
%% ============================================================
\section{Brunnermeier--Niepelt Equivalence}

% --- Frame 9: The Equivalence Question ---
\begin{frame}[t]{The Equivalence Question}
\begin{columns}[T]
\column{0.48\textwidth}
\textbf{The Core Question}
\begin{itemize}
\item If people swap bank deposits for CBDC, does it matter for the economy?
\item Modigliani--Miller (a theorem showing that, under ideal conditions, how a firm is financed does not affect its value) analogy: does the \emph{composition} of money matter?
\item Brunnermeier \& Niepelt (2019) answer: under specific conditions, NO
\end{itemize}

\column{0.48\textwidth}
\textbf{Intuition}
\begin{itemize}
\item Deposits fund bank lending. If deposits leave for CBDC, banks lose funding
\item BUT: central bank can lend those CBDC funds back to banks (pass-through funding)
\item Net effect: banks' funding source changes, but total credit unchanged
\item Like refinancing a mortgage---same house, different lender
\end{itemize}
\end{columns}

\bottomnote{Brunnermeier \& Niepelt (2019), Journal of Monetary Economics. An equivalence result for private vs public money}
\end{frame}

% --- Frame 10: The Equivalence Theorem ---
\begin{frame}[t]{The Equivalence Theorem}
\begin{columns}[T]
\column{0.48\textwidth}
\textbf{Theorem (Simplified)}
\begin{itemize}
\item If: (1)~central bank provides pass-through funding at same terms, (2)~no bank runs, (3)~no friction differences
\item Then: swapping deposits for CBDC leaves prices, output, and allocation unchanged
\item Formally: equilibrium under $(D, 0_{CBDC})$ $=$ equilibrium under $(D-x, x_{CBDC})$ for any $x$
\end{itemize}

\column{0.48\textwidth}
\textbf{Worked Example}
\begin{itemize}
\item Economy: 100B deposits, 0 CBDC
\item Swap: 30B deposits move to CBDC
\item Central bank lends 30B back to banks at same rate
\item Banks: 70B deposits $+$ 30B CB funding $=$ 100B total funding
\item Lending unchanged: still 100B
\item Outcome identical
\end{itemize}
\end{columns}

\bottomnote{The theorem shows CBDC need not cause a credit crunch---IF the central bank provides pass-through funding}
\end{frame}

% --- Frame 11: When Equivalence Breaks Down ---
\begin{frame}[t]{When Equivalence Breaks Down}
\begin{center}
\includegraphics[width=0.55\textwidth]{07_brunnermeier_niepelt_equivalence/chart.pdf}
\end{center}

\begin{itemize}
\item Panel~(a): Balance sheet mechanics showing how pass-through funding preserves total bank funding despite deposit-to-CBDC migration
\item Panel~(b): Parameter space showing where equivalence holds (green) vs breaks down (red)---frictions, bank runs, and funding term mismatches break the result
\item Real-world implication: most economies have frictions that partially break equivalence, making CBDC design choices consequential
\end{itemize}

\bottomnote{Equivalence breaks when: banks face funding cost premium, depositors have heterogeneous preferences, or bank runs are possible}
\end{frame}

% --- Frame 12: Policy Implications of (Non-)Equivalence ---
\begin{frame}[t]{Policy Implications of (Non-)Equivalence}
\begin{columns}[T]
\column{0.48\textwidth}
\textbf{If Equivalence Holds}
\begin{itemize}
\item CBDC is neutral for financial system
\item Design choices are about convenience, not stability
\item Central bank can freely issue CBDC without worrying about credit
\item Holding limits unnecessary
\end{itemize}

\column{0.48\textwidth}
\textbf{If Equivalence Breaks (Likely)}
\begin{itemize}
\item CBDC design matters for credit supply
\item Holding limits protect against disintermediation
\item Pass-through funding design is critical
\item Rate setting affects bank profitability
\end{itemize}

\vspace{0.3em}
\textbf{Worked example:} if bank funding premium $= 0.5\%$ after CBDC, lending rate rises $0.5\%$, credit contracts by $\sim$10\%
\end{columns}

\bottomnote{Most CBDC researchers believe equivalence partially breaks---hence the careful design with holding limits and no interest}
\end{frame}

%% ============================================================
%% SECTION 4: Optimal CBDC Interest Rate (4 slides)
%% ============================================================
\section{Optimal CBDC Interest Rate}

% --- Frame 13: The Optimization Problem ---
\begin{frame}[t]{The Optimization Problem}
\begin{columns}[T]
\column{0.48\textwidth}
\textbf{Central Bank's Objective}
\begin{itemize}
\item Maximize welfare:
$$W = \text{GDP gain} + \text{Inclusion gain} - \text{Disintermediation cost}$$
\item GDP gain from Barrdear \& Kumhof (2022): $\Delta Y = f(\theta)$ where $\theta$ = CBDC share of money supply
\item Calibration: $\theta = 30\%$ $\Rightarrow$ $\Delta Y = +3\%$ GDP permanently
\end{itemize}

\column{0.48\textwidth}
\textbf{The Trade-off}
\begin{itemize}
\item Higher $r_{CBDC}$ $\Rightarrow$ more CBDC adoption ($\theta$ rises)
\item More adoption $\Rightarrow$ GDP benefits from lower transaction costs
\item But also $\Rightarrow$ deposit flight, bank funding stress, potential credit contraction
\item Optimal $r_{CBDC}$ balances these forces
\end{itemize}

\vspace{0.3em}
\textbf{Worked example:} if $r_{CBDC}$ too high (3.5\%), disintermediation cost dominates ($W = -3.451$). If $r_{CBDC} = 0\%$, $W = 0$ (no benefit). Sweet spot near 0.8\%
\end{columns}

\bottomnote{Based on Barrdear \& Kumhof (2022), JEDC. DSGE model calibrated to pre-2008 US economy}
\end{frame}

% --- Frame 14: Deriving the Optimal Rate ---
\begin{frame}[t]{Deriving the Optimal Rate}
\begin{columns}[T]
\column{0.48\textwidth}
\textbf{Simplified Welfare Function}
\begin{itemize}
\item $W(r) = B(r) - C(r)$
\item $B(r) = k_1 \sqrt{r}$ (concave: diminishing returns to higher CBDC rate)
\item $C(r) = k_2 r^2$ (convex: accelerating disintermediation costs)
\item FOC: $\dfrac{dW}{dr} = \dfrac{dB}{dr} - \dfrac{dC}{dr} = 0$
\item $\dfrac{dB}{dr} = \dfrac{k_1}{2\sqrt{r}}$;\quad $\dfrac{dC}{dr} = 2k_2 r$
\end{itemize}

\column{0.48\textwidth}
\textbf{Worked Example}
\begin{itemize}
\item Parameters: $k_1 = 14.3$, $k_2 = 5{,}000$
\item Setting FOC: $\dfrac{14.3}{2\sqrt{r}} = 2 \times 5{,}000 \times r$
\item $14.3 = 20{,}000 \cdot r^{3/2}$
\item $r^{3/2} = 0.000715$
\item $r^* = 0.000715^{2/3} = 0.008 = 0.8\%$
\item Check: $W(0.8\%) = 14.3\sqrt{0.008} - 5{,}000 \times 0.008^2$
\item $= 1.279 - 0.320 = 0.959$
\item Compare: $W(0\%) = 0$, $W(3.5\%) = 2.674 - 6.125 = -3.451$
\end{itemize}
\end{columns}

\bottomnote{The optimal CBDC rate is typically low---explaining why most central banks propose 0\% initially}
\end{frame}

% --- Frame 15: Visualizing the Optimum ---
\begin{frame}[t]{Visualizing the Optimum}
\begin{center}
\includegraphics[width=0.55\textwidth]{08_optimal_cbdc_rate/chart.pdf}
\end{center}

\begin{itemize}
\item The welfare curve shows an inverted-U shape: welfare rises with low CBDC rates, peaks at $r^* = 0.8\%$, then turns sharply negative as disintermediation costs dominate
\item The optimal rate $r^*$ sits where marginal benefit equals marginal cost---the classic optimization condition
\item Sensitivity: the optimal rate shifts left (lower) when the banking sector is fragile, and right (higher) when financial inclusion is poor
\end{itemize}

\bottomnote{Key policy result: the optimal CBDC rate is positive but modest, typically 0--1\% depending on economy characteristics}
\end{frame}

% --- Frame 16: Sensitivity Analysis ---
\begin{frame}[t]{Sensitivity Analysis}
\begin{columns}[T]
\column{0.48\textwidth}
\textbf{Factors Pushing $r^*$ Higher}
\begin{itemize}
\item High unbanked rate (strong inclusion gains)
\item Inefficient banking sector (high monopoly rents)
\item Strong digital infrastructure (low CBDC deployment costs)
\item Example: Nigeria (62\% unbanked) $\Rightarrow$ $r^*$ higher than EU (2\% unbanked)
\end{itemize}

\column{0.48\textwidth}
\textbf{Factors Pushing $r^*$ Lower}
\begin{itemize}
\item Fragile banking sector (high disintermediation risk)
\item High bank concentration (severe credit contraction)
\item Low digital literacy (CBDC adoption slow anyway)
\item Example: EU (concentrated banking) $\Rightarrow$ $r^*$ lower, hence 0\% initial rate
\end{itemize}
\end{columns}

\bottomnote{The Digital Euro at 0\% is consistent with optimal policy for an economy with low unbanked rate and fragile bank margins}
\end{frame}

%% ============================================================
%% SECTION 5: Welfare Analysis (4 slides)
%% ============================================================
\section{Welfare Analysis}

% --- Frame 17: Deposit Market Welfare -- Setup ---
\begin{frame}[t]{Deposit Market Welfare -- Setup}
\begin{columns}[T]
\column{0.48\textwidth}
\textbf{Consumer Surplus (CS)}
\begin{itemize}
\item CS = depositor welfare = benefit depositors get above the rate they receive
\item Monopoly bank pays $r_D^* <$ competitive rate $\Rightarrow$ low CS
\item $r_{max} = r_L$ in the competitive limit (at $r_L$ the bank earns zero spread)
\end{itemize}

\column{0.48\textwidth}
\textbf{Producer Surplus (PS) and Deadweight Loss (DWL)}
\begin{itemize}
\item $PS = (r_L - r_D) \cdot D(r_D)$ (bank profit, with $FC = 0$)
\item $DWL = \frac{1}{2}(r_D^{comp} - r_D^{mono})(D^{comp} - D^{mono})$
\item Monopoly $\Rightarrow$ high PS, low CS, positive DWL
\end{itemize}

\vspace{0.3em}
\textbf{Worked example (pre-CBDC):}\\
$DWL = \frac{1}{2} \times (0.035 - 0.004) \times (12.4\text{T} - 6.2\text{T}) = 96.1$B

\vspace{0.2em}
\textbf{Post-CBDC:} $DWL = \frac{1}{2} \times (0.035 - 0.01) \times (12.4\text{T} - 7.4\text{T}) = 62.5$B

\vspace{0.2em}
DWL reduction: $96.1 - 62.5 = 33.6$B welfare gain
\end{columns}

\bottomnote{Welfare analysis uses standard consumer/producer surplus framework applied to the deposit market}
\end{frame}

% --- Frame 18: Welfare Decomposition Chart ---
\begin{frame}[t]{Welfare Decomposition Chart}
\begin{center}
\includegraphics[width=0.55\textwidth]{09_welfare_surplus_analysis/chart.pdf}
\end{center}

\begin{itemize}
\item Panel~(a): Pre-CBDC monopoly---large producer surplus, small consumer surplus, significant deadweight loss triangle ($DWL = 96.1$B)
\item Panel~(b): Post-CBDC---CBDC floor at 1\% compresses monopoly margin, expanding consumer surplus and reducing deadweight loss to 62.5B
\item Net welfare change: DWL falls by 33.6B---the pie gets bigger, not just redistributed
\end{itemize}

\bottomnote{Deadweight loss reduction is the key welfare argument for CBDC---it corrects monopoly distortion in the deposit market}
\end{frame}

% --- Frame 19: Who Wins, Who Loses? ---
\begin{frame}[t]{Who Wins, Who Loses?}
\begin{columns}[T]
\column{0.48\textwidth}
\textbf{Winners}

\vspace{0.3em}
\begin{adjustbox}{max width=\textwidth}
\begin{tabular}{lrl}
\toprule
\textbf{Group} & \textbf{Gain} & \textbf{Mechanism} \\
\midrule
Depositors & +33.6B & DWL recaptured, higher rates \\
Unbanked & New access & Digital payment inclusion \\
Treasury & Seigniorage & Revenue on CBDC issuance \\
\bottomrule
\end{tabular}
\end{adjustbox}

\column{0.48\textwidth}
\textbf{Losers}

\vspace{0.3em}
\begin{adjustbox}{max width=\textwidth}
\begin{tabular}{lrl}
\toprule
\textbf{Group} & \textbf{Loss} & \textbf{Mechanism} \\
\midrule
Bank shareholders & $-7.2$B & Margin compression \\
Privacy advocates & Hard to quantify & Surveillance risk \\
\bottomrule
\end{tabular}
\end{adjustbox}

\vspace{0.5em}
Key point: net welfare positive because DWL reduction (33.6B) far exceeds profit loss (7.2B)
\end{columns}

\bottomnote{Like any policy reform, CBDC creates winners and losers. The welfare case depends on total gains exceeding total losses}
\end{frame}

% --- Frame 20: Welfare Under Uncertainty ---
\begin{frame}[t]{Welfare Under Uncertainty}
\begin{columns}[T]
\column{0.48\textwidth}
\textbf{The Tail Risk}
\begin{itemize}
\item Normal times: CBDC improves welfare (deposit market competition)
\item Crisis times: CBDC may amplify instability (instant flight to safety)
\item $E[W] = P(\text{normal}) \cdot W_{\text{normal}} + P(\text{crisis}) \cdot W_{\text{crisis}}$
\end{itemize}

\column{0.48\textwidth}
\textbf{Numerical Example}
\begin{itemize}
\item $P(\text{normal}) = 0.95$, $W_{\text{normal}} = +33.6$B
\item $P(\text{crisis}) = 0.05$, $W_{\text{crisis}} = -400$B (severe bank run loss)
\item $E[W] = 0.95 \times 33.6 + 0.05 \times (-400) = 31.92 - 20.0 = +11.92$B
\end{itemize}

\vspace{0.3em}
With holding limits: $W_{\text{crisis}} = -100$B (capped flight)
\begin{itemize}
\item $E[W] = 0.95 \times 33.6 + 0.05 \times (-100) = 31.92 - 5.0 = +26.92$B
\item Holding limits more than double expected welfare!
\end{itemize}
\end{columns}

\bottomnote{Holding limits are welfare-optimal because they cap the downside risk of digital bank runs while preserving normal-time benefits}
\end{frame}

%% ============================================================
%% SECTION 6: International CBDC Game Theory (4 slides)
%% ============================================================
\section{International CBDC Game Theory}

% --- Frame 21: Closing Comic ---
\begin{frame}[t]{The Standards Problem in International Finance}
\begin{center}
\textit{[XKCD \#927: ``Standards'']}\\[0.5em]
\small Source: xkcd.com/927 by Randall Munroe, CC BY-NC 2.5\\[1em]
\normalsize ``International CBDC competition risks the same outcome---more currencies, not more interoperability. Let's model why.''
\end{center}
\bottomnote{The game theory of CBDC competition explains why coordination is hard but necessary.}
\end{frame}

% --- Frame 22: Currency Competition as a Game ---
\begin{frame}[t]{Currency Competition as a Game}
\begin{columns}[T]
\column{0.48\textwidth}
\textbf{Setup}
\begin{itemize}
\item Two countries (A and B) each decide: Launch CBDC or Wait
\item Payoffs depend on both countries' decisions
\item First-mover advantage ($\gamma$ = bonus payoff for launching while rival waits): early launcher may capture cross-border payments
\item Network effects (the phenomenon where a product becomes more valuable as more people use it): more users $\Rightarrow$ more valuable
\item Based on Benigno, Schilling \& Uhlig (2022) framework
\end{itemize}

\column{0.48\textwidth}
\textbf{Payoff Drivers}
\begin{itemize}
\item Benefits of launching: seigniorage, sanctions power, policy autonomy, payment efficiency
\item Costs of launching: development cost, bank disruption, privacy backlash
\item Strategic interaction: if rival launches first, your currency faces substitution pressure
\end{itemize}
\end{columns}

\bottomnote{International CBDC competition is a strategic game---each country's optimal action depends on what others do}
\end{frame}

% --- Frame 23: The CBDC Game -- Payoff Matrix ---
\begin{frame}[t]{The CBDC Game -- Payoff Matrix}
\begin{center}
\includegraphics[width=0.55\textwidth]{10_cbdc_game_theory/chart.pdf}
\end{center}

\begin{itemize}
\item Panel~(a): The payoff matrix reveals a Prisoner's Dilemma structure---both countries launching is NOT the best joint outcome, but each has an incentive to launch regardless
\item Panel~(b): Reaction functions show the Nash equilibrium (named after John Nash---the outcome where no country can improve its payoff by unilaterally changing strategy)
\item The Nash equilibrium (both launch) may be welfare-inferior to coordination (both wait), explaining calls for international CBDC standards
\end{itemize}

\bottomnote{Nash equilibrium predicts mutual CBDC launch; international coordination could achieve the Pareto-superior outcome}
\end{frame}

% --- Frame 24: Cross-Border Settlement ---
\begin{frame}[t]{Cross-Border Settlement}
\begin{center}
\includegraphics[width=0.55\textwidth]{11_cross_border_settlement/chart.pdf}
\end{center}

\begin{itemize}
\item Panel~(a): Multi-CBDC settlement dramatically reduces cost ($-83\%$), time ($-99.8\%$), and counterparty risk ($-75\%$) compared to correspondent banking (the traditional system where banks maintain accounts at each other to process cross-border payments)
\item Panel~(b): The cost waterfall shows where savings come from---eliminating intermediary bank fees and FX markup generates the largest gains
\item Benigno, Schilling \& Uhlig (2022): cross-border CBDC use enforces interest rate synchronization, limiting monetary policy autonomy---an extension of the impossible trinity (a country cannot simultaneously maintain a fixed exchange rate, free capital movement, and independent monetary policy)
\end{itemize}

\bottomnote{Source: BIS (2022) data. Wholesale CBDC reduces cross-border payment costs from 6\% to under 1\%}
\end{frame}

%% ============================================================
%% SECTION 7: Synthesis, Policy Implications & References (3 slides)
%% ============================================================
\section{Synthesis and Policy Implications}

% --- Frame 25: Model Synthesis ---
\begin{frame}[t]{Model Synthesis}
\begin{columns}[T]
\column{0.48\textwidth}
\textbf{Converging Insights}
\begin{itemize}
\item Andolfatto: CBDC is pro-competitive (0.4\% $\rightarrow$ 1.0\%, +1.2T deposits)
\item Brunnermeier--Niepelt: no credit crunch if pass-through
\item Barrdear--Kumhof: optimal rate is low but positive ($r^* = 0.8\%$)
\item Game theory: international coordination needed (Prisoner's Dilemma)
\end{itemize}

\column{0.48\textwidth}
\textbf{Policy Design Principles}

\vspace{0.3em}
\begin{adjustbox}{max width=\textwidth}
\begin{tabular}{ll}
\toprule
\textbf{Principle} & \textbf{Model Basis} \\
\midrule
Start with 0\% rate & Barrdear--Kumhof: uncertainty favors caution \\
Holding limits essential & Welfare: doubles expected welfare \\
Pass-through funding & Brunnermeier--Niepelt conditions \\
International coordination & Benigno et al.: Prisoner's Dilemma \\
Two-tier distribution & Andolfatto: preserves bank role \\
\bottomrule
\end{tabular}
\end{adjustbox}
\end{columns}

\bottomnote{All four models converge on a common recommendation: cautious CBDC introduction with strong safeguards}
\end{frame}

% --- Frame 26: Open Questions ---
\begin{frame}[t]{Open Questions}
\begin{columns}[T]
\column{0.48\textwidth}
\textbf{Unresolved Issues}
\begin{itemize}
\item Optimal holding limit level (3{,}000 EUR? 5{,}000? Dynamic?)
\item Long-run bank adaptation (do banks find new business models?)
\item Privacy-efficiency trade-off (can zero-knowledge proofs resolve it?)
\item Digital divide implications (does CBDC worsen inequality?)
\end{itemize}

\column{0.48\textwidth}
\textbf{Your Assignment Connection}
\begin{itemize}
\item These models are the foundation for Exercise~6 (Disintermediation Calculator)
\item Exercise~7 (Tiered Remuneration) applies the optimal rate framework
\item Quiz questions 9--14 test understanding of these models
\item Research paper topic: any of the open questions
\end{itemize}
\end{columns}

\bottomnote{CBDCs remain an active research frontier. The models here provide the analytical toolkit for evaluating new proposals}
\end{frame}

% --- Frame 27: References ---
\begin{frame}[t]{References}
\small
\begin{itemize}
\setlength{\itemsep}{2pt}
\item Andolfatto, D.\ (2021). Assessing the Impact of Central Bank Digital Currency on Private Banks. \textit{The Economic Journal}, 131(634), 525--540.
\item Barrdear, J.\ \& Kumhof, M.\ (2022). The macroeconomics of central bank digital currencies. \textit{Journal of Economic Dynamics and Control}, 142, 104148.
\item Benigno, P., Schilling, L.M.\ \& Uhlig, H.\ (2022). Cryptocurrencies, currency competition, and the impossible trinity. \textit{Journal of International Economics}, 136, 103601.
\item Brunnermeier, M.K.\ \& Niepelt, D.\ (2019). On the equivalence of private and public money. \textit{Journal of Monetary Economics}, 106, 27--41.
\item BIS (2022). Project mBridge: Connecting economies through CBDC. \textit{BIS Innovation Hub}.
\item Bindseil, U.\ (2020). Tiered CBDC and the financial system. \textit{ECB Working Paper No.\ 2351}.
\end{itemize}

\bottomnote{Full references for all models and data sources cited in this lecture}
\end{frame}

%% ============================================================
%% APPENDIX (3 slides)
%% ============================================================
\appendix
\section{Appendix}

% --- Frame 28 (A1): Complete Notation Table ---
\begin{frame}[t]{Appendix A1: Complete Notation Table}
\begin{center}
\begin{adjustbox}{max width=0.95\textwidth, max totalheight=0.78\textheight}
\begin{tabular}{lll}
\toprule
\textbf{Symbol} & \textbf{Meaning} & \textbf{Section} \\
\midrule
$D$ & Bank deposits (EUR trillions) & Andolfatto \\
$r_D$ & Bank deposit interest rate & Andolfatto \\
$r_{CBDC}$ & CBDC interest rate & Andolfatto / Optimal Rate \\
$r_L$ & Bank lending rate & Andolfatto \\
$r_{max}$ & Maximum deposit rate (competitive limit) & Welfare \\
$a$ & Inelastic deposit base (EUR trillions) & Andolfatto \\
$b$ & Deposit rate sensitivity (EUR trillions per unit rate) & Andolfatto \\
$\Pi_B$ & Bank profit (EUR billions) & Andolfatto \\
$FC$ & Fixed costs & Andolfatto \\
$U(c)$ & Utility of consumption & Welfare \\
$\alpha$ & Risk aversion parameter & Welfare \\
$W$ & Aggregate welfare & Welfare / Optimal Rate \\
$B(r)$ & Benefit function of CBDC rate & Optimal Rate \\
$C(r)$ & Cost function of CBDC rate & Optimal Rate \\
$k_1$ & Benefit scaling parameter ($= 14.3$) & Optimal Rate \\
$k_2$ & Cost scaling parameter ($= 5{,}000$) & Optimal Rate \\
$CS$ & Consumer surplus (depositor welfare) & Welfare \\
$PS$ & Producer surplus (bank profit) & Welfare \\
$DWL$ & Deadweight loss & Welfare \\
$L(D)$ & Lending as function of deposits & Equivalence \\
$\sigma$ & Volatility / uncertainty parameter & Welfare / Game Theory \\
$Y$ & GDP / output & Optimal Rate \\
$\theta$ & CBDC share of money supply & Optimal Rate \\
$\gamma$ & First-mover advantage payoff bonus & Game Theory \\
$V_i$ & Country $i$ payoff in CBDC game & Game Theory \\
$\delta$ & Discount factor (present value weight) & Game Theory \\
\bottomrule
\end{tabular}
\end{adjustbox}
\end{center}

\bottomnote{Reference page for all mathematical notation used in this lecture}
\end{frame}

% --- Frame 29 (A2): Andolfatto Full Derivation ---
\begin{frame}[t]{Appendix A2: Andolfatto Full Derivation}
\begin{columns}[T]
\column{0.48\textwidth}
\textbf{Pre-CBDC FOC}
\begin{align*}
\Pi_B &= (r_L - r_D)(a + b \cdot r_D) \\
&= r_L \cdot a + r_L \cdot b \cdot r_D - a \cdot r_D - b \cdot r_D^2
\end{align*}
\begin{align*}
\frac{d\Pi_B}{dr_D} &= r_L \cdot b - a - 2b \cdot r_D = 0
\end{align*}
$$r_D^* = \frac{r_L \cdot b - a}{2b} = \frac{r_L - a/b}{2}$$

\vspace{0.3em}
With $a = 5.4$, $b = 200$, $r_L = 0.035$:
$$r_D^* = \frac{0.035 - 0.027}{2} = 0.004$$

\column{0.48\textwidth}
\textbf{Post-CBDC (Kuhn--Tucker)}
\begin{itemize}
\item $\max \Pi_B$ \quad s.t.\ $r_D \geq r_{CBDC}$
\item If $r_D^* < r_{CBDC}$: constraint binds, $r_D = r_{CBDC}$
\item If $r_D^* \geq r_{CBDC}$: constraint slack, monopoly solution unchanged
\item With $r_{CBDC} = 0.01 > r_D^* = 0.004$: binds.
$$r_D = 0.01$$
\end{itemize}

\vspace{0.3em}
Post-CBDC equilibrium:
\begin{align*}
D &= 5.4 + 200 \times 0.01 = 7.4\text{T} \\
\Pi_B &= (0.035 - 0.01) \times 7{,}400\text{B} = 185.0\text{B}
\end{align*}
\end{columns}

\bottomnote{Complete derivation for students who want the mathematical details}
\end{frame}

% --- Frame 30 (A3): Brunnermeier-Niepelt Sufficient Conditions ---
\begin{frame}[t]{Appendix A3: Brunnermeier--Niepelt Sufficient Conditions}
\begin{columns}[T]
\column{0.48\textwidth}
\textbf{Three Sufficient Conditions}
\begin{enumerate}
\item \textbf{Pass-through funding:} CB lends to banks at rate $r_{CB} = r_D$ (no funding cost premium)
\item \textbf{No bank runs:} depositors do not panic-withdraw (Diamond--Dybvig stability)
\item \textbf{No friction asymmetry:} CBDC and deposits face identical regulatory, tax, and convenience costs
\end{enumerate}

\column{0.48\textwidth}
\textbf{Proof Sketch}
\begin{itemize}
\item Under conditions 1--3, bank balance sheets adjust: deposits $\downarrow$, CB funding $\uparrow$, total funding unchanged
\item Loan supply unchanged $\Rightarrow$ real allocation unchanged
\item Modigliani--Miller logic: liability composition is irrelevant when frictions are absent
\item Policy implication: making conditions hold (e.g., credible pass-through) is the design goal
\end{itemize}
\end{columns}

\bottomnote{Formal conditions from Brunnermeier \& Niepelt (2019), Journal of Monetary Economics, 106, 27--41}
\end{frame}

\end{document}
