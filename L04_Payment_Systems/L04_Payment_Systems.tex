\documentclass[8pt,aspectratio=169]{beamer}
\usetheme{Madrid}
\usepackage{graphicx}
\usepackage{booktabs}
\usepackage{adjustbox}
\usepackage{multicol}
\usepackage{amsmath}

% Color definitions
\definecolor{mlblue}{RGB}{0,102,204}
\definecolor{mlpurple}{RGB}{51,51,178}
\definecolor{mllavender}{RGB}{173,173,224}
\definecolor{mllavender2}{RGB}{193,193,232}
\definecolor{mllavender3}{RGB}{204,204,235}
\definecolor{mllavender4}{RGB}{214,214,239}
\definecolor{mlorange}{RGB}{255, 127, 14}
\definecolor{mlgreen}{RGB}{44, 160, 44}
\definecolor{mlred}{RGB}{214, 39, 40}
\definecolor{mlgray}{RGB}{127, 127, 127}

\definecolor{lightgray}{RGB}{240, 240, 240}
\definecolor{midgray}{RGB}{180, 180, 180}

\setbeamercolor{palette primary}{bg=mllavender3,fg=mlpurple}
\setbeamercolor{palette secondary}{bg=mllavender2,fg=mlpurple}
\setbeamercolor{palette tertiary}{bg=mllavender,fg=white}
\setbeamercolor{palette quaternary}{bg=mlpurple,fg=white}

\setbeamercolor{structure}{fg=mlpurple}
\setbeamercolor{section in toc}{fg=mlpurple}
\setbeamercolor{subsection in toc}{fg=mlblue}
\setbeamercolor{title}{fg=mlpurple}
\setbeamercolor{frametitle}{fg=mlpurple,bg=mllavender3}
\setbeamercolor{block title}{bg=mllavender2,fg=mlpurple}
\setbeamercolor{block body}{bg=mllavender4,fg=black}

\setbeamertemplate{navigation symbols}{}
\setbeamertemplate{itemize items}[circle]
\setbeamertemplate{enumerate items}[default]
\setbeamersize{text margin left=5mm,text margin right=5mm}

\newcommand{\bottomnote}[1]{%
\vfill
\vspace{-2mm}
\textcolor{mllavender2}{\rule{\textwidth}{0.4pt}}
\vspace{1mm}
\footnotesize
\textbf{#1}
}

\title{Payment Systems Economics}
\subtitle{L04: Economics of Value Transfer\\[0.3em]\small Why sending money abroad costs 6\% and takes 3 days---and how technology is changing that}
\author{Economics of Digital Finance}
\institute{BSc Course}
\date{}

\begin{document}

% Title slide
\begin{frame}[plain]
\titlepage
\end{frame}

% Outline
\begin{frame}[t]{Lesson Overview}
\begin{columns}[T]
\column{0.48\textwidth}
\textbf{Today's Topics}
\begin{enumerate}
\item Economics of payment systems
\item Network effects and adoption
\item Two-sided market theory (how platforms like Visa balance merchants and consumers)
\item Cross-border payment challenges
\item Financial inclusion
\end{enumerate}

\column{0.48\textwidth}
\textbf{Learning Objectives}
\begin{itemize}
\item Apply network economics to payments
\item Analyze interchange fee economics (the hidden fees merchants pay every time you use a credit card)
\item Understand correspondent banking costs
\item Evaluate digital solutions
\end{itemize}
\end{columns}

\bottomnote{Payment systems are infrastructure for economic activity; their economics matter. No prior economics required---all terms explained.}
\end{frame}

% Payment System Basics
\begin{frame}[t]{Economics of Payment Systems}
\begin{columns}[T]
\column{0.48\textwidth}
\textbf{Economic Functions}

Payment systems enable:
\begin{itemize}
\item Value transfer between parties
\item Settlement of obligations \textit{(the final, irreversible transfer of money between parties---once settled, it cannot be undone)}
\item Support for economic transactions
\end{itemize}

\vspace{0.3em}
\textbf{Key Economic Properties}
\begin{itemize}
\item Network goods (products that become more valuable as more people use them)
\item Infrastructure characteristics (shared systems everyone uses, like roads or power grids)
\item Significant fixed costs (costs that don\'t change with volume---building the system costs the same for 1 or 1 million payments)
\end{itemize}

\column{0.48\textwidth}
\textbf{System Types}

\vspace{0.3em}
\textbf{Large Value (Wholesale)}
\begin{itemize}
\item Fedwire (US system handling \$5 trillion/day---mostly between banks), TARGET2 (EU equivalent)
\item RTGS (Real-Time Gross Settlement): each payment settles individually and immediately
\item Low volume, high value
\end{itemize}

\vspace{0.3em}
\textbf{Retail}
\begin{itemize}
\item Cards, ACH (Automated Clearing House---processes payroll deposits, bill payments, and bank transfers in batches, usually overnight), instant payments
\item High volume, lower value
\item Consumer-facing
\end{itemize}
\end{columns}

\bottomnote{Payment systems exhibit natural monopoly characteristics (one provider can serve everyone cheaper) due to network effects}
\end{frame}

% Network Effects
\begin{frame}[t]{Network Effects in Payments}
\begin{columns}[T]
\column{0.48\textwidth}
\textbf{Direct Network Effects}

Value increases with users:
$$V(n) = n \cdot v(n)$$
\textit{(Total value = number of users times value per user.)}

where $v(n)$ is per-user value. \textbf{The key insight:} $v(n)$ \textit{increases} with $n$. For a regular product (e.g., a toaster), value per unit is constant. For a network (e.g., Venmo), each user becomes more valuable as more people join---a virtuous cycle.

\vspace{0.3em}
\textbf{Metcalfe's Law (simplified)}
$$V \propto n^2$$
\textit{($\propto$ means ``is proportional to'' or ``grows with'')}
\textit{(Value grows with the square of users: 10 users = 100 units of value, 100 users = 10,000 units. This explains why dominant networks stay dominant.)}
\textit{(Real example: Venmo 2012: 10,000 users. Venmo 2023: 90 million users. Same app, 9,000x more useful.)}
\begin{itemize}
\item Each user can transact with $n-1$ others
\item Creates positive feedback loops (success breeds success---more users attract more users)
\end{itemize}

\column{0.48\textwidth}
\textbf{Implications for Payments}

\vspace{0.3em}
\textbf{Adoption Dynamics}
\begin{itemize}
\item Critical mass threshold (minimum users needed before network becomes self-sustaining)
\item Tipping points (moments when adoption suddenly accelerates)
\textit{(Rogers (1962) showed: below $\sim$16\% (``early adopters''), must convince each user individually. Above 16\%: social proof kicks in and adoption accelerates)}
\item Winner-take-most markets (where one or two players capture most market share)
\end{itemize}

\vspace{0.3em}
\textbf{Entry Barriers}
\begin{itemize}
\item Incumbents (existing dominant players) have user base advantage
\item New entrants need to ``buy'' network
\item Interoperability \textit{(ability for different payment systems to work together)} can reduce barriers
\end{itemize}
\end{columns}

\bottomnote{Network effects explain why few payment networks dominate---and why you use Venmo/PayPal not because they're best, but because your friends use them}
\end{frame}

% Network Effects Chart (Metcalfe's Law)
\begin{frame}[t]{Network Value Growth: Metcalfe's Law vs.\ Alternatives}
\begin{center}
\includegraphics[width=0.62\textwidth]{04_network_effects_metcalfe/chart.pdf}
\end{center}

\begin{itemize}
\item \textbf{Metcalfe's Law} ($V \propto n^2$) predicts explosive growth---network with strong effects reaches critical mass (the switching-cost threshold) far sooner than linear growth
\item The \textbf{Odlyzko-Tilly} model ($V \propto n \log n$) is more conservative: not every new connection is equally valuable, so growth is fast but not quadratic
\item \textbf{Practical implication:} Whichever model holds, payment networks with genuine network effects dominate within hundreds of users, not millions---explaining early-mover advantage
\end{itemize}

\bottomnote{Chart compares three growth models; vertical lines mark where each model first exceeds the switching-cost threshold}
\end{frame}

% Payment Adoption
\begin{frame}[t]{Payment Technology Adoption Patterns}
\begin{center}
\includegraphics[width=0.65\textwidth]{02_payment_adoption/chart.pdf}
\end{center}

\bottomnote{S-curve adoption: slow start, rapid growth after critical mass (\textasciitilde16\%---once 1 in 6 people use something, social proof accelerates adoption), then saturation}
\end{frame}

% Two-Sided Markets: The Concept
\begin{frame}[t]{Two-Sided Market Theory: The Concept}
\begin{columns}[T]
\column{0.48\textwidth}
\textbf{The Platform Model}

Card networks connect:
\begin{itemize}
\item Side 1: Cardholders (consumers)
\item Side 2: Merchants (acceptors)
\item Platform: Visa, Mastercard
\end{itemize}

\vspace{0.3em}
\textbf{Cross-Side Network Effects}

\textit{(Cross-side: each side benefits when the OTHER side grows---different from direct network effects where users benefit from more of the SAME type of user)}
\begin{itemize}
\item More merchants $\rightarrow$ more cardholders
\item More cardholders $\rightarrow$ more merchants
\item Chicken-and-egg problem
\end{itemize}

\column{0.48\textwidth}
\textbf{Key Insight}

Price structure matters, not just level:
\begin{itemize}
\item Subsidize price-sensitive side
\item Charge price-insensitive side
\item ``Get both sides on board''
\end{itemize}

\textit{(Fee breakdown: Of \$2 merchant fee on \$100 purchase, \$1.80 goes to your bank as interchange, \$0.20 to Visa)}

\vspace{0.3em}
\textbf{Beyond Payments}

\textit{(This framework explains Uber, Airbnb, Amazon, dating apps---any platform connecting two groups)}
\end{columns}

\bottomnote{Rochet \& Tirole (2003): Two-sided markets require analysis beyond standard economics}
\end{frame}

% Two-Sided Markets: Rochet-Tirole Pricing Model
\begin{frame}[t]{Two-Sided Market Theory: Rochet-Tirole Pricing Model}
\begin{columns}[T]
\column{0.48\textwidth}
\textbf{Rochet-Tirole Model} \textit{(Tirole received the 2014 Nobel Prize in Economics)}

Total price constraint:
$$p_B + p_S = c + m$$
\textit{where $c$ is total cost per transaction ($c = c_B + c_S$, split between serving buyers and sellers) and $m$ is the platform's profit margin.}

\vspace{0.3em}
Optimal price ratio:
$$\frac{p_B - c_B}{p_S - c_S} = \frac{\eta_S}{\eta_B}$$
\textit{(Charge more to the side that's less price-sensitive.)}

where $\eta$ = price elasticity (how much demand changes when price changes---elasticity of 2 means a 1\% price increase causes a 2\% demand drop)

\column{0.48\textwidth}
\textbf{Worked Example}

\smallskip\textbf{Example:} Suppose merchant elasticity $\eta_S = 0.5$ (merchants are trapped---they must accept cards) and consumer elasticity $\eta_B = 2.0$ (consumers can switch to cash).

$$\frac{p_B - c_B}{p_S - c_S} = \frac{0.5}{2.0} = 0.25$$

This means the markup on consumers should be \textbf{one-quarter} of the markup on merchants. Result: cardholders pay zero or negative fees (rewards!) while merchants pay 2--3\%.

\vspace{0.3em}
\textit{Real-world check: Visa charges cardholders \$0 annual fee (often with cashback), but merchants pay 1.5--3\% per transaction. The math matches reality.}
\end{columns}

\bottomnote{The elasticity ratio explains why YOUR card is free but the coffee shop pays Visa 2\% on every sale}
\end{frame}

% Two-Sided Market Pricing Chart
\begin{frame}[t]{Interchange Fee Optimization: Balancing Both Sides}
\begin{center}
\includegraphics[width=0.62\textwidth]{05_two_sided_market_pricing/chart.pdf}
\end{center}

\begin{itemize}
\item As the interchange fee rises, \textbf{merchant acceptance falls} (green) but \textbf{consumer adoption rises} (blue, because higher fees fund better rewards)
\item \textbf{Platform profit} (purple) peaks at the optimal fee where the product of both sides is maximized---set the fee too low and consumers leave; too high and merchants leave
\item The red line marks the profit-maximizing interchange fee---in practice, Visa/Mastercard set fees near this optimum
\end{itemize}

\bottomnote{This chart visualizes the Rochet-Tirole model: the platform's problem is finding the fee that keeps BOTH sides on board}
\end{frame}

% Interchange Fees
\begin{frame}[t]{Interchange Fee Economics}
\begin{columns}[T]
\column{0.48\textwidth}
\textbf{The Fee Flow}

\begin{enumerate}
\item Consumer pays \$100
\item Merchant receives \$97-98
\item Interchange: 1.5-2\% to issuer (bank that gave you the card)
\item Network fee: 0.1-0.2\%
\item Acquirer (merchant's bank) margin: 0.2-0.5\%
\end{enumerate}

\vspace{0.3em}
\textbf{Economic Rationale}
\begin{itemize}
\item Issuer bears fraud risk (refunds you when your card is stolen---needs compensation for that risk)
\item Subsidizes cardholder rewards
\item Balances two-sided market
\end{itemize}

\column{0.48\textwidth}
\textbf{Regulatory Debate}

\vspace{0.3em}
\textbf{Against High Interchange}
\begin{itemize}
\item Merchants pass costs to prices
\item Regressive (poorer cash users subsidize wealthier card users through higher prices)
\textit{(Mechanism: Merchant raises all prices 2\%; cash customer pays higher price but gets no rewards)}
\item Centralized fee-setting (interchange fees are set by Visa/Mastercard, not negotiated between individual banks---critics argue this resembles price-fixing)
\end{itemize}

\vspace{0.3em}
\textbf{For Market Rates}
\begin{itemize}
\item Funds card benefits
\item Network competition exists
\item Caps reduce innovation
\end{itemize}
\end{columns}

\bottomnote{EU capped interchange at 0.2-0.3\%; US Durbin Amendment \textit{(2010 law)} capped \textbf{debit} at 0.05\%{}+21c but left \textbf{credit uncapped}---debit was capped because issuers bear less risk (money is already in your account), while credit remains uncapped because banks bear default risk.}
\end{frame}

% Cross-Border Payments
\begin{frame}[t]{Cross-Border Payment Challenges}
\begin{columns}[T]
\column{0.48\textwidth}
\textbf{The Problem}

As we discussed in L03 (CBDCs), cross-border payments are:
\begin{itemize}
\item Expensive: 6\%{}+ average cost (send \$100, only \$94 arrives)---now we examine \textit{why}
\item Slow: 2-5 days settlement
\item Opaque: uncertain fees
\item Fragmented: many intermediaries
\end{itemize}

\vspace{0.3em}
\textbf{Root Causes}
\begin{itemize}
\item Lack of common infrastructure
\item Regulatory fragmentation
\item Legacy technology
\item Correspondent banking model (payments routed through intermediary banks)
\textit{(Each bank: maintains relationship, holds capital, performs compliance, takes profit = costs multiply)}
\end{itemize}

\column{0.48\textwidth}
\textbf{Economic Inefficiencies}

\vspace{0.3em}
\textbf{FX (Foreign Exchange) Costs}
\begin{itemize}
\item Wide bid-ask spreads (gap between buy and sell prices)
\item Hidden markups in rates
\item Multiple conversions
\end{itemize}

\vspace{0.3em}
\textbf{Compliance Costs}
\begin{itemize}
\item AML (Anti-Money Laundering) / KYC (Know Your Customer) checks at each hop
\item Sanctions screening (checking if recipients are on government blacklists of terrorists or banned countries)
\item Data format inconsistencies
\end{itemize}
\end{columns}

\bottomnote{G20 target: reduce average cost to 3\% by 2027; currently at 6\%{}+}
\end{frame}

% Correspondent Banking
\begin{frame}[t]{Correspondent Banking Network}
\begin{center}
\includegraphics[width=0.58\textwidth]{03_correspondent_banking/chart.pdf}
\end{center}

\bottomnote{Hub-and-spoke model adds costs and delays; each intermediary takes fees. Example: \$1,000 US to Kenya = \$80+ lost across 4 intermediaries}
\end{frame}

% Remittance Costs
\begin{frame}[t]{Remittance Costs: The Scale of Inefficiency}
\begin{center}
\includegraphics[width=0.62\textwidth]{01_remittance_costs/chart.pdf}
\end{center}

\bottomnote{High-cost corridors (Africa) hurt poorest populations most; digital can reduce costs}
\end{frame}

% Settlement Systems
\begin{frame}[t]{Settlement System Design}
\begin{columns}[T]
\column{0.48\textwidth}
\textbf{RTGS (Real-Time Gross Settlement)}

Characteristics:
\begin{itemize}
\item Immediate, final settlement
\item Each transaction settled individually
\item High liquidity requirement (banks must hold more cash)
\end{itemize}

\vspace{0.3em}
Examples: Fedwire, TARGET2

\vspace{0.3em}
Trade-offs:
\begin{itemize}
\item[\textcolor{mlgreen}{+}] Eliminates settlement risk (risk one party pays but other defaults)
\textit{(Your rent payment: RTGS = clears instantly; DNS = might take days, risk if bank fails mid-batch)}
\item[\textcolor{mlred}{-}] High liquidity cost
\end{itemize}

\column{0.48\textwidth}
\textbf{DNS (Deferred Net Settlement)}

Characteristics:
\begin{itemize}
\item Batch settlement at intervals
\item Payments netted (offsetting opposite payments so only difference settles)
\item Lower liquidity needs
\textit{(Example: A owes B \$5M, B owes A \$4M. DNS: settle net \$1M in one transfer. RTGS: two separate transfers (\$5M + \$4M = \$9M total flowing through the system))}
\end{itemize}

\vspace{0.3em}
Examples: ACH, many retail systems

\vspace{0.3em}
Trade-offs:
\begin{itemize}
\item[\textcolor{mlgreen}{+}] Liquidity efficient
\item[\textcolor{mlred}{-}] Settlement risk until batch
\end{itemize}
\end{columns}

\bottomnote{Hybrid systems combine benefits: queue management \textit{(holding payments in line until funds arrive)}, partial netting \textit{(offsetting some payments while settling others individually)} with RTGS speed}
\end{frame}

% Instant Payments
\begin{frame}[t]{Instant Payments: Economic Impact}
\begin{columns}[T]
\column{0.48\textwidth}
\textbf{Global Rollout}

Major systems:
\begin{itemize}
\item UK: Faster Payments (2008)
\item India: UPI (Unified Payments Interface, 2016)
\item EU: SEPA (Single Euro Payments Area) Instant (2017)
\item US: FedNow (2023)---a \textbf{public} system competing with the private RTP network
\textit{(Same public-vs-private question as CBDCs in L03. US was late: thousands of fragmented banks made coordination hard; existing systems were ``good enough'' for incumbents)}
\end{itemize}

\vspace{0.3em}
\textbf{Design Features}
\begin{itemize}
\item 24/7/365 availability
\item Settlement in seconds
\item Irrevocable payments \textit{(cannot be reversed or cancelled once sent)}
\end{itemize}

\column{0.48\textwidth}
\textbf{Economic Benefits}

For consumers:
\begin{itemize}
\item Improved cash flow management
\item Emergency transfers
\item P2P (person-to-person) payments
\end{itemize}

For businesses:
\begin{itemize}
\item Working capital (cash needed for daily operations) optimization
\item Reduced float costs (money sitting in transit earns interest for intermediaries---instant payments eliminate this)
\item Real-time reconciliation \textit{(matching payments to invoices instantly)}
\end{itemize}

For economy:
\begin{itemize}
\item Velocity of money (how fast money circulates---faster = more economic activity) increase
\item Reduced payment friction
\end{itemize}
\end{columns}

\bottomnote{India's UPI: 10+ billion transactions/month; transformed payment landscape}
\end{frame}

% Financial Inclusion
\begin{frame}[t]{Financial Inclusion Economics}
\textit{Building on the L03 inclusion discussion, here we examine M-Pesa---financial inclusion WITHOUT a CBDC.}
\begin{columns}[T]
\column{0.48\textwidth}
\textbf{The Unbanked}

1.4 billion adults lack accounts:
\begin{itemize}
\item Documentation barriers
\item Physical access (branches)
\item Minimum balance requirements
\item Trust and literacy issues
\end{itemize}

\vspace{0.3em}
\textbf{Economic Costs}
\begin{itemize}
\item Check cashing fees (2-5\%)
\item No savings accumulation
\item Excluded from credit (cannot get loans, credit cards, or mortgages without a bank account)
\textit{(Even in rich countries: 5.4\% of US households are unbanked and pay \$40+/month in check-cashing fees)}
\end{itemize}

\column{0.48\textwidth}
\textbf{Mobile Money Success}

M-Pesa (Kenya) model:
\begin{itemize}
\item Agent network (not branches)
\item Phone-based (no smartphone needed)
\item Low-value, low-cost transactions
\textit{(M-Pesa fees: Sending \$8 costs \$0.09 (1.1\%) vs Western Union \$5 minimum = 62\% on small amounts)}
\end{itemize}

\vspace{0.3em}
\textbf{Impact Evidence}
\begin{itemize}
\item Suri \& Jack (2016): 2\% poverty reduction (approximately 400,000 Kenyan households lifted out of poverty)
\item Women especially benefited
\item Improved risk sharing \textit{(spreading financial risks across family and community members)}
\end{itemize}

\textit{Why it works: When someone's crops fail, relatives can send money instantly via phone. Before M-Pesa, they'd have to physically travel with cash or use expensive services.}
\end{columns}

\bottomnote{Mobile money shows technology can reduce barriers; requires complementary ecosystem}
\end{frame}

% Digital Solutions
\begin{frame}[t]{Digital Solutions to Payment Inefficiencies}
\begin{columns}[T]
\column{0.48\textwidth}
\textbf{Blockchain-Based}
\begin{itemize}
\item Ripple/XRP \textit{(cross-border payment company using blockchain technology)} for cross-border
\item Stablecoins for remittances
\textit{(How it works: Buy stablecoins, send instantly, recipient converts locally---\$1-2 total vs \$6 traditional)}
\item DeFi (Decentralized Finance---financial services on blockchain without traditional banks) payment rails (using blockchain networks as infrastructure for moving money)
\end{itemize}

Advantages:
\begin{itemize}
\item Bypass correspondent banking \textit{(sending value directly via blockchain instead of routing through multiple intermediary banks)}
\item 24/7 operation
\item Lower intermediary costs
\end{itemize}

\column{0.48\textwidth}
\textbf{Traditional Innovation}
\begin{itemize}
\item SWIFT gpi (Global Payments Innovation) improvements
\item ISO 20022 (new global messaging standard for financial data)
\textit{(Structured data enables automation---like the difference between email and handwritten letters)}
\item Linked instant payment systems
\end{itemize}

\vspace{0.3em}
\textbf{Comparison}
\begin{itemize}
\item Blockchain: disruptive but immature
\item Traditional: incremental but reliable
\item Likely hybrid outcome
\end{itemize}
\end{columns}

\bottomnote{Competition between approaches benefits users; both have role to play}
\end{frame}

% Competition and Regulation
\begin{frame}[t]{Competition and Regulation in Payments}
\begin{columns}[T]
\column{0.48\textwidth}
\textbf{Market Structure Issues}
\begin{itemize}
\item Natural monopoly tendencies \textit{(markets where one provider can serve everyone at lower cost than multiple competitors---common in payment infrastructure)}
\item High barriers to entry
\item Winner-take-most dynamics
\end{itemize}

\vspace{0.3em}
\textbf{Regulatory Responses}
\begin{itemize}
\item Interchange caps (EU, US)
\item Open banking mandates (regulations requiring banks to share customer data with authorized apps---e.g., EU's PSD2 (2018) forced banks to let apps like Revolut access your account data)
\item Access to payment systems
\end{itemize}

\column{0.48\textwidth}
\textbf{Big Tech Entry}
\begin{itemize}
\item Apple Pay, Google Pay
\item Alipay, WeChat Pay
\item Meta (stablecoin attempt---the Diem/Libra project, abandoned due to regulatory opposition in 2022)
\textit{(Why Big Tech is different: They already have your phone, your data, your habits---banks never had that relationship)}
\end{itemize}

\vspace{0.3em}
\textbf{Policy Concerns}
\begin{itemize}
\item Data concentration
\item Systemic risk (risk that one failure causes chain reaction---if a major payment system fails, the entire economy could freeze)
\item Competitive fairness
\end{itemize}
\end{columns}

\bottomnote{Regulators balance innovation promotion against financial stability and competition}
\end{frame}

% Key Takeaways
\begin{frame}[t]{Key Takeaways}
\begin{columns}[T]
\column{0.48\textwidth}
\textbf{Main Conclusions}
\begin{enumerate}
\item Payment systems are network goods
\item Two-sided markets require special analysis
\item Cross-border payments remain inefficient
\item Digital innovation offers solutions
\end{enumerate}

\column{0.48\textwidth}
\textbf{Economic Framework}
\begin{itemize}
\item Network effects and critical mass
\item Two-sided market pricing
\item Correspondent banking costs
\item Financial inclusion economics
\end{itemize}
\end{columns}

\vspace{0.5em}
\textbf{Core Insight}

Payment system economics explains both why incumbents dominate and why digital disruption is difficult but potentially transformative.

\textit{Memorable anchor: India's UPI processes 10+ billion transactions/month with near-zero fees---proof that digital can scale.}

\bottomnote{Next lesson: Platform Economics and Token Economics}
\end{frame}

% Key Terms Part 1
\begin{frame}[t]{Key Terms (1/2)}
\begin{columns}[T]
\column{0.48\textwidth}
\textbf{RTGS (Real-Time Gross Settlement)}
Payment system settling transactions individually and immediately, eliminating settlement risk.

\vspace{0.3em}
\textbf{DNS (Deferred Net Settlement)}
System accumulating transactions and settling net positions at end of day, requiring less liquidity but more risk.

\vspace{0.3em}
\textbf{Interchange Fee}
Fee paid by merchant's bank to cardholder's bank for each card transaction---funds rewards programs.

\vspace{0.3em}
\textbf{Correspondent Banking}
Arrangement where banks hold accounts with each other to facilitate cross-border payments through intermediaries.

\vspace{0.3em}
\textbf{Two-Sided Market}
Platform connecting two user groups (e.g., cardholders and merchants) who provide value to each other.

\column{0.48\textwidth}
\textbf{Network Effects}
Value of payment network increases as more users and merchants participate---why dominant networks stay dominant.

\vspace{0.3em}
\textbf{Natural Monopoly}
Market where one provider can serve everyone at lower cost than multiple competitors---common in payment infrastructure.

\vspace{0.3em}
\textbf{Critical Mass}
Minimum number of users needed before a network becomes self-sustaining---typically around 16\% adoption.

\vspace{0.3em}
\textbf{Issuer}
Bank that provides cards to consumers and extends credit (e.g., the bank that gave you your Visa card).

\vspace{0.3em}
\textbf{Acquirer}
Bank that processes card payments for merchants and deposits funds into their accounts.
\end{columns}

\bottomnote{Terms continued on next slide}
\end{frame}

% Key Terms Part 2
\begin{frame}[t]{Key Terms (2/2)}
\begin{columns}[T]
\column{0.48\textwidth}
\textbf{Settlement Risk}
Risk that one party delivers but the other defaults---eliminated by RTGS, present in deferred systems.

\vspace{0.3em}
\textbf{Liquidity}
Having cash available when needed---RTGS requires banks to hold more liquidity than netting systems.

\vspace{0.3em}
\textbf{Float}
Money in transit between accounts---delay benefits whoever holds it (earns interest) at the other's expense.

\vspace{0.3em}
\textbf{Remittances}
Money sent home by workers abroad---often to developing countries, with high fees (average 6\%).

\vspace{0.3em}
\textbf{Netting}
Offsetting payments in opposite directions so only the net difference settles---reduces liquidity needs.

\column{0.48\textwidth}
\textbf{Price Elasticity}
How much demand changes when price changes. An elasticity of 2 means a 1\% price increase causes a 2\% drop in demand. Card users are less price-sensitive (low elasticity) than merchants (higher elasticity).

\vspace{0.3em}
\textbf{Bid-Ask Spread}
Difference between buy and sell prices for currency---wider spreads mean higher hidden costs.

\vspace{0.3em}
\textbf{Metcalfe's Law}
A network's value grows with the square of its users---10x users means 100x value, explaining network dominance.

\vspace{0.3em}
\textbf{Financial Inclusion}
Ensuring all people have access to useful and affordable financial services---1.4 billion adults lack bank accounts.

\vspace{0.3em}
\textbf{Mobile Money}
Financial services via mobile phone without a bank account---M-Pesa in Kenya pioneered this model.
\end{columns}

\bottomnote{Payment system economics involves trade-offs between efficiency, risk, and access}
\end{frame}

% References
\begin{frame}[t]{Further Reading}
\begin{columns}[T]
\column{0.48\textwidth}
\textbf{Academic Papers}
\begin{itemize}
\item Rochet \& Tirole (2003): ``Platform Competition in Two-Sided Markets''
\item Suri \& Jack (2016): ``The Long-Run Poverty and Gender Impacts of Mobile Money''
\item Kahn \& Roberds (2009): ``Payments Settlement: Tiering in Private and Public Systems''
\end{itemize}

\column{0.48\textwidth}
\textbf{Policy Reports}
\begin{itemize}
\item BIS -- Bank for International Settlements (2020): ``Enhancing Cross-border Payments''
\item FSB -- Financial Stability Board (2020): ``Cross-border Payments Roadmap''
\item World Bank Remittance Reports
\end{itemize}
\end{columns}

\bottomnote{All readings available on course platform}
\end{frame}

\end{document}
