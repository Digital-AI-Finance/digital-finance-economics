\documentclass[8pt,aspectratio=169]{beamer}
\usetheme{Madrid}
\usepackage{graphicx}
\usepackage{booktabs}
\usepackage{adjustbox}
\usepackage{multicol}
\usepackage{amsmath}

\definecolor{mlblue}{RGB}{0,102,204}
\definecolor{mlpurple}{RGB}{51,51,178}
\definecolor{mllavender}{RGB}{173,173,224}
\definecolor{mllavender2}{RGB}{193,193,232}
\definecolor{mllavender3}{RGB}{204,204,235}
\definecolor{mllavender4}{RGB}{214,214,239}
\definecolor{mlorange}{RGB}{255, 127, 14}
\definecolor{mlgreen}{RGB}{44, 160, 44}
\definecolor{mlred}{RGB}{214, 39, 40}
\definecolor{mlgray}{RGB}{127, 127, 127}
\definecolor{lightgray}{RGB}{240, 240, 240}
\definecolor{midgray}{RGB}{180, 180, 180}

\setbeamercolor{palette primary}{bg=mllavender3,fg=mlpurple}
\setbeamercolor{palette secondary}{bg=mllavender2,fg=mlpurple}
\setbeamercolor{palette tertiary}{bg=mllavender,fg=white}
\setbeamercolor{palette quaternary}{bg=mlpurple,fg=white}
\setbeamercolor{structure}{fg=mlpurple}
\setbeamercolor{section in toc}{fg=mlpurple}
\setbeamercolor{subsection in toc}{fg=mlblue}
\setbeamercolor{title}{fg=mlpurple}
\setbeamercolor{frametitle}{fg=mlpurple,bg=mllavender3}
\setbeamercolor{block title}{bg=mllavender2,fg=mlpurple}
\setbeamercolor{block body}{bg=mllavender4,fg=black}
\setbeamertemplate{navigation symbols}{}
\setbeamertemplate{itemize items}[circle]
\setbeamertemplate{enumerate items}[default]
\setbeamersize{text margin left=5mm,text margin right=5mm}

\newcommand{\bottomnote}[1]{%
\vfill
\vspace{-2mm}
\textcolor{mllavender2}{\rule{\textwidth}{0.4pt}}
\vspace{1mm}
\footnotesize
\textbf{#1}
}

\title{Payment Systems: Mathematical Models and Welfare Optimization}
\subtitle{L04 Extended: Formalizing Network Economics of Value Transfer\\[0.3em]\normalsize From Rochet-Tirole interchange to RTGS liquidity optimization}
\author{Economics of Digital Finance}
\institute{BSc Course}
\date{}

\begin{document}

%% ============================================================
%% SECTION 1: Bridge from Basic Lecture (4 frames: 1 title + 3 content)
%% ============================================================
\section{Bridge from Basic Lecture}

% --- Frame 0 (title) ---
\begin{frame}[plain]
\titlepage
\end{frame}

% --- Frame 1 ---
\begin{frame}[t]{Welcome Back: From Concepts to Math}
\begin{center}
\textit{[XKCD \#1838: ``Machine Learning'']}\\[0.5em]
\small Source: xkcd.com/1838 by Randall Munroe, CC BY-NC 2.5\\[1em]
\normalsize ``You've learned what payment systems do. Today we learn to \textbf{optimize} them.''
\end{center}

\vspace{0.5em}
\begin{center}
\textit{``If you can't model it, you can't regulate it. If you can't regulate it, you can't protect consumers.''}\\
\small --- Adapted from Lord Kelvin
\end{center}

\bottomnote{This lecture formalizes the economic models behind the payment system concepts introduced in the basic lecture.}
\end{frame}

% --- Frame 2 ---
\begin{frame}[t]{From Concepts to Models}
\begin{columns}[T]
\column{0.48\textwidth}
\textbf{What We Know}
\begin{enumerate}
\item Network effects drive payment adoption
\item Two-sided markets require special pricing (platforms must ``get both sides on board'')
\item Cross-border payments are expensive (6\%+ average)
\item Correspondent banking adds intermediary costs
\end{enumerate}

\column{0.48\textwidth}
\textbf{What We'll Formalize}
\begin{enumerate}
\item Rochet-Tirole (2003): optimal platform pricing with cross-side externalities
\item Interchange welfare: who benefits from fee caps?
\item RTGS vs DNS: liquidity cost trade-offs
\item Suri \& Jack (2016): M-Pesa causal impact via difference-in-differences
\end{enumerate}
\end{columns}

\bottomnote{This lecture builds mathematical foundations for the payment concepts introduced in the basic L04 lecture}
\end{frame}

% --- Frame 3: NO GREEK ---
\begin{frame}[t]{Mathematical Toolkit}
\begin{columns}[T]
\column{0.48\textwidth}
\textbf{Key Mathematical Concepts}
\begin{itemize}
\item Optimization (choosing the best outcome given constraints)
\item Welfare analysis (measuring total benefit to society, not just one firm)
\item Game theory (strategic interaction where each player's best action depends on others' choices)
\item Causal inference (proving that X \textit{caused} Y, not just correlated with it)
\end{itemize}

\column{0.48\textwidth}
\textbf{Notation Preview}

\vspace{0.3em}
\begin{adjustbox}{max width=\textwidth}
\begin{tabular}{ll}
\toprule
\textbf{Symbol} & \textbf{Meaning} \\
\midrule
$p_B, p_S$ & Buyer / seller price \\
$c$ & Platform cost per transaction \\
$m$ & Platform profit margin \\
$n_B, n_S$ & Number of buyers / sellers \\
$V$ & Network value \\
$L$ & Liquidity required \\
$W$ & Total welfare (societal benefit) \\
$IF$ & Interchange fee (\% per transaction) \\
\bottomrule
\end{tabular}
\end{adjustbox}

\vspace{0.3em}
\small All notation uses plain letters. Greek symbols introduced one at a time starting next slide.
\end{columns}

\bottomnote{All derivations use BSc-level calculus. Full notation table including Greek symbols in Appendix A1}
\end{frame}

%% ============================================================
%% SECTION 2: Rochet-Tirole Two-Sided Market Optimization (5 frames)
%% ============================================================
\section{Rochet-Tirole Two-Sided Market Optimization}

% --- Frame 4 ---
\begin{frame}[t]{The Platform's Problem -- Setup}
\begin{columns}[T]
\column{0.48\textwidth}
\textbf{Model Setup}

A platform (e.g., Visa) connects buyers (B) and sellers (S):
\begin{itemize}
\item Revenue from buyers: $p_B \times n_B$
\item Revenue from sellers: $p_S \times n_S$
\item Cost of serving buyers: $c_B = 0.5$ per user
\item Cost of serving sellers: $c_S = 0.3$ per user
\end{itemize}

\vspace{0.3em}
\textbf{New symbol:} $\pi$ = platform profit (Greek letter ``pi'')
$$\pi = (p_B - c_B) \cdot n_B + (p_S - c_S) \cdot n_S$$

\column{0.48\textwidth}
\textbf{Demand Functions (with Cross-Side Effects)}

Each side's demand depends on the \textit{other} side's size:
\begin{itemize}
\item $n_B = a_B - b_B \cdot p_B + d_B \cdot n_S$
\item $n_S = a_S - b_S \cdot p_S + d_S \cdot n_B$
\end{itemize}

\vspace{0.3em}
\textbf{Parameters:}
\begin{itemize}
\item $a_B = 100$, $b_B = 20$, $d_B = 0.3$
\item $a_S = 80$, $b_S = 15$, $d_S = 0.2$
\end{itemize}

\vspace{0.3em}
\textit{$d_B = 0.3$ means: for every additional seller, 0.3 more buyers join. This is the cross-side externality.}
\end{columns}

\bottomnote{Based on Rochet \& Tirole (2003, 2006). $\pi$ (pi) is standard notation for profit in economics}
\end{frame}

% --- Frame 5 ---
\begin{frame}[t]{Optimal Price Structure Derivation}
\begin{columns}[T]
\column{0.48\textwidth}
\textbf{First-Order Conditions}

Setting $\dfrac{d\pi}{dp_B} = 0$ and $\dfrac{d\pi}{dp_S} = 0$ yields Lerner-type pricing conditions:

\vspace{0.3em}
\textbf{New symbol:} $\varepsilon$ = elasticity (Greek ``epsilon''), measuring how sensitive demand is to price

$$\frac{p_B - c_B}{p_B} = \frac{1}{\varepsilon_B} - \frac{d_B \cdot n_S}{p_B \cdot b_B}$$

The second term is the \textbf{cross-side adjustment}: the platform internalizes that charging buyers less attracts more sellers.

\column{0.48\textwidth}
\textbf{Worked Example}

With $\varepsilon_B = 2.0$ (buyers are price-sensitive) and $\varepsilon_S = 0.5$ (sellers are captive):

$$\frac{p_B - c_B}{p_S - c_S} = \frac{\varepsilon_S}{\varepsilon_B} = \frac{0.5}{2.0} = 0.25$$

\vspace{0.3em}
\textbf{Interpretation:} The markup on buyers should be one-quarter of the markup on sellers. The platform subsidizes the price-sensitive side (buyers get rewards) and charges the captive side (merchants pay 2--3\%).

\vspace{0.3em}
\textit{Real-world check: Visa charges cardholders \$0 (often with cashback), but merchants pay 1.5--3\%. The math matches reality.}
\end{columns}

\bottomnote{The Lerner index measures market power: higher index = more markup above cost. The cross-side term is the Rochet-Tirole innovation}
\end{frame}

% --- Frame 6 ---
\begin{frame}[t]{Cross-Side Externality Internalization}
\begin{columns}[T]
\column{0.48\textwidth}
\textbf{Key Insight: Level vs Structure}

Rochet-Tirole's central result: in two-sided markets, the \textbf{price structure} (how the total price is split between sides) matters separately from the \textbf{price level} (the total price).

\vspace{0.3em}
\textbf{Cross-side externalities} occur when one side's participation increases value for the other side:
\begin{itemize}
\item More merchants $\to$ card is more useful $\to$ more consumers join
\item More consumers $\to$ merchants cannot afford to refuse cards $\to$ more merchants join
\end{itemize}

This differs from \textbf{standard externalities} (e.g., pollution) because both sides benefit, and the platform can internalize it through pricing.

\column{0.48\textwidth}
\textbf{The Subsidy Logic}

\vspace{0.3em}
\begin{adjustbox}{max width=\textwidth}
\begin{tabular}{lcc}
\toprule
& \textbf{Buyers} & \textbf{Sellers} \\
\midrule
Elasticity & High (2.0) & Low (0.5) \\
If overcharged & Switch to cash & Must accept \\
Platform strategy & Subsidize & Charge \\
Real-world price & \$0 + rewards & 1.5--3\% \\
\bottomrule
\end{tabular}
\end{adjustbox}

\vspace{0.3em}
\textbf{Why not charge both?} Losing the price-sensitive side collapses the whole market (chicken-and-egg problem). Subsidizing buyers creates a positive externality for sellers through increased transaction volume.

\vspace{0.3em}
\textit{Analogy: nightclubs let women in free (price-sensitive, attract the other side) and charge men (captive, high willingness to pay).}
\end{columns}

\bottomnote{Cross-side externalities distinguish two-sided markets from ordinary markets. Standard antitrust (focusing on one side) can get the wrong answer}
\end{frame}

% --- Frame 7 ---
\begin{frame}[t]{Rochet-Tirole Platform Optimization}
\begin{center}
\includegraphics[width=0.55\textwidth]{06_rochet_tirole_platform_optimization/chart.pdf}
\end{center}

\begin{itemize}
\item Panel~(a): Contour plot of platform profit over $(p_B, p_S)$ space. The monopoly optimum (red dot) charges sellers more than buyers, reflecting the elasticity ratio. The social optimum (green square) lowers both prices
\item Panel~(b): Welfare decomposition shows the deadweight loss (DWL) from monopoly pricing---the welfare destroyed by the platform's market power. Socially optimal pricing recovers this DWL
\end{itemize}

\bottomnote{Rochet \& Tirole (2003, 2006). Monopoly pricing distorts the price structure away from the social optimum}
\end{frame}

% --- Frame 8 ---
\begin{frame}[t]{Comparison: Monopoly vs Socially Optimal Pricing}
\begin{columns}[T]
\column{0.48\textwidth}
\textbf{Monopoly Overcharges the Captive Side}
\begin{itemize}
\item Monopoly platform sets $p_S$ above marginal cost because merchants have low elasticity ($\varepsilon_S = 0.5$)
\item Buyer price may be below cost (subsidized via rewards)
\item Total welfare is lower because some efficient transactions do not occur (merchants priced out)
\end{itemize}

\vspace{0.3em}
\textbf{Why Can't Merchants Refuse?}
\begin{itemize}
\item Consumer expectation: ``everywhere I shop accepts Visa''
\item Competitive pressure: if your competitor accepts cards, you must too
\item The ``must-take'' card phenomenon
\end{itemize}

\column{0.48\textwidth}
\textbf{Regulator's Problem}

Maximize total welfare:
$$W = CS_B + CS_S + PS$$
where $CS$ = consumer surplus (benefit to each side above what they pay), $PS$ = producer surplus (platform profit).

\vspace{0.3em}
The monopolist maximizes only $PS$, ignoring $CS_B + CS_S$.

\vspace{0.3em}
\textbf{Key difference:} The social planner internalizes cross-side externalities differently---subsidizes \textit{both} sides more because the network effects benefit everyone.

\vspace{0.3em}
\textit{For multi-platform dynamics (what happens when Visa and Mastercard compete for merchants), see L05 Extended.}
\end{columns}

\bottomnote{Regulators must analyze BOTH sides of the market. Capping merchant fees without considering consumer effects can reduce welfare}
\end{frame}

%% ============================================================
%% SECTION 3: Interchange Fee Regulation and Welfare (4 frames)
%% ============================================================
\section{Interchange Fee Regulation and Welfare}

% --- Frame 9 ---
\begin{frame}[t]{Welfare Analysis of Interchange Caps}
\begin{columns}[T]
\column{0.48\textwidth}
\textbf{Welfare Framework}

Total welfare under interchange fee $IF$:
$$W = CS_B + CS_S + PS_{\text{platform}}$$

where:
\begin{itemize}
\item $CS_B$ = consumer surplus (cardholders benefit from rewards funded by IF)
\item $CS_S$ = merchant surplus (lower IF $\to$ lower merchant costs $\to$ higher surplus)
\item $PS$ = platform/issuer profit
\end{itemize}

\vspace{0.3em}
\textbf{New symbol:} $\Delta$ = change in welfare (Greek ``delta'')

$\Delta W = W_{\text{cap}} - W_{\text{unregulated}}$

\column{0.48\textwidth}
\textbf{Three Regimes Compared}

\vspace{0.3em}
\begin{adjustbox}{max width=\textwidth}
\begin{tabular}{lcc}
\toprule
\textbf{Regime} & \textbf{IF Rate} & \textbf{Example} \\
\midrule
Unregulated & 2.0\% & US credit cards \\
EU Cap & 0.3\% & EU IFR (2015) \\
Zero Cap & 0\% & Theoretical ideal \\
\bottomrule
\end{tabular}
\end{adjustbox}

\vspace{0.3em}
\textbf{The Trade-off:}
\begin{itemize}
\item Lower IF $\to$ merchants pay less $\to$ $CS_S$ rises
\item Lower IF $\to$ fewer card rewards $\to$ $CS_B$ may fall
\item Lower IF $\to$ less issuer profit $\to$ $PS$ falls
\item Net effect depends on elasticities
\end{itemize}

\textbf{Calibration:} 1 trillion transactions, $\varepsilon_B = 2.0$, $\varepsilon_S = 0.5$
\end{columns}

\bottomnote{The EU IFR (Interchange Fee Regulation, 2015) capped interchange at 0.2\% (debit) and 0.3\% (credit). The US left credit uncapped}
\end{frame}

% --- Frame 10 ---
\begin{frame}[t]{Interchange Cap Welfare Effects}
\begin{center}
\includegraphics[width=0.55\textwidth]{07_interchange_welfare_decomposition/chart.pdf}
\end{center}

\begin{itemize}
\item Panel~(a): Welfare decomposition under three regimes. The EU cap shifts surplus from platforms/issuers to merchants while modestly reducing consumer rewards
\item Panel~(b): Net welfare change relative to the unregulated baseline. The EU cap achieves a near-optimal balance---further reduction to zero actually \textit{reduces} welfare by eliminating the card network's ability to subsidize consumers
\end{itemize}

\bottomnote{The EU 0.3\% cap is near-optimal: it reduces merchant costs without destroying the two-sided market mechanism}
\end{frame}

% --- Frame 11 ---
\begin{frame}[t]{Who Really Pays? Pass-Through Analysis}
\begin{columns}[T]
\column{0.48\textwidth}
\textbf{Merchant Cost Pass-Through}

When merchants pay interchange fees, they raise retail prices:
$$\frac{dP_{\text{retail}}}{dIF} = \rho$$
where $\rho$ = pass-through rate (what fraction of the fee increase the merchant passes to consumers in higher prices).

\vspace{0.3em}
\textbf{Empirical estimates:} $\rho \approx 0.5$--$0.7$ (merchants pass 50--70\% of interchange to all consumers via higher prices).

\vspace{0.3em}
\textit{If Visa charges merchants 2\% and $\rho = 0.6$, retail prices rise 1.2\% for EVERYONE---including cash customers.}

\column{0.48\textwidth}
\textbf{Regressive Effects on Cash Users}

\vspace{0.3em}
The cross-subsidy is regressive (benefits the rich, hurts the poor):
\begin{itemize}
\item Card users: pay 1.2\% higher prices, receive 1.5\% cashback $\to$ \textbf{net gain: +0.3\%}
\item Cash users: pay 1.2\% higher prices, receive \$0 cashback $\to$ \textbf{net loss: $-1.2\%$}
\end{itemize}

\vspace{0.3em}
Cash users are disproportionately:
\begin{itemize}
\item Lower income
\item Unbanked or underbanked
\item Elderly or rural populations
\end{itemize}

\vspace{0.3em}
\textit{Schuh et al.\ (2010): average US cash-using household transfers \$149/year to card-using households through this mechanism.}
\end{columns}

\bottomnote{Interchange fees create a hidden regressive transfer from cash users to card users---a key argument for caps}
\end{frame}

% --- Frame 12 ---
\begin{frame}[t]{Optimal Interchange Rate}
\begin{columns}[T]
\column{0.48\textwidth}
\textbf{Deriving the Optimum}

Maximize $W(IF)$ by setting $\dfrac{dW}{dIF} = 0$:

\vspace{0.3em}
Marginal benefit of IF:
$$\frac{dCS_B}{dIF} > 0 \quad\text{(funds consumer rewards)}$$

Marginal cost of IF:
$$\frac{dCS_S}{dIF} < 0 \quad\text{(raises merchant costs)}$$

At the optimum:
$$\frac{dCS_B}{dIF} + \frac{dCS_S}{dIF} + \frac{dPS}{dIF} = 0$$

\column{0.48\textwidth}
\textbf{Worked Example}

With our calibration ($\varepsilon_B = 2.0$, $\varepsilon_S = 0.5$):

\vspace{0.3em}
\begin{adjustbox}{max width=\textwidth}
\begin{tabular}{lcc}
\toprule
\textbf{IF Rate} & \textbf{Net Welfare} & \textbf{Notes} \\
\midrule
0\% & Lower & No rewards, low adoption \\
0.3\% (EU) & Near-optimal & Good balance \\
0.8\% & Theoretical max & Hard to implement \\
2.0\% (US) & Sub-optimal & Merchant distortion \\
\bottomrule
\end{tabular}
\end{adjustbox}

\vspace{0.3em}
The optimal rate $IF^* \approx 0.8\%$ sits between the EU cap (0.3\%) and the US market rate (2.0\%).

\vspace{0.3em}
\textit{Implication: The EU cap is conservative but welfare-improving. The US rate is too high.}
\end{columns}

\bottomnote{The optimal interchange rate balances consumer rewards against merchant cost distortion. Neither zero nor 2\% is optimal}
\end{frame}

%% ============================================================
%% SECTION 4: RTGS vs DNS Liquidity Models (4 frames)
%% ============================================================
\section{RTGS vs DNS Liquidity Models}

% --- Frame 13 ---
\begin{frame}[t]{Liquidity Cost Model -- Setup}
\begin{columns}[T]
\column{0.48\textwidth}
\textbf{RTGS (Real-Time Gross Settlement)}

Each payment needs full liquidity upfront:
$$L_{\text{RTGS}} = \sum_{i,j} |P_{ij}|$$
where $P_{ij}$ = payment from bank $i$ to bank $j$.

\vspace{0.3em}
\textit{With $N$ banks and average bilateral flow of 1B, total gross liquidity = $N(N-1) \times 1$B.}

\vspace{0.3em}
\textbf{Cost:} Banks must hold this as reserves at the central bank, earning below-market returns. Liquidity cost = opportunity cost of holding reserves.

\column{0.48\textwidth}
\textbf{DNS (Deferred Net Settlement)}

Batch netting reduces liquidity needs:
$$L_{\text{DNS}} = \sum_{i} |\text{Net}_i|$$
where $\text{Net}_i = \sum_j P_{ij} - \sum_j P_{ji}$ (each bank's net position).

\vspace{0.3em}
\textbf{Netting Ratio:}
$$NR = 1 - \frac{L_{\text{DNS}}}{L_{\text{RTGS}}}$$

\textit{NR = 0.82 means DNS uses only 18\% of RTGS liquidity---82\% savings.}

\vspace{0.3em}
\textbf{Empirical fit:} $NR = 0.7 + 0.05 \ln(N)$

More banks $\to$ more netting opportunities $\to$ higher NR (logarithmic improvement).
\end{columns}

\bottomnote{Kahn \& Roberds (2009), BIS (2005). The liquidity-risk trade-off is the fundamental design choice in payment systems}
\end{frame}

% --- Frame 14 ---
\begin{frame}[t]{Netting Efficiency}
\begin{columns}[T]
\column{0.48\textwidth}
\textbf{Worked Example: 10 Banks}

\vspace{0.3em}
\begin{adjustbox}{max width=\textwidth}
\begin{tabular}{lr}
\toprule
\textbf{Metric} & \textbf{Value} \\
\midrule
Banks ($N$) & 10 \\
Bilateral pairs & $10 \times 9 = 90$ \\
Average bilateral flow & \$1B \\
Gross liquidity (RTGS) & \$90B \\
Netting ratio & $0.7 + 0.05\ln(10) = 0.815$ \\
Net liquidity (DNS) & $90 \times (1-0.815) = \$16.7$B \\
Liquidity savings & \$73.3B (82\%) \\
\bottomrule
\end{tabular}
\end{adjustbox}

\vspace{0.3em}
\textit{At 5\% opportunity cost of capital, saving \$73.3B in liquidity saves \$3.7B/year in holding costs.}

\column{0.48\textwidth}
\textbf{Why Not Always Use DNS?}

\vspace{0.3em}
\textbf{Settlement risk:} DNS batches payments until end of day. If a bank fails mid-day:
\begin{itemize}
\item All its pending payments unwind
\item Counterparties face unexpected shortfalls
\item Systemic contagion possible
\end{itemize}

\vspace{0.3em}
\textbf{Hybrid systems} combine benefits:
\begin{itemize}
\item Continuous netting with RTGS finality
\item Queue management (holding payments until funds arrive)
\item Central bank intraday credit (lending during the day, repaid by close)
\end{itemize}

\vspace{0.3em}
\textit{Example: TARGET2 (EU) uses RTGS with liquidity-saving mechanisms that achieve partial netting.}
\end{columns}

\bottomnote{Hybrid systems achieve 60--70\% of full netting savings while maintaining real-time finality}
\end{frame}

% --- Frame 15 ---
\begin{frame}[t]{RTGS vs DNS Liquidity Comparison}
\begin{center}
\includegraphics[width=0.55\textwidth]{08_rtgs_dns_liquidity/chart.pdf}
\end{center}

\begin{itemize}
\item Panel~(a): Liquidity required rises quadratically with $N$ for RTGS (every bank must fund every bilateral flow), but grows much more slowly for DNS thanks to netting. At $N=50$, DNS saves 89\% of liquidity
\item Panel~(b): The netting ratio $NR = 0.7 + 0.05\ln(N)$ improves logarithmically---adding banks always helps, but with diminishing returns. At $N=10$, the worked-example value $NR = 0.815$ matches
\end{itemize}

\bottomnote{Kahn \& Roberds (2009). The choice between RTGS and DNS is a liquidity-risk trade-off, not a clear dominance}
\end{frame}

% --- Frame 16 ---
\begin{frame}[t]{Intraday Liquidity and Gridlock}
\begin{columns}[T]
\column{0.48\textwidth}
\textbf{Gridlock: The Coordination Failure}

In RTGS, banks may \textbf{strategically delay} payments:
\begin{itemize}
\item Bank A waits for Bank B's payment before sending its own
\item Bank B does the same
\item Result: mutual waiting (gridlock), no payments settle
\end{itemize}

\vspace{0.3em}
This is a \textbf{coordination game}---both banks prefer to settle, but each prefers the other to go first (to use incoming funds rather than costly reserves).

\vspace{0.3em}
\textit{Gridlock costs: BIS estimates 5--15\% of daily payment value can be delayed by gridlock in systems without resolution mechanisms.}

\column{0.48\textwidth}
\textbf{Resolution Mechanisms}

\vspace{0.3em}
\begin{adjustbox}{max width=\textwidth}
\begin{tabular}{ll}
\toprule
\textbf{Mechanism} & \textbf{How It Works} \\
\midrule
Priority queues & Process time-critical payments first \\
Partial netting & Net a subset of queued payments \\
Intraday credit & Central bank lends during the day \\
LSM (Liquidity & Match offsetting payments \\
Saving Mechanism) & in the queue simultaneously \\
\bottomrule
\end{tabular}
\end{adjustbox}

\vspace{0.3em}
\textbf{Central bank intraday credit:}
\begin{itemize}
\item Free or very low cost (encourages timely payment)
\item Must be repaid by close of day
\item Collateralized (backed by government bonds)
\end{itemize}

\vspace{0.3em}
\textit{Fedwire provides free intraday credit with a cap---banks can overdraw their Fed account during the day.}
\end{columns}

\bottomnote{Gridlock is a coordination failure, not a liquidity shortage. Central bank mechanisms solve the game-theoretic problem}
\end{frame}

%% ============================================================
%% SECTION 5: M-Pesa Impact Estimation (4 frames)
%% ============================================================
\section{M-Pesa Impact Estimation}

% --- Frame 17 ---
\begin{frame}[t]{M-Pesa Natural Experiment -- Setup}
\begin{columns}[T]
\column{0.48\textwidth}
\textbf{Suri \& Jack (2016, \textit{Science})}

Study design: \textbf{difference-in-differences} (DID), a method that compares changes over time between a treatment group and a control group to isolate causal effects.

\vspace{0.3em}
\textbf{Treatment group:} Households near an M-Pesa agent (easy access to mobile money).

\textbf{Control group:} Households far from agents (limited access).

\vspace{0.3em}
\textbf{Why this works:} Agent placement was driven by Safaricom's commercial rollout strategy (population density, road access), not by poverty levels. This creates \textit{quasi-random} variation in access.

\column{0.48\textwidth}
\textbf{Data and Scale}

\vspace{0.3em}
\begin{adjustbox}{max width=\textwidth}
\begin{tabular}{lr}
\toprule
\textbf{Feature} & \textbf{Detail} \\
\midrule
Sample size & $\sim$194,000 households \\
Country & Kenya \\
Period & 2008--2014 \\
Outcome & Poverty rate (\%) \\
Treatment & Proximity to M-Pesa agent \\
\bottomrule
\end{tabular}
\end{adjustbox}

\vspace{0.3em}
\textbf{M-Pesa basics:}
\begin{itemize}
\item Mobile money via basic phone (no smartphone needed)
\item Agent network (not bank branches)
\item Send/receive money, pay bills, save
\item Transaction cost: $\sim$1\% (vs 5--10\% alternatives)
\end{itemize}
\end{columns}

\bottomnote{Suri \& Jack (2016). M-Pesa launched in 2007; by 2014 it was used by 80\%+ of Kenyan adults}
\end{frame}

% --- Frame 18 ---
\begin{frame}[t]{Identification Strategy}
\begin{columns}[T]
\column{0.48\textwidth}
\textbf{DID Formula}

$$Y_{it} = \alpha + \beta \cdot T_i + \gamma \cdot P_t + \delta \cdot (T_i \times P_t) + \varepsilon_{it}$$

where:
\begin{itemize}
\item $Y_{it}$ = poverty rate for household $i$ at time $t$
\item $T_i = 1$ if treatment (near agent), 0 if control
\item $P_t = 1$ if post-treatment period
\item \textbf{New symbol:} $\delta$ = treatment effect (Greek ``delta'')---the causal impact of M-Pesa access
\item $\varepsilon_{it}$ = error term
\end{itemize}

\column{0.48\textwidth}
\textbf{Key Assumption: Parallel Trends}

For DID to give a \textbf{causal} estimate, the two groups must have been trending similarly \textit{before} M-Pesa arrived:
\begin{itemize}
\item If both groups' poverty was falling at the same rate pre-2010, then any \textit{divergence} post-2010 is caused by M-Pesa
\item If trends differed pre-treatment, the estimate is biased
\end{itemize}

\vspace{0.3em}
\textbf{Suri \& Jack verify:} poverty trends in treatment and control areas were statistically parallel from 2008 to 2010 (pre-rollout). See Panel~(a) in the chart.

\vspace{0.3em}
\textit{This is the gold standard for causal inference when randomized experiments are not feasible.}
\end{columns}

\bottomnote{DID is the workhorse of empirical economics. The parallel trends assumption is its Achilles' heel---if it fails, the estimate is invalid}
\end{frame}

% --- Frame 19 ---
\begin{frame}[t]{M-Pesa Poverty Impact}
\begin{center}
\includegraphics[width=0.55\textwidth]{09_mpesa_did_poverty/chart.pdf}
\end{center}

\begin{itemize}
\item Panel~(a): Pre-treatment parallel trends validate the DID design. Post-treatment, the treatment group's poverty rate drops 4pp (34\%$\to$30\%) while the control drops only 1pp (33\%$\to$32\%). The shaded area is the DID effect
\item Panel~(b): The DID estimate $\delta = -3$pp with 95\% CI [$-4.96$, $-1.04$]pp---statistically significant. Poverty fell 3 percentage points \textit{more} in areas with M-Pesa access, controlling for background trends
\end{itemize}

\bottomnote{Suri \& Jack (2016, Science). DID = (30-34) - (32-33) = -4 - (-1) = -3pp. This is a CAUSAL estimate, not mere correlation}
\end{frame}

% --- Frame 20 ---
\begin{frame}[t]{Welfare Implications}
\begin{columns}[T]
\column{0.48\textwidth}
\textbf{Scale of Impact}

Treatment poverty: 34\% $\to$ 30\% ($-4$pp raw change)\\
Control poverty: 33\% $\to$ 32\% ($-1$pp raw change)\\
DID = $(30-34) - (32-33) = -4 - (-1) = -3$pp net causal effect

\vspace{0.3em}
With $\sim$194,000 households in Kenya near M-Pesa agents:
\begin{itemize}
\item 3\% of 194,000 $\approx$ 5,820 households directly
\item Extrapolated nationally: $\sim$196,000 households lifted from poverty (Suri \& Jack estimate)
\end{itemize}

\vspace{0.3em}
\textit{This makes M-Pesa one of the most impactful financial innovations for poverty reduction ever measured.}

\column{0.48\textwidth}
\textbf{Mechanisms}

\vspace{0.3em}
\textbf{Why does mobile money reduce poverty?}
\begin{enumerate}
\item \textbf{Improved risk-sharing:} families send money instantly when crops fail or illness strikes
\item \textbf{Women's empowerment:} 185,000 women moved from farming to business occupations (Suri \& Jack 2016)
\item \textbf{Consumption smoothing:} households maintain spending during income shocks instead of cutting food or school fees
\item \textbf{Savings:} mobile accounts provide a safe place to accumulate small amounts
\end{enumerate}

\vspace{0.3em}
\textit{The poverty reduction effect was larger for female-headed households ($-4.2$pp vs $-2.1$pp for male-headed).}
\end{columns}

\bottomnote{M-Pesa demonstrates that payment infrastructure is not just plumbing---it is a poverty reduction tool with measurable causal impact}
\end{frame}

%% ============================================================
%% SECTION 6: Cross-Border Payment Efficiency (4 frames)
%% ============================================================
\section{Cross-Border Payment Efficiency}

% --- Frame 21 ---
\begin{frame}[t]{Cross-Border Payments: The Last Frontier}
\begin{center}
\textit{[XKCD \#949: ``File Transfer'']}\\[0.5em]
\small Source: xkcd.com/949 by Randall Munroe, CC BY-NC 2.5\\[1em]
\normalsize ``It's easier to send a 1GB file to Mars than \$100 to Manila.\\At least the file doesn't lose 6\% on the way.''
\end{center}

\vspace{0.5em}
\begin{center}
Cross-border payments remain the most expensive, slowest, and most opaque part of the global financial system. Today we decompose \textit{why} and model how new technology addresses each cost component.
\end{center}

\bottomnote{G20 Cross-Border Payments Roadmap (2020) targets: cost below 3\%, speed under 1 hour, transparency by 2027}
\end{frame}

% --- Frame 22 ---
\begin{frame}[t]{Cross-Border Cost Decomposition}
\begin{columns}[T]
\column{0.48\textwidth}
\textbf{Total Cost = Four Components}

$$TC = TC_{FX} + TC_{\text{comply}} + TC_{\text{intermed}} + TC_{\text{settle}}$$

\vspace{0.3em}
\begin{adjustbox}{max width=\textwidth}
\begin{tabular}{lp{5cm}}
\toprule
\textbf{Component} & \textbf{What It Is} \\
\midrule
$TC_{FX}$ & Foreign exchange markup (gap between market rate and rate you get) \\
$TC_{\text{comply}}$ & Compliance costs (AML/KYC checks at each intermediary) \\
$TC_{\text{intermed}}$ & Intermediary fees (each correspondent bank takes a cut) \\
$TC_{\text{settle}}$ & Settlement costs (moving money between central banks) \\
\bottomrule
\end{tabular}
\end{adjustbox}

\column{0.48\textwidth}
\textbf{How Each Technology Helps}

\vspace{0.3em}
\begin{adjustbox}{max width=\textwidth}
\begin{tabular}{lcccc}
\toprule
& \textbf{FX} & \textbf{Comply} & \textbf{Intermed} & \textbf{Settle} \\
\midrule
SWIFT gpi & -- & -- & \checkmark & \checkmark \\
Stablecoin & \checkmark & -- & \checkmark & \checkmark \\
Multi-CBDC & \checkmark & \checkmark & \checkmark & \checkmark \\
\bottomrule
\end{tabular}
\end{adjustbox}

\vspace{0.3em}
\textbf{Key insight:} No single technology eliminates all costs. Multi-CBDC addresses all four components because central banks can:
\begin{itemize}
\item Provide direct FX conversion
\item Standardize compliance
\item Eliminate intermediaries
\item Settle atomically (all-or-nothing, no partial settlement risk)
\end{itemize}
\end{columns}

\bottomnote{BIS (2020). Each cost component requires a different technological solution. Correspondent banking has the worst score on all four}
\end{frame}

% --- Frame 23 ---
\begin{frame}[t]{Cross-Border Payment Efficiency}
\begin{center}
\includegraphics[width=0.55\textwidth]{10_cross_border_cost_waterfall/chart.pdf}
\end{center}

\begin{itemize}
\item Panel~(a): Stacked cost decomposition. Correspondent banking (6.3\%) is dominated by FX markup (2.5\%) and intermediary fees (1.5\%). Each newer technology eliminates or compresses these components. Multi-CBDC achieves 0.5\% total cost
\item Panel~(b): Settlement speed varies by 1,440x---from 3 days (correspondent banking) to 3 minutes (multi-CBDC). The G20 3\% cost target is only met by stablecoin and multi-CBDC channels
\end{itemize}

\bottomnote{BIS (2020), World Bank Remittance Prices. The G20 3\% cost target (red dashed line) is achievable with new technology}
\end{frame}

% --- Frame 24 ---
\begin{frame}[t]{Network Adoption Threshold}
\begin{center}
\includegraphics[width=0.55\textwidth]{11_network_adoption_subsidy/chart.pdf}
\end{center}

\begin{itemize}
\item Panel~(a): Bifurcation diagram showing how subsidies push networks past the critical mass threshold. With weak network effects ($\sigma=0.5$), large subsidies are needed. With strong effects ($\sigma=1.5$), a small push triggers self-sustaining adoption
\item Panel~(b): The optimal subsidy decreases with network effect strength---networks with strong effects need less initial investment because positive feedback loops do the work. Diminishing returns above $\sigma \approx 1.5$
\end{itemize}

\bottomnote{Katz \& Shapiro (1985), Cabral (2011). This model explains why UPI (strong effects) succeeded with modest government subsidy while many payment apps (weak effects) failed}
\end{frame}

%% ============================================================
%% SECTION 7: Synthesis and Policy Implications (3 frames)
%% ============================================================
\section{Synthesis and Policy Implications}

% --- Frame 25 ---
\begin{frame}[t]{Model Synthesis}
\begin{columns}[T]
\column{0.48\textwidth}
\textbf{Converging Insights}
\begin{itemize}
\item \textbf{Rochet-Tirole:} Platform pricing must account for cross-side externalities. Subsidize the elastic side
\item \textbf{Interchange regulation:} EU 0.3\% cap is near-optimal. Eliminates regressive cash-to-card transfer
\item \textbf{RTGS/DNS:} Netting saves 80\%+ liquidity but introduces settlement risk. Hybrid systems optimize the trade-off
\item \textbf{M-Pesa DID:} $\delta = -3$pp poverty reduction, causally identified. Payment infrastructure reduces poverty
\item \textbf{Cross-border:} Multi-CBDC reduces costs from 6.3\% to 0.5\% across all four components
\end{itemize}

\column{0.48\textwidth}
\textbf{Policy Design Principles}

\vspace{0.3em}
\begin{adjustbox}{max width=\textwidth}
\begin{tabular}{ll}
\toprule
\textbf{Principle} & \textbf{Model Basis} \\
\midrule
Cap interchange & Rochet-Tirole + welfare \\
Promote interop & Network adoption model \\
Invest in netting & RTGS/DNS trade-off \\
Subsidize inclusion & M-Pesa evidence \\
Coordinate cross-border & BIS cost decomposition \\
\bottomrule
\end{tabular}
\end{adjustbox}

\vspace{0.3em}
\textbf{Common thread:} Payment systems are infrastructure with network effects. Market outcomes are often suboptimal. Targeted regulation improves welfare without destroying the two-sided market mechanism.
\end{columns}

\bottomnote{All five models converge: payment systems need careful regulation that respects network economics, not laissez-faire or heavy-handed control}
\end{frame}

% --- Frame 26 ---
\begin{frame}[t]{Open Questions}
\begin{columns}[T]
\column{0.48\textwidth}
\textbf{Unresolved Issues}
\begin{itemize}
\item What is the globally optimal interchange rate? (Varies by market structure, elasticities, and pass-through rates)
\item How should multi-CBDC interoperability be governed? (BIS mBridge vs bilateral agreements)
\item Should Big Tech be licensed as payment providers? (Data concentration vs innovation)
\item Can stablecoins deliver cross-border cost reduction at scale without regulatory arbitrage?
\end{itemize}

\column{0.48\textwidth}
\textbf{Assignment Connections}
\begin{itemize}
\item Exercise 1: Compute Rochet-Tirole optimal prices with different elasticities
\item Exercise 2: DID estimation with M-Pesa data
\item Exercise 3: RTGS vs DNS liquidity calculator
\item Quiz questions test interchange welfare reasoning
\item Research paper: any open question above
\end{itemize}

\vspace{0.3em}
\textbf{Key skill:} After this lecture, you can model the welfare effects of payment system regulation using two-sided market theory and causal inference.
\end{columns}

\bottomnote{Payment systems economics is an active research frontier. These models provide the analytical toolkit for evaluating new proposals}
\end{frame}

% --- Frame 27 ---
\begin{frame}[t]{References}
\small
\begin{itemize}
\setlength{\itemsep}{2pt}
\item Rochet, J.-C.\ \& Tirole, J.\ (2003). Platform Competition in Two-Sided Markets. \textit{Journal of the European Economic Association}, 1(4), 990--1029.
\item Rochet, J.-C.\ \& Tirole, J.\ (2006). Two-Sided Markets: A Progress Report. \textit{The RAND Journal of Economics}, 37(3), 645--667.
\item Rochet, J.-C.\ \& Tirole, J.\ (2011). Must-Take Cards: Merchant Discounts and Avoided Costs. \textit{Journal of the European Economic Association}, 9(3), 462--495.
\item EU Interchange Fee Regulation (2015). Regulation (EU) 2015/751.
\item Kahn, C.M.\ \& Roberds, W.\ (2009). Payments Settlement: Tiering in Private and Public Systems. \textit{Journal of Money, Credit and Banking}, 41(5), 855--884.
\item BIS (2005). New Developments in Large-Value Payment Systems. \textit{CPSS Publications No.\ 67}.
\item BIS (2020). Enhancing Cross-border Payments. \textit{Stage 3 Roadmap}.
\item Suri, T.\ \& Jack, W.\ (2016). The Long-Run Poverty and Gender Impacts of Mobile Money. \textit{Science}, 354(6317), 1288--1292.
\item Katz, M.L.\ \& Shapiro, C.\ (1985). Network Externalities, Competition, and Compatibility. \textit{American Economic Review}, 75(3), 424--440.
\item Cabral, L.\ (2011). Dynamic Price Competition with Network Effects. \textit{Review of Economic Studies}, 78(1), 83--111.
\end{itemize}

\bottomnote{Full references for all models and data sources cited in this lecture}
\end{frame}

%% ============================================================
%% APPENDIX (3 frames)
%% ============================================================
\appendix
\section{Appendix}

% --- Frame 28 (A1): Complete Notation Table ---
\begin{frame}[t]{Appendix A1: Complete Notation Table}
\begin{center}
\begin{adjustbox}{max width=0.95\textwidth, max totalheight=0.78\textheight}
\begin{tabular}{lll}
\toprule
\textbf{Symbol} & \textbf{Meaning} & \textbf{Section} \\
\midrule
$p_B, p_S$ & Buyer / seller price & Rochet-Tirole \\
$c_B, c_S$ & Cost of serving buyers / sellers & Rochet-Tirole \\
$n_B, n_S$ & Number of buyers / sellers & Rochet-Tirole \\
$a_B, a_S$ & Demand intercept (base demand without price/network effects) & Rochet-Tirole \\
$b_B, b_S$ & Price sensitivity of demand & Rochet-Tirole \\
$d_B, d_S$ & Cross-side externality strength & Rochet-Tirole \\
$\pi$ & Platform profit (Greek ``pi'') & Rochet-Tirole \\
$\varepsilon_B, \varepsilon_S$ & Price elasticity of demand (Greek ``epsilon'') & Rochet-Tirole / Interchange \\
$IF$ & Interchange fee rate & Interchange \\
$W$ & Total welfare ($= CS_B + CS_S + PS$) & Interchange / Synthesis \\
$CS_B, CS_S$ & Consumer surplus for buyers / sellers & Interchange / Welfare \\
$PS$ & Producer surplus (platform profit) & Interchange / Welfare \\
$\Delta$ & Change in a variable (Greek ``delta'') & Interchange \\
$\rho$ & Pass-through rate (fraction of fee passed to retail prices) & Interchange \\
$L$ & Liquidity required & RTGS/DNS \\
$P_{ij}$ & Payment from bank $i$ to bank $j$ & RTGS/DNS \\
$NR$ & Netting ratio ($= 1 - L_{\text{DNS}}/L_{\text{RTGS}}$) & RTGS/DNS \\
$N$ & Number of banks & RTGS/DNS \\
$Y_{it}$ & Outcome (poverty rate) for unit $i$ at time $t$ & M-Pesa DID \\
$T_i$ & Treatment indicator (1 = near agent) & M-Pesa DID \\
$P_t$ & Post-treatment indicator & M-Pesa DID \\
$\delta$ & DID treatment effect ($= -3$pp) & M-Pesa DID \\
$\sigma$ & Network effect strength & Network Adoption \\
$s$ & Subsidy level & Network Adoption \\
$x$ & Adoption rate & Network Adoption \\
$\delta_d$ & Decay / churn rate & Network Adoption \\
$TC$ & Total cross-border cost & Cross-Border \\
\bottomrule
\end{tabular}
\end{adjustbox}
\end{center}

\bottomnote{Reference page for all mathematical notation used in this lecture. Greek letters introduced gradually through the slides}
\end{frame}

% --- Frame 29 (A2): Rochet-Tirole FOC Derivation ---
\begin{frame}[t]{Appendix A2: Rochet-Tirole FOC Derivation}
\begin{columns}[T]
\column{0.48\textwidth}
\textbf{Simultaneous Demand System}

Solve for equilibrium demands:
\begin{align*}
n_B &= a_B - b_B p_B + d_B n_S \\
n_S &= a_S - b_S p_S + d_S n_B
\end{align*}

Substituting the second into the first:
$$n_B(1 - d_B d_S) = (a_B - b_B p_B) + d_B(a_S - b_S p_S)$$

$$n_B^* = \frac{a_B - b_B p_B + d_B(a_S - b_S p_S)}{1 - d_B d_S}$$

With $d_B d_S = 0.3 \times 0.2 = 0.06$, the denominator $= 0.94$ (system is stable since $d_B d_S < 1$).

\column{0.48\textwidth}
\textbf{Profit Maximization}

$$\pi = (p_B - c_B) n_B^*(p_B, p_S) + (p_S - c_S) n_S^*(p_B, p_S)$$

FOC with respect to $p_B$:
$$\frac{\partial \pi}{\partial p_B} = n_B^* + (p_B - c_B)\frac{\partial n_B^*}{\partial p_B} + (p_S - c_S)\frac{\partial n_S^*}{\partial p_B} = 0$$

The third term captures the \textbf{cross-side effect}: raising $p_B$ reduces $n_B$, which reduces $n_S$ (via $d_S$), which reduces seller revenue.

\vspace{0.3em}
This is why monopoly two-sided pricing differs from standard monopoly: the platform internalizes the indirect effect on the other side.
\end{columns}

\bottomnote{Full derivation for students who want the mathematical details. The cross-side term is the key innovation of Rochet-Tirole}
\end{frame}

% --- Frame 30 (A3): DID Estimation Conditions ---
\begin{frame}[t]{Appendix A3: DID Estimation Conditions}
\begin{columns}[T]
\column{0.48\textwidth}
\textbf{Three Key Assumptions}
\begin{enumerate}
\item \textbf{Parallel trends:} In the absence of treatment, treated and control groups would have followed the same trajectory. Verified by checking pre-treatment trends are statistically indistinguishable.

\item \textbf{SUTVA} (Stable Unit Treatment Value Assumption): One household's treatment does not affect another's outcome. Potentially violated if M-Pesa creates spillovers to control areas.

\item \textbf{No anticipation:} Households did not change behavior \textit{before} M-Pesa arrived. Verified by checking for pre-treatment jumps.
\end{enumerate}

\column{0.48\textwidth}
\textbf{Threats to Validity}

\vspace{0.3em}
\begin{adjustbox}{max width=\textwidth}
\begin{tabular}{lp{4.5cm}}
\toprule
\textbf{Threat} & \textbf{How Addressed} \\
\midrule
Selection bias & Agent placement driven by commercial factors, not poverty \\
Spillovers & Reduced-form estimates are lower bounds (spillovers dilute effect) \\
Time-varying confounds & Placebo tests on unrelated outcomes show null effects \\
Attrition & Panel balanced; attrition uncorrelated with treatment \\
\bottomrule
\end{tabular}
\end{adjustbox}

\vspace{0.3em}
\textbf{Why DID, not RCT?}\\
\textit{A randomized controlled trial (RCT) would randomly assign M-Pesa access. Ethically and practically impossible at national scale. DID exploits the natural variation in rollout timing.}
\end{columns}

\bottomnote{Suri \& Jack (2016) is considered one of the strongest DID designs in development economics due to the quasi-random agent rollout}
\end{frame}

\end{document}
