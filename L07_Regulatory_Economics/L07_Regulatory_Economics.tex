\documentclass[8pt,aspectratio=169]{beamer}
\usetheme{Madrid}
\usepackage{graphicx}
\usepackage{booktabs}
\usepackage{adjustbox}
\usepackage{multicol}
\usepackage{amsmath}

% Color definitions
\definecolor{mlblue}{RGB}{0,102,204}
\definecolor{mlpurple}{RGB}{51,51,178}
\definecolor{mllavender}{RGB}{173,173,224}
\definecolor{mllavender2}{RGB}{193,193,232}
\definecolor{mllavender3}{RGB}{204,204,235}
\definecolor{mllavender4}{RGB}{214,214,239}
\definecolor{mlorange}{RGB}{255, 127, 14}
\definecolor{mlgreen}{RGB}{44, 160, 44}
\definecolor{mlred}{RGB}{214, 39, 40}
\definecolor{mlgray}{RGB}{127, 127, 127}

\definecolor{lightgray}{RGB}{240, 240, 240}
\definecolor{midgray}{RGB}{180, 180, 180}

% Apply custom colors to Madrid theme
\setbeamercolor{palette primary}{bg=mllavender3,fg=mlpurple}
\setbeamercolor{palette secondary}{bg=mllavender2,fg=mlpurple}
\setbeamercolor{palette tertiary}{bg=mllavender,fg=white}
\setbeamercolor{palette quaternary}{bg=mlpurple,fg=white}

\setbeamercolor{structure}{fg=mlpurple}
\setbeamercolor{section in toc}{fg=mlpurple}
\setbeamercolor{subsection in toc}{fg=mlblue}
\setbeamercolor{title}{fg=mlpurple}
\setbeamercolor{frametitle}{fg=mlpurple,bg=mllavender3}
\setbeamercolor{block title}{bg=mllavender2,fg=mlpurple}
\setbeamercolor{block body}{bg=mllavender4,fg=black}

\setbeamertemplate{navigation symbols}{}
\setbeamertemplate{itemize items}[circle]
\setbeamertemplate{enumerate items}[default]
\setbeamersize{text margin left=5mm,text margin right=5mm}

\newcommand{\bottomnote}[1]{%
\vfill
\vspace{-2mm}
\textcolor{mllavender2}{\rule{\textwidth}{0.4pt}}
\vspace{1mm}
\footnotesize
\textbf{#1}
}

\title{Regulatory Economics of Digital Finance}
\subtitle{L07: Costs, Benefits, and Arbitrage}
\author{Economics of Digital Finance}
\institute{BSc Course}
\date{}

\begin{document}

% Title slide
\begin{frame}[plain]
\titlepage
\end{frame}

% Outline
\begin{frame}[t]{Lesson Overview}
\begin{columns}[T]
\column{0.48\textwidth}
\textbf{Today's Topics}
\begin{enumerate}
\item Why regulate digital finance?
\item Cost-benefit analysis of regulation
\item The regulatory perimeter problem
\item Compliance costs and economies of scale
\item Regulatory arbitrage dynamics
\item Consumer protection frameworks
\item International coordination challenges
\end{enumerate}

\column{0.48\textwidth}
\textbf{Learning Objectives}
\begin{itemize}
\item Apply market failure analysis to digital finance
\item Evaluate regulatory costs vs. benefits
\item Understand regulatory arbitrage incentives
\item Assess consumer protection trade-offs
\item Analyze international coordination problems
\end{itemize}
\end{columns}

\bottomnote{This lesson applies regulatory economics frameworks to understand when and how to regulate digital finance}
\end{frame}

% Why Regulate Digital Finance?
\begin{frame}[t]{Why Regulate Digital Finance?}
\begin{columns}[T]
\column{0.48\textwidth}
\textbf{Market Failures in Digital Finance}

\textit{Market failure}: when free markets fail to allocate resources efficiently.

\vspace{0.3em}
\textbf{1. Information Asymmetry (one party knows more)}
\begin{itemize}
\item Complex technical protocols
\item Opaque risk disclosures
\item Retail investor sophistication gaps
\end{itemize}

\vspace{0.3em}
\textbf{2. Externalities (costs/benefits affecting uninvolved parties)}
\begin{itemize}
\item Systemic risk contagion (cascading failures)
\item Money laundering spillovers
\item Environmental costs (energy use)
\end{itemize}

\column{0.48\textwidth}
\textbf{3. Natural Monopoly}
\begin{itemize}
\item Network effects in payment systems
\item Infrastructure control
\item Data monopolies
\end{itemize}

\vspace{0.3em}
\textbf{4. Consumer Protection (shielding individuals from exploitation)}
\begin{itemize}
\item Fraud and scams
\item Irreversible transactions
\item Custody and loss risks
\end{itemize}

\vspace{0.3em}
\textbf{Public Interest Rationale}
\begin{itemize}
\item Financial stability
\item Consumer protection
\item Market integrity
\end{itemize}
\end{columns}

\bottomnote{Stigler (1971): Regulation can serve public interest or private capture---evidence determines which}
\end{frame}

% Cost-Benefit Analysis
\begin{frame}[t]{Cost-Benefit Analysis of Digital Finance Regulation}
\begin{center}
\includegraphics[width=0.65\textwidth]{01_regulation_cost_benefit/chart.pdf}
\end{center}

\vspace{0.3em}
\textbf{Marginal Analysis}: Optimal regulation occurs where the additional (marginal) cost of one more unit of regulation equals its additional benefit. Too little regulation leaves market failures uncorrected; too much creates deadweight loss (value destroyed by excessive rules).

\bottomnote{Harberger (1964): Minimize deadweight loss---the welfare triangle representing foregone efficient transactions}
\end{frame}

% Regulatory Benefits
\begin{frame}[t]{Benefits of Regulating Digital Finance}
\begin{columns}[T]
\column{0.48\textwidth}
\textbf{Quantifiable Benefits}
\begin{itemize}
\item Reduced fraud losses
\item Lower systemic risk probability
\item Improved market efficiency
\item Enhanced consumer confidence
\end{itemize}

\vspace{0.3em}
\textbf{Akerlof's Lemons Problem (Adverse Selection)}

Without regulation:
\begin{itemize}
\item Good projects cannot signal quality
\item Bad projects dominate (adverse selection: bad drives out good)
\item Market unravels
\end{itemize}

\textit{Example}: FTX collapse (2022)---no licensing requirements allowed fraudulent exchange to mix customer funds.

\column{0.48\textwidth}
\textbf{Difficult-to-Measure Benefits}
\begin{itemize}
\item Financial inclusion effects
\item Innovation spillovers
\item Reduced inequality
\item Social trust in institutions
\end{itemize}

\vspace{0.3em}
\textbf{Regulatory Certification}
\begin{itemize}
\item Licensing signals quality
\item Reduces information costs
\item Supports market development
\end{itemize}
\end{columns}

\bottomnote{Akerlof (1970): Information asymmetry can cause market failure---regulation can restore efficiency}
\end{frame}

% Regulatory Costs
\begin{frame}[t]{Costs of Regulating Digital Finance}
\begin{columns}[T]
\column{0.48\textwidth}
\textbf{Direct Costs}
\begin{itemize}
\item Compliance infrastructure
\item Legal and audit expenses
\item Reporting systems
\item Staff training
\end{itemize}

\vspace{0.3em}
\textbf{Administrative Burden}
\begin{itemize}
\item Licensing application fees
\item Ongoing supervisory costs
\item RegTech (Regulatory Technology---software for compliance automation)
\end{itemize}

\column{0.48\textwidth}
\textbf{Indirect Costs}
\begin{itemize}
\item Innovation slowdown
\item Market entry barriers
\item Reduced competition
\item Geographic restrictions
\end{itemize}

\vspace{0.3em}
\textbf{Harberger Deadweight Loss (Welfare Triangle)}

\begin{itemize}
\item Foregone efficient transactions
\item Producer and consumer surplus loss (value destroyed)
\item Optimal regulation minimizes total loss
\end{itemize}

The welfare loss triangle: area between supply and demand curves representing transactions killed by regulation.
\end{columns}

\bottomnote{Harberger (1964): Regulation creates deadweight loss when marginal cost exceeds marginal benefit}
\end{frame}

% Regulatory Perimeter
\begin{frame}[t]{The Regulatory Perimeter Problem}
\begin{center}
\includegraphics[width=0.65\textwidth]{02_regulatory_perimeter/chart.pdf}
\end{center}

\vspace{0.3em}
\textbf{Regulatory Perimeter}: The boundary between regulated and unregulated activities. Firms just outside the perimeter avoid compliance costs but pose similar risks.

\bottomnote{Determining what falls inside vs. outside the regulatory perimeter is a critical policy choice}
\end{frame}

% Perimeter Challenges
\begin{frame}[t]{Challenges in Defining the Regulatory Perimeter}
\begin{columns}[T]
\column{0.48\textwidth}
\textbf{Functional vs. Institutional Regulation}

\vspace{0.3em}
\textbf{Institutional Approach}
\begin{itemize}
\item Regulate entities (banks, exchanges)
\item Clear jurisdictional boundaries
\item Legacy financial system model
\end{itemize}

\vspace{0.3em}
\textbf{Functional Approach}
\begin{itemize}
\item Regulate activities (lending, custody)
\item Technology-neutral
\item Better for digital finance
\end{itemize}

\column{0.48\textwidth}
\textbf{Perimeter Ambiguities}
\begin{itemize}
\item Are DeFi (Decentralized Finance) protocols financial institutions?
\item Are NFTs (Non-Fungible Tokens) securities or commodities?
\item Are DAOs (Decentralized Autonomous Organizations) legal entities?
\item Are validators financial intermediaries?
\end{itemize}

\vspace{0.3em}
\textbf{Economic Trade-offs}
\begin{itemize}
\item Broad perimeter: Higher costs, more certainty
\item Narrow perimeter: Lower costs, arbitrage risk
\end{itemize}
\end{columns}

\bottomnote{Same risk, same regulation principle suggests functional approach for digital finance}
\end{frame}

% Compliance Costs and Scale
\begin{frame}[t]{Compliance Costs and Economies of Scale}
\begin{center}
\includegraphics[width=0.65\textwidth]{03_compliance_cost_scale/chart.pdf}
\end{center}

\bottomnote{Stigler (1971): Regulation favors large firms through economies of scale in compliance---creates barrier to entry}
\end{frame}

% Economies of Scale Analysis
\begin{frame}[t]{Regulatory Economies of Scale: Winners and Losers}
\begin{columns}[T]
\column{0.48\textwidth}
\textbf{Why Compliance Costs Are Fixed}
\begin{itemize}
\item Core systems: KYC (Know Your Customer) / AML (Anti-Money Laundering) / reporting
\item Legal and compliance teams
\item Technology infrastructure
\item Audit and certification
\end{itemize}

\vspace{0.3em}
\textbf{Stigler's Insight}
\begin{itemize}
\item Large firms spread fixed costs over more users
\item Average cost per user falls with scale
\item Small entrants face higher unit costs
\end{itemize}

\column{0.48\textwidth}
\textbf{Market Structure Consequences}
\begin{itemize}
\item Concentration and consolidation
\item Reduced innovation from startups
\item Incumbent protection
\item Barriers to entry
\end{itemize}

\vspace{0.3em}
\textbf{Policy Responses}
\begin{itemize}
\item Tiered regulation (proportionality)
\item Regulatory sandboxes (controlled experimentation zones)
\item Shared compliance infrastructure
\item RegTech innovation support
\end{itemize}
\end{columns}

\bottomnote{Regulation can unintentionally create barriers that protect incumbents at the expense of competition}
\end{frame}

% Regulatory Arbitrage
\begin{frame}[t]{Regulatory Arbitrage: Game Theory Perspective}
\begin{center}
\includegraphics[width=0.65\textwidth]{04_regulatory_arbitrage_game/chart.pdf}
\end{center}

\vspace{0.3em}
\textbf{Regulatory Arbitrage}: Moving to jurisdictions with looser rules to avoid compliance costs.

\textbf{Nash Equilibrium}: Outcome where no player can improve by changing strategy alone. Here: all jurisdictions compete by lowering standards.

\bottomnote{Regulatory arbitrage is a Nash equilibrium when jurisdictions compete for mobile capital}
\end{frame}

% Arbitrage Mechanisms
\begin{frame}[t]{Mechanisms of Regulatory Arbitrage in Digital Finance}
\begin{columns}[T]
\column{0.48\textwidth}
\textbf{Types of Arbitrage}

\vspace{0.3em}
\textbf{1. Jurisdictional Arbitrage}
\begin{itemize}
\item Offshore exchange registration
\item Tax haven incorporation
\item Regulatory shopping
\end{itemize}

\textit{Example}: Bitcoin mining operations moved from China (2021 ban) to Kazakhstan, then to US/Canada when Kazakhstan tightened rules.

\vspace{0.3em}
\textbf{2. Structural Arbitrage}
\begin{itemize}
\item Legal entity classification gaming
\item Product redesign to avoid rules
\item Functional unbundling
\end{itemize}

\column{0.48\textwidth}
\textbf{3. Temporal Arbitrage}
\begin{itemize}
\item Operating before rules finalized
\item Moving to new jurisdictions preemptively
\item Exploiting regulatory lag
\end{itemize}

\vspace{0.3em}
\textbf{Economic Consequences}
\begin{itemize}
\item Undermines regulatory effectiveness
\item Race to the bottom in standards
\item Regulatory fragmentation
\item Enforcement challenges
\end{itemize}
\end{columns}

\bottomnote{Digital finance's global and borderless nature amplifies regulatory arbitrage incentives}
\end{frame}

% Game Theory of Arbitrage
\begin{frame}[t]{The Race to the Bottom: Game Theory of Regulatory Competition}
\begin{columns}[T]
\column{0.48\textwidth}
\textbf{Standard Prisoner's Dilemma}

\vspace{0.3em}
Two jurisdictions:
\begin{itemize}
\item Cooperate: Harmonized standards
\item Defect: Lax regulation to attract firms
\end{itemize}

\vspace{0.3em}
\textbf{Nash Equilibrium}
\begin{itemize}
\item Both defect (lax regulation)
\item Suboptimal outcome for global welfare
\item Coordination failure
\end{itemize}

\column{0.48\textwidth}
\textbf{Solutions to Coordination Failure}

\vspace{0.3em}
\textbf{1. International Agreements}
\begin{itemize}
\item FATF (Financial Action Task Force) standards for AML
\item Basel accords for capital
\item IOSCO (International Organization of Securities Commissions) principles
\end{itemize}

\textit{Example}: FATF grey-listing (e.g., UAE 2022) forces jurisdictions to strengthen AML rules or face financial exclusion.

\vspace{0.3em}
\textbf{2. Extraterritorial Enforcement}
\begin{itemize}
\item Long-arm jurisdiction
\item Market access conditionality
\item Reciprocity requirements
\end{itemize}
\end{columns}

\bottomnote{Without coordination, jurisdictional competition creates a race to the bottom in regulatory standards}
\end{frame}

% Consumer Protection
\begin{frame}[t]{Consumer Protection: Information Asymmetry Framework}
\begin{center}
\includegraphics[width=0.65\textwidth]{05_consumer_protection_information/chart.pdf}
\end{center}

\bottomnote{Spence (1973): Disclosure requirements can signal quality and mitigate information asymmetry}
\end{frame}

% Consumer Protection Details
\begin{frame}[t]{Consumer Protection in Digital Finance: Policy Tools}
\begin{columns}[T]
\column{0.48\textwidth}
\textbf{Information Remedies}

\vspace{0.3em}
\textbf{Disclosure Requirements}
\begin{itemize}
\item Risk warnings
\item Fee transparency
\item Performance metrics
\item Conflict of interest disclosures
\end{itemize}

\vspace{0.3em}
\textbf{Financial Literacy}
\begin{itemize}
\item Education campaigns
\item Suitability assessments
\item Cooling-off periods
\end{itemize}

\column{0.48\textwidth}
\textbf{Conduct Remedies}

\vspace{0.3em}
\textbf{Behavioral Rules}
\begin{itemize}
\item Advertising restrictions
\item Prohibited practices (pump-and-dump)
\item Custody standards (self-custody: holding own private keys)
\item Fiduciary duties
\end{itemize}

\textit{Example}: QuadrigaCX collapse (2019)---\$190M lost when CEO died with sole custody of private keys. No custody standards.

\vspace{0.3em}
\textbf{Redress Mechanisms}
\begin{itemize}
\item Dispute resolution
\item Compensation schemes
\item Whistleblower protections
\end{itemize}
\end{columns}

\bottomnote{Effective consumer protection balances information remedies with conduct regulation}
\end{frame}

% Information Asymmetry Analysis
\begin{frame}[t]{When Does Disclosure Fail? Limits of Information Remedies}
\begin{columns}[T]
\column{0.48\textwidth}
\textbf{Theoretical Limits}

\vspace{0.3em}
\textbf{Bounded Rationality}
\begin{itemize}
\item Information overload
\item Complexity of protocols
\item Cognitive biases
\end{itemize}

\vspace{0.3em}
\textbf{Behavioral Biases}
\begin{itemize}
\item Overconfidence
\item FOMO (fear of missing out)
\item Herd behavior
\end{itemize}

\textit{Moral Hazard}: Taking more risk because someone else bears the cost (e.g., expecting bailouts).

\column{0.48\textwidth}
\textbf{When Conduct Rules Are Needed}
\begin{itemize}
\item Sophisticated fraud schemes
\item Systemic risk externalities
\item Irreversible harm
\item Vulnerable populations
\end{itemize}

\vspace{0.3em}
\textbf{Proportionality Principle}
\begin{itemize}
\item Retail vs. institutional investors
\item Size and complexity thresholds
\item Risk-based approach
\end{itemize}
\end{columns}

\bottomnote{Disclosure alone is insufficient when complexity exceeds consumer capacity to process information}
\end{frame}

% International Coordination
\begin{frame}[t]{International Coordination: Challenges and Solutions}
\begin{columns}[T]
\column{0.48\textwidth}
\textbf{Why Coordination Is Hard}

\vspace{0.3em}
\textbf{Sovereignty Concerns}
\begin{itemize}
\item Differing policy priorities
\item Regulatory culture variation
\item Political economy constraints
\end{itemize}

\vspace{0.3em}
\textbf{Heterogeneous Preferences}
\begin{itemize}
\item Financial stability vs. innovation
\item Privacy vs. law enforcement
\item Consumer protection vs. market access
\end{itemize}

\column{0.48\textwidth}
\textbf{Coordination Mechanisms}

\vspace{0.3em}
\textbf{Standard-Setting Bodies}
\begin{itemize}
\item FSB (Financial Stability Board---global coordination body)
\item BCBS (Basel Committee on Banking Supervision---capital standards)
\item FATF for AML/CFT (Combating Financing of Terrorism)
\item IOSCO for securities
\end{itemize}

\vspace{0.3em}
\textbf{Success Factors}
\begin{itemize}
\item Soft law and principles
\item Peer review mechanisms
\item Market access incentives
\end{itemize}
\end{columns}

\bottomnote{International coordination requires balancing sovereignty with collective action to address cross-border risks}
\end{frame}

% Coordination Examples
\begin{frame}[t]{Case Studies in International Coordination}
\begin{columns}[T]
\column{0.48\textwidth}
\textbf{Success: FATF Travel Rule}

\vspace{0.3em}
Coordination achieved through:
\begin{itemize}
\item Clear standards: Travel Rule (originator/beneficiary info for transactions >1000 USD)
\item Peer review and grey-listing
\item Private sector engagement
\item Technology-neutral approach
\end{itemize}

\vspace{0.3em}
\textbf{Remaining Challenges}
\begin{itemize}
\item Implementation heterogeneity
\item Enforcement gaps
\item DeFi application
\end{itemize}

\column{0.48\textwidth}
\textbf{Challenge: Stablecoin Regulation}

\vspace{0.3em}
Fragmentation due to:
\begin{itemize}
\item Divergent classifications (money, security, e-money)
\item Reserve requirements variation
\item Redemption right differences
\item Systemic risk thresholds
\end{itemize}

\vspace{0.3em}
\textbf{Needed Reforms}
\begin{itemize}
\item Harmonized definitions
\item Mutual recognition agreements
\item Cross-border resolution frameworks
\end{itemize}
\end{columns}

\bottomnote{Successful coordination requires alignment on definitions, standards, and enforcement mechanisms}
\end{frame}

% Future Regulatory Frameworks
\begin{frame}[t]{Future Regulatory Frameworks for Digital Finance}
\begin{columns}[T]
\column{0.48\textwidth}
\textbf{Principles-Based vs. Rules-Based}

\vspace{0.3em}
\textbf{Principles-Based (UK Model)}
\begin{itemize}
\item Flexible and adaptive
\item High-level objectives
\item Supervisory judgment
\item Better for rapid innovation
\end{itemize}

\vspace{0.3em}
\textbf{Rules-Based (US Model)}
\begin{itemize}
\item Precise requirements
\item Legal certainty
\item Easier enforcement
\item Lower supervisory discretion
\end{itemize}

\column{0.48\textwidth}
\textbf{Emerging Approaches}

\vspace{0.3em}
\textbf{Regulatory Sandboxes}
\begin{itemize}
\item Controlled experimentation
\item Learning by doing
\item Reduced barriers for innovation
\end{itemize}

\vspace{0.3em}
\textbf{Embedded Supervision}
\begin{itemize}
\item Real-time compliance monitoring
\item Automated reporting (RegTech)
\item Smart contract-based rules
\item SupTech (Supervisory Technology---tools for regulators to monitor markets)
\end{itemize}
\end{columns}

\bottomnote{Future regulation will likely blend principles-based flexibility with technology-enabled precision}
\end{frame}

% Optimal Regulation Framework
\begin{frame}[t]{Designing Optimal Regulation: Economic Framework}
\begin{columns}[T]
\column{0.48\textwidth}
\textbf{Marginal Analysis}

\vspace{0.3em}
Optimal regulation occurs where:
\begin{itemize}
\item Marginal benefit = Marginal cost
\item Deadweight loss minimized
\item Net social welfare maximized
\end{itemize}

\vspace{0.3em}
\textbf{Context-Specific Factors}
\begin{itemize}
\item Market maturity
\item Systemic importance
\item Consumer sophistication
\item Cross-border exposure
\end{itemize}

\column{0.48\textwidth}
\textbf{Design Principles}

\vspace{0.3em}
\textbf{1. Proportionality}
\begin{itemize}
\item Regulation matches risk level
\item Tiered approaches for different sizes
\end{itemize}

\vspace{0.3em}
\textbf{2. Technology-Neutrality}
\begin{itemize}
\item Functional not institutional
\item Avoid picking technology winners
\end{itemize}

\vspace{0.3em}
\textbf{3. Adaptive Regulation}
\begin{itemize}
\item Review and adjust
\item Sunset provisions
\item Experimentation and learning
\end{itemize}
\end{columns}

\bottomnote{Optimal regulation is dynamic, proportional, and evidence-based}
\end{frame}

% Summary
\begin{frame}[t]{Summary and Key Takeaways}
\begin{columns}[T]
\column{0.48\textwidth}
\textbf{What We Covered}
\begin{enumerate}
\item Market failures justify regulation
\item Cost-benefit framework for analysis
\item Regulatory perimeter challenges
\item Compliance costs favor large firms
\item Regulatory arbitrage dynamics
\item Consumer protection trade-offs
\item International coordination needs
\end{enumerate}

\column{0.48\textwidth}
\textbf{Key Economic Insights}
\begin{itemize}
\item Harberger: Minimize deadweight loss
\item Stigler: Regulation creates scale economies
\item Akerlof: Information asymmetry market failure
\item Game theory: Coordination challenges
\end{itemize}

\vspace{0.3em}
\textbf{Looking Ahead}
\begin{itemize}
\item L08: Synthesis of all four lenses
\item Integration across lessons
\end{itemize}
\end{columns}

\vspace{0.5em}
\textbf{Core Message}

Regulatory economics provides tools to design efficient regulation: balancing market failure correction against compliance costs, while managing arbitrage and coordination challenges.

\bottomnote{Next lesson: Synthesis and Integration of Economic Frameworks}
\end{frame}

% Key Terms
\begin{frame}[t]{Key Terms (1/3)}
\begin{columns}[T]
\column{0.48\textwidth}
\textbf{Market Failure}
Situation where free markets fail to allocate resources efficiently, justifying regulatory intervention.

\vspace{0.3em}
\textbf{Information Asymmetry}
Situation where one party has more information than another, potentially leading to adverse selection or moral hazard.

\vspace{0.3em}
\textbf{Adverse Selection}
When information asymmetry causes bad products/actors to dominate markets (Akerlof's lemons problem).

\vspace{0.3em}
\textbf{Moral Hazard}
Taking excessive risk because someone else bears the cost (e.g., expecting bailouts).

\vspace{0.3em}
\textbf{Externality}
Cost or benefit affecting parties not directly involved in a transaction.

\vspace{0.3em}
\textbf{Systemic Risk}
Risk that one failure triggers cascading failures across the financial system.

\column{0.48\textwidth}
\textbf{Regulatory Perimeter}
Boundary defining which activities and entities fall under regulatory oversight.

\vspace{0.3em}
\textbf{Regulatory Arbitrage}
Exploiting differences in rules across jurisdictions or asset classifications to avoid regulation.

\vspace{0.3em}
\textbf{Regulatory Capture}
When regulated industry gains control over its regulator, leading to rules favoring industry over public interest.

\vspace{0.3em}
\textbf{Deadweight Loss}
Economic inefficiency representing value destroyed by regulation or market distortions (Harberger triangle).

\vspace{0.3em}
\textbf{Compliance Costs}
Direct and indirect expenses of meeting regulatory requirements.

\vspace{0.3em}
\textbf{Consumer Protection}
Rules and standards designed to shield individuals from exploitation, fraud, or harm.

\end{columns}

\bottomnote{Regulatory economics provides framework for evaluating when and how to regulate digital finance}
\end{frame}

% Key Terms Page 2
\begin{frame}[t]{Key Terms (2/3)}
\begin{columns}[T]
\column{0.48\textwidth}
\textbf{Nash Equilibrium}
Game theory outcome where no player can improve by changing strategy alone; all players' choices are mutual best responses.

\vspace{0.3em}
\textbf{Prisoner's Dilemma}
Situation where individual rationality leads to collectively suboptimal outcome (e.g., regulatory race to the bottom).

\vspace{0.3em}
\textbf{Proportionality}
Principle that regulatory intensity should match risk level and firm size/complexity.

\vspace{0.3em}
\textbf{Sandbox}
Controlled regulatory environment allowing limited experimentation with relaxed rules for innovation.

\vspace{0.3em}
\textbf{RegTech}
Regulatory Technology---software tools automating compliance (KYC, AML, reporting).

\vspace{0.3em}
\textbf{SupTech}
Supervisory Technology---tools enabling regulators to monitor markets in real-time.

\column{0.48\textwidth}
\textbf{KYC (Know Your Customer)}
Identity verification process required to prevent fraud and money laundering.

\vspace{0.3em}
\textbf{AML (Anti-Money Laundering)}
Rules preventing financial system use for laundering illicit funds.

\vspace{0.3em}
\textbf{Travel Rule}
FATF requirement to transmit originator/beneficiary information for transactions exceeding threshold (typically USD 1000).

\vspace{0.3em}
\textbf{Grey-listing}
FATF designation of jurisdictions with strategic AML/CFT deficiencies, triggering enhanced scrutiny.

\vspace{0.3em}
\textbf{Self-Custody}
Holding private keys directly without intermediary (vs. custodial services like exchanges).

\vspace{0.3em}
\textbf{Prudential Regulation}
Rules ensuring financial institutions' safety and soundness (capital, liquidity, risk management).

\end{columns}

\bottomnote{Understanding these terms is essential for analyzing regulatory economics of digital finance}
\end{frame}

% Key Terms Page 3
\begin{frame}[t]{Key Terms (3/3)}
\begin{columns}[T]
\column{0.48\textwidth}
\textbf{Technology-Neutral Regulation}
Rules focused on function/activity rather than specific technology, avoiding picking winners.

\vspace{0.3em}
\textbf{Functional Regulation}
Regulating activities (lending, custody) rather than entity types (banks, exchanges).

\vspace{0.3em}
\textbf{Institutional Regulation}
Regulating entity types with clear jurisdictional boundaries (traditional banking model).

\vspace{0.3em}
\textbf{Same Risk, Same Regulation}
Principle that activities posing similar risks should face equivalent regulatory treatment.

\vspace{0.3em}
\textbf{Extraterritorial Jurisdiction}
Applying domestic laws to foreign entities accessing domestic markets.

\column{0.48\textwidth}
\textbf{Principles-Based Regulation}
Flexible approach setting high-level objectives with supervisory judgment (UK model).

\vspace{0.3em}
\textbf{Rules-Based Regulation}
Precise requirements approach with clear enforcement but less flexibility (US model).

\vspace{0.3em}
\textbf{FATF (Financial Action Task Force)}
Global standard-setter for AML/CFT policies and international coordination.

\vspace{0.3em}
\textbf{FSB (Financial Stability Board)}
International body coordinating financial regulation to address systemic risks.

\vspace{0.3em}
\textbf{Economies of Scale}
Cost advantages achieved with increased production volume; in regulation, large firms spread fixed compliance costs over more users.

\end{columns}

\bottomnote{Mastery of terminology enables precise analysis of regulatory trade-offs}
\end{frame}

% Further Reading
\begin{frame}[t]{Further Reading}
\begin{columns}[T]
\column{0.48\textwidth}
\textbf{Foundational Papers}
\begin{itemize}
\item Stigler (1971): ``The Theory of Economic Regulation''
\item Akerlof (1970): ``The Market for Lemons''
\item Harberger (1964): ``The Measurement of Waste''
\item Peltzman (1976): ``Toward a More General Theory of Regulation''
\end{itemize}

\column{0.48\textwidth}
\textbf{Digital Finance Applications}
\begin{itemize}
\item FSB (2022): ``Assessment of Risks to Financial Stability from Crypto-assets''
\item Zetzsche et al. (2020): ``The Markets in Crypto-Assets Regulation (MiCA)''
\item Auer \& Claessens (2020): ``Regulating Big Tech in Finance''
\end{itemize}
\end{columns}

\bottomnote{All readings available on course platform}
\end{frame}

\end{document}
