\documentclass[8pt,aspectratio=169]{beamer}
\usetheme{Madrid}
\usepackage{graphicx}
\usepackage{booktabs}
\usepackage{adjustbox}
\usepackage{multicol}
\usepackage{amsmath}

\definecolor{mlblue}{RGB}{0,102,204}
\definecolor{mlpurple}{RGB}{51,51,178}
\definecolor{mllavender}{RGB}{173,173,224}
\definecolor{mllavender2}{RGB}{193,193,232}
\definecolor{mllavender3}{RGB}{204,204,235}
\definecolor{mllavender4}{RGB}{214,214,239}
\definecolor{mlorange}{RGB}{255, 127, 14}
\definecolor{mlgreen}{RGB}{44, 160, 44}
\definecolor{mlred}{RGB}{214, 39, 40}
\definecolor{mlgray}{RGB}{127, 127, 127}

\definecolor{lightgray}{RGB}{240, 240, 240}
\definecolor{midgray}{RGB}{180, 180, 180}

\setbeamercolor{palette primary}{bg=mllavender3,fg=mlpurple}
\setbeamercolor{palette secondary}{bg=mllavender2,fg=mlpurple}
\setbeamercolor{palette tertiary}{bg=mllavender,fg=white}
\setbeamercolor{palette quaternary}{bg=mlpurple,fg=white}

\setbeamercolor{structure}{fg=mlpurple}
\setbeamercolor{section in toc}{fg=mlpurple}
\setbeamercolor{subsection in toc}{fg=mlblue}
\setbeamercolor{title}{fg=mlpurple}
\setbeamercolor{frametitle}{fg=mlpurple,bg=mllavender3}
\setbeamercolor{block title}{bg=mllavender2,fg=mlpurple}
\setbeamercolor{block body}{bg=mllavender4,fg=black}

\setbeamertemplate{navigation symbols}{}
\setbeamertemplate{itemize items}[circle]
\setbeamertemplate{enumerate items}[default]
\setbeamersize{text margin left=5mm,text margin right=5mm}

\newcommand{\bottomnote}[1]{%
\vfill
\vspace{-2mm}
\textcolor{mllavender2}{\rule{\textwidth}{0.4pt}}
\vspace{1mm}
\footnotesize
\textbf{#1}
}

\title{Regulatory Economics: Mathematical Models and Game Theory Analysis}
\subtitle{L07 Extended: Formalizing Market Failures, Regulatory Games, and Optimal Design\\[0.3em]\normalsize From Harberger welfare triangles to FATF coordination game theory}
\author{Economics of Digital Finance}
\institute{BSc Course}
\date{}

\begin{document}

%% ===== SECTION 1: Bridge from Basic Lecture (4 frames: 1 title + 3 content) =====
\section{Bridge from Basic Lecture}

% Frame 1: Title slide
\begin{frame}[plain]
\titlepage
\end{frame}

% Frame 2: Lesson overview
\begin{frame}[t]{Lesson Overview: Why Formalize Regulation?}

\textbf{Connection to L07 Basic:} The basic lecture introduced regulatory concepts qualitatively. This extended lecture formalizes them with mathematical models, enabling precise welfare analysis and policy comparison.

\vspace{0.3em}
\begin{columns}[T]
\column{0.48\textwidth}
\textbf{Today's Topics}
\begin{enumerate}
\item Harberger welfare analysis of regulation costs
\item Akerlof lemons model applied to crypto
\item Regulatory sandbox evaluation metrics
\item FATF coordination game theory
\item RegTech cost optimization
\item Multi-criteria framework comparison
\end{enumerate}

\column{0.48\textwidth}
\textbf{Learning Objectives}
\begin{itemize}
\item Calculate deadweight loss from regulatory distortions
\item Model information asymmetry in token markets
\item Evaluate sandbox efficiency with data
\item Solve coordination games for international regulation
\item Optimize compliance technology investment
\item Compare regulatory frameworks across stakeholder perspectives
\end{itemize}
\end{columns}

\bottomnote{This extended lecture builds mathematical foundations for the qualitative concepts introduced in L07 Basic}
\end{frame}

% Frame 3: Mathematical toolkit (NO Greek)
\begin{frame}[t]{Mathematical Toolkit for Regulatory Analysis}

\textbf{Core notation used throughout this lecture:}

\vspace{0.3em}
\begin{columns}[T]
\column{0.48\textwidth}
\textbf{Welfare Variables}
\begin{itemize}
\item W = total social welfare (sum of all gains and losses in society)
\item CS = consumer surplus (benefit consumers get above what they pay)
\item PS = producer surplus (revenue firms earn above their costs)
\item DWL = deadweight loss (welfare destroyed by market distortions---value that nobody captures)
\end{itemize}

\vspace{0.3em}
\textbf{Market Variables}
\begin{itemize}
\item p = market price
\item q = quantity traded or quality level
\item c = cost of compliance per unit
\end{itemize}

\column{0.48\textwidth}
\textbf{Regulatory Variables}
\begin{itemize}
\item V = transaction volume (number of compliance checks)
\item s = signal quality (how well a test reveals true quality)
\item r = regulation intensity (strictness of rules, from 0 = none to high)
\end{itemize}

\vspace{0.3em}
\textbf{Key Relationships}
\begin{itemize}
\item W = CS + PS $-$ DWL (welfare accounting identity)
\item DWL = 0.5 $\times$ price change $\times$ quantity change (the Harberger triangle)
\item Higher r raises compliance costs but may reduce fraud losses
\end{itemize}

\vspace{0.3em}
\textbf{Goal:} Find the regulation intensity r that maximizes W
\end{columns}

\bottomnote{All notation is standard Latin letters; Greek symbols will be introduced one at a time in later slides}
\end{frame}

% Frame 4: Introducing tau and DWL formula
\begin{frame}[t]{Deadweight Loss from Regulatory Burden}

\textbf{New symbol:} Let $\tau$ (tau, the Greek letter) represent the tax or regulatory burden imposed on each transaction---the per-unit compliance cost that acts like a tax.

\vspace{0.3em}
\begin{columns}[T]
\column{0.48\textwidth}
\textbf{The Harberger Triangle}

When regulation imposes cost $\tau$ per transaction:
\begin{itemize}
\item Price rises by $\Delta p$ (part of $\tau$ passed to consumers)
\item Quantity falls by $\Delta q$ (some transactions no longer worth doing)
\item \textbf{Deadweight loss:}
$$\text{DWL} = \frac{1}{2} \cdot \Delta p \cdot \Delta q$$
\end{itemize}

\textbf{Example:} KYC (Know Your Customer---identity verification) costs \$25 per customer. If this deters 10\% of a 100{,}000-customer market:
$$\text{DWL} = \frac{1}{2} \times \$25 \times 10{,}000 = \$125{,}000$$

But if KYC prevents \$500{,}000 in fraud, net welfare is positive: \$500{,}000 $-$ \$125{,}000 $-$ \$250{,}000 (direct cost) = \$125{,}000 gain.

\column{0.48\textwidth}
\textbf{Why DWL Matters for Crypto}

\vspace{0.3em}
\begin{itemize}
\item DWL is not a transfer---it is value destroyed
\item Crypto regulation creates DWL when:
\begin{itemize}
\item Compliance costs deter legitimate innovation
\item Small projects cannot afford licensing
\item Cross-border friction reduces trade
\end{itemize}
\item But \textit{absence} of regulation also creates losses:
\begin{itemize}
\item Fraud losses (scams, rug pulls)
\item Market manipulation
\item Systemic risk from unregulated leverage
\end{itemize}
\end{itemize}

\vspace{0.3em}
\textbf{The Regulator's Problem:}

Choose $\tau$ to maximize:
$$W(\tau) = \text{Fraud prevented}(\tau) - \text{DWL}(\tau) - \text{Direct cost}(\tau)$$
\end{columns}

\bottomnote{Harberger (1964): DWL grows with the square of $\tau$---doubling regulatory burden quadruples the deadweight loss}
\end{frame}

%% ===== SECTION 2: Harberger Welfare Analysis (4 frames) =====
\section{Harberger Welfare Analysis of Regulation}

% Frame 5: Marginal benefit/cost framework
\begin{frame}[t]{Optimal Regulation: When to Stop Adding Rules}

\textbf{Framework:} Regulation intensity r has diminishing marginal benefit (MB) and increasing marginal cost (MC). Optimal regulation is where MB = MC.

\vspace{0.3em}
\begin{columns}[T]
\column{0.48\textwidth}
\textbf{Welfare Function}
$$W(r) = B(r) - C(r)$$

where:
\begin{itemize}
\item $B(r)$ = total benefits (fraud prevented, consumer confidence gained, market stability improved)
\item $C(r)$ = total costs (compliance spending, innovation deterred, DWL from reduced activity)
\end{itemize}

\textbf{Optimality condition:}
$$\frac{dW}{dr} = 0 \implies MB(r^*) = MC(r^*)$$

\column{0.48\textwidth}
\textbf{Worked Example: KYC Regulation}

\vspace{0.3em}
\begin{tabular}{lcc}
\toprule
\textbf{Metric} & \textbf{Per customer} & \textbf{Total (10K)} \\
\midrule
KYC cost & \$25 & \$250{,}000 \\
Fraud prevented & \$150 & \$1{,}500{,}000 \\
Net benefit & \$125 & \$1{,}250{,}000 \\
\midrule
DWL (deterred) & --- & \$125{,}000 \\
\textbf{Net welfare} & --- & \textbf{\$1{,}125{,}000} \\
\bottomrule
\end{tabular}

\vspace{0.3em}
At this level, $MB = \$150 > MC = \$25 + \text{DWL}$, so adding more KYC checks is still welfare-improving. But as $r$ increases, MB falls (fewer frauds left to catch) and MC rises (more legitimate users deterred).
\end{columns}

\bottomnote{$W(r) = B(r) - C(r)$; the optimal regulation intensity $r^*$ equates marginal benefit to marginal cost}
\end{frame}

% Frame 6: Chart - Regulatory welfare optimization
\begin{frame}[t]{Regulatory Welfare Optimization}
\begin{center}
\includegraphics[width=0.85\textwidth]{06_regulatory_welfare_optimization/chart.pdf}
\end{center}

\bottomnote{$MB(r) = \frac{100}{1+2r}$, $MC(r) = 10 + 15r$. The inverted-U welfare curve peaks at $r^*$ where MB = MC; over-regulation destroys more value than it creates}
\end{frame}

% Frame 7: Comparative statics of regulation
\begin{frame}[t]{Comparative Statics: What Shifts Optimal Regulation?}

\begin{columns}[T]
\column{0.48\textwidth}
\textbf{Factors Shifting Optimal $r^*$ Higher}

\vspace{0.3em}
\begin{itemize}
\item \textbf{Higher fraud rate:} More benefit from each unit of regulation. When 80\% of ICOs are scams, the MB curve shifts up---more regulation is justified.
\item \textbf{Lower compliance technology cost:} RegTech reduces MC, so the MC curve shifts down, allowing more regulation before costs exceed benefits.
\item \textbf{Larger market:} Fixed regulatory costs spread over more participants, reducing per-unit MC.
\end{itemize}

\column{0.48\textwidth}
\textbf{Factors Shifting Optimal $r^*$ Lower}

\vspace{0.3em}
\begin{itemize}
\item \textbf{Innovation sensitivity:} If regulation deters valuable innovation (high DWL per unit of $r$), the MC curve shifts up, pulling $r^*$ down.
\item \textbf{Regulatory arbitrage:} If firms can easily relocate to less-regulated jurisdictions, the MB curve shifts down (regulation catches fewer bad actors).
\item \textbf{Market maturity:} As crypto markets develop self-regulatory mechanisms, MB from government rules decreases.
\end{itemize}

\vspace{0.3em}
\textbf{Policy Implication:}

Different countries should have different $r^*$ values depending on their market conditions, enforcement capacity, and innovation goals.
\end{columns}

\bottomnote{Comparative statics: any change that raises MB or lowers MC increases optimal regulation intensity $r^*$, and vice versa}
\end{frame}

% Frame 8: Akerlof introduction with mu
\begin{frame}[t]{Information Asymmetry: The Lemons Problem}

\textbf{New symbol:} Let $\mu$ (mu, the Greek letter) represent the average quality in the market.

\vspace{0.3em}
\begin{columns}[T]
\column{0.48\textwidth}
\textbf{Akerlof (1970) Setup}

\vspace{0.3em}
\begin{itemize}
\item Project quality $q$ drawn from uniform distribution: $q \sim U[0,1]$
\item Sellers know true quality; buyers only know the distribution
\item Average quality: $\mu = E[q] = 0.5$
\item Buyers willing to pay $p = \mu$ (expected value)
\end{itemize}

\vspace{0.3em}
\textbf{The Unraveling}
\begin{enumerate}
\item Buyers offer $p = 0.5$ (the average)
\item Sellers with $q > 0.5$ exit (their project is worth more than the offered price)
\item Remaining: $q \sim U[0, 0.5]$, so $E[q | q < 0.5] = 0.25$
\item Buyers revise offer to $p = 0.25$
\item Sellers with $q > 0.25$ exit \ldots
\item Market collapses to only the worst projects
\end{enumerate}

\column{0.48\textwidth}
\textbf{Why This Applies to Crypto}

\vspace{0.3em}
\begin{itemize}
\item Token buyers cannot verify project quality
\item Good projects are indistinguishable from scams
\item The ICO boom of 2017--2018 saw exactly this dynamic
\item Result: only ``lemons'' (low-quality projects) remain
\end{itemize}

\vspace{0.3em}
\textbf{Regulatory Solutions}
\begin{itemize}
\item \textbf{Disclosure requirements:} Force sellers to reveal information, improving signal quality $s$
\item \textbf{Quality floors:} Set minimum quality $q_{\min}$ (licensing, audits)---excludes projects below threshold
\item \textbf{Signaling:} Voluntary compliance as quality signal (e.g., registering with a regulator)
\end{itemize}
\end{columns}

\bottomnote{Akerlof (1970): when buyers cannot observe quality, adverse selection drives good sellers out---only ``lemons'' remain}
\end{frame}

%% ===== SECTION 3: Akerlof Lemons Model in Crypto (4 frames) =====
\section{Akerlof Lemons Model in Crypto}

% Frame 9: ICO market evidence
\begin{frame}[t]{ICO Market as a Lemons Market}

\begin{columns}[T]
\column{0.48\textwidth}
\textbf{Empirical Evidence}

\vspace{0.3em}
\textbf{The 2017--2018 ICO Boom:}
\begin{itemize}
\item Over 5{,}000 ICOs (Initial Coin Offerings---a fundraising method where a project sells tokens to investors) raised \$25B+
\item \textbf{80\% were classified as scams} (Dowlat \& Hodapp, 2018)
\item Average investor return: $-$87\% (Benedetti \& Kostovetsky, 2021)
\item Classic adverse selection: good projects could not credibly distinguish themselves
\end{itemize}

\vspace{0.3em}
\textbf{The Unraveling in Practice}
\begin{enumerate}
\item Early ICOs raised millions with just a whitepaper
\item Scammers flooded in (low cost of entry)
\item Legitimate projects moved to alternative fundraising (VCs, IEOs)
\item ICO market collapsed by late 2018
\end{enumerate}

\column{0.48\textwidth}
\textbf{Mathematical Analysis}

\vspace{0.3em}
With $q \sim U[0,1]$ and no regulation:
\begin{itemize}
\item Round 1: $E[q] = 0.5$, buyers offer $p_1 = 0.5$
\item Sellers with $q > 0.5$ leave (50\% of market exits)
\item Round 2: $E[q | q < 0.5] = 0.25$, offer $p_2 = 0.25$
\item Round 3: $E[q | q < 0.25] = 0.125$, offer $p_3 = 0.125$
\item \ldots converges to $p \to 0$, volume $\to 0$
\end{itemize}

\vspace{0.3em}
\textbf{With Quality Floor $q_{\min}$:}
$$E[q \mid q \geq q_{\min}] = \frac{1 + q_{\min}}{2}$$

If regulators set $q_{\min} = 0.4$ (basic due diligence, code audit):
$$E[q] = \frac{1 + 0.4}{2} = 0.7$$

Market stabilizes at higher quality, but volume drops to $1 - 0.4 = 0.6$ (60\% of projects remain eligible).
\end{columns}

\bottomnote{Dowlat \& Hodapp (2018): 80\% of ICOs were scams; the market for lemons predicts exactly this adverse selection outcome}
\end{frame}

% Frame 10: Chart - Crypto lemons
\begin{frame}[t]{Akerlof Lemons Model: Unraveling and Quality Floors}
\begin{center}
\includegraphics[width=0.85\textwidth]{07_akerlof_crypto_lemons/chart.pdf}
\end{center}

\bottomnote{Panel (a): Without regulation, market unravels as good projects exit. Panel (b): Quality floor $q_{\min}$ raises average quality but reduces volume}
\end{frame}

% Frame 11: Welfare analysis of disclosure regulation
\begin{frame}[t]{Welfare Analysis: Optimal Disclosure Requirements}

\begin{columns}[T]
\column{0.48\textwidth}
\textbf{Disclosure as Signal}

\vspace{0.3em}
Mandatory disclosure (financial audits, team backgrounds, code reviews) improves the signal quality $s$ that buyers observe.

\vspace{0.3em}
\textbf{Effect on Market:}
\begin{itemize}
\item Perfect signal ($s = 1$): buyers observe true $q$, no adverse selection, all projects priced at true value
\item No signal ($s = 0$): full lemons problem, market collapses
\item Partial signal ($0 < s < 1$): some separation of good from bad projects
\end{itemize}

\vspace{0.3em}
\textbf{Welfare from Disclosure:}
$$W(s) = \underbrace{\text{Market quality gain}}_{\text{from separating good/bad}} - \underbrace{\text{Disclosure cost}}_{\text{audits, legal, reporting}}$$

\column{0.48\textwidth}
\textbf{The MiCA Approach}

\vspace{0.3em}
The EU's Markets in Crypto-Assets Regulation (MiCA, enacted 2023) mandates:
\begin{itemize}
\item Whitepaper with standardized disclosures
\item Reserve proof for stablecoins
\item Management fitness tests
\item Ongoing reporting requirements
\end{itemize}

\vspace{0.3em}
\textbf{Trade-off Table}
\begin{tabular}{lcc}
\toprule
\textbf{Metric} & \textbf{No reg.} & \textbf{MiCA} \\
\midrule
Avg quality $\mu$ & 0.15 & 0.65 \\
Volume & 100\% & 60\% \\
Fraud rate & 80\% & $\sim$15\% \\
Entry cost & \$0 & $\sim$\$100K \\
\bottomrule
\end{tabular}

\vspace{0.3em}
MiCA improves quality dramatically but raises barriers for small innovators.
\end{columns}

\bottomnote{Disclosure regulation increases signal quality $s$, reducing adverse selection; MiCA is the most comprehensive crypto disclosure framework}
\end{frame}

% Frame 12: Signaling equilibrium
\begin{frame}[t]{Spence Signaling in Crypto: Voluntary Compliance}

\begin{columns}[T]
\column{0.48\textwidth}
\textbf{Spence (1973) Signaling Model}

\vspace{0.3em}
High-quality projects can signal their type by undertaking costly actions that low-quality projects would find unprofitable:

\begin{itemize}
\item \textbf{Signal cost for high-quality:} $c_H$ (e.g., \$200K for code audit + legal review)
\item \textbf{Signal cost for low-quality:} $c_L > c_H$ (scammers must fabricate evidence, hire actors---higher marginal cost)
\item \textbf{Separating equilibrium:} Only high-quality projects signal if $c_H < p_H - p_L < c_L$
\end{itemize}

\vspace{0.3em}
The signal works because it is differentially costly: legitimate projects already have auditable code and real teams.

\column{0.48\textwidth}
\textbf{Crypto Signaling Mechanisms}

\vspace{0.3em}
\begin{tabular}{lcc}
\toprule
\textbf{Signal} & \textbf{Cost} & \textbf{Effective?} \\
\midrule
Code audit & \$50--200K & Strong \\
VC backing & Equity dilution & Strong \\
Reg.\ registration & \$100K+ & Moderate \\
Bug bounty & Variable & Moderate \\
Token lockup & Opportunity cost & Weak \\
Celebrity endorsement & \$50K+ & Very weak \\
\bottomrule
\end{tabular}

\vspace{0.3em}
\textbf{Key insight:} Effective signals are those where cost is \textit{correlated with quality}. Celebrity endorsements fail because scams can afford them equally easily.

\vspace{0.3em}
Regulation can amplify signaling by certifying specific signals (e.g., registered auditor requirement).
\end{columns}

\bottomnote{Spence (1973): costly signaling can solve adverse selection when signal cost is negatively correlated with quality}
\end{frame}

%% ===== SECTION 4: Regulatory Sandbox Evaluation (5 frames) =====
\section{Regulatory Sandbox Evaluation}

% Frame 13: Sandbox concept and metrics
\begin{frame}[t]{Regulatory Sandboxes: Concept and Evaluation Framework}

\begin{columns}[T]
\column{0.48\textwidth}
\textbf{What is a Regulatory Sandbox?}

\vspace{0.3em}
A sandbox is a controlled testing environment where fintech firms can operate under relaxed regulations for a limited period. The regulator monitors outcomes to learn whether new rules are needed.

\vspace{0.3em}
\textbf{Key Parameters}
\begin{itemize}
\item Cohort size: number of firms admitted per round
\item Duration: typically 6--24 months
\item Restrictions: customer limits, capital requirements
\item Exit: graduate to full license, extend, or exit
\end{itemize}

\vspace{0.3em}
\textbf{Economic Rationale}
\begin{itemize}
\item Reduces regulatory uncertainty (lowers barrier to innovation)
\item Provides regulators with real-world data
\item Enables ``learning by doing'' for both sides
\end{itemize}

\column{0.48\textwidth}
\textbf{Evaluation Metrics}

\vspace{0.3em}
\begin{tabular}{ll}
\toprule
\textbf{Metric} & \textbf{Definition} \\
\midrule
Graduation rate & \% of firms receiving \\
 & full license \\
Cost per firm & Regulator spend / firms \\
Efficiency & Graduation / cost \\
Innovation index & New products launched \\
Consumer harm & Complaints per firm \\
\bottomrule
\end{tabular}

\vspace{0.3em}
\textbf{Global Sandbox Landscape}
\begin{itemize}
\item First: UK FCA (2016)
\item Over 50 jurisdictions by 2024
\item Crypto-focused: Abu Dhabi ADGM, MAS, HK SFC
\item Variant: ``regulatory beach'' (lighter, e.g., El Salvador)
\end{itemize}
\end{columns}

\bottomnote{Regulatory sandboxes reduce uncertainty for innovators while generating evidence for policymakers}
\end{frame}

% Frame 14: Chart - Sandbox performance scatter
\begin{frame}[t]{Sandbox Performance: Graduation vs. Cost}
\begin{center}
\includegraphics[width=0.85\textwidth]{08_sandbox_performance_scatter/chart.pdf}
\end{center}

\bottomnote{Data from FCA (2023), Cornelli et al.\ (2020). Efficiency = graduation rate / cost per firm. Bubble size = cohort size}
\end{frame}

% Frame 15: Sandbox design trade-offs
\begin{frame}[t]{Sandbox Design: Economic Trade-offs}

\begin{columns}[T]
\column{0.48\textwidth}
\textbf{Strictness vs. Innovation}

\vspace{0.3em}
\begin{itemize}
\item \textbf{Strict sandbox} (Abu Dhabi ADGM): high entry bar (\$3M cost/firm), fewer firms, but higher quality graduates
\item \textbf{Open sandbox} (Australia ASIC): low bar (\$0.8M), more firms, but lower graduation rate
\item \textbf{Balanced} (UK FCA): moderate cost (\$2M), largest cohort (50 firms), highest graduation (80\%)
\end{itemize}

\vspace{0.3em}
\textbf{The Screening Function}

A sandbox serves as a screening device (a mechanism that reveals firm quality through self-selection): firms that apply and succeed are more likely to be high-quality. This is valuable information for regulators even if the firm does not graduate.

\column{0.48\textwidth}
\textbf{Lessons from Global Experience}

\vspace{0.3em}
\begin{enumerate}
\item \textbf{Scale matters:} Fixed regulatory costs mean larger cohorts are more efficient per firm
\item \textbf{Clear exit criteria:} Firms need to know what ``graduation'' requires upfront
\item \textbf{Mutual learning:} The regulator must actually update rules based on sandbox evidence (many do not)
\item \textbf{Consumer protection:} Sandbox users must be informed they are testing unproven products
\end{enumerate}

\vspace{0.3em}
\textbf{Success Factors} (Cornelli et al., 2020):
\begin{itemize}
\item Dedicated sandbox team at regulator
\item Public reporting of outcomes
\item Cross-border recognition (graduates accepted in other jurisdictions)
\end{itemize}
\end{columns}

\bottomnote{Sandbox efficiency depends on cohort size, exit clarity, and whether regulators update rules based on evidence}
\end{frame}

% Frame 16: Sandbox cost-benefit model
\begin{frame}[t]{Cost-Benefit Model for Sandbox Programs}

\begin{columns}[T]
\column{0.48\textwidth}
\textbf{Regulator's Decision}

\vspace{0.3em}
Should a jurisdiction establish a sandbox? Compare:

\vspace{0.3em}
\textbf{Benefits:}
\begin{itemize}
\item Innovation attracted: firms that would otherwise go elsewhere (estimated at 15--30 per sandbox cohort)
\item Regulatory learning: reduces future policy mistakes
\item Revenue: licensing fees from graduates
\item Reputation: signals ``innovation-friendly'' jurisdiction
\end{itemize}

\vspace{0.3em}
\textbf{Costs:}
\begin{itemize}
\item Setup: \$2--5M (technology, staffing, legal framework)
\item Per-cohort operating: \$0.5--1M
\item Risk: consumer harm from sandbox failures
\item Opportunity cost: regulatory staff diverted
\end{itemize}

\column{0.48\textwidth}
\textbf{Break-even Calculation}

\vspace{0.3em}
For UK FCA sandbox (per cohort of 50):
\begin{itemize}
\item Setup cost (amortized): \$1M
\item Operating cost: \$1M
\item Total cost: \$2M
\item Graduation: 40 firms $\times$ \$50K license fee = \$2M
\item Tax revenue from graduates: \$5M+ annually
\item Innovation premium: attracts 200+ fintech firms to London
\end{itemize}

\vspace{0.3em}
\textbf{Net present value:} Strongly positive when accounting for ecosystem effects.

\vspace{0.3em}
\textbf{Key insight:} The sandbox pays for itself through licensing fees alone; innovation and ecosystem effects are bonus welfare gains. Jurisdictions without sandboxes lose firms to those that have them---a coordination game.
\end{columns}

\bottomnote{Sandbox programs are net positive for most jurisdictions; the main risk is doing them poorly (no learning, weak exit criteria)}
\end{frame}

% Frame 17: FATF introduction with delta_g
\begin{frame}[t]{International Regulatory Coordination: The FATF Game}

\textbf{New symbol:} Let $\delta_g$ (delta-g, a Greek letter) represent the discount factor (how much a country values future payoffs relative to today; $\delta_g = 0.9$ means future periods are worth 90\% of the present).

\vspace{0.3em}
\begin{columns}[T]
\column{0.48\textwidth}
\textbf{The Coordination Problem}

\vspace{0.3em}
The FATF (Financial Action Task Force---the international body setting AML/CFT standards) faces a classic coordination game:
\begin{itemize}
\item Each country chooses: \textbf{Comply} (C) or \textbf{Defect} (D)
\item Compliance is costly but creates mutual benefits
\item Defection is individually tempting (save costs, attract business) but collectively harmful
\end{itemize}

\vspace{0.3em}
\textbf{Stage Game Payoffs}

\begin{tabular}{lcc}
\toprule
& \textbf{Other: C} & \textbf{Other: D} \\
\midrule
\textbf{Country: C} & (5, 5) & (3, 7) \\
\textbf{Country: D} & (7, 3) & (2, 2) \\
\bottomrule
\end{tabular}

\column{0.48\textwidth}
\textbf{Repeated Game Analysis}

\vspace{0.3em}
In a one-shot game, defection dominates (7 > 5). But FATF evaluations repeat every 5--8 years, creating a repeated game.

\vspace{0.3em}
\textbf{Trigger Strategy:}
\begin{itemize}
\item Start by cooperating
\item If the other country defects, switch to defection forever (punishment)
\item Cooperation sustained if $\delta_g$ is high enough
\end{itemize}

\vspace{0.3em}
\textbf{Cooperation condition:}
$$\delta_g > \frac{7 - 5}{7 - 2} = \frac{2}{5} = 0.4$$

Since most countries have $\delta_g > 0.8$ (they care about long-term reputation), cooperation is sustainable---but only with credible punishment (greylisting).
\end{columns}

\bottomnote{FATF coordination is a repeated prisoner's dilemma; cooperation requires $\delta_g > 0.4$ and credible punishment via greylisting}
\end{frame}

%% ===== SECTION 5: FATF Coordination Game Theory (4 frames) =====
\section{FATF Coordination Game Theory}

% Frame 18: Detection probability and expected payoffs
\begin{frame}[t]{FATF Detection and Expected Payoffs}

\begin{columns}[T]
\column{0.48\textwidth}
\textbf{Detection Probability}

\vspace{0.3em}
FATF mutual evaluations detect non-compliance with probability $p = 0.8$ (high detection rate due to peer review, on-site visits, and quantitative benchmarks).

\vspace{0.3em}
\textbf{Payoff Structure with Detection:}
\begin{itemize}
\item \textbf{Comply:} Payoff = 5 (certain)
\item \textbf{Defect, caught} ($p = 0.8$): Payoff = 1 (greylisted: capital flight, restricted banking access)
\item \textbf{Defect, escaped} ($1-p = 0.2$): Payoff = 7 (save compliance costs, attract illicit finance)
\end{itemize}

\vspace{0.3em}
\textbf{Expected payoff from defection:}
$$E[\text{Defect}] = 0.2 \times 7 + 0.8 \times 1 = 2.2$$

Since $2.2 < 5$, \textbf{compliance is rational even in a one-shot game} when detection is sufficiently high.

\column{0.48\textwidth}
\textbf{Critical Detection Threshold}

\vspace{0.3em}
Defection is rational only when:
$$E[\text{Defect}] > E[\text{Comply}]$$
$$(1-p) \times 7 + p \times 1 > 5$$
$$7 - 6p > 5$$
$$p < \frac{1}{3} \approx 0.33$$

So compliance breaks down only if detection falls below 33\%. FATF's current $p = 0.8$ provides a large safety margin.

\vspace{0.3em}
\textbf{Greylisting in Practice (2024):}
\begin{itemize}
\item 21 countries currently greylisted
\item Average GDP growth penalty: $-$0.4\% (Masciandaro, 2022)
\item Average time on greylist: 3--5 years
\item 85\% eventually achieve compliance
\end{itemize}
\end{columns}

\bottomnote{With detection $p = 0.8$: $E[\text{Defect}] = 2.2 < 5 = E[\text{Comply}]$; compliance is individually rational, not just collectively optimal}
\end{frame}

% Frame 19: Chart - FATF game tree
\begin{frame}[t]{FATF Coordination Game: Decision Tree and Payoffs}
\begin{center}
\includegraphics[width=0.85\textwidth]{09_fatf_greylisting_game_tree/chart.pdf}
\end{center}

\bottomnote{Panel (a): Game tree with detection $p = 0.8$. Panel (b): Expected payoff comparison; cooperation sustained when $\delta_g > 0.4$}
\end{frame}

% Frame 20: Cross-jurisdictional coordination
\begin{frame}[t]{Cross-Jurisdictional Regulatory Coordination}

\begin{columns}[T]
\column{0.48\textwidth}
\textbf{The Regulatory Race}

\vspace{0.3em}
Without coordination, jurisdictions face two races:

\vspace{0.3em}
\textbf{Race to the bottom:}
\begin{itemize}
\item Compete for crypto business by lowering standards
\item Attracts illicit finance alongside legitimate firms
\item Increases systemic risk globally
\item Example: Some Caribbean jurisdictions offered ``crypto licenses'' with minimal oversight
\end{itemize}

\vspace{0.3em}
\textbf{Race to the top:}
\begin{itemize}
\item Compete for legitimacy by raising standards
\item Attracts institutional investors
\item May deter early-stage innovation
\item Example: Swiss FINMA, Singapore MAS
\end{itemize}

\column{0.48\textwidth}
\textbf{FATF as Coordination Mechanism}

\vspace{0.3em}
FATF resolves the coordination failure by:
\begin{enumerate}
\item \textbf{Setting minimum standards:} 40 Recommendations create a floor
\item \textbf{Monitoring compliance:} Mutual evaluations every 5--8 years
\item \textbf{Punishing defection:} Greylist/blacklist with real economic consequences
\item \textbf{Rewarding compliance:} Reputation effects attract legitimate business
\end{enumerate}

\vspace{0.3em}
\textbf{Crypto-Specific Challenge:}

The ``Travel Rule'' (FATF Recommendation 16) requires VASPs (Virtual Asset Service Providers---crypto exchanges, wallets, custodians) to share sender/receiver information. This is difficult for decentralized protocols where there is no intermediary to collect information.

\vspace{0.3em}
\textit{For multi-criteria policy scoring across jurisdictions, see L08 Extended.}
\end{columns}

\bottomnote{FATF resolves the international coordination game through monitoring, minimum standards, and credible punishment (greylisting)}
\end{frame}

% Frame 21: Nash equilibrium analysis
\begin{frame}[t]{Nash Equilibrium Analysis of Regulatory Competition}

\begin{columns}[T]
\column{0.48\textwidth}
\textbf{Three Possible Equilibria}

\vspace{0.3em}
\textbf{1. Race to the Bottom (Bad Equilibrium)}
\begin{itemize}
\item All jurisdictions defect (D, D)
\item Payoff: (2, 2) for each
\item Occurs when $\delta_g < 0.4$ or punishment not credible
\item Real-world analog: pre-2019 crypto regulation
\end{itemize}

\vspace{0.3em}
\textbf{2. Coordination (Good Equilibrium)}
\begin{itemize}
\item All comply (C, C)
\item Payoff: (5, 5) for each
\item Sustained by repeated game + FATF monitoring
\item Real-world analog: post-2020 convergence on Travel Rule
\end{itemize}

\column{0.48\textwidth}
\textbf{3. Asymmetric (Mixed)}
\begin{itemize}
\item Some comply, some defect
\item Defectors attract regulatory arbitrage
\item Compliers bear higher costs without full benefits
\item Real-world analog: current state (MiCA vs.\ unregulated jurisdictions)
\end{itemize}

\vspace{0.3em}
\textbf{Moving Toward Coordination}

\vspace{0.3em}
\begin{tabular}{lc}
\toprule
\textbf{Mechanism} & \textbf{Effect} \\
\midrule
FATF greylisting & Raises cost of D \\
Trade agreements & Links compliance to trade \\
Technology (RegTech) & Lowers cost of C \\
Information sharing & Raises detection $p$ \\
\bottomrule
\end{tabular}

\vspace{0.3em}
All four mechanisms push $r^*$ toward more coordination by either raising the cost of defection or lowering the cost of compliance.
\end{columns}

\bottomnote{The good equilibrium (C, C) requires credible punishment, high detection, and low compliance costs---all of which FATF and RegTech improve}
\end{frame}

%% ===== SECTION 6: RegTech and Compliance Automation (4 frames) =====
\section{RegTech and Compliance Automation}

% Frame 22: RegTech cost functions
\begin{frame}[t]{RegTech: Compliance Cost Functions}

\begin{columns}[T]
\column{0.48\textwidth}
\textbf{Three Compliance Approaches}

\vspace{0.3em}
Each approach has a different cost structure (fixed setup + variable per-transaction cost):

\vspace{0.3em}
\begin{tabular}{lcc}
\toprule
\textbf{Approach} & \textbf{Fixed} & \textbf{Variable} \\
\midrule
Manual & \$500K & \$5/txn \\
Basic RegTech & \$1.5M & \$1/txn \\
AI RegTech & \$3M & \$0.20/txn \\
\bottomrule
\end{tabular}

\vspace{0.3em}
\textbf{Cost Functions:}
\begin{itemize}
\item $C_{\text{manual}}(V) = 500{,}000 + 5V$
\item $C_{\text{basic}}(V) = 1{,}500{,}000 + V$
\item $C_{\text{AI}}(V) = 3{,}000{,}000 + 0.2V$
\end{itemize}

where $V$ = number of transactions to screen per year.

\column{0.48\textwidth}
\textbf{Crossover Points}

\vspace{0.3em}
\textbf{Manual = Basic:}
$$500{,}000 + 5V = 1{,}500{,}000 + V$$
$$4V = 1{,}000{,}000 \implies V_1 = 250{,}000$$

\textbf{Basic = AI:}
$$1{,}500{,}000 + V = 3{,}000{,}000 + 0.2V$$
$$0.8V = 1{,}500{,}000 \implies V_2 = 1{,}875{,}000$$

\vspace{0.3em}
\textbf{Optimal Choice:}
\begin{itemize}
\item $V < 250{,}000$: manual cheapest (small exchanges, startups)
\item $250K < V < 1.875M$: basic RegTech (mid-size firms)
\item $V > 1.875M$: AI RegTech (large exchanges processing millions of transactions)
\end{itemize}

This explains why Binance (\$50B+ daily volume) invests heavily in AI compliance while small exchanges use spreadsheets.
\end{columns}

\bottomnote{Technology choice depends on transaction volume; AI RegTech only breaks even above $V_2 = 1.875M$ transactions}
\end{frame}

% Frame 23: Chart - RegTech cost analysis
\begin{frame}[t]{RegTech Cost Analysis: Technology Choice Optimization}
\begin{center}
\includegraphics[width=0.85\textwidth]{10_regtech_cost_analysis/chart.pdf}
\end{center}

\bottomnote{Panel (a): Cost curves cross at $V_1 = 250K$ and $V_2 = 1.875M$. Panel (b): Savings vs.\ manual process increase dramatically with volume}
\end{frame}

% Frame 24: Chart - Regulatory framework comparison
\begin{frame}[t]{Regulatory Framework Comparison: Multi-Criteria Analysis}
\begin{center}
\includegraphics[width=0.85\textwidth]{11_regulatory_framework_comparison/chart.pdf}
\end{center}

\bottomnote{MCDA scores: MiCA [7,8,9,7,5], SEC [4,7,5,9,4], MAS [9,6,7,6,7], None [10,1,1,1,10]. Winner depends on stakeholder weights}
\end{frame}

% Frame 25: RegTech innovation and future
\begin{frame}[t]{RegTech Innovation: From Rules to Algorithms}

\begin{columns}[T]
\column{0.48\textwidth}
\textbf{RegTech Categories}

\vspace{0.3em}
\begin{itemize}
\item \textbf{Identity verification:} KYC/AML automation (e.g., Chainalysis, Elliptic)---tracing blockchain transactions to identify suspicious patterns
\item \textbf{Transaction monitoring:} Real-time screening of transactions against sanctions lists and risk models
\item \textbf{Reporting:} Automated regulatory filings (e.g., suspicious activity reports)
\item \textbf{Risk assessment:} AI-based risk scoring of customers, transactions, counterparties
\end{itemize}

\vspace{0.3em}
\textbf{Market Size:} RegTech market projected at \$55B by 2028 (Grand View Research, 2024), up from \$12B in 2023.

\column{0.48\textwidth}
\textbf{Impact on Regulatory Design}

\vspace{0.3em}
RegTech changes the optimal regulatory equilibrium:

\begin{enumerate}
\item \textbf{Lowers MC curve:} Cheaper compliance $\to$ more regulation is welfare-improving
\item \textbf{Raises detection $p$:} Better monitoring $\to$ stronger deterrence in FATF game
\item \textbf{Reduces DWL:} Automated compliance has lower per-transaction friction
\item \textbf{Enables proportionality:} Risk-based approaches possible at scale
\end{enumerate}

\vspace{0.3em}
\textbf{The ``Embedded Regulation'' Vision:}

Future where regulatory compliance is built into the technology itself. Smart contracts that automatically enforce rules, blockchain analytics that flag suspicious activity in real-time, and programmable money that cannot be used for sanctioned purposes.
\end{columns}

\bottomnote{RegTech lowers the MC of compliance, shifting optimal regulation intensity $r^*$ higher and strengthening coordination incentives}
\end{frame}

%% ===== SECTION 7: Synthesis and Policy Implications (3 frames) =====
\section{Synthesis and Policy Implications}

% Frame 26: Integrated framework
\begin{frame}[t]{Integrated Framework: Connecting All Models}

\begin{columns}[T]
\column{0.48\textwidth}
\textbf{How the Models Fit Together}

\vspace{0.3em}
\begin{enumerate}
\item \textbf{Harberger} answers: How much regulation? ($r^*$ where MB = MC)
\item \textbf{Akerlof} answers: Why regulate? (adverse selection destroys markets without quality floors)
\item \textbf{Sandbox} answers: How to regulate uncertain markets? (test before scaling)
\item \textbf{FATF game} answers: How to coordinate across borders? (repeated game + punishment)
\item \textbf{RegTech} answers: How to reduce compliance costs? (technology lowers MC)
\item \textbf{MCDA} answers: How to compare frameworks? (weight criteria by stakeholder)
\end{enumerate}

\column{0.48\textwidth}
\textbf{The Regulatory Design Loop}

\vspace{0.3em}
\begin{enumerate}
\item Identify market failure (Akerlof: adverse selection)
\item Design intervention (quality floor, disclosure)
\item Calculate optimal intensity (Harberger: $r^*$)
\item Test in sandbox (measure efficiency)
\item Coordinate internationally (FATF game)
\item Reduce costs with technology (RegTech)
\item Evaluate and compare (MCDA)
\item Return to step 1 with updated data
\end{enumerate}

\vspace{0.3em}
\textbf{Key principle:} Regulation should be evidence-based and iterative, not static. The sandbox model embodies this: test, learn, adjust.
\end{columns}

\bottomnote{Optimal regulation combines: right intensity (Harberger), right target (Akerlof), right testing (sandbox), right coordination (FATF), right technology (RegTech)}
\end{frame}

% Frame 27: Policy recommendations
\begin{frame}[t]{Policy Recommendations from the Models}

\begin{columns}[T]
\column{0.48\textwidth}
\textbf{For Regulators}

\vspace{0.3em}
\begin{enumerate}
\item \textbf{Set quality floors, not bans:} Akerlof shows that minimum standards (licensing, audits) preserve market function while reducing fraud
\item \textbf{Invest in RegTech:} Every dollar spent on compliance technology lowers the MC curve, enabling more effective regulation at lower cost
\item \textbf{Use sandboxes for uncertainty:} When the right rule is unknown, test before mandating
\item \textbf{Coordinate via FATF:} The repeated game structure means greylisting works---but only if detection remains high
\end{enumerate}

\column{0.48\textwidth}
\textbf{For Industry}

\vspace{0.3em}
\begin{enumerate}
\item \textbf{Signal quality early:} Code audits, regulatory registration, and VC backing are credible signals that differentiate from scams
\item \textbf{Invest in compliance at the right scale:} Use the RegTech crossover analysis to choose the right technology tier
\item \textbf{Engage with sandboxes:} Sandbox participation is both a learning opportunity and a quality signal
\item \textbf{Support coordination:} Industry benefits from FATF compliance through reduced uncertainty and access to institutional capital
\end{enumerate}
\end{columns}

\bottomnote{Good regulation is an investment that generates returns through market confidence, fraud reduction, and institutional participation}
\end{frame}

% Frame 28: Key takeaways
\begin{frame}[t]{Key Takeaways}

\begin{columns}[T]
\column{0.48\textwidth}
\textbf{Core Results}
\begin{enumerate}
\item Optimal regulation intensity $r^*$ balances marginal benefit against marginal cost (Harberger)
\item Without quality standards, adverse selection drives good projects out (Akerlof)
\item Sandboxes are net positive when designed with clear exit criteria and learning mechanisms
\item FATF coordination works because detection $p = 0.8$ makes defection irrational ($E[\text{D}] = 2.2 < 5$)
\item RegTech shifts the optimal equilibrium toward more regulation at lower cost
\end{enumerate}

\column{0.48\textwidth}
\textbf{Quantitative Results}
\begin{itemize}
\item DWL grows with $\tau^2$: doubling regulatory burden quadruples deadweight loss
\item Quality floor $q_{\min} = 0.4$ raises avg quality from 0.15 to 0.70
\item FATF cooperation threshold: $\delta_g > 0.4$
\item RegTech crossovers: $V_1 = 250K$, $V_2 = 1.875M$
\item MiCA scores highest for consumer and regulator stakeholders; MAS for startups
\end{itemize}

\vspace{0.3em}
\textbf{Next Lecture:} L08 Extended will apply multi-criteria decision analysis to broader digital finance policy questions.
\end{columns}

\bottomnote{Regulation is a design problem with an optimal solution; mathematical models help find it rather than relying on ideology}
\end{frame}

%% ===== SECTION 8: Appendix (3 frames) =====
\section{Appendix}

% Frame 29: Key terms
\begin{frame}[t]{Key Terms}

\begin{columns}[T]
\column{0.48\textwidth}
\textbf{Deadweight Loss (DWL)}
Welfare destroyed by market distortions---value that nobody captures; the Harberger triangle.

\vspace{0.3em}
\textbf{Adverse Selection}
When uninformed buyers cannot distinguish quality, leading high-quality sellers to exit the market.

\vspace{0.3em}
\textbf{Lemons Problem}
Akerlof's (1970) model showing how information asymmetry causes market unraveling.

\vspace{0.3em}
\textbf{Regulatory Sandbox}
A controlled testing environment where fintech firms operate under relaxed rules for a limited period.

\vspace{0.3em}
\textbf{FATF}
Financial Action Task Force---the international body setting AML/CFT (Anti-Money Laundering / Countering the Financing of Terrorism) standards.

\vspace{0.3em}
\textbf{Greylisting}
FATF's punishment for non-compliant countries: increased monitoring, restricted banking access, capital flight.

\column{0.48\textwidth}
\textbf{RegTech}
Regulatory technology---software that automates compliance processes like KYC, transaction monitoring, and reporting.

\vspace{0.3em}
\textbf{Trigger Strategy}
In a repeated game, cooperate until the opponent defects, then punish forever; sustains cooperation when $\delta_g$ is high enough.

\vspace{0.3em}
\textbf{Quality Floor}
A minimum quality standard ($q_{\min}$) imposed by regulation; prevents the worst projects from entering the market.

\vspace{0.3em}
\textbf{MCDA}
Multi-Criteria Decision Analysis---a method for comparing options across multiple weighted dimensions.

\vspace{0.3em}
\textbf{MiCA}
Markets in Crypto-Assets Regulation---the EU's comprehensive framework for crypto regulation (enacted 2023).

\vspace{0.3em}
\textbf{VASP}
Virtual Asset Service Provider---a crypto exchange, wallet, or custodian subject to FATF Travel Rule requirements.
\end{columns}

\bottomnote{These terms form the vocabulary for regulatory economics in digital finance}
\end{frame}

% Frame 30: Mathematical summary
\begin{frame}[t]{Mathematical Summary}

\begin{columns}[T]
\column{0.48\textwidth}
\textbf{Welfare Optimization}
$$W(r) = \int_0^r [MB(s) - MC(s)]\,ds$$
$$MB(r) = \frac{a}{1+br}, \quad MC(r) = c + dr$$
$$r^* \text{ where } MB(r^*) = MC(r^*)$$

\vspace{0.3em}
\textbf{Deadweight Loss}
$$\text{DWL} = \frac{1}{2} \cdot \Delta p \cdot \Delta q \propto \tau^2$$

\vspace{0.3em}
\textbf{Akerlof Unraveling}
$$q \sim U[0,1], \quad E[q | q < p] = \frac{p}{2}$$
$$\text{With floor } q_{\min}: \quad E[q] = \frac{1 + q_{\min}}{2}$$

\vspace{0.3em}
\textbf{RegTech Crossovers}
$$V_1 = \frac{F_{\text{basic}} - F_{\text{manual}}}{m_{\text{manual}} - m_{\text{basic}}} = 250{,}000$$
$$V_2 = \frac{F_{\text{AI}} - F_{\text{basic}}}{m_{\text{basic}} - m_{\text{AI}}} = 1{,}875{,}000$$

\column{0.48\textwidth}
\textbf{FATF Game}
\begin{tabular}{lcc}
\toprule
& \textbf{C} & \textbf{D} \\
\midrule
\textbf{C} & (5, 5) & (3, 7) \\
\textbf{D} & (7, 3) & (2, 2) \\
\bottomrule
\end{tabular}

\vspace{0.3em}
$$E[\text{Defect}] = (1-p) \times 7 + p \times 1$$
$$= 0.2 \times 7 + 0.8 \times 1 = 2.2 < 5$$

\vspace{0.3em}
\textbf{Cooperation Threshold:}
$$\delta_g > \frac{7 - 5}{7 - 2} = 0.4$$

\vspace{0.3em}
\textbf{Detection Threshold:}
$$p > \frac{7 - 5}{7 - 1} = \frac{1}{3} \approx 0.33$$

\vspace{0.3em}
\textbf{Sandbox Efficiency:}
$$\text{Efficiency}_i = \frac{\text{Graduation rate}_i}{\text{Cost per firm}_i}$$
\end{columns}

\bottomnote{All formulas use standard welfare economics, game theory, and decision analysis---no specialized mathematical background required}
\end{frame}

% Frame 31: Further reading
\begin{frame}[t]{Further Reading}

\begin{columns}[T]
\column{0.48\textwidth}
\textbf{Foundational Papers}
\begin{itemize}
\item Akerlof (1970): ``The Market for `Lemons' ''
\item Harberger (1964): ``Taxation, Resource Allocation, and Welfare''
\item Posner (1974): ``Theories of Economic Regulation''
\item Spence (1973): ``Job Market Signaling''
\item Stigler (1971): ``The Theory of Economic Regulation''
\end{itemize}

\vspace{0.3em}
\textbf{Crypto Regulation}
\begin{itemize}
\item Auer, Cornelli \& Frost (2020): ``Rise of the Central Bank Digital Currencies''
\item Cornelli et al.\ (2020): ``Fintech and Big Tech Credit''
\item Dowlat \& Hodapp (2018): ``ICO Quality: Development \& Trading''
\end{itemize}

\column{0.48\textwidth}
\textbf{Regulatory Policy}
\begin{itemize}
\item FSB (2023): ``High-level Recommendations for Crypto-Asset Regulation''
\item FATF (2023): ``Updated Guidance for Virtual Assets and VASPs''
\item FCA (2023): ``Regulatory Sandbox Annual Report''
\item Masciandaro (2022): ``The Governance of Financial Supervision''
\end{itemize}

\vspace{0.3em}
\textbf{RegTech}
\begin{itemize}
\item Deloitte (2023): ``RegTech Universe Report''
\item Arner, Barberis \& Buckley (2017): ``FinTech, RegTech, and the Reconceptualization of Financial Regulation''
\item Grand View Research (2024): ``RegTech Market Size Report''
\end{itemize}
\end{columns}

\bottomnote{All readings available on course platform; foundational papers are essential for understanding regulatory economics}
\end{frame}

\end{document}
