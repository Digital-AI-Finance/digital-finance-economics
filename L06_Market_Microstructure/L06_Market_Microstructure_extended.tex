\documentclass[8pt,aspectratio=169]{beamer}
\usetheme{Madrid}
\usepackage{graphicx}
\usepackage{booktabs}
\usepackage{adjustbox}
\usepackage{multicol}
\usepackage{amsmath}

\definecolor{mlblue}{RGB}{0,102,204}
\definecolor{mlpurple}{RGB}{51,51,178}
\definecolor{mllavender}{RGB}{173,173,224}
\definecolor{mllavender2}{RGB}{193,193,232}
\definecolor{mllavender3}{RGB}{204,204,235}
\definecolor{mllavender4}{RGB}{214,214,239}
\definecolor{mlorange}{RGB}{255, 127, 14}
\definecolor{mlgreen}{RGB}{44, 160, 44}
\definecolor{mlred}{RGB}{214, 39, 40}
\definecolor{mlgray}{RGB}{127, 127, 127}
\definecolor{lightgray}{RGB}{240, 240, 240}
\definecolor{midgray}{RGB}{180, 180, 180}

\setbeamercolor{palette primary}{bg=mllavender3,fg=mlpurple}
\setbeamercolor{palette secondary}{bg=mllavender2,fg=mlpurple}
\setbeamercolor{palette tertiary}{bg=mllavender,fg=white}
\setbeamercolor{palette quaternary}{bg=mlpurple,fg=white}
\setbeamercolor{structure}{fg=mlpurple}
\setbeamercolor{section in toc}{fg=mlpurple}
\setbeamercolor{subsection in toc}{fg=mlblue}
\setbeamercolor{title}{fg=mlpurple}
\setbeamercolor{frametitle}{fg=mlpurple,bg=mllavender3}
\setbeamercolor{block title}{bg=mllavender2,fg=mlpurple}
\setbeamercolor{block body}{bg=mllavender4,fg=black}
\setbeamertemplate{navigation symbols}{}
\setbeamertemplate{itemize items}[circle]
\setbeamertemplate{enumerate items}[default]
\setbeamersize{text margin left=5mm,text margin right=5mm}

\newcommand{\bottomnote}[1]{%
\vfill
\vspace{-2mm}
\textcolor{mllavender2}{\rule{\textwidth}{0.4pt}}
\vspace{1mm}
\footnotesize
\textbf{#1}
}

\title{Market Microstructure: Mathematical Models and DeFi Analysis}
\subtitle{L06 Extended: Formalizing AMM Design, Price Discovery, and LP Profitability\\[0.3em]\normalsize From Kyle's lambda to concentrated liquidity and MEV game theory}
\author{Economics of Digital Finance}
\institute{BSc Course}
\date{}

\begin{document}

%% ===== SECTION 1: Bridge from Basic Lecture (4 frames: 1 title + 3) =====
\section{Bridge from Basic Lecture}

% Frame 1 (title)
\begin{frame}[plain]
\titlepage
\end{frame}

% Frame 2
\begin{frame}[t]{L06 Extended: What This Lecture Adds}

\textbf{In the basic L06 lecture} you learned the constant product formula $x \cdot y = k$, impermanent loss (IL), and MEV (Maximal Extractable Value---profit from reordering transactions). This extended lecture formalizes those ideas mathematically and introduces new models.

\vspace{0.3em}
\begin{columns}[T]
\column{0.48\textwidth}
\textbf{New Mathematical Tools}
\begin{enumerate}
\item Kyle's price impact model (quantifies adverse selection)
\item Concentrated liquidity mechanics (Uniswap v3)
\item StableSwap invariant (Curve Finance)
\item LP profitability frontier (fee vs IL trade-off)
\item MEV game theory (sandwich attack optimization)
\item DEX vs CEX multi-dimensional comparison
\end{enumerate}

\column{0.48\textwidth}
\textbf{Why Formalize?}
\begin{itemize}
\item Qualitative intuition from L06 basic is necessary but not sufficient for design or policy
\item Mathematical models let us predict outcomes: ``If volatility doubles, do LPs still profit?''
\item DeFi protocols are themselves mathematical---understanding the math lets you audit the code
\end{itemize}

\vspace{0.3em}
\textbf{Prerequisites}
\begin{itemize}
\item L06 basic lecture (AMM, IL, MEV definitions)
\item Algebra and basic calculus (derivatives)
\item No prior knowledge of market microstructure theory required
\end{itemize}
\end{columns}

\bottomnote{This lecture bridges qualitative understanding from L06 basic with formal models used in DeFi research}
\end{frame}

% Frame 3 (NO GREEK)
\begin{frame}[t]{Notation and Variable Reference}

\textbf{We use plain-letter notation throughout this lecture.} Greek letters are introduced only where standard in the literature and always with explicit definitions.

\vspace{0.3em}
\begin{columns}[T]
\column{0.48\textwidth}
\textbf{AMM Variables}
\begin{itemize}
\item $x$, $y$: token reserves in a liquidity pool
\item $k$: constant product invariant ($x \cdot y = k$)
\item $P = y/x$: spot price (units of $y$ per unit of $x$)
\item $L$: liquidity (geometric mean of reserves, $L = \sqrt{k}$)
\item $r$: price ratio $P_{\text{final}}/P_{\text{initial}}$
\item $f$: fee rate (e.g., 0.003 for 0.3\%)
\item $IL$: impermanent loss $= 2\sqrt{r}/(1+r) - 1$
\end{itemize}

\column{0.48\textwidth}
\textbf{Trading Variables}
\begin{itemize}
\item $Q$: trade quantity (number of tokens)
\item $Q_f$: front-runner's trade size in a sandwich attack
\item $fee$: total fee paid on a trade ($= f \times Q \times P$)
\item $gas$: blockchain transaction cost (in USD)
\item $V$: trading volume; $TVL$: total value locked in pool
\end{itemize}

\vspace{0.3em}
\textbf{Efficiency Variables}
\begin{itemize}
\item $IS$: information share (venue's contribution to price discovery)
\item $A$: amplification factor (Curve's StableSwap parameter)
\item $D$: StableSwap invariant parameter
\end{itemize}
\end{columns}

\bottomnote{All variables use plain letters; Greek symbols are introduced with definitions when they appear in Frames 4+ onward}
\end{frame}

%% ===== SECTION 2: Kyle Model Extensions for Crypto (4 frames) =====
\section{Kyle Model Extensions for Crypto}

% Frame 4 (introduce Greek lambda)
\begin{frame}[t]{Kyle's Lambda: Formalizing Price Impact}
\begin{columns}[T]
\column{0.48\textwidth}
\textbf{The Kyle (1985) Model}

Kyle introduced $\lambda$ (``lambda''), which measures how much prices move per unit traded. $\lambda$ is the \textit{price impact coefficient}---the core concept in market microstructure.

$$\lambda = \frac{\sigma_v}{2\sigma_u}$$

where:
\begin{itemize}
\item $\sigma_v$ = standard deviation of the asset's true value changes (how much private information moves the price)
\item $\sigma_u$ = standard deviation of noise trading volume (uninformed, random trading)
\item Higher $\sigma_v$ (more insider info) $\to$ higher $\lambda$
\item Higher $\sigma_u$ (more noise trading) $\to$ lower $\lambda$
\end{itemize}

\column{0.48\textwidth}
\textbf{Crypto vs Traditional Markets}

\vspace{0.3em}
\begin{tabular}{lcc}
\toprule
\textbf{Market} & $\sigma_v$ & $\lambda$ \\
\midrule
S\&P 500 & 0.01 & 0.0125 \\
AAPL & 0.016 & 0.02 \\
BTC (Binance) & 0.024 & 0.06 \\
BTC (Coinbase) & 0.03 & 0.075 \\
ETH (Uniswap) & 0.048 & 0.12 \\
SOL (DEX) & 0.06 & 0.15 \\
\bottomrule
\end{tabular}

\vspace{0.3em}
\textbf{Key insight:} Crypto markets have $\lambda$ values 5--12$\times$ higher than traditional equities. This means a 1-unit trade moves crypto prices 5--12$\times$ more.

\textbf{Why?} Higher $\sigma_v$ (protocol hacks, governance votes, delistings known to insiders) and lower $\sigma_u$ (less passive index-fund flow).
\end{columns}

\bottomnote{Kyle (1985): $\lambda = \sigma_v / (2\sigma_u)$ measures adverse selection; crypto $\lambda$ is 5--12x TradFi due to higher info asymmetry}
\end{frame}

% Frame 5
\begin{frame}[t]{From Lambda to Execution Cost}
\begin{columns}[T]
\column{0.48\textwidth}
\textbf{Price Impact of a Trade}

For a trade of size $Q$, Kyle's model predicts the price moves by:
$$\Delta P = \lambda \cdot Q$$

\textbf{Total execution cost} (what you pay above the fair price):
$$C(Q) = \lambda \cdot Q^2 / 2$$

This is quadratic: doubling your trade size quadruples your execution cost. This is why large traders split orders across time.

\vspace{0.3em}
\textbf{Worked example:} Buy 10 ETH on Uniswap ($\lambda = 0.12$, $P_0 = \$1{,}800$).

Price impact: $\Delta P = 0.12 \times 10 = \$1.20$/ETH.

Execution cost: $C = 0.12 \times 10^2 / 2 = \$6$ above fair value.

As a percentage: $6 / (10 \times 1800) = 0.033\%$.

\column{0.48\textwidth}
\textbf{Comparing Venues}

\vspace{0.3em}
\begin{tabular}{lcc}
\toprule
\textbf{Venue} & $\lambda$ & \textbf{Cost (10 ETH)} \\
\midrule
Binance & 0.06 & \$3.00 \\
Coinbase & 0.075 & \$3.75 \\
Uniswap v3 & 0.12 & \$6.00 \\
\bottomrule
\end{tabular}

\vspace{0.3em}
Binance is cheapest because it has deepest order books (lowest $\lambda$). Uniswap costs 2$\times$ more per trade, but offers permissionless access and censorship resistance.

\vspace{0.3em}
\textbf{AMMs vs Order Books}

In an AMM, $\lambda$ is not fixed---it depends on pool depth:
$$\lambda_{\text{AMM}} = \frac{P}{2 \cdot x}$$
where $x$ is the token reserve. Deeper pools $\to$ lower $\lambda$.

For Uniswap v3 concentrated liquidity, effective $\lambda$ can be much lower within the active range.
\end{columns}

\bottomnote{Execution cost $C = \lambda Q^2 / 2$ is quadratic in trade size; venue choice depends on $\lambda$ vs access trade-off}
\end{frame}

% Frame 6 (Chart 06)
\begin{frame}[t]{Kyle Lambda: Traditional vs Crypto Markets}
\begin{center}
\includegraphics[width=0.72\textwidth]{06_kyle_lambda_comparison/chart.pdf}
\end{center}

\bottomnote{Kyle (1985), Barbon \& Ranaldo (2022): crypto $\lambda$ reflects higher $\sigma_v$ (insider info) and lower $\sigma_u$ (less passive flow)}
\end{frame}

% Frame 7
\begin{frame}[t]{Implications for DeFi Protocol Design}
\begin{columns}[T]
\column{0.48\textwidth}
\textbf{Reducing Lambda in AMMs}

Protocol designers can lower $\lambda$ (reduce price impact) through:
\begin{enumerate}
\item \textbf{Deeper pools:} More TVL reduces $\lambda_{\text{AMM}} = P/(2x)$
\item \textbf{Concentrated liquidity:} Uniswap v3 concentrates reserves near the current price, effectively increasing $x$ in the active range
\item \textbf{Dynamic fees:} Raise fees during high-volatility periods (when $\sigma_v$ spikes) to compensate LPs for increased adverse selection
\item \textbf{Batch auctions:} CoW Protocol collects orders over a time window and executes at a single clearing price, eliminating front-running
\end{enumerate}

\column{0.48\textwidth}
\textbf{Lambda and LP Profitability}

High $\lambda$ is a double-edged sword for LPs:
\begin{itemize}
\item \textbf{Pro:} Higher $\lambda$ $\to$ more fee revenue per trade (trades generate larger price moves, earning LPs more in percentage terms)
\item \textbf{Con:} Higher $\lambda$ $\to$ more adverse selection loss (informed traders systematically extract value from the pool)
\end{itemize}

\vspace{0.3em}
\textbf{The Informed Trader Problem}

Kyle's model assumes one informed trader, one market maker, and noise traders. In DeFi:
\begin{itemize}
\item \textbf{Informed traders} include MEV searchers, arbitrageurs, and insiders
\item \textbf{Market makers} are passive LPs (cannot adjust quotes in real time)
\item LPs are structurally disadvantaged: they always trade against informed flow
\end{itemize}
\end{columns}

\bottomnote{Protocol design directly controls $\lambda$; the challenge is reducing adverse selection without eliminating LP incentives}
\end{frame}

%% ===== SECTION 3: Concentrated Liquidity: Uniswap v3 (5 frames) =====
\section{Concentrated Liquidity: Uniswap v3}

% Frame 8 (introduce Greek eta)
\begin{frame}[t]{Concentrated Liquidity: The Capital Efficiency Revolution}
\begin{columns}[T]
\column{0.48\textwidth}
\textbf{From v2 to v3}

Uniswap v2 spreads liquidity uniformly across all prices from 0 to infinity. Most of this liquidity sits idle---ETH will never trade at \$0.01 or \$10 million.

Uniswap v3 lets LPs choose a price range $[P_{\text{low}}, P_{\text{high}}]$, concentrating their capital where it is actually used.

\vspace{0.3em}
\textbf{Capital Efficiency Multiplier $\eta$}

We define $\eta$ (``eta'') as the capital efficiency gain from concentrating liquidity:
$$\eta = \frac{1}{\sqrt{1+r} - \sqrt{1-r}}$$
where $r$ is the half-width of the price range as a fraction of current price.

\textbf{Example:} $\pm 5\%$ range ($r = 0.05$):
$$\eta = \frac{1}{\sqrt{1.05} - \sqrt{0.95}} = \frac{1}{1.0247 - 0.9747} = 20.0\times$$

\$100K concentrated = \$2M full-range equivalent.

\column{0.48\textwidth}
\textbf{Efficiency by Range Width}

\vspace{0.3em}
\begin{tabular}{ccc}
\toprule
\textbf{Range} & $\eta$ & \textbf{Equivalent} \\
\midrule
$\pm 1\%$ & 100x & \$100K $=$ \$10M \\
$\pm 2\%$ & 50x & \$100K $=$ \$5M \\
$\pm 5\%$ & 20x & \$100K $=$ \$2M \\
$\pm 10\%$ & 10x & \$100K $=$ \$1M \\
$\pm 20\%$ & 5x & \$100K $=$ \$500K \\
$\pm 50\%$ & 2x & \$100K $=$ \$200K \\
$\pm 100\%$ & 1x & Full range (v2) \\
\bottomrule
\end{tabular}

\vspace{0.3em}
\textbf{The catch:} If the price moves outside your range, you earn \textbf{zero fees} and hold 100\% of the depreciating token. Narrow ranges amplify both gains and losses.
\end{columns}

\bottomnote{Adams et al.\ (2021): $\eta = 1/(\sqrt{1+r} - \sqrt{1-r})$ measures capital efficiency; $\pm$5\% range gives 20x efficiency}
\end{frame}

% Frame 9
\begin{frame}[t]{How Concentrated Liquidity Works Mathematically}
\begin{columns}[T]
\column{0.48\textwidth}
\textbf{Virtual Reserves}

In v3, an LP providing liquidity in $[P_a, P_b]$ creates ``virtual reserves'' that behave like a v2 pool within that range:

$$x_{\text{virtual}} = L \left(\frac{1}{\sqrt{P}} - \frac{1}{\sqrt{P_b}}\right)$$
$$y_{\text{virtual}} = L \left(\sqrt{P} - \sqrt{P_a}\right)$$

where $L = \sqrt{k}$ is the liquidity parameter. The constant product $x \cdot y = k$ still holds locally, but with $L$ amplified by the concentration factor $\eta$.

\vspace{0.3em}
\textbf{Price Representation}

v3 uses $\sqrt{P}$ (square root of price) as the fundamental variable, not $P$ itself. This makes the math simpler: swaps change $\sqrt{P}$ linearly in the trade amount.

\column{0.48\textwidth}
\textbf{Tick System}

Prices are discretized into ``ticks'' (minimum price increments):
$$P(i) = 1.0001^i$$

Each tick represents a 0.01\% price change. This logarithmic spacing means:
\begin{itemize}
\item Low prices have small absolute tick sizes
\item High prices have large absolute tick sizes
\item Same percentage precision everywhere
\end{itemize}

\vspace{0.3em}
\textbf{Active Liquidity}

Only LPs whose range includes the current price earn fees. As price moves through ticks:
\begin{itemize}
\item Entering a tick: LP position activates
\item Leaving a tick: LP position deactivates
\item The pool's effective depth changes at every tick boundary
\end{itemize}
\end{columns}

\bottomnote{v3 uses virtual reserves and tick-based discretization; $\sqrt{P}$ is the native variable for swap computation}
\end{frame}

% Frame 10 (Chart 07)
\begin{frame}[t]{Capital Efficiency vs Range Risk}
\begin{center}
\includegraphics[width=0.72\textwidth]{07_concentrated_liquidity_efficiency/chart.pdf}
\end{center}

\bottomnote{Adams et al.\ (2021): Narrower ranges give exponentially higher $\eta$ but exponentially higher out-of-range risk}
\end{frame}

% Frame 11
\begin{frame}[t]{Active LP Management in v3}
\begin{columns}[T]
\column{0.48\textwidth}
\textbf{The JIT Liquidity Problem}

Just-in-time (JIT) liquidity providers deposit for a single block to capture fees from a known large trade, then withdraw immediately.

\textbf{Impact on passive LPs:}
\begin{itemize}
\item JIT providers dilute fee share for the block
\item Passive LPs earn fewer fees per dollar deposited
\item JIT effectively front-runs passive LP fee income
\end{itemize}

\vspace{0.3em}
\textbf{Rebalancing Costs}

Active LPs must frequently adjust their range as price moves. Each adjustment costs:
\begin{itemize}
\item Gas fees (\$5--\$50 on Ethereum mainnet)
\item Potential slippage on the rebalancing trade
\item Realized impermanent loss (becomes permanent)
\end{itemize}

\column{0.48\textwidth}
\textbf{Professional vs Retail LPs}

\vspace{0.3em}
\begin{tabular}{lcc}
\toprule
& \textbf{Professional} & \textbf{Retail} \\
\midrule
Range width & $\pm 2$--5\% & $\pm 20$--50\% \\
Rebalancing & Minutes & Days/never \\
Gas budget & High & Low \\
Fee capture & 60--80\% & 20--40\% \\
IL management & Hedged & Unhedged \\
\bottomrule
\end{tabular}

\vspace{0.3em}
Adams et al.\ (2024) find that professional LPs capture the majority of Uniswap v3 fees. Retail LPs who set-and-forget wide ranges underperform relative to simply holding the tokens.

\textbf{Implication:} v3 shifted LP from a passive activity to a professional skill, similar to traditional market making.
\end{columns}

\bottomnote{v3 professionalized LP provision; JIT liquidity and rebalancing costs disadvantage passive retail LPs}
\end{frame}

% Frame 12
\begin{frame}[t]{Concentrated Impermanent Loss: Amplified Risk}
\begin{columns}[T]
\column{0.48\textwidth}
\textbf{IL Amplification Formula}

For a concentrated position with efficiency $\eta$:
$$IL_{\text{conc}} = IL_{\text{v2}} \times \eta$$

where $IL_{\text{v2}} = \frac{2\sqrt{r}}{1+r} - 1$ is standard impermanent loss and $r = P_{\text{final}} / P_{\text{initial}}$.

\vspace{0.3em}
\textbf{Worked Example}

ETH moves 10\% ($r = 1.10$):
\begin{itemize}
\item v2 IL: $\frac{2\sqrt{1.10}}{1 + 1.10} - 1 = -0.12\%$ (small)
\item $\pm 5\%$ range ($\eta = 20$): $IL_{\text{conc}} = -0.12\% \times 20 = -2.4\%$
\item $\pm 2\%$ range ($\eta = 50$): $IL_{\text{conc}} = -0.12\% \times 50 = -6.0\%$
\end{itemize}

For a 5$\times$ concentration ($\eta = 5$, range $\pm 20\%$):
\begin{itemize}
\item v2 IL at 10\% move: $-0.12\%$
\item Concentrated: $-0.12\% \times 5 = -0.6\%$
\item At larger moves: 20\% move gives $-0.6\% \times 5 = -3.0\%$
\end{itemize}

\column{0.48\textwidth}
\textbf{Risk-Return Comparison}

\vspace{0.3em}
\begin{tabular}{lccc}
\toprule
& \textbf{v2} & \textbf{5x} & \textbf{20x} \\
\midrule
Range & Full & $\pm 20\%$ & $\pm 5\%$ \\
$\eta$ & 1 & 5 & 20 \\
Fee mult. & 1x & 5x & 20x \\
IL (10\%) & 0.12\% & 0.6\% & 2.4\% \\
IL (50\%) & 2.0\% & 10\% & 40\% \\
\bottomrule
\end{tabular}

\vspace{0.3em}
\textbf{Key insight:} Concentration amplifies both fee income and IL by the same factor $\eta$. The net effect depends on the ratio of fee income to IL:
\begin{itemize}
\item High volume, low volatility $\to$ concentration helps
\item Low volume, high volatility $\to$ concentration hurts
\end{itemize}

\textbf{Decision rule:} Concentrate only if $\frac{f \cdot V/TVL}{\sigma^2} > \text{threshold}$, where $\sigma$ is the asset's volatility.
\end{columns}

\bottomnote{Concentrated IL $= IL_{\text{v2}} \times \eta$; 10\% price move with 5x concentration: 0.6\% loss (vs 0.12\% in v2)}
\end{frame}

%% ===== SECTION 4: Curve StableSwap Invariant (4 frames) =====
\section{Curve StableSwap Invariant}

% Frame 13
\begin{frame}[t]{The StableSwap Invariant: Best of Both Worlds}
\begin{columns}[T]
\column{0.48\textwidth}
\textbf{Three AMM Invariants}

\vspace{0.3em}
\textbf{1. Constant-Sum:} $x + y = D$
\begin{itemize}
\item Zero slippage (price always = 1)
\item But: reserves can be fully depleted (one token goes to zero)
\end{itemize}

\textbf{2. Constant-Product:} $x \cdot y = k$
\begin{itemize}
\item Infinite liquidity (always tradable)
\item But: high slippage for large trades
\end{itemize}

\textbf{3. StableSwap (Curve):} Interpolates between both
$$A \cdot n^n \cdot \sum x_i + D = A \cdot n^n \cdot D + \frac{D^{n+1}}{n^n \cdot \prod x_i}$$

For $n = 2$ tokens:
$$4A(x + y) + D = 4AD + \frac{D^3}{4xy}$$

\column{0.48\textwidth}
\textbf{The Amplification Parameter $A$}

$A$ controls how ``stableswap-like'' vs ``constant-product-like'' the curve is:
\begin{itemize}
\item $A = 0$: pure constant product (high slippage)
\item $A = 1$: mild concentration near peg
\item $A = 100$: very flat near peg (Curve default for stablecoins)
\item $A \to \infty$: approaches constant sum (zero slippage but fragile)
\end{itemize}

\vspace{0.3em}
\textbf{Why This Matters}

Stablecoin pairs (USDC/USDT) rarely deviate from 1:1. Curve exploits this by concentrating liquidity near the peg:
\begin{itemize}
\item 100$\times$ lower slippage than Uniswap for same-peg swaps
\item Still provides safety: if one token depegs, curve transitions smoothly to constant-product behavior
\end{itemize}
\end{columns}

\bottomnote{Egorov (2019): StableSwap invariant with amplification $A$ achieves 100x lower slippage for pegged assets}
\end{frame}

% Frame 14
\begin{frame}[t]{StableSwap: Numerical Example}
\begin{columns}[T]
\column{0.48\textwidth}
\textbf{Setup:} Pool with 100M USDC and 100M USDT.

$D = 200$M, $A = 100$.

\vspace{0.3em}
\textbf{Swap 1M USDC for USDT}

\begin{tabular}{lcc}
\toprule
\textbf{AMM Type} & \textbf{USDT Out} & \textbf{Slippage} \\
\midrule
Constant-Sum & 1{,}000{,}000 & 0.000\% \\
StableSwap & 999{,}990 & 0.001\% \\
Constant-Prod & 999{,}010 & 0.099\% \\
\bottomrule
\end{tabular}

\vspace{0.3em}
StableSwap gives 99$\times$ less slippage than constant product for this stablecoin swap.

\column{0.48\textwidth}
\textbf{Swap 10M USDC (10\% of pool)}

\begin{tabular}{lcc}
\toprule
\textbf{AMM Type} & \textbf{USDT Out} & \textbf{Slippage} \\
\midrule
Constant-Sum & 10{,}000{,}000 & 0.000\% \\
StableSwap & 9{,}999{,}100 & 0.009\% \\
Constant-Prod & 9{,}090{,}909 & 9.09\% \\
\bottomrule
\end{tabular}

\vspace{0.3em}
At 10\% of pool, constant product loses 9\% to slippage. StableSwap loses only 0.009\%.

\vspace{0.3em}
\textbf{When StableSwap fails:} If USDC depegs to \$0.90, the pool becomes unbalanced. StableSwap transitions to constant-product behavior, protecting LPs from complete loss but increasing slippage dramatically.
\end{columns}

\bottomnote{StableSwap efficiency depends on assets remaining pegged; depeg events trigger constant-product fallback behavior}
\end{frame}

% Frame 15 (Chart 08)
\begin{frame}[t]{AMM Invariant Comparison}
\begin{center}
\includegraphics[width=0.72\textwidth]{08_amm_invariant_comparison/chart.pdf}
\end{center}

\bottomnote{Egorov (2019): StableSwap (green) hugs the constant-sum line near the peg but curves like constant-product far from it}
\end{frame}

% Frame 16
\begin{frame}[t]{Choosing the Right AMM Design}
\begin{columns}[T]
\column{0.48\textwidth}
\textbf{Design Decision Matrix}

\vspace{0.3em}
\begin{tabular}{lcc}
\toprule
\textbf{Asset Pair} & \textbf{Best AMM} & \textbf{Why} \\
\midrule
USDC/USDT & StableSwap & Near-peg \\
ETH/USDC & Uni v3 conc. & Moderate vol \\
MEME/ETH & Uni v2 full & High vol \\
wBTC/BTC & StableSwap & Near-peg \\
\bottomrule
\end{tabular}

\vspace{0.3em}
\textbf{The Specialization Trend}

Different AMM designs optimize for different volatility regimes:
\begin{itemize}
\item Low volatility (stablecoins): StableSwap dominates
\item Medium volatility (majors): Concentrated liquidity
\item High volatility (memecoins): Full-range v2 safest for LPs
\end{itemize}

\column{0.48\textwidth}
\textbf{Hybrid Approaches}

Newer protocols combine multiple invariants:
\begin{itemize}
\item \textbf{Curve v2:} Dynamic $A$ that adjusts to market conditions
\item \textbf{Maverick:} LPs choose directional concentration (bet on price direction)
\item \textbf{Ambient (CrocSwap):} Combines concentrated and ambient (full-range) liquidity in one pool
\end{itemize}

\vspace{0.3em}
\textbf{Open Research Question}

Can an AMM dynamically switch invariants based on real-time volatility? This would require an oracle (external price feed) and introduces new attack surfaces (oracle manipulation).

The fundamental tension: \textit{efficiency requires assumptions about asset behavior; robustness requires minimal assumptions.}
\end{columns}

\bottomnote{No single AMM design dominates; optimal choice depends on asset volatility, correlation, and LP sophistication}
\end{frame}

%% ===== SECTION 5: LP Profitability Analysis (4 frames) =====
\section{LP Profitability Analysis}

% Frame 17
\begin{frame}[t]{The LP Profitability Equation}
\begin{columns}[T]
\column{0.48\textwidth}
\textbf{Net LP Return}

An LP's net return combines fee income and impermanent loss:
$$R_{\text{net}} = \underbrace{f \cdot \frac{V}{TVL} \cdot T}_{\text{fee income}} + \underbrace{IL(r)}_{\text{impermanent loss (negative)}}$$

where:
\begin{itemize}
\item $f$: fee rate (0.05\%, 0.3\%, or 1\%)
\item $V/TVL$: daily volume relative to pool size
\item $T$: number of days
\item $IL(r) = \frac{2\sqrt{r}}{1+r} - 1$ (always $\leq 0$)
\end{itemize}

\vspace{0.3em}
\textbf{Breakeven condition:} $R_{\text{net}} = 0$ when:
$$f \cdot \frac{V}{TVL} \cdot T = |IL(r)|$$

\column{0.48\textwidth}
\textbf{Empirical Data (Adams et al., 2024)}

\vspace{0.3em}
\begin{tabular}{lcc}
\toprule
\textbf{Pool} & \textbf{Avg APR} & \textbf{\% LPs losing} \\
\midrule
ETH/USDC 0.3\% & 12.5\% & 49.5\% \\
ETH/USDC 0.05\% & 8.2\% & 55.1\% \\
WBTC/ETH 0.3\% & 5.1\% & 62.3\% \\
USDC/USDT 0.01\% & 2.8\% & 28.4\% \\
\bottomrule
\end{tabular}

\vspace{0.3em}
\textbf{Striking finding:} Nearly half of all Uniswap v3 LPs lose money compared to simply holding their tokens. The profitable LPs tend to be:
\begin{itemize}
\item Professional market makers
\item Active rebalancers
\item Concentrated in high-volume pools
\end{itemize}
\end{columns}

\bottomnote{$R_{\text{net}} = f \cdot V/TVL \cdot T + IL$; Adams et al.\ (2024) find 49.5\% of v3 LPs underperform holding}
\end{frame}

% Frame 18
\begin{frame}[t]{Fee Tier Selection and Volume Dynamics}
\begin{columns}[T]
\column{0.48\textwidth}
\textbf{Uniswap v3 Fee Tiers}

\vspace{0.3em}
\begin{tabular}{cll}
\toprule
\textbf{Fee} & \textbf{Pairs} & \textbf{$V/TVL$} \\
\midrule
0.01\% & Stablecoins & 5--20 \\
0.05\% & Major pairs & 2--5 \\
0.30\% & Standard & 0.5--2 \\
1.00\% & Exotic/volatile & 0.1--0.5 \\
\bottomrule
\end{tabular}

\vspace{0.3em}
Higher fee tiers attract less volume but earn more per trade. The market finds an equilibrium:
\begin{itemize}
\item If a pool's fee is too high, traders route to lower-fee alternatives
\item If too low, LPs exit (fees do not cover IL)
\item Competition among fee tiers is a coordination game
\end{itemize}

\column{0.48\textwidth}
\textbf{Volume-to-TVL Ratio}

$V/TVL$ is the key metric for LP profitability:

\vspace{0.3em}
\textbf{High $V/TVL$ (good for LPs):}
\begin{itemize}
\item Popular trading pairs (ETH/USDC)
\item High market volatility (more trading)
\item Low TVL (less competition among LPs)
\end{itemize}

\textbf{Low $V/TVL$ (bad for LPs):}
\begin{itemize}
\item Obscure pairs
\item Calm markets
\item Oversubscribed pools (too much TVL)
\end{itemize}

\vspace{0.3em}
\textbf{Paradox:} When markets are volatile, both fee income \textit{and} IL increase. The question is: which grows faster? For moderate volatility, fees win. For extreme moves, IL dominates.
\end{columns}

\bottomnote{Fee tier selection is a coordination game; $V/TVL$ ratio determines LP profitability more than fee rate alone}
\end{frame}

% Frame 19 (Chart 09)
\begin{frame}[t]{LP Profitability Frontier}
\begin{center}
\includegraphics[width=0.72\textwidth]{09_lp_profitability_frontier/chart.pdf}
\end{center}

\bottomnote{Adams et al.\ (2024), Lambert et al.\ (2022): LPs profitable only below breakeven volatility for their fee tier}
\end{frame}

% Frame 20
\begin{frame}[t]{LP Strategies and Hedging}
\begin{columns}[T]
\column{0.48\textwidth}
\textbf{Strategy 1: Passive Full-Range}
\begin{itemize}
\item Set wide range, collect fees
\item Low maintenance, low gas costs
\item Returns: $\sim$5--15\% APR minus IL
\item \textbf{Risk:} Unhedged IL exposure
\end{itemize}

\vspace{0.3em}
\textbf{Strategy 2: Active Concentrated}
\begin{itemize}
\item Narrow range ($\pm$2--5\%), frequent rebalancing
\item High fee capture, high gas costs
\item Returns: $\sim$20--50\% APR minus IL minus gas
\item \textbf{Risk:} Out-of-range losses, gas costs
\end{itemize}

\vspace{0.3em}
\textbf{Strategy 3: Delta-Hedged LP}
\begin{itemize}
\item LP position + perpetual futures hedge
\item Eliminates directional exposure
\item Isolates pure fee income
\item \textbf{Risk:} Funding rate cost, basis risk
\end{itemize}

\column{0.48\textwidth}
\textbf{Hedging IL with Perpetuals}

An LP in ETH/USDC is long ETH (from the pool) and short ETH (from IL). To hedge:
\begin{itemize}
\item Short $\Delta$ ETH via perpetual swap
\item $\Delta$ changes with price (gamma exposure)
\item Must continuously rebalance hedge
\end{itemize}

\vspace{0.3em}
\textbf{Cost-Benefit Analysis}

\vspace{0.3em}
\begin{tabular}{lcc}
\toprule
\textbf{Strategy} & \textbf{Gross} & \textbf{Net} \\
\midrule
Passive & 12\% & 5--8\% \\
Active & 35\% & 15--25\% \\
Hedged & 12\% & 8--10\% \\
\bottomrule
\end{tabular}

\vspace{0.3em}
Hedged LP removes IL risk but at the cost of futures funding rates (typically 5--15\% annualized). Professional LPs use this approach when funding rates are below fee income.
\end{columns}

\bottomnote{Optimal LP strategy depends on risk tolerance, capital, and ability to actively manage positions}
\end{frame}

%% ===== SECTION 6: MEV Game Theory and DeFi Efficiency (4 frames) =====
\section{MEV Game Theory and DeFi Efficiency}

% Frame 21
\begin{frame}[t]{MEV as a Game: Searcher Competition}
\begin{columns}[T]
\column{0.48\textwidth}
\textbf{The MEV Game Structure}

MEV extraction is a multi-player game:
\begin{itemize}
\item \textbf{Players:} Searchers (bots scanning the mempool for profit opportunities), builders (assembling blocks), validators (proposing blocks)
\item \textbf{Strategy:} Choose which transactions to include, in what order, and how much to bid
\item \textbf{Payoff:} Profit from reordering minus gas and priority bids
\end{itemize}

\vspace{0.3em}
\textbf{Priority Gas Auction (PGA)}

Searchers compete by bidding higher gas prices. In equilibrium, competition drives profits toward zero:
$$\pi_{\text{net}} = \pi_{\text{gross}} - c_{\text{PGA}}$$

where $c_{\text{PGA}}$ is the auction cost. With $N$ competing searchers, expected profit per searcher approaches $\pi_{\text{gross}} / N$.

\column{0.48\textwidth}
\textbf{Flashbots and PBS}

Flashbots introduced Proposer-Builder Separation (PBS):
\begin{enumerate}
\item Searchers submit ``bundles'' (ordered transaction sets) to builders
\item Builders assemble optimal blocks from bundles
\item Validators choose the highest-paying block
\item Most MEV revenue flows to validators
\end{enumerate}

\vspace{0.3em}
\textbf{Welfare Implications}
\begin{itemize}
\item \textbf{Without Flashbots:} PGA creates network congestion (wasted gas) and front-running visible in mempool
\item \textbf{With Flashbots:} MEV extraction is more efficient (less wasted gas) but still extracts value from users
\item \textbf{Net effect:} Flashbots reduced gas waste by $\sim$40\% but did not eliminate MEV costs for traders
\end{itemize}
\end{columns}

\bottomnote{MEV is a competitive game; Flashbots/PBS reduce waste but do not eliminate extraction from users}
\end{frame}

% Frame 22 (cross-ref to L07)
\begin{frame}[t]{MEV Taxonomy and Welfare Analysis}
\begin{columns}[T]
\column{0.48\textwidth}
\textbf{Beneficial MEV}
\begin{itemize}
\item \textbf{Arbitrage:} Aligns prices across venues, improving efficiency. Arbitrage MEV is socially useful because it makes prices more accurate.
\item \textbf{Liquidations:} Maintains protocol solvency in lending markets (Aave, Compound). Without MEV-motivated liquidators, bad debt accumulates.
\end{itemize}

\vspace{0.3em}
\textbf{Harmful MEV}
\begin{itemize}
\item \textbf{Sandwich attacks:} Pure wealth transfer from victims to attackers with no social benefit
\item \textbf{Front-running:} Trades ahead of user orders, increasing execution cost
\item \textbf{Time-bandit attacks:} Re-mining past blocks to extract MEV (threatens consensus security)
\end{itemize}

\column{0.48\textwidth}
\textbf{MEV Flow of Funds}

\vspace{0.3em}
\begin{tabular}{lc}
\toprule
\textbf{Recipient} & \textbf{Share} \\
\midrule
Validators & 60--80\% \\
Searchers & 10--20\% \\
Builders & 5--10\% \\
Burned (gas) & 5--10\% \\
\bottomrule
\end{tabular}

\vspace{0.3em}
\textbf{Policy Question}

Should regulators treat sandwich attacks as market manipulation? Traditional finance prohibits front-running. DeFi operates in a regulatory gray zone.

\vspace{0.3em}
\textit{For regulatory response to MEV, see L07 Extended (Regulatory Economics of Digital Finance).}
\end{columns}

\bottomnote{Some MEV (arbitrage, liquidations) is socially useful; sandwich attacks are pure extraction; regulation remains unsettled}
\end{frame}

% Frame 23 (Chart 10)
\begin{frame}[t]{MEV Extraction: Optimal Sandwich Profit}
\begin{center}
\includegraphics[width=0.72\textwidth]{10_mev_extraction_profit/chart.pdf}
\end{center}

\bottomnote{Daian et al.\ (2020), Qin et al.\ (2022): interior maximum exists because AMM fees + PGA costs eventually dominate revenue}
\end{frame}

% Frame 24 (Chart 11)
\begin{frame}[t]{DEX vs CEX: Multi-Dimensional Quality Comparison}
\begin{center}
\includegraphics[width=0.72\textwidth]{11_dex_cex_radar_comparison/chart.pdf}
\end{center}

\bottomnote{Barbon \& Ranaldo (2022): CEX excels at execution quality; DEX excels at transparency and censorship resistance}
\end{frame}

%% ===== SECTION 7: Synthesis and Policy Implications (3 frames) =====
\section{Synthesis and Policy Implications}

% Frame 25
\begin{frame}[t]{The DeFi Microstructure Trilemma}
\begin{columns}[T]
\column{0.48\textwidth}
\textbf{Three Desirable Properties}

DeFi market design faces a trilemma---you can optimize for at most two of three properties:

\begin{enumerate}
\item \textbf{Capital efficiency:} Low slippage per dollar of liquidity deployed
\item \textbf{LP safety:} Low impermanent loss and adverse selection risk
\item \textbf{Decentralization:} Permissionless, non-custodial, censorship-resistant
\end{enumerate}

\vspace{0.3em}
\textbf{Trade-offs in Practice}
\begin{itemize}
\item Uniswap v3: High efficiency + decentralization, but LP risk
\item Curve: Moderate efficiency + LP safety (for pegged assets), decentralized
\item CEX: High efficiency + LP safety, but centralized
\end{itemize}

\column{0.48\textwidth}
\textbf{The Impossibility}

Why can't we have all three?
\begin{itemize}
\item Capital efficiency requires concentrating liquidity (narrowing range), which amplifies IL
\item LP safety requires wide ranges or hedging, which reduces efficiency
\item Decentralization means no circuit breakers, no order cancellation during crashes
\end{itemize}

\vspace{0.3em}
\textbf{Emerging Solutions}

\begin{itemize}
\item \textbf{Oracle-based AMMs:} Use external price feeds to reduce adverse selection (at cost of oracle dependency)
\item \textbf{Intent-based trading:} Users specify desired outcome; solvers compete to execute optimally
\item \textbf{Application-specific ordering:} Protocols control transaction ordering within their domain
\end{itemize}
\end{columns}

\bottomnote{DeFi faces a capital efficiency--LP safety--decentralization trilemma; no current design achieves all three}
\end{frame}

% Frame 26
\begin{frame}[t]{What Traditional Finance Can Learn from DeFi}
\begin{columns}[T]
\column{0.48\textwidth}
\textbf{Innovations Worth Adopting}
\begin{itemize}
\item \textbf{Transparent pricing formulas:} AMM math is open-source; traditional market makers use proprietary algorithms
\item \textbf{Permissionless LP:} Anyone can provide liquidity; traditional markets require broker-dealer licenses
\item \textbf{24/7 trading:} No market hours, no holidays. Crypto showed demand exists for always-on markets.
\item \textbf{Composability:} DeFi protocols can be combined (``money legos''). TradFi systems are siloed.
\end{itemize}

\column{0.48\textwidth}
\textbf{What DeFi Can Learn from TradFi}
\begin{itemize}
\item \textbf{Circuit breakers:} TradFi halts trading during extreme moves; DeFi liquidates ruthlessly
\item \textbf{Best execution rules:} Brokers must route orders to best venue; DeFi has no such obligation
\item \textbf{Market surveillance:} TradFi monitors for manipulation; DeFi relies on code (which can be exploited)
\item \textbf{Insurance and compensation:} TradFi has deposit insurance; DeFi hacks result in total loss
\end{itemize}

\vspace{0.3em}
\textbf{Convergence Thesis}

Hybrid systems combining TradFi regulation with DeFi infrastructure may offer the best of both worlds. Examples: tokenized securities on regulated DEXs, KYC-gated AMM pools.
\end{columns}

\bottomnote{TradFi and DeFi each have structural advantages; convergence toward hybrid models is likely}
\end{frame}

% Frame 27
\begin{frame}[t]{Key Takeaways and Exam Preparation}
\begin{columns}[T]
\column{0.48\textwidth}
\textbf{Core Models to Know}
\begin{enumerate}
\item \textbf{Kyle's $\lambda$:} $\lambda = \sigma_v / (2\sigma_u)$ measures price impact; crypto $\lambda$ is 5--12$\times$ TradFi
\item \textbf{Concentrated efficiency:} $\eta = 1/(\sqrt{1+r} - \sqrt{1-r})$; narrow ranges amplify both fees and IL by $\eta$
\item \textbf{StableSwap:} Curve's $A$ parameter interpolates between constant-sum and constant-product
\item \textbf{LP frontier:} $R_{\text{net}} = f \cdot V/TVL \cdot T + IL$; 49.5\% of v3 LPs lose
\item \textbf{MEV sandwich:} Profit has interior maximum due to AMM fees + PGA costs
\end{enumerate}

\column{0.48\textwidth}
\textbf{Conceptual Questions}
\begin{itemize}
\item Why is crypto $\lambda$ higher than TradFi?
\item When does concentrated liquidity help vs.\ hurt LPs?
\item Why does StableSwap dominate for pegged assets?
\item Under what conditions are LPs profitable?
\item Is MEV always harmful? What is the social cost?
\item Can the trilemma be resolved?
\end{itemize}

\vspace{0.3em}
\textbf{Connection to Other Lectures}
\begin{itemize}
\item L05: Platform economics $\to$ why DEXs have network effects
\item L06 basic: Qualitative AMM/MEV $\to$ this lecture formalizes
\item L07: Regulation $\to$ policy responses to MEV and market manipulation
\end{itemize}
\end{columns}

\bottomnote{Master the five core models and their comparative statics; connect to platform economics (L05) and regulation (L07)}
\end{frame}

%% ===== SECTION 8: Appendix (3 frames) =====
\section*{Appendix}

% Frame 28
\begin{frame}[t]{Appendix A: Mathematical Derivations}
\begin{columns}[T]
\column{0.48\textwidth}
\textbf{Deriving IL from Constant Product}

Start with $x \cdot y = k$ and an LP deposit at price $P_0 = y_0/x_0$.

After price changes to $P_1 = r \cdot P_0$:
$$x_1 = \sqrt{k/P_1} = x_0 / \sqrt{r}$$
$$y_1 = \sqrt{k \cdot P_1} = y_0 \cdot \sqrt{r}$$

LP value: $V_{LP} = x_1 \cdot P_1 + y_1 = 2y_0\sqrt{r}$.

Hold value: $V_{hold} = x_0 \cdot P_1 + y_0 = y_0(r + 1)$.

$$IL = \frac{V_{LP}}{V_{hold}} - 1 = \frac{2\sqrt{r}}{1+r} - 1$$

\column{0.48\textwidth}
\textbf{Kyle's Lambda Derivation (Simplified)}

Market maker observes total order flow $Q = Q_{\text{informed}} + Q_{\text{noise}}$.

Bayesian update: $E[V | Q] = \bar{V} + \lambda \cdot Q$.

Equilibrium $\lambda$ minimizes market maker expected loss:
$$\lambda = \frac{\text{Cov}(V, Q)}{\text{Var}(Q)} = \frac{\sigma_v}{2\sigma_u}$$

This is the OLS regression coefficient of value on order flow. Higher $\sigma_v$ (more informed trading) increases $\lambda$; higher $\sigma_u$ (more noise) decreases it.
\end{columns}

\bottomnote{Appendix: IL derivation follows from constant product rebalancing; Kyle's $\lambda$ is Bayesian signal extraction}
\end{frame}

% Frame 29
\begin{frame}[t]{Appendix B: StableSwap Derivation and Curve v2}
\begin{columns}[T]
\column{0.48\textwidth}
\textbf{StableSwap Invariant Derivation}

Egorov (2019) combines constant-sum and constant-product:
$$\chi \cdot D^{n-1} \cdot \sum x_i + \prod x_i = \chi \cdot D^n + \left(\frac{D}{n}\right)^n$$

where $\chi = \frac{A \cdot \prod x_i}{(D/n)^n}$ is a dynamic leverage factor.

For $n = 2$, near the peg ($x \approx y \approx D/2$), $\chi \approx A$, and the curve is nearly flat (constant-sum-like).

Far from peg ($x \gg y$ or $y \gg x$), $\chi \to 0$, and the invariant reduces to $xy = (D/2)^2$ (constant-product).

\column{0.48\textwidth}
\textbf{Curve v2: Dynamic Peg}

Curve v2 (2021) extends StableSwap to volatile pairs by:
\begin{enumerate}
\item Using an internal oracle to track the ``peg'' price
\item Dynamically adjusting $A$ and $D$ to center liquidity around the oracle price
\item Concentrating liquidity like Uniswap v3 but automatically
\end{enumerate}

\vspace{0.3em}
\textbf{Comparison}
\begin{itemize}
\item \textbf{Uniswap v3:} LPs manually set ranges
\item \textbf{Curve v2:} Protocol automatically adjusts concentration
\item \textbf{Trade-off:} Automation vs LP control
\end{itemize}
\end{columns}

\bottomnote{Appendix: StableSwap uses dynamic leverage $\chi$ that transitions from constant-sum to constant-product}
\end{frame}

% Frame 30 (Frame 31 coming next)
\begin{frame}[t]{Appendix C: Further Reading and Research Frontier}
\begin{columns}[T]
\column{0.48\textwidth}
\textbf{Foundational Papers}
\begin{itemize}
\item Kyle (1985): ``Continuous Auctions and Insider Trading'' --- price impact model
\item Glosten \& Milgrom (1985): ``Bid, Ask and Transaction Prices'' --- spread model
\item Hasbrouck (1995): ``One Security, Many Markets'' --- information share
\item Adams et al.\ (2021): ``Uniswap v3 Core'' --- concentrated liquidity whitepaper
\end{itemize}

\vspace{0.3em}
\textbf{DeFi Market Microstructure}
\begin{itemize}
\item Egorov (2019): ``StableSwap'' --- Curve invariant
\item Adams et al.\ (2024): ``Am I too Liquidity?'' --- LP profitability
\item Lambert et al.\ (2022): ``Uniswap v3: The Universal AMM''
\end{itemize}

\column{0.48\textwidth}
\textbf{MEV Research}
\begin{itemize}
\item Daian et al.\ (2020): ``Flash Boys 2.0'' --- MEV taxonomy
\item Qin et al.\ (2022): ``Quantifying Blockchain Extractable Value''
\item Barbon \& Ranaldo (2022): ``On the Quality of Cryptocurrency Markets''
\end{itemize}

\vspace{0.3em}
\textbf{Research Frontier}
\begin{itemize}
\item Loss-vs-rebalancing (LVR): new framework for LP costs (Milionis et al., 2022)
\item Mechanism design for MEV redistribution
\item Cross-chain arbitrage and bridge MEV
\item Dynamic fee mechanisms with oracle integration
\end{itemize}
\end{columns}

\bottomnote{All readings available on course platform; research frontier includes LVR, MEV redistribution, and cross-chain design}
\end{frame}

% Frame 31
\begin{frame}[t]{Appendix D: Glossary of Extended Terms}
\begin{columns}[T]
\column{0.48\textwidth}
\textbf{$\lambda$ (Kyle's Lambda)}
Price impact coefficient: how much the price moves per unit of order flow. Higher $\lambda$ = less liquid market.

\vspace{0.3em}
\textbf{$\eta$ (Capital Efficiency)}
Multiplier from concentrating liquidity. $\eta = 20$ means \$1 of concentrated liquidity provides the depth of \$20 full-range.

\vspace{0.3em}
\textbf{Amplification Factor ($A$)}
Curve parameter controlling curvature. Higher $A$ = flatter curve near peg = lower stablecoin slippage.

\vspace{0.3em}
\textbf{PGA (Priority Gas Auction)}
Competitive bidding among MEV searchers for transaction ordering priority. Drives MEV profits toward zero.

\vspace{0.3em}
\textbf{PBS (Proposer-Builder Separation)}
Architecture separating block building from block proposing. Searchers $\to$ builders $\to$ validators.

\column{0.48\textwidth}
\textbf{LVR (Loss-Versus-Rebalancing)}
Alternative to IL that measures LP cost as the difference between LP returns and a continuously rebalanced portfolio. More accurate than IL for measuring true LP costs.

\vspace{0.3em}
\textbf{JIT (Just-In-Time) Liquidity}
Providing liquidity for a single block to capture fees from a known trade. Dilutes passive LP income.

\vspace{0.3em}
\textbf{Virtual Reserves}
In Uniswap v3, the effective reserves within a concentrated range. Virtual reserves are larger than actual deposits by factor $\eta$.

\vspace{0.3em}
\textbf{Tick}
Minimum price increment in Uniswap v3, defined as $P(i) = 1.0001^i$. Each tick represents a 0.01\% price change (basis point).

\vspace{0.3em}
\textbf{Intent-Based Trading}
Users specify desired trade outcomes; solvers compete to execute optimally. Emerging MEV mitigation approach.
\end{columns}

\bottomnote{Master these terms for exam preparation; connect each to its mathematical model and economic intuition}
\end{frame}

\end{document}
