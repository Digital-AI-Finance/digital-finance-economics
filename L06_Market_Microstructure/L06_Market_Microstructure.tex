\documentclass[8pt,aspectratio=169]{beamer}
\usetheme{Madrid}
\usepackage{graphicx}
\usepackage{booktabs}
\usepackage{adjustbox}
\usepackage{multicol}
\usepackage{amsmath}

% Color definitions
\definecolor{mlblue}{RGB}{0,102,204}
\definecolor{mlpurple}{RGB}{51,51,178}
\definecolor{mllavender}{RGB}{173,173,224}
\definecolor{mllavender2}{RGB}{193,193,232}
\definecolor{mllavender3}{RGB}{204,204,235}
\definecolor{mllavender4}{RGB}{214,214,239}
\definecolor{mlorange}{RGB}{255, 127, 14}
\definecolor{mlgreen}{RGB}{44, 160, 44}
\definecolor{mlred}{RGB}{214, 39, 40}
\definecolor{mlgray}{RGB}{127, 127, 127}

\definecolor{lightgray}{RGB}{240, 240, 240}
\definecolor{midgray}{RGB}{180, 180, 180}

% Apply custom colors to Madrid theme
\setbeamercolor{palette primary}{bg=mllavender3,fg=mlpurple}
\setbeamercolor{palette secondary}{bg=mllavender2,fg=mlpurple}
\setbeamercolor{palette tertiary}{bg=mllavender,fg=white}
\setbeamercolor{palette quaternary}{bg=mlpurple,fg=white}

\setbeamercolor{structure}{fg=mlpurple}
\setbeamercolor{section in toc}{fg=mlpurple}
\setbeamercolor{subsection in toc}{fg=mlblue}
\setbeamercolor{title}{fg=mlpurple}
\setbeamercolor{frametitle}{fg=mlpurple,bg=mllavender3}
\setbeamercolor{block title}{bg=mllavender2,fg=mlpurple}
\setbeamercolor{block body}{bg=mllavender4,fg=black}

\setbeamertemplate{navigation symbols}{}
\setbeamertemplate{itemize items}[circle]
\setbeamertemplate{enumerate items}[default]
\setbeamersize{text margin left=5mm,text margin right=5mm}

\newcommand{\bottomnote}[1]{%
\vfill
\vspace{-2mm}
\textcolor{mllavender2}{\rule{\textwidth}{0.4pt}}
\vspace{1mm}
\footnotesize
\textbf{#1}
}

\title{Market Microstructure in Digital Finance}
\subtitle{L06: AMMs, Order Books, and Price Discovery}
\author{Economics of Digital Finance}
\institute{BSc Course}
\date{}

\begin{document}

% Title slide
\begin{frame}[plain]
\titlepage
\end{frame}

% Outline
\begin{frame}[t]{Lesson Overview}
\begin{columns}[T]
\column{0.48\textwidth}
\textbf{Today's Topics}
\begin{enumerate}
\item Traditional vs. DeFi market structure
\item Automated Market Makers (AMMs)
\item Order book mechanics and depth
\item Price discovery in fragmented markets
\item Impermanent loss for liquidity providers
\item MEV (Maximal Extractable Value---profit from reordering transactions) and front-running economics
\end{enumerate}

\column{0.48\textwidth}
\textbf{Learning Objectives}
\begin{itemize}
\item Compare order book and AMM mechanisms
\item Analyze liquidity provision economics
\item Understand price discovery in crypto markets
\item Evaluate MEV extraction strategies
\item Apply microstructure theory to DeFi
\end{itemize}
\end{columns}

\bottomnote{This lesson applies market microstructure theory to understand digital asset trading}
\end{frame}

% Traditional Market Structure
\begin{frame}[t]{Traditional vs. DeFi Market Structure}
\begin{columns}[T]
\column{0.48\textwidth}
\textbf{Centralized Exchanges (CEXs)}

\vspace{0.3em}
\textbf{Characteristics}
\begin{itemize}
\item CLOB (Central Limit Order Book---matches buy/sell orders by price)
\item Custodial trading
\item Professional market makers
\item High-frequency trading infrastructure
\end{itemize}

\vspace{0.3em}
\textbf{Examples}
\begin{itemize}
\item Binance, Coinbase, Kraken
\item NYSE, Nasdaq (traditional)
\end{itemize}

\column{0.48\textwidth}
\textbf{Decentralized Exchanges (DEXs)}

\vspace{0.3em}
\textbf{Characteristics}
\begin{itemize}
\item Automated market makers (AMMs)
\item Non-custodial trading
\item Permissionless liquidity provision
\item On-chain settlement
\end{itemize}

\vspace{0.3em}
\textbf{Examples}
\begin{itemize}
\item Uniswap, Curve, Balancer
\item PancakeSwap, SushiSwap
\end{itemize}
\end{columns}

\bottomnote{Economic question: Which structure provides better price discovery and lower transaction costs?}
\end{frame}

% Order Book Mechanics
\begin{frame}[t]{Order Book Mechanics and Liquidity}
\begin{columns}[T]
\column{0.48\textwidth}
\textbf{Order Book Structure}

\vspace{0.3em}
\textbf{Key Components}
\begin{itemize}
\item Bid side (buy orders)
\item Ask side (sell orders)
\item Spread: ask price - bid price
\item Depth: volume at each price level
\end{itemize}

\vspace{0.3em}
\textbf{Example}
If you buy 100 ETH and the order book is thin, you might pay \$1800 for first 50 and \$1850 for next 50.

\vspace{0.3em}
\textbf{Price Formation}
\begin{itemize}
\item Limit orders provide liquidity
\item Market orders consume liquidity
\item Spread compensates market makers
\end{itemize}

\column{0.48\textwidth}
\textbf{Economic Theory}

\vspace{0.3em}
Kyle (1985) model:
\begin{itemize}
\item Price impact: $\lambda = \frac{\sigma_v}{2\sigma_u}$ (higher $\lambda$ = bigger price move per trade)
\item Informed trading depth
\item Adverse selection (informed traders profit at uninformed traders' expense)
\end{itemize}

\vspace{0.3em}
Glosten-Milgrom (1985):
\begin{itemize}
\item Bid-ask spread reflects asymmetry
\item Sequential trade learning
\end{itemize}
\end{columns}

\bottomnote{Traditional microstructure assumes informed vs. uninformed traders; crypto markets add MEV extractors}
\end{frame}

% Order Book Depth Chart
\begin{frame}[t]{Order Book Depth Visualization}
\begin{center}
\includegraphics[width=0.65\textwidth]{02_order_book_depth/chart.pdf}
\end{center}

\bottomnote{Deeper order books reduce price impact for large trades; depth is a key liquidity metric}
\end{frame}

% AMM Introduction
\begin{frame}[t]{Automated Market Makers: The Constant Product Formula}
\begin{columns}[T]
\column{0.48\textwidth}
\textbf{How AMMs Work}

\vspace{0.3em}
\textbf{Constant Product Formula}
$$x \cdot y = k$$

where:
\begin{itemize}
\item $x$: reserve of token A
\item $y$: reserve of token B
\item $k$: constant (invariant)
\end{itemize}

This formula sets prices automatically: as traders buy token A, $x$ decreases, so $y$ must increase (keeping $k$ constant), which raises the price $P = \frac{y}{x}$.

\vspace{0.3em}
\textbf{Price Determination}
$$P = \frac{y}{x}$$

\textbf{Example:} In a pool with 100 ETH and 180,000 USDC, the price is $\frac{180,000}{100} = 1800$ USDC/ETH.

\column{0.48\textwidth}
\textbf{Economic Properties}

\vspace{0.3em}
\textbf{Advantages}
\begin{itemize}
\item Always provides liquidity
\item Permissionless participation
\item No order matching needed
\item Transparent pricing
\end{itemize}

\vspace{0.3em}
\textbf{Disadvantages}
\begin{itemize}
\item Slippage (price change between order and execution) on large trades
\item Impermanent loss for LPs (Liquidity Providers---people who deposit tokens)
\item Capital inefficiency
\item MEV vulnerability
\end{itemize}
\end{columns}

\bottomnote{Uniswap pioneered constant product AMM; now the dominant DeFi trading mechanism}
\end{frame}

% AMM Chart
\begin{frame}[t]{Constant Product Curve and Price Impact}
\begin{center}
\includegraphics[width=0.65\textwidth]{01_amm_constant_product/chart.pdf}
\end{center}

\bottomnote{Larger trades move farther along the curve, experiencing greater price impact (slippage)}
\end{frame}

% Liquidity Provision Economics
\begin{frame}[t]{Economics of Liquidity Provision}
\begin{columns}[T]
\column{0.48\textwidth}
\textbf{LP Revenue Streams}

\vspace{0.3em}
\textbf{Fee Income}
\begin{itemize}
\item Uniswap v2: 0.3\% per trade
\item Uniswap v3: 0.05\%, 0.3\%, 1\% tiers
\item Proportional to pool share
\end{itemize}

\vspace{0.3em}
\textbf{Incentive Programs}
\begin{itemize}
\item Liquidity mining rewards
\item Protocol token emissions
\item Governance rights
\end{itemize}

\column{0.48\textwidth}
\textbf{LP Costs and Risks}

\vspace{0.3em}
\textbf{Impermanent Loss}
\begin{itemize}
\item Loss from price divergence
\item Compared to holding assets
\item Amplified by volatility
\end{itemize}

\textbf{Example:} If you provide ETH/USDC liquidity and ETH doubles in price, you lose 5.7\% vs just holding ETH and USDC.

\vspace{0.3em}
\textbf{Other Risks}
\begin{itemize}
\item Smart contract risk
\item Gas costs for rebalancing
\item MEV extraction
\end{itemize}
\end{columns}

\bottomnote{LPs face a trade-off: fee income vs. impermanent loss; only profitable if fees exceed IL}
\end{frame}

% Impermanent Loss Chart
\begin{frame}[t]{Impermanent Loss Analysis}
\begin{center}
\includegraphics[width=0.65\textwidth]{04_impermanent_loss/chart.pdf}
\end{center}

\bottomnote{IL (Impermanent Loss) increases non-linearly with price divergence; 2x price change = 5.7\% loss, 5x change = 25.5\% loss}
\end{frame}

% Price Discovery
\begin{frame}[t]{Price Discovery in Fragmented Markets}
\begin{columns}[T]
\column{0.48\textwidth}
\textbf{Market Fragmentation}

\vspace{0.3em}
\textbf{Sources of Fragmentation}
\begin{itemize}
\item Multiple DEXs (Uniswap, Curve, etc.)
\item Multiple CEXs (Binance, Coinbase, etc.)
\item Cross-chain markets
\item Different trading pairs
\end{itemize}

\vspace{0.3em}
\textbf{Arbitrage (profiting from price differences) Mechanism}
\begin{itemize}
\item Arbitrageurs exploit price differences
\item Drive convergence across venues
\item Extract value from inefficiencies
\end{itemize}

\column{0.48\textwidth}
\textbf{Information Share}

\vspace{0.3em}
Hasbrouck (1995) information share (measures how much each venue contributes to price discovery---finding the "true" price):
$$IS_i = \frac{\text{Variance contribution of venue } i}{\text{Total price discovery variance}}$$

Higher $IS_i$ means venue $i$ leads in price discovery.

\vspace{0.3em}
\textbf{Empirical Findings}
\begin{itemize}
\item CEXs dominate price discovery
\item DEXs lag by seconds to minutes
\item Arbitrage costs determine efficiency
\item MEV complicates traditional metrics
\end{itemize}
\end{columns}

\bottomnote{Crypto markets test traditional price discovery theory due to permissionless arbitrage and MEV}
\end{frame}

% Price Discovery Chart
\begin{frame}[t]{Price Discovery Across Fragmented Venues}
\begin{center}
\includegraphics[width=0.65\textwidth]{03_price_discovery_fragmented/chart.pdf}
\end{center}

\bottomnote{Information shares vary by market conditions; CEXs typically lead, DEXs follow with lag}
\end{frame}

% MEV Introduction
\begin{frame}[t]{Maximal Extractable Value (MEV)}
\begin{columns}[T]
\column{0.48\textwidth}
\textbf{What is MEV?}

\vspace{0.3em}
\textbf{Definition}
\begin{itemize}
\item Value extractable by ordering transactions
\item Enabled by block producer control
\item Zero-sum redistribution (mostly)
\end{itemize}

\vspace{0.3em}
\textbf{Types of MEV}
\begin{itemize}
\item Front-running (trading ahead of someone else's order)
\item Back-running
\item Sandwich attacks (buying before and selling after a victim's trade)
\item Liquidations
\item Arbitrage
\end{itemize}

\column{0.48\textwidth}
\textbf{Economic Impact}

\vspace{0.3em}
\textbf{MEV Magnitude}
\begin{itemize}
\item \$600M+ extracted since 2020
\item 5-10\% of some DEX trades
\item Growing with DeFi adoption
\end{itemize}

\vspace{0.3em}
\textbf{Welfare Effects}
\begin{itemize}
\item Transfers wealth from traders to extractors
\item Increases transaction costs
\item May improve price efficiency
\item Potential consensus instability
\end{itemize}
\end{columns}

\bottomnote{MEV is a new phenomenon requiring new economic models beyond traditional market microstructure}
\end{frame}

% MEV Sandwich Attack Chart
\begin{frame}[t]{Anatomy of a Sandwich Attack}
\begin{center}
\includegraphics[width=0.65\textwidth]{05_mev_sandwich_attack/chart.pdf}
\end{center}

\bottomnote{Searchers front-run victim's buy, then back-run with sell, extracting value via price manipulation}
\end{frame}

% MEV Economics
\begin{frame}[t]{Economics of MEV Extraction}
\begin{columns}[T]
\column{0.48\textwidth}
\textbf{MEV Supply Chain}

\vspace{0.3em}
\textbf{Actors}
\begin{itemize}
\item Searchers: find MEV opportunities
\item Builders: construct blocks
\item Validators: propose blocks
\item Users: suffer MEV extraction
\end{itemize}

\vspace{0.3em}
\textbf{Competition Dynamics}
\begin{itemize}
\item Priority gas auctions (PGAs)
\item Flashbots auction mechanism
\item MEV-Boost infrastructure
\end{itemize}

\column{0.48\textwidth}
\textbf{Mitigation Strategies}

\vspace{0.3em}
\textbf{Protocol-Level}
\begin{itemize}
\item Encrypted mempools
\item Fair ordering protocols
\item Batch auctions
\item Threshold encryption
\end{itemize}

\vspace{0.3em}
\textbf{Application-Level}
\begin{itemize}
\item MEV-protected RPCs
\item Slippage tolerance limits
\item Time-weighted average pricing
\item Private order flow
\end{itemize}
\end{columns}

\bottomnote{MEV creates tension between validator revenue and user welfare; no clear solution yet}
\end{frame}

% Market Efficiency
\begin{frame}[t]{Market Efficiency in Crypto Markets}
\begin{columns}[T]
\column{0.48\textwidth}
\textbf{Efficiency Metrics}

\vspace{0.3em}
\textbf{Spread and Depth}
\begin{itemize}
\item Bid-ask spreads wider than TradFi (Traditional Finance---banks, stock exchanges)
\item Lower depth for most pairs
\item Improving over time
\end{itemize}

\vspace{0.3em}
\textbf{Price Discovery Speed}
\begin{itemize}
\item Fast arbitrage (seconds)
\item Cross-exchange efficiency
\item 24/7 trading advantage
\end{itemize}

\column{0.48\textwidth}
\textbf{Informational Efficiency}

\vspace{0.3em}
\textbf{Challenges}
\begin{itemize}
\item High retail participation
\item Limited fundamental anchors
\item Sentiment-driven volatility
\item Manipulation concerns
\end{itemize}

\vspace{0.3em}
\textbf{Advantages}
\begin{itemize}
\item Transparent on-chain data
\item Permissionless arbitrage
\item No trading halts
\item Global liquidity pools
\end{itemize}
\end{columns}

\bottomnote{Crypto markets exhibit mixed efficiency: fast arbitrage but high volatility and manipulation risk}
\end{frame}

% Comparing AMM and Order Book
\begin{frame}[t]{AMM vs. Order Book: Economic Comparison}
\begin{columns}[T]
\column{0.48\textwidth}
\textbf{Automated Market Makers}

\vspace{0.3em}
\textbf{Strengths}
\begin{itemize}
\item Guaranteed liquidity (always tradable)
\item Simple passive LP participation
\item Transparent pricing formula
\item No counterparty matching needed
\end{itemize}

\vspace{0.3em}
\textbf{Weaknesses}
\begin{itemize}
\item High price impact for large trades
\item Impermanent loss risk
\item Capital inefficiency (idle reserves)
\item MEV vulnerability
\end{itemize}

\column{0.48\textwidth}
\textbf{Order Books}

\vspace{0.3em}
\textbf{Strengths}
\begin{itemize}
\item Better for large trades (lower slippage)
\item Sophisticated order types (limit, stop)
\item Professional market making
\item Familiar interface
\end{itemize}

\vspace{0.3em}
\textbf{Weaknesses}
\begin{itemize}
\item Requires active market makers
\item Liquidity can disappear
\item Higher technical complexity
\item Custodial risk on CEXs
\end{itemize}
\end{columns}

\bottomnote{Optimal structure depends on asset liquidity, trade size, and user preferences; hybrid approaches emerging}
\end{frame}

% Advanced AMM Designs
\begin{frame}[t]{Advanced AMM Mechanisms}
\begin{columns}[T]
\column{0.48\textwidth}
\textbf{Uniswap v3: Concentrated Liquidity}

\vspace{0.3em}
\textbf{Innovation}
\begin{itemize}
\item LPs choose price ranges
\item Higher capital efficiency
\item Customizable fee tiers
\end{itemize}

\vspace{0.3em}
\textbf{Trade-offs}
\begin{itemize}
\item Active management required
\item Higher impermanent loss risk
\item Complexity barrier
\end{itemize}

\column{0.48\textwidth}
\textbf{Curve: Stableswap}

\vspace{0.3em}
\textbf{Innovation}
\begin{itemize}
\item Optimized for low-volatility pairs
\item Hybrid constant sum + constant product
\item Lower slippage for stablecoins
\end{itemize}

\vspace{0.3em}
\textbf{Balancer: Weighted Pools}
\begin{itemize}
\item Multi-asset pools
\item Customizable weights
\item Index fund functionality
\end{itemize}
\end{columns}

\bottomnote{AMM innovation continues; trend toward capital efficiency and specialization by asset type}
\end{frame}

% Empirical Evidence
\begin{frame}[t]{Empirical Evidence on DeFi Market Microstructure}
\begin{columns}[T]
\column{0.48\textwidth}
\textbf{Key Findings}

\vspace{0.3em}
\textbf{Liquidity and Trading Costs}
\begin{itemize}
\item DEX spreads 2-5x CEX spreads
\item Improving with liquidity growth
\item Uniswap v3 narrows gap
\end{itemize}

\vspace{0.3em}
\textbf{Impermanent Loss}
\begin{itemize}
\item Most LPs lose vs. HODL (Hold On for Dear Life---crypto slang for holding)
\item Fee income often insufficient
\item Incentive programs crucial
\end{itemize}

\column{0.48\textwidth}
\textbf{MEV Impact}

\vspace{0.3em}
\textbf{Magnitude}
\begin{itemize}
\item 5-10\% implicit cost on some trades
\item Concentrated in large trades
\item Growing with DeFi TVL (Total Value Locked---money deposited in DeFi)
\end{itemize}

\vspace{0.3em}
\textbf{Price Discovery}
\begin{itemize}
\item CEXs lead by 30-60 seconds
\item Arbitrage profitable despite costs
\item Cross-chain discovery slower
\end{itemize}
\end{columns}

\bottomnote{Research from Lehar \& Parlour (2021), Barbon \& Ranaldo (2022), Adams et al. (2024) on AMMs and MEV}
\end{frame}

% Future Directions
\begin{frame}[t]{Future Directions in DeFi Microstructure}
\begin{columns}[T]
\column{0.48\textwidth}
\textbf{Technical Innovation}

\vspace{0.3em}
\textbf{Emerging Mechanisms}
\begin{itemize}
\item Dynamic fees (volatility-adjusted)
\item Just-in-time liquidity
\item Intent-based architectures
\item ZK-rollup order books
\end{itemize}

\vspace{0.3em}
\textbf{MEV Solutions}
\begin{itemize}
\item Encrypted mempools
\item Fair ordering protocols
\item MEV redistribution to users
\end{itemize}

\column{0.48\textwidth}
\textbf{Economic Research Needs}

\vspace{0.3em}
\textbf{Open Questions}
\begin{itemize}
\item Optimal LP compensation design
\item MEV welfare impact measurement
\item Cross-chain price discovery
\item Decentralized governance efficiency
\end{itemize}

\vspace{0.3em}
\textbf{Policy Implications}
\begin{itemize}
\item Market manipulation regulation
\item Investor protection in DeFi
\item Systemic risk from MEV
\end{itemize}
\end{columns}

\bottomnote{DeFi market microstructure is rapidly evolving; economic theory must adapt to new mechanisms}
\end{frame}

% Key Takeaways
\begin{frame}[t]{Key Takeaways}
\begin{columns}[T]
\column{0.48\textwidth}
\textbf{Core Concepts}
\begin{enumerate}
\item AMMs use constant product formula for automated trading
\item Order books rely on active market makers
\item Impermanent loss is key LP risk
\item MEV extracts value via transaction ordering
\item Price discovery faster but less efficient than TradFi
\end{enumerate}

\column{0.48\textwidth}
\textbf{Economic Insights}
\begin{itemize}
\item Trade-offs between decentralization and efficiency
\item Liquidity provision requires compensation for IL
\item MEV creates new welfare considerations
\item Market fragmentation enables arbitrage
\item Innovation continues with v3, Curve, etc.
\end{itemize}
\end{columns}

\vspace{0.5em}
\textbf{Application to Practice}

Market microstructure theory helps evaluate DEX design, understand LP economics, and assess efficiency of crypto trading venues.

\bottomnote{Next lesson: Regulatory Economics of Digital Finance}
\end{frame}

% Key Terms
\begin{frame}[t]{Key Terms (1/2)}
\begin{columns}[T]
\column{0.48\textwidth}
\textbf{AMM (Automated Market Maker)}
Smart contract providing liquidity through algorithmic pricing (e.g., constant product formula $x \cdot y = k$).

\vspace{0.3em}
\textbf{Order Book}
Traditional market structure listing buy and sell orders at various prices.

\vspace{0.3em}
\textbf{Liquidity Provider (LP)}
Party depositing assets into AMM pool to enable trading, earning fees in return.

\vspace{0.3em}
\textbf{Impermanent Loss}
Loss LPs experience when asset prices diverge from deposit ratio, compared to simply holding.

\vspace{0.3em}
\textbf{MEV (Maximal Extractable Value)}
Profit extractable by reordering, inserting, or censoring transactions within a block.

\vspace{0.3em}
\textbf{Price Discovery}
Process by which market determines asset prices through trading activity and information aggregation.

\vspace{0.3em}
\textbf{Market Microstructure}
How trading actually works at the detailed level (order matching, price formation, liquidity provision).

\column{0.48\textwidth}
\textbf{Adverse Selection}
Informed traders profit at uninformed traders' expense, causing market makers to widen spreads.

\vspace{0.3em}
\textbf{Slippage}
Price change between when you submit an order and when it executes, especially on large trades.

\vspace{0.3em}
\textbf{Price Impact}
How much your trade moves the market price; larger trades have higher price impact.

\vspace{0.3em}
\textbf{Arbitrage}
Profiting from price differences across markets (e.g., buying ETH on one exchange and selling on another).

\vspace{0.3em}
\textbf{Front-Running}
Trading ahead of someone else's order to profit from the expected price change.

\vspace{0.3em}
\textbf{Sandwich Attack}
MEV strategy: buy before victim's trade, then sell after, profiting from price manipulation.

\vspace{0.3em}
\textbf{CLOB (Central Limit Order Book)}
Order book matching system that pairs buy and sell orders by price-time priority.
\end{columns}

\bottomnote{Market microstructure concepts are essential for understanding DeFi efficiency and risks}
\end{frame}

\begin{frame}[t]{Key Terms (2/2)}
\begin{columns}[T]
\column{0.48\textwidth}
\textbf{Constant Product Formula}
AMM pricing rule $x \cdot y = k$ that automatically adjusts prices as traders swap tokens.

\vspace{0.3em}
\textbf{TradFi (Traditional Finance)}
Banks, stock exchanges, and conventional financial institutions (vs. DeFi).

\vspace{0.3em}
\textbf{TVL (Total Value Locked)}
Total amount of money deposited in DeFi protocols, a measure of adoption and liquidity.

\vspace{0.3em}
\textbf{Gas Fees}
Transaction fees paid to blockchain validators for processing transactions (high gas = expensive trades).

\column{0.48\textwidth}
\textbf{Block Time}
Time between new blocks added to blockchain (e.g., 12 seconds on Ethereum); determines MEV opportunities.

\vspace{0.3em}
\textbf{Bid-Ask Spread}
Difference between highest buy price and lowest sell price; narrower spreads = more liquid markets.

\vspace{0.3em}
\textbf{Order Book Depth}
Volume of buy/sell orders at each price level; deeper books absorb large trades without price impact.

\vspace{0.3em}
\textbf{HODL (Hold On for Dear Life)}
Crypto slang for holding assets long-term instead of trading; origin of "impermanent loss vs. HODL" comparison.
\end{columns}

\bottomnote{Understanding these terms is crucial for analyzing DeFi trading efficiency and risks}
\end{frame}

% References
\begin{frame}[t]{Further Reading}
\begin{columns}[T]
\column{0.48\textwidth}
\textbf{Foundational Papers}
\begin{itemize}
\item Kyle (1985): ``Continuous Auctions and Insider Trading''
\item Hasbrouck (1995): ``One Security, Many Markets''
\item Adams et al. (2024): ``Uniswap v3: The Economics of Concentrated Liquidity''
\item Lehar \& Parlour (2021): ``Systemic Fragility in Decentralized Markets''
\end{itemize}

\column{0.48\textwidth}
\textbf{MEV Research}
\begin{itemize}
\item Daian et al. (2020): ``Flash Boys 2.0''
\item Qin et al. (2022): ``Quantifying MEV on Ethereum''
\item Barbon \& Ranaldo (2022): ``On the Quality of Cryptocurrency Markets''
\end{itemize}
\end{columns}

\bottomnote{All readings available on course platform}
\end{frame}

\end{document}
