\documentclass[8pt,aspectratio=169]{beamer}
\usetheme{Madrid}
\usepackage{graphicx}
\usepackage{booktabs}
\usepackage{adjustbox}
\usepackage{multicol}
\usepackage{amsmath}

% Color definitions
\definecolor{mlblue}{RGB}{0,102,204}
\definecolor{mlpurple}{RGB}{51,51,178}
\definecolor{mllavender}{RGB}{173,173,224}
\definecolor{mllavender2}{RGB}{193,193,232}
\definecolor{mllavender3}{RGB}{204,204,235}
\definecolor{mllavender4}{RGB}{214,214,239}
\definecolor{mlorange}{RGB}{255, 127, 14}
\definecolor{mlgreen}{RGB}{44, 160, 44}
\definecolor{mlred}{RGB}{214, 39, 40}
\definecolor{mlgray}{RGB}{127, 127, 127}

\definecolor{lightgray}{RGB}{240, 240, 240}
\definecolor{midgray}{RGB}{180, 180, 180}

% Apply custom colors to Madrid theme
\setbeamercolor{palette primary}{bg=mllavender3,fg=mlpurple}
\setbeamercolor{palette secondary}{bg=mllavender2,fg=mlpurple}
\setbeamercolor{palette tertiary}{bg=mllavender,fg=white}
\setbeamercolor{palette quaternary}{bg=mlpurple,fg=white}

\setbeamercolor{structure}{fg=mlpurple}
\setbeamercolor{section in toc}{fg=mlpurple}
\setbeamercolor{subsection in toc}{fg=mlblue}
\setbeamercolor{title}{fg=mlpurple}
\setbeamercolor{frametitle}{fg=mlpurple,bg=mllavender3}
\setbeamercolor{block title}{bg=mllavender2,fg=mlpurple}
\setbeamercolor{block body}{bg=mllavender4,fg=black}

\setbeamertemplate{navigation symbols}{}
\setbeamertemplate{itemize items}[circle]
\setbeamertemplate{enumerate items}[default]
\setbeamersize{text margin left=5mm,text margin right=5mm}

\newcommand{\bottomnote}[1]{%
\vfill
\vspace{-2mm}
\textcolor{mllavender2}{\rule{\textwidth}{0.4pt}}
\vspace{1mm}
\footnotesize
\textbf{#1}
}

\title{Market Microstructure in Digital Finance}
\subtitle{L06: When you swap ETH for USDC on Uniswap, who sets the price---and who profits?}
\author{Economics of Digital Finance}
\institute{BSc Course}
\date{}

\begin{document}

% Title slide
\begin{frame}[plain]
\titlepage
\end{frame}

% Outline
\begin{frame}[t]{Lesson Overview}

\textbf{Connection to L05:} In Lesson 5 we studied how platforms compete and tokens create incentives. Today we look inside the trading mechanisms---how do prices form, and who provides liquidity?

Market microstructure is the study of how trading works at the detailed level: how orders are matched, how prices form, and who provides liquidity.

\vspace{0.3em}
\begin{columns}[T]
\column{0.48\textwidth}
\textbf{Today's Topics}
\begin{enumerate}
\item Traditional vs. DeFi market structure
\item Automated Market Makers (AMMs)
\item Order book mechanics and depth
\item Price discovery in fragmented markets
\item Impermanent loss for liquidity providers
\item MEV (Maximal Extractable Value---profit from reordering transactions) and front-running economics
\end{enumerate}

\column{0.48\textwidth}
\textbf{Learning Objectives}
\begin{itemize}
\item Compare order book and AMM mechanisms
\item Analyze liquidity provision economics
\item Understand price discovery in crypto markets
\item Evaluate MEV extraction strategies
\item Apply microstructure theory to DeFi
\end{itemize}
\end{columns}

\bottomnote{This lesson applies market microstructure theory to understand digital asset trading}
\end{frame}

% Traditional Market Structure
\begin{frame}[t]{Traditional vs. DeFi Market Structure}
\begin{columns}[T]
\column{0.48\textwidth}
\textbf{Centralized Exchanges (CEXs)}

\vspace{0.3em}
\textbf{Characteristics}
\begin{itemize}
\item CLOB (Central Limit Order Book---matches buy/sell orders by price)
\item Custodial trading (the exchange holds your assets on your behalf)
\item Professional market makers (firms that continuously offer to buy and sell, providing liquidity)
\item High-frequency trading (automated trading at microsecond speeds) infrastructure
\end{itemize}

\vspace{0.3em}
\textbf{Examples}
\begin{itemize}
\item Binance, Coinbase, Kraken
\item NYSE, Nasdaq (traditional)
\end{itemize}

\column{0.48\textwidth}
\textbf{Decentralized Exchanges (DEXs)}

\vspace{0.3em}
\textbf{Characteristics}
\begin{itemize}
\item Automated market makers (AMMs)
\item Non-custodial trading (you keep control of your own assets)
\item Permissionless liquidity provision
\item On-chain settlement (transactions recorded directly on the blockchain)
\end{itemize}

\vspace{0.3em}
\textbf{Examples}
\begin{itemize}
\item Uniswap, Curve, Balancer
\item PancakeSwap, SushiSwap
\end{itemize}
\end{columns}

\bottomnote{Economic question: Which structure provides better price discovery and lower transaction costs?}
\end{frame}

% Order Book Mechanics (Part 1: Basics)
\begin{frame}[t]{Order Book Basics}
\begin{columns}[T]
\column{0.48\textwidth}
\textbf{Order Book Structure}

\vspace{0.3em}
\textbf{Key Components}
\begin{itemize}
\item Bid side (buy orders)
\item Ask side (sell orders)
\item Spread: ask price $-$ bid price
\item Depth: volume at each price level
\end{itemize}

\vspace{0.3em}
\textbf{Two Order Types}
\begin{itemize}
\item \textbf{Limit orders} (offers at a specific price, waiting in the book)---provide liquidity
\item \textbf{Market orders} (buy or sell immediately at the best available price)---consume liquidity
\item The spread compensates market makers for the risk of holding inventory
\end{itemize}

\column{0.48\textwidth}
\textbf{Price Impact Example}

If you buy 100 ETH and the order book is thin (few orders near current price), you might pay \$1{,}800 for the first 50 and \$1{,}850 for the next 50. This is \textbf{price impact}: your trade moved the price against you.

\vspace{0.3em}
\textbf{Adverse Selection}

Informed traders (those with private information about the asset's true value) buy when the market underprices and sell when it overprices, systematically profiting at market makers' expense. Market makers respond by widening spreads to protect themselves---making trading costlier for everyone.

\vspace{0.3em}
\textbf{Why This Matters for Crypto}

Crypto markets have higher information asymmetry than traditional markets: insiders may know about hacks, delistings, or protocol upgrades before the public.
\end{columns}

\bottomnote{Order books match buyers and sellers by price priority; market makers earn the spread but face adverse selection risk}
\end{frame}

% Order Book Mechanics (Part 2: Price Impact Theory)
\begin{frame}[t]{Price Impact Theory}
\begin{columns}[T]
\column{0.48\textwidth}
\textbf{Kyle (1985) Model}

Price impact per unit traded:
$$\lambda = \frac{\sigma_v}{2\sigma_u}$$

where:
\begin{itemize}
\item $\sigma_v$ = std.\ dev.\ of the asset's true value (how much private information moves the price)
\item $\sigma_u$ = std.\ dev.\ of noise trading (uninformed trading volume)
\item Higher $\lambda$ = bigger price move per unit traded
\end{itemize}

\smallskip\textbf{Example:} For Bitcoin: $\sigma_v = 0.015$ (high information asymmetry), $\sigma_u = 0.10$ (moderate noise trading). $\lambda = \frac{0.015}{2 \times 0.10} = 0.075$.

For S\&P 500: $\sigma_v = 0.005$, $\sigma_u = 0.20$. $\lambda = \frac{0.005}{2 \times 0.20} = 0.0125$.

Bitcoin trades move prices 6$\times$ more per unit---reflecting higher insider risk.

\column{0.48\textwidth}
\textbf{Glosten-Milgrom (1985)}

\vspace{0.3em}
\begin{itemize}
\item Bid-ask spread reflects information asymmetry
\item Sequential trade learning: market makers update beliefs about the asset's true value after each trade
\item More informed traders $\to$ wider spreads
\end{itemize}

\vspace{0.3em}
\textbf{Crypto vs.\ Traditional Markets}

\vspace{0.3em}
\begin{tabular}{lcc}
\toprule
\textbf{Metric} & \textbf{BTC} & \textbf{S\&P 500} \\
\midrule
$\lambda$ & 0.075 & 0.0125 \\
Spread & 0.05\% & 0.01\% \\
Depth & Low & Deep \\
\bottomrule
\end{tabular}

\vspace{0.3em}
Crypto markets add a new actor beyond Kyle's model: \textbf{MEV extractors} who profit from transaction ordering, not information.
\end{columns}

\bottomnote{Kyle's $\lambda$ measures price impact per unit traded; deeper order books and more noise trading reduce $\lambda$}
\end{frame}

% Order Book Depth Chart
\begin{frame}[t]{Order Book Depth Visualization}
\begin{center}
\includegraphics[width=0.65\textwidth]{02_order_book_depth/chart.pdf}
\end{center}

\bottomnote{Kyle's $\lambda$ measures price impact per unit traded; deeper order books reduce $\lambda$. Compare Bitcoin ($\lambda \approx 0.075$) vs. S\&P 500 ($\lambda \approx 0.0125$)}
\end{frame}

% AMM Introduction
\begin{frame}[t]{Automated Market Makers: The Constant Product Formula}
\begin{columns}[T]
\column{0.48\textwidth}
\textbf{How AMMs Work}

\vspace{0.3em}
\textbf{Constant Product Formula}
$$x \cdot y = k$$

where:
\begin{itemize}
\item $x$: reserve of token A
\item $y$: reserve of token B
\item $k$: constant (invariant)
\end{itemize}

This formula sets prices automatically: as traders buy token A, $x$ decreases, so $y$ must increase (keeping $k$ constant), which raises the price $P = \frac{y}{x}$.

\vspace{0.3em}
\textbf{Price Determination}
$$P = \frac{y}{x}$$

\textbf{Example:} In a pool with 100 ETH and 180,000 USDC, the price is $\frac{180{,}000}{100} = 1{,}800$ USDC/ETH, with $k = 100 \times 180{,}000 = 18{,}000{,}000$.

\smallskip\textbf{Worked trade:} Buy 10 ETH. Pool goes from 100 to 90 ETH. New USDC reserve: $\frac{18{,}000{,}000}{90} = 200{,}000$. You pay $200{,}000 - 180{,}000 = 20{,}000$ USDC for 10 ETH $= \$2{,}000$/ETH. That is 11\% above the spot price of \$1{,}800---this is \textit{slippage}.

\column{0.48\textwidth}
\textbf{Economic Properties}

\vspace{0.3em}
\textbf{Advantages}
\begin{itemize}
\item Always provides liquidity
\item Permissionless participation
\item No order matching needed
\item Transparent pricing
\end{itemize}

\vspace{0.3em}
\textbf{Disadvantages}
\begin{itemize}
\item Slippage (price change between order and execution) on large trades
\item Impermanent loss for LPs (Liquidity Providers---people who deposit tokens)
\item Capital inefficiency (in Uniswap v2, only $\sim$5\% of deposited capital is actively used near the current price; the rest sits idle at extreme prices)
\item MEV vulnerability
\end{itemize}
\end{columns}

\bottomnote{Uniswap pioneered constant product AMM; now the dominant DeFi trading mechanism}
\end{frame}

% AMM Chart
\begin{frame}[t]{Constant Product Curve and Price Impact}
\begin{center}
\includegraphics[width=0.65\textwidth]{01_amm_constant_product/chart.pdf}
\end{center}

\bottomnote{Larger trades move farther along the curve, experiencing greater price impact (slippage)}
\end{frame}

% Liquidity Provision Economics
\begin{frame}[t]{Economics of Liquidity Provision}
\begin{columns}[T]
\column{0.48\textwidth}
\textbf{LP Revenue Streams}

\vspace{0.3em}
\textbf{Fee Income}
\begin{itemize}
\item Uniswap v2: 0.3\% per trade
\item Uniswap v3: 0.05\%, 0.3\%, 1\% tiers
\item Proportional to pool share
\end{itemize}

\vspace{0.3em}
\textbf{Incentive Programs}
\begin{itemize}
\item Liquidity mining rewards (free tokens given to LPs as incentive)
\item Protocol token emissions (new tokens created and distributed as rewards)
\item Governance rights
\end{itemize}

\column{0.48\textwidth}
\textbf{LP Costs and Risks}

\vspace{0.3em}
\textbf{Impermanent Loss}
\begin{itemize}
\item Loss from price divergence
\item Compared to holding assets
\item Amplified by volatility
\end{itemize}

IL formula: $IL = \frac{2\sqrt{r}}{1+r} - 1$ where $r = P_{\text{final}}/P_{\text{initial}}$ (ETH dollar price at withdrawal / at deposit; USDC assumed stable).

\textit{Why $\sqrt{r}$?} The constant product formula forces the LP's ETH quantity to scale as $1/\sqrt{r}$---so if ETH doubles, the pool rebalances by selling ETH, leaving the LP with $\sqrt{2} \times$ original value while a holder gets $1.5\times$. The difference is IL.

\textbf{Example:} ETH doubles ($r=2$): $IL = \frac{2\sqrt{2}}{3} - 1 = -5.7\%$ vs.\ holding. ETH 5$\times$ ($r=5$): $IL = -25.5\%$.

\vspace{0.3em}
\textbf{Other Risks}
\begin{itemize}
\item Smart contract (self-executing code on the blockchain) risk
\item Gas costs for rebalancing
\item MEV extraction
\end{itemize}
\end{columns}

\bottomnote{LPs face a trade-off: fee income vs. impermanent loss; only profitable if fees exceed IL}
\end{frame}

% Impermanent Loss Chart
\begin{frame}[t]{Impermanent Loss Analysis}
\begin{center}
\includegraphics[width=0.65\textwidth]{04_impermanent_loss/chart.pdf}
\end{center}

\bottomnote{IL (Impermanent Loss) increases non-linearly with price divergence; 2x price change = 5.7\% loss, 5x change = 25.5\% loss}
\end{frame}

% Price Discovery
\begin{frame}[t]{Price Discovery in Fragmented Markets}
\begin{columns}[T]
\column{0.48\textwidth}
\textbf{Market Fragmentation}

\vspace{0.3em}
\textbf{Sources of Fragmentation}
\begin{itemize}
\item Multiple DEXs (Uniswap, Curve, etc.)
\item Multiple CEXs (Binance, Coinbase, etc.)
\item Cross-chain markets (trading across different blockchains, e.g., Ethereum and Solana)
\item Different trading pairs
\end{itemize}

\vspace{0.3em}
\textbf{Arbitrage (profiting from price differences) Mechanism}
\begin{itemize}
\item Arbitrageurs exploit price differences
\item Drive convergence across venues
\item Extract value from inefficiencies
\end{itemize}

\column{0.48\textwidth}
\textbf{Information Share}

\vspace{0.3em}
Hasbrouck (1995) information share (measures how much each venue contributes to price discovery---finding the ``true'' price):
$$IS_i = \frac{(\beta_i \cdot \sigma_i)^2}{\sum_j (\beta_j \cdot \sigma_j)^2}$$

where $\beta_i$ is venue $i$'s price-impact coefficient and $\sigma_i$ is its innovation variance. Higher $IS_i$ means venue $i$ leads in price discovery---it incorporates new information first.

\smallskip\textbf{Note:} IS measures \textit{variance contribution}, not temporal order. A venue can have high IS without always moving first.

\textbf{Example:} If CEX contributes 64\% of price variance, DEX 28\%, and P2P 8\%, then $IS_{CEX} = 0.64$, $IS_{DEX} = 0.28$, $IS_{P2P} = 0.08$. The CEX leads price discovery.

\vspace{0.3em}
\textbf{Empirical Findings}
\begin{itemize}
\item CEXs dominate price discovery
\item DEXs lag by seconds to minutes
\item Arbitrage costs determine efficiency
\item MEV complicates traditional metrics
\end{itemize}
\end{columns}

\bottomnote{Crypto markets test traditional price discovery theory due to permissionless arbitrage and MEV}
\end{frame}

% Price Discovery Chart
\begin{frame}[t]{Price Discovery Across Fragmented Venues}
\begin{center}
\includegraphics[width=0.65\textwidth]{03_price_discovery_fragmented/chart.pdf}
\end{center}

\bottomnote{CEXs typically lead price discovery with $\sim$60--65\% information share; DEXs follow with seconds-to-minutes lag}
\end{frame}

% MEV Introduction
\begin{frame}[t]{Maximal Extractable Value (MEV)}
\begin{columns}[T]
\column{0.48\textwidth}
\textbf{What is MEV?}

\vspace{0.3em}
\textbf{Definition}
\begin{itemize}
\item Value extractable by ordering transactions
\item Enabled by block producer (the entity assembling the next block) control
\item Zero-sum redistribution (mostly)
\end{itemize}

\vspace{0.3em}
\textbf{Types of MEV}
\begin{itemize}
\item Front-running (trading ahead of someone else's order)
\item Back-running (placing a trade immediately \textit{after} someone else's large trade to profit from the resulting price change)
\item Sandwich attacks (buying before and selling after a victim's trade)
\item Liquidations
\item Arbitrage
\end{itemize}

\column{0.48\textwidth}
\textbf{Economic Impact}

\vspace{0.3em}
\textbf{MEV Magnitude}
\begin{itemize}
\item Over \$1.5B+ extracted on Ethereum alone since 2020 (Flashbots, 2024)
\item 5-10\% of some DEX trades
\item Growing with DeFi adoption
\end{itemize}

\vspace{0.3em}
\textbf{Welfare Effects}
\begin{itemize}
\item Transfers wealth from traders to extractors
\item Increases transaction costs
\item May improve price efficiency
\item Potential consensus instability
\end{itemize}
\end{columns}

\bottomnote{MEV is a new phenomenon requiring new economic models beyond traditional market microstructure}
\end{frame}

% MEV Sandwich Attack Chart
\begin{frame}[t]{Anatomy of a Sandwich Attack}
\begin{center}
\includegraphics[width=0.65\textwidth]{05_mev_sandwich_attack/chart.pdf}
\end{center}

\bottomnote{Full sandwich profit: $\pi = Q_f(P'-P) - Q_f \cdot f_{\text{AMM}} \cdot (P + P') - 2 \cdot \text{gas} - c_{\text{PGA}}$ where $f_{\text{AMM}}$ is the AMM fee (e.g., 0.3\%) and $c_{\text{PGA}}$ is the priority gas auction cost (the bribe to get the transaction ordered correctly). Profitable only when price impact exceeds all costs.}
\end{frame}

% MEV Economics (Part 1: Supply Chain)
\begin{frame}[t]{MEV Supply Chain}
\begin{columns}[T]
\column{0.48\textwidth}
\textbf{Actors in the MEV Pipeline}

\vspace{0.3em}
\begin{enumerate}
\item \textbf{Searchers} (bots scanning the mempool for profitable transaction orderings): find MEV opportunities
\item \textbf{Builders} (entities assembling transactions into candidate blocks): construct blocks
\item \textbf{Validators} (nodes that propose and finalize blocks on the network): propose blocks
\item \textbf{Users}: suffer MEV extraction
\end{enumerate}

\vspace{0.3em}
\textbf{Competition Dynamics}
\begin{itemize}
\item Priority gas auctions (PGAs---users bid up gas fees to get their transactions processed first)
\item Flashbots auction (private marketplace where searchers bid for block inclusion without spamming the network)
\item MEV-Boost (middleware connecting validators to block builders for fair MEV distribution)
\end{itemize}

\column{0.48\textwidth}
\textbf{Sandwich Profit: Full Calculation}

Pool: 100 ETH / 180{,}000 USDC ($k = 18$M).

\vspace{0.2em}
\textbf{Step 1 -- Front-run:} Buy 5 ETH.

New USDC: $18{,}000{,}000 / 95 = 189{,}474$. Cost: 9{,}474 USDC.

New price: $189{,}474/95 = \$1{,}994$/ETH.

\vspace{0.2em}
\textbf{Step 2 -- Victim trades:} Buys 5 ETH at inflated price.

\vspace{0.2em}
\textbf{Step 3 -- Back-run:} Sell 5 ETH back.

After victim: 85 ETH / $18$M$/85 = 211{,}765$ USDC.

Sell 5 ETH: pool $\to$ 90 ETH / $18$M$/90 = 200{,}000$ USDC.

Receive: $211{,}765 - 200{,}000 = 11{,}765$ USDC.

\vspace{0.2em}
\textbf{Profit:} $11{,}765 - 9{,}474 = \$2{,}291$ minus gas ($\sim$\$6) minus 0.3\% AMM fees ($\sim$\$63) $\approx \$2{,}222$.
\end{columns}

\bottomnote{MEV profit depends on pool depth, trade size, and competition among searchers; most MEV is competed away via PGAs}
\end{frame}

% MEV Economics (Part 2: Mitigation)
\begin{frame}[t]{MEV Mitigation Strategies}
\begin{columns}[T]
\column{0.48\textwidth}
\textbf{Protocol-Level Solutions}

\vspace{0.3em}
\begin{itemize}
\item \textbf{Encrypted mempools}---hiding pending transactions so searchers cannot see them; the mempool is the waiting area for unconfirmed transactions
\item \textbf{Fair ordering protocols}---process transactions in the order received, not by gas price
\item \textbf{Batch auctions}---collect all trades in a time window and execute at one clearing price
\item \textbf{Threshold encryption}---multiple parties must cooperate to decrypt transaction contents
\end{itemize}

\column{0.48\textwidth}
\textbf{Application-Level Solutions}

\vspace{0.3em}
\begin{itemize}
\item \textbf{MEV-protected RPCs}---private endpoints for submitting transactions invisible to searchers
\item \textbf{Slippage tolerance limits}---reject trades if price moves too much
\item \textbf{TWAP orders} (time-weighted average pricing)---split a large trade over time to reduce impact
\item \textbf{Private order flow}---route trades through channels hidden from MEV searchers
\end{itemize}

\vspace{0.3em}
\textbf{The Fundamental Tension}

MEV creates revenue for validators (which secures the network) but imposes costs on users. Eliminating MEV entirely may reduce network security. The research frontier seeks to \textit{redistribute} MEV back to users rather than eliminate it.
\end{columns}

\bottomnote{MEV creates tension between validator revenue and user welfare; no clear solution yet}
\end{frame}

% Market Efficiency
\begin{frame}[t]{Market Efficiency in Crypto Markets}

\textbf{What is market efficiency?} The Efficient Market Hypothesis (EMH) states that prices quickly reflect all available information---so nobody can consistently ``beat the market.'' We ask: how efficient are crypto markets compared to traditional ones?

\begin{columns}[T]
\column{0.48\textwidth}
\textbf{Efficiency Metrics}

\vspace{0.3em}
\textbf{Spread and Depth}
\begin{itemize}
\item Bid-ask spreads wider than TradFi (Traditional Finance---banks, stock exchanges)
\item Lower depth for most pairs
\item Improving over time
\end{itemize}

\vspace{0.3em}
\textbf{Price Discovery Speed}
\begin{itemize}
\item Fast arbitrage (seconds)
\item Cross-exchange efficiency
\item 24/7 trading advantage
\end{itemize}

\column{0.48\textwidth}
\textbf{Informational Efficiency}

\vspace{0.3em}
\textbf{Challenges}
\begin{itemize}
\item High retail participation
\item Limited fundamental anchors
\item Sentiment-driven volatility
\item Manipulation concerns
\end{itemize}

\vspace{0.3em}
\textbf{Advantages}
\begin{itemize}
\item Transparent on-chain data
\item Permissionless arbitrage
\item No trading halts
\item Global liquidity pools
\end{itemize}
\end{columns}

\bottomnote{Crypto markets exhibit mixed efficiency: fast arbitrage but high volatility and manipulation risk}
\end{frame}

% Comparing AMM and Order Book
\begin{frame}[t]{AMM vs. Order Book: Economic Comparison}
\begin{columns}[T]
\column{0.48\textwidth}
\textbf{Automated Market Makers}

\vspace{0.3em}
\textbf{Strengths}
\begin{itemize}
\item Guaranteed liquidity (always tradable)
\item Simple passive LP participation
\item Transparent pricing formula
\item No counterparty matching needed
\end{itemize}

\vspace{0.3em}
\textbf{Weaknesses}
\begin{itemize}
\item High price impact for large trades
\item Impermanent loss risk
\item Capital inefficiency (idle reserves)
\item MEV vulnerability
\end{itemize}

\column{0.48\textwidth}
\textbf{Order Books}

\vspace{0.3em}
\textbf{Strengths}
\begin{itemize}
\item Better for large trades (lower slippage)
\item Sophisticated order types (limit, stop)
\item Professional market making
\item Familiar interface
\end{itemize}

\vspace{0.3em}
\textbf{Weaknesses}
\begin{itemize}
\item Requires active market makers
\item Liquidity can disappear
\item Higher technical complexity
\item Custodial risk on CEXs
\end{itemize}
\end{columns}

\bottomnote{Optimal structure depends on asset liquidity, trade size, and user preferences; hybrid approaches emerging}
\end{frame}

% Advanced AMM Designs
\begin{frame}[t]{Advanced AMM Mechanisms}
\begin{columns}[T]
\column{0.48\textwidth}
\textbf{Uniswap v3: Concentrated Liquidity}

\vspace{0.3em}
\textbf{Innovation}
\begin{itemize}
\item LPs choose price ranges
\item Higher capital efficiency
\item Customizable fee tiers
\end{itemize}

\vspace{0.3em}
\textbf{Trade-offs}
\begin{itemize}
\item Active management required
\item Higher impermanent loss risk
\item Complexity barrier
\end{itemize}

\smallskip\textbf{Example:} \$100K concentrated in the \$1{,}700--\$1{,}900 range provides the same depth as \$2M spread across all prices in v2---a 20$\times$ capital efficiency improvement.

\column{0.48\textwidth}
\textbf{Curve: Stableswap}

\vspace{0.3em}
\textbf{Innovation}
\begin{itemize}
\item Optimized for low-volatility pairs
\item Blends constant sum ($x+y=k$, zero slippage but can run dry) with constant product ($xy=k$, infinite liquidity but high slippage). Near the peg, Curve behaves like constant sum; far from it, like constant product.
\item Result: 100$\times$ lower slippage for stablecoin-to-stablecoin swaps
\end{itemize}

\vspace{0.3em}
\textbf{Balancer: Weighted Pools}
\begin{itemize}
\item Multi-asset pools
\item Customizable weights
\item Index fund functionality
\end{itemize}
\end{columns}

\bottomnote{AMM innovation continues; trend toward capital efficiency and specialization by asset type}
\end{frame}

% Empirical Evidence
\begin{frame}[t]{Empirical Evidence on DeFi Market Microstructure}
\begin{columns}[T]
\column{0.48\textwidth}
\textbf{Key Findings}

\vspace{0.3em}
\textbf{Liquidity and Trading Costs}
\begin{itemize}
\item DEX spreads 2--5$\times$ CEX spreads (Lehar \& Parlour, 2021)
\item Improving with liquidity growth
\item Uniswap v3 narrows gap significantly
\end{itemize}

\vspace{0.3em}
\textbf{Impermanent Loss}
\begin{itemize}
\item $\sim$49.5\% of LPs lose vs.\ HODL (Hold On for Dear Life---crypto slang for holding) (Adams et al., 2024)
\item Fee income often insufficient to offset IL
\item Incentive programs crucial for LP retention
\end{itemize}

\column{0.48\textwidth}
\textbf{MEV Impact}

\vspace{0.3em}
\textbf{Magnitude}
\begin{itemize}
\item 5--10\% implicit cost on some trades (Qin et al., 2022)
\item Concentrated in large trades
\item Growing with DeFi TVL (Total Value Locked---money deposited in DeFi)
\end{itemize}

\vspace{0.3em}
\textbf{Price Discovery}
\begin{itemize}
\item CEXs lead by 30--60 seconds (Barbon \& Ranaldo, 2022)
\item Arbitrage profitable despite gas costs
\item Cross-chain discovery slower due to bridge latency
\end{itemize}
\end{columns}

\bottomnote{Research from Lehar \& Parlour (2021), Barbon \& Ranaldo (2022), Adams et al. (2024) on AMMs and MEV}
\end{frame}

% Future Directions
\begin{frame}[t]{Future Directions in DeFi Microstructure}
\begin{columns}[T]
\column{0.48\textwidth}
\textbf{Technical Innovation}

\vspace{0.3em}
\textbf{Emerging Mechanisms}
\begin{itemize}
\item Dynamic fees (volatility-adjusted)
\item Just-in-time liquidity (providing liquidity for a single block to capture fees)
\item Intent-based architectures (users specify desired outcomes, solvers compete to execute)
\item ZK-rollup order books (ZK-rollups bundle hundreds of trades into one blockchain transaction using cryptographic proofs, cutting costs by 50--100$\times$ while inheriting Ethereum's security)
\end{itemize}

\vspace{0.3em}
\textbf{MEV Solutions}
\begin{itemize}
\item Encrypted mempools
\item Fair ordering protocols
\item MEV redistribution to users
\end{itemize}

\column{0.48\textwidth}
\textbf{Economic Research Needs}

\vspace{0.3em}
\textbf{Open Questions}
\begin{itemize}
\item Optimal LP compensation design
\item MEV welfare impact measurement
\item Cross-chain price discovery
\item Decentralized governance efficiency
\end{itemize}

\vspace{0.3em}
\textbf{Policy Implications}
\begin{itemize}
\item Market manipulation regulation
\item Investor protection in DeFi
\item Systemic risk from MEV
\end{itemize}
\end{columns}

\bottomnote{DeFi market microstructure is rapidly evolving; economic theory must adapt to new mechanisms}
\end{frame}

% Key Takeaways
\begin{frame}[t]{Key Takeaways}
\begin{columns}[T]
\column{0.48\textwidth}
\textbf{Core Concepts}
\begin{enumerate}
\item AMMs use constant product formula for automated trading
\item Order books rely on active market makers
\item Impermanent loss is key LP risk
\item MEV extracts value via transaction ordering
\item Price discovery: faster arbitrage (24/7 trading) but less informationally efficient (prices can diverge from fundamentals longer than in TradFi)
\end{enumerate}

\column{0.48\textwidth}
\textbf{Economic Insights}
\begin{itemize}
\item Trade-offs between decentralization and efficiency
\item Liquidity provision requires compensation for IL
\item MEV creates new welfare considerations
\item Market fragmentation enables arbitrage
\item Innovation continues with v3, Curve, etc.
\end{itemize}
\end{columns}

\vspace{0.5em}
\textbf{Application to Practice}

Market microstructure theory helps evaluate DEX design, understand LP economics, and assess efficiency of crypto trading venues.

\bottomnote{Next lesson: Regulatory Economics of Digital Finance}
\end{frame}

% Key Terms
\begin{frame}[t]{Key Terms (1/2)}
\begin{columns}[T]
\column{0.48\textwidth}
\textbf{AMM (Automated Market Maker)}
Smart contract providing liquidity through algorithmic pricing (e.g., constant product formula $x \cdot y = k$).

\vspace{0.3em}
\textbf{Order Book}
Traditional market structure listing buy and sell orders at various prices.

\vspace{0.3em}
\textbf{Liquidity Provider (LP)}
Party depositing assets into AMM pool to enable trading, earning fees in return.

\vspace{0.3em}
\textbf{Impermanent Loss}
Loss LPs experience when asset prices diverge from deposit ratio, compared to simply holding.

\vspace{0.3em}
\textbf{MEV (Maximal Extractable Value)}
Profit extractable by reordering, inserting, or censoring transactions within a block.

\vspace{0.3em}
\textbf{Price Discovery}
Process by which market determines asset prices through trading activity and information aggregation.

\vspace{0.3em}
\textbf{Market Microstructure}
How trading actually works at the detailed level (order matching, price formation, liquidity provision).

\column{0.48\textwidth}
\textbf{Adverse Selection}
Informed traders profit at uninformed traders' expense, causing market makers to widen spreads.

\vspace{0.3em}
\textbf{Slippage}
Price change between when you submit an order and when it executes, especially on large trades.

\vspace{0.3em}
\textbf{Price Impact}
How much your trade moves the market price; larger trades have higher price impact.

\vspace{0.3em}
\textbf{Arbitrage}
Profiting from price differences across markets (e.g., buying ETH on one exchange and selling on another).

\vspace{0.3em}
\textbf{Front-Running}
Trading ahead of someone else's order to profit from the expected price change.

\vspace{0.3em}
\textbf{Sandwich Attack}
MEV strategy: buy before victim's trade, then sell after, profiting from price manipulation.

\vspace{0.3em}
\textbf{CLOB (Central Limit Order Book)}
Order book matching system that pairs buy and sell orders by price-time priority.
\end{columns}

\bottomnote{Market microstructure concepts are essential for understanding DeFi efficiency and risks}
\end{frame}

\begin{frame}[t]{Key Terms (2/2)}
\begin{columns}[T]
\column{0.48\textwidth}
\textbf{Constant Product Formula}
AMM pricing rule $x \cdot y = k$ that automatically adjusts prices as traders swap tokens.

\vspace{0.3em}
\textbf{TradFi (Traditional Finance)}
Banks, stock exchanges, and conventional financial institutions (vs. DeFi).

\vspace{0.3em}
\textbf{TVL (Total Value Locked)}
Total amount of money deposited in DeFi protocols, a measure of adoption and liquidity.

\vspace{0.3em}
\textbf{Gas Fees}
Transaction fees paid to blockchain validators for processing transactions (high gas = expensive trades).

\column{0.48\textwidth}
\textbf{Block Time}
Time between new blocks added to blockchain (e.g., 12 seconds on Ethereum); determines MEV opportunities.

\vspace{0.3em}
\textbf{Bid-Ask Spread}
Difference between highest buy price and lowest sell price; narrower spreads = more liquid markets.

\vspace{0.3em}
\textbf{Order Book Depth}
Volume of buy/sell orders at each price level; deeper books absorb large trades without price impact.

\vspace{0.3em}
\textbf{HODL (Hold On for Dear Life)}
Crypto slang for holding assets long-term instead of trading; origin of "impermanent loss vs. HODL" comparison.

\vspace{0.3em}
\textbf{Mempool}
The waiting area for unconfirmed transactions before they are included in a block; MEV searchers monitor mempools for opportunities.

\vspace{0.3em}
\textbf{Concentrated Liquidity}
Uniswap v3 innovation allowing LPs to provide liquidity within a chosen price range, improving capital efficiency.
\end{columns}

\bottomnote{Understanding these terms is crucial for analyzing DeFi trading efficiency and risks}
\end{frame}

% References
\begin{frame}[t]{Further Reading}
\begin{columns}[T]
\column{0.48\textwidth}
\textbf{Foundational Papers}
\begin{itemize}
\item Kyle (1985): ``Continuous Auctions and Insider Trading''
\item Hasbrouck (1995): ``One Security, Many Markets''
\item Adams et al. (2024): ``Uniswap v3: The Economics of Concentrated Liquidity''
\item Lehar \& Parlour (2021): ``Systemic Fragility in Decentralized Markets''
\end{itemize}

\column{0.48\textwidth}
\textbf{MEV Research}
\begin{itemize}
\item Daian et al. (2020): ``Flash Boys 2.0''
\item Qin et al. (2022): ``Quantifying MEV on Ethereum''
\item Barbon \& Ranaldo (2022): ``On the Quality of Cryptocurrency Markets''
\end{itemize}
\end{columns}

\bottomnote{All readings available on course platform}
\end{frame}

\end{document}
