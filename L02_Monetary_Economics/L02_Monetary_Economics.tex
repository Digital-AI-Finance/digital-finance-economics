\documentclass[8pt,aspectratio=169]{beamer}
\usetheme{Madrid}
\usepackage{graphicx}
\usepackage{booktabs}
\usepackage{adjustbox}
\usepackage{multicol}
\usepackage{amsmath}

% Color definitions
\definecolor{mlblue}{RGB}{0,102,204}
\definecolor{mlpurple}{RGB}{51,51,178}
\definecolor{mllavender}{RGB}{173,173,224}
\definecolor{mllavender2}{RGB}{193,193,232}
\definecolor{mllavender3}{RGB}{204,204,235}
\definecolor{mllavender4}{RGB}{214,214,239}
\definecolor{mlorange}{RGB}{255, 127, 14}
\definecolor{mlgreen}{RGB}{44, 160, 44}
\definecolor{mlred}{RGB}{214, 39, 40}
\definecolor{mlgray}{RGB}{127, 127, 127}

\definecolor{lightgray}{RGB}{240, 240, 240}
\definecolor{midgray}{RGB}{180, 180, 180}

\setbeamercolor{palette primary}{bg=mllavender3,fg=mlpurple}
\setbeamercolor{palette secondary}{bg=mllavender2,fg=mlpurple}
\setbeamercolor{palette tertiary}{bg=mllavender,fg=white}
\setbeamercolor{palette quaternary}{bg=mlpurple,fg=white}

\setbeamercolor{structure}{fg=mlpurple}
\setbeamercolor{section in toc}{fg=mlpurple}
\setbeamercolor{subsection in toc}{fg=mlblue}
\setbeamercolor{title}{fg=mlpurple}
\setbeamercolor{frametitle}{fg=mlpurple,bg=mllavender3}
\setbeamercolor{block title}{bg=mllavender2,fg=mlpurple}
\setbeamercolor{block body}{bg=mllavender4,fg=black}

\setbeamertemplate{navigation symbols}{}
\setbeamertemplate{itemize items}[circle]
\setbeamertemplate{enumerate items}[default]
\setbeamersize{text margin left=5mm,text margin right=5mm}

\newcommand{\bottomnote}[1]{%
\vfill
\vspace{-2mm}
\textcolor{mllavender2}{\rule{\textwidth}{0.4pt}}
\vspace{1mm}
\footnotesize
\textbf{#1}
}

\title{Monetary Economics of Digital Currencies}
\subtitle{L02: Money Theory Meets Cryptocurrency}
\author{Economics of Digital Finance}
\institute{BSc Course}
\date{}

\begin{document}

% Title slide
\begin{frame}[plain]
\titlepage
\end{frame}

% Outline
\begin{frame}[t]{Lesson Overview}
\begin{columns}[T]
\column{0.48\textwidth}
\textbf{Today's Topics}
\begin{enumerate}
\item Functions of money revisited
\item Quantity theory in digital age
\item Cryptocurrencies as money
\item Stablecoin economics
\item Currency substitution
\end{enumerate}

\column{0.48\textwidth}
\textbf{Learning Objectives}
\begin{itemize}
\item Apply monetary theory to digital currencies
\item Assess crypto against money functions
\item Analyze stablecoin stability mechanisms
\item Understand Gresham's Law implications
\end{itemize}
\end{columns}

\bottomnote{Monetary economics provides rigorous framework for evaluating digital currencies}
\end{frame}

% Functions of Money Deep Dive
\begin{frame}[t]{Functions of Money: Economic Analysis}
\begin{columns}[T]
\column{0.48\textwidth}
\textbf{Medium of Exchange}

Economic rationale:
\begin{itemize}
\item Eliminates barter inefficiency
\item Reduces search and matching costs
\item Transaction cost = $c_b - c_m$ where $c_m \ll c_b$
\end{itemize}

\vspace{0.3em}
Requirements:
\begin{itemize}
\item Acceptability (network effect)
\item Divisibility
\item Portability
\end{itemize}

\column{0.48\textwidth}
\textbf{Unit of Account}

Economic rationale:
\begin{itemize}
\item Reduces cognitive costs
\item With $n$ goods: $\frac{n(n-1)}{2} \rightarrow n-1$ prices
\item Enables economic calculation
\end{itemize}

\vspace{0.3em}
\textbf{Store of Value}

Requirements:
\begin{itemize}
\item Stable purchasing power
\item Low volatility: $\sigma_{\text{money}} < \sigma_{\text{goods}}$
\item Inflation protection
\end{itemize}
\end{columns}

\bottomnote{Trade-offs exist: a good medium of exchange may not be ideal store of value}
\end{frame}

% Quantity Theory
\begin{frame}[t]{Quantity Theory of Money in Digital Age}
\begin{columns}[T]
\column{0.48\textwidth}
\textbf{Classical Equation of Exchange}
$$MV = PY$$
\begin{itemize}
\item $M$ = Money supply
\item $V$ = Velocity of circulation
\item $P$ = Price level
\item $Y$ = Real output
\end{itemize}

\vspace{0.3em}
\textbf{Implications}
\begin{itemize}
\item If $V$ stable: $\Delta M \rightarrow \Delta P$
\item Seigniorage = $\frac{\dot{M}}{P}$
\item Inflation tax on money holders
\end{itemize}

\column{0.48\textwidth}
\textbf{Digital Currency Complications}

Bitcoin example:
\begin{itemize}
\item $M$ fixed at 21 million (deflationary)
\item $V$ highly volatile and hard to measure
\item Which $P$? (Crypto priced in fiat)
\end{itemize}

\vspace{0.3em}
\textbf{Velocity Puzzle}
\begin{itemize}
\item Traditional money: $V \approx 5-7$
\item Bitcoin: $V$ varies 2-20+
\item Stablecoins: Very high turnover
\end{itemize}
\end{columns}

\bottomnote{Fixed supply creates deflationary pressure; incompatible with MoE function}
\end{frame}

% Bitcoin Volatility
\begin{frame}[t]{Bitcoin as Money: The Volatility Problem}
\begin{center}
\includegraphics[width=0.65\textwidth]{01_bitcoin_volatility/chart.pdf}
\end{center}

\bottomnote{Bitcoin's 50-85\% volatility vs EUR/USD's 5-10\% makes it unsuitable as unit of account}
\end{frame}

% Cryptocurrency Assessment
\begin{frame}[t]{Cryptocurrencies as Money: Economic Assessment}
\begin{columns}[T]
\column{0.48\textwidth}
\textbf{Medium of Exchange: Grade C-}
\begin{itemize}
\item Limited merchant acceptance
\item High transaction costs (at times)
\item 10-60 min confirmation times
\item Scalability trilemma
\end{itemize}

\vspace{0.3em}
\textbf{Unit of Account: Grade F}
\begin{itemize}
\item Extreme volatility
\item ``Menu cost'' of repricing
\item No contracts denominated in BTC
\end{itemize}

\column{0.48\textwidth}
\textbf{Store of Value: Grade C}
\begin{itemize}
\item Long-term appreciation (but volatile)
\item Digital gold narrative
\item Correlation with risk assets
\end{itemize}

\vspace{0.3em}
\textbf{Yermack (2015) Conclusion}

``Bitcoin behaves more like a speculative investment than a currency''
\begin{itemize}
\item Low correlation with major currencies
\item High correlation with tech stocks
\item Driven by speculation, not trade
\end{itemize}
\end{columns}

\bottomnote{Bitcoin fails key money functions; better characterized as speculative asset}
\end{frame}

% Money Functions Matrix
\begin{frame}[t]{Money Functions: Comparative Assessment}
\begin{center}
\includegraphics[width=0.60\textwidth]{03_money_functions_matrix/chart.pdf}
\end{center}

\bottomnote{CBDCs designed to achieve high scores across all functions; stablecoins compromise}
\end{frame}

% Stablecoin Economics
\begin{frame}[t]{Stablecoin Economics: Design and Mechanisms}
\begin{columns}[T]
\column{0.48\textwidth}
\textbf{Types by Collateral}

\vspace{0.3em}
\textbf{1. Fiat-backed (USDT, USDC)}
\begin{itemize}
\item 1:1 reserve in bank accounts
\item Trust in issuer and audits
\item Redemption guarantee
\end{itemize}

\vspace{0.3em}
\textbf{2. Crypto-backed (DAI)}
\begin{itemize}
\item Over-collateralized (150\%+)
\item Smart contract enforcement
\item Liquidation mechanisms
\end{itemize}

\column{0.48\textwidth}
\textbf{3. Algorithmic (failed: UST)}
\begin{itemize}
\item No collateral backing
\item Arbitrage-based stability
\item Prone to death spirals
\end{itemize}

\vspace{0.5em}
\textbf{Economic Trade-offs}
\begin{itemize}
\item Capital efficiency vs. safety
\item Centralization vs. transparency
\item Scalability vs. collateral needs
\end{itemize}
\end{columns}

\bottomnote{Terra/UST collapse (2022) showed algorithmic designs are inherently fragile}
\end{frame}

% Stablecoin Market
\begin{frame}[t]{Stablecoin Market Evolution}
\begin{center}
\includegraphics[width=0.65\textwidth]{02_stablecoin_market/chart.pdf}
\end{center}

\bottomnote{Market concentrated in USDT; Terra collapse caused \$40B+ losses in May 2022}
\end{frame}

% Gresham's Law
\begin{frame}[t]{Gresham's Law and Currency Substitution}
\begin{columns}[T]
\column{0.48\textwidth}
\textbf{Gresham's Law}

``Bad money drives out good''
\begin{itemize}
\item When two currencies circulate at fixed rate
\item Undervalued currency hoarded
\item Overvalued currency spent
\end{itemize}

\vspace{0.3em}
\textbf{Digital Application}
\begin{itemize}
\item Bitcoin hoarded (``HODL'')
\item Stablecoins used for transactions
\item Self-fulfilling: reduces velocity
\end{itemize}

\column{0.48\textwidth}
\textbf{Currency Substitution}

Dollarization analogy:
\begin{itemize}
\item Weak local currency replaced
\item ``Crypto-ization'' in high-inflation countries
\item Argentina, Venezuela, Turkey cases
\end{itemize}

\vspace{0.3em}
\textbf{Economic Consequences}
\begin{itemize}
\item Loss of monetary policy autonomy
\item Seigniorage transfer abroad
\item Financial stability risks
\end{itemize}
\end{columns}

\bottomnote{Currency competition creates both opportunities and risks for monetary systems}
\end{frame}

% Money Demand
\begin{frame}[t]{Money Demand for Digital Currencies}
\begin{columns}[T]
\column{0.48\textwidth}
\textbf{Traditional Money Demand}
$$M^d/P = L(Y, i)$$
\begin{itemize}
\item $L_Y > 0$: Transaction motive
\item $L_i < 0$: Opportunity cost
\item Baumol-Tobin inventory model
\end{itemize}

\vspace{0.3em}
\textbf{Portfolio Approach}
$$M^d = f(W, r_m, r_b, \pi^e, \sigma)$$
\begin{itemize}
\item Wealth effect
\item Relative returns
\item Inflation expectations
\end{itemize}

\column{0.48\textwidth}
\textbf{Crypto Money Demand}

Additional factors:
\begin{itemize}
\item Speculative motive dominates
\item Network effects matter
\item Regulatory risk premium
\end{itemize}

\vspace{0.3em}
\textbf{Empirical Challenges}
\begin{itemize}
\item What is ``crypto money supply''?
\item How to measure crypto velocity?
\item Multiple exchanges, prices
\end{itemize}
\end{columns}

\bottomnote{Traditional money demand models require significant adaptation for crypto analysis}
\end{frame}

% Seigniorage
\begin{frame}[t]{Seigniorage in Digital Currency Systems}
\begin{columns}[T]
\column{0.48\textwidth}
\textbf{Traditional Seigniorage}
$$S = \frac{\dot{M}}{P} = \frac{\Delta M}{M} \cdot \frac{M}{P}$$
\begin{itemize}
\item Revenue from money creation
\item Accrues to central bank/government
\item Inflation tax on money holders
\end{itemize}

\vspace{0.3em}
\textbf{Bitcoin ``Seigniorage''}
\begin{itemize}
\item Block rewards to miners
\item Declining over time (halvings)
\item Dissipated in mining costs
\end{itemize}

\column{0.48\textwidth}
\textbf{Stablecoin Seigniorage}
\begin{itemize}
\item Interest on reserves kept by issuer
\item Tether earns billions annually
\item Users bear opportunity cost
\end{itemize}

\vspace{0.3em}
\textbf{Policy Implications}
\begin{itemize}
\item Who captures monetary rents?
\item Private vs. public money trade-offs
\item CBDC: Returns seigniorage to public
\end{itemize}
\end{columns}

\bottomnote{Stablecoin issuers capture seigniorage that would otherwise accrue to governments}
\end{frame}

% Monetary Policy Implications
\begin{frame}[t]{Monetary Policy Implications}
\begin{columns}[T]
\column{0.48\textwidth}
\textbf{Transmission Mechanism Risks}
\begin{itemize}
\item Crypto reduces money multiplier
\item Interest rate channel weakened
\item Bank reserves less relevant
\end{itemize}

\vspace{0.3em}
\textbf{Financial Stability}
\begin{itemize}
\item Pro-cyclical crypto prices
\item Contagion from crypto crashes
\item Interconnection with TradFi
\end{itemize}

\column{0.48\textwidth}
\textbf{Central Bank Responses}
\begin{itemize}
\item CBDC development (defensive)
\item Stablecoin regulation
\item Reserve requirements for crypto banks
\end{itemize}

\vspace{0.3em}
\textbf{Long-term Questions}
\begin{itemize}
\item Can crypto coexist with fiat?
\item Optimal regulatory perimeter?
\item International coordination needs?
\end{itemize}
\end{columns}

\bottomnote{Central banks view crypto growth as potential challenge to monetary sovereignty}
\end{frame}

% Key Takeaways
\begin{frame}[t]{Key Takeaways}
\begin{columns}[T]
\column{0.48\textwidth}
\textbf{Main Conclusions}
\begin{enumerate}
\item Bitcoin fails core money functions due to volatility
\item Stablecoins are ``money-like'' but carry risks
\item Quantity theory applies but needs adaptation
\item Seigniorage distribution is policy issue
\end{enumerate}

\column{0.48\textwidth}
\textbf{Economic Framework}
\begin{itemize}
\item Money functions: MoE, UoA, SoV
\item Quantity theory: $MV = PY$
\item Gresham's Law and hoarding
\item Currency substitution dynamics
\end{itemize}
\end{columns}

\vspace{0.5em}
\textbf{Core Insight}

Monetary economics reveals why cryptocurrencies struggle as money: they optimize for speculation, not monetary functions. Stablecoins address some issues but create new ones.

\bottomnote{Next lesson: Central Bank Digital Currencies (CBDCs)}
\end{frame}

% References
\begin{frame}[t]{Further Reading}
\begin{columns}[T]
\column{0.48\textwidth}
\textbf{Academic Papers}
\begin{itemize}
\item Yermack (2015): ``Is Bitcoin a Real Currency?''
\item Gorton \& Zhang (2023): ``Taming Wildcat Stablecoins''
\item Brunnermeier et al. (2019): ``The Digitalization of Money''
\end{itemize}

\column{0.48\textwidth}
\textbf{Policy Analysis}
\begin{itemize}
\item BIS (2022): ``The Future Monetary System''
\item IMF (2023): ``Elements of Effective Crypto Policies''
\item ECB (2022): ``Stablecoin Assessment''
\end{itemize}
\end{columns}

\bottomnote{All readings available on course platform}
\end{frame}

\end{document}
