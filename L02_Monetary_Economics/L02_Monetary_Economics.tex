\documentclass[8pt,aspectratio=169]{beamer}
\usetheme{Madrid}
\usepackage{graphicx}
\usepackage{booktabs}
\usepackage{adjustbox}
\usepackage{multicol}
\usepackage{amsmath}

% Color definitions
\definecolor{mlblue}{RGB}{0,102,204}
\definecolor{mlpurple}{RGB}{51,51,178}
\definecolor{mllavender}{RGB}{173,173,224}
\definecolor{mllavender2}{RGB}{193,193,232}
\definecolor{mllavender3}{RGB}{204,204,235}
\definecolor{mllavender4}{RGB}{214,214,239}
\definecolor{mlorange}{RGB}{255, 127, 14}
\definecolor{mlgreen}{RGB}{44, 160, 44}
\definecolor{mlred}{RGB}{214, 39, 40}
\definecolor{mlgray}{RGB}{127, 127, 127}

\definecolor{lightgray}{RGB}{240, 240, 240}
\definecolor{midgray}{RGB}{180, 180, 180}

\setbeamercolor{palette primary}{bg=mllavender3,fg=mlpurple}
\setbeamercolor{palette secondary}{bg=mllavender2,fg=mlpurple}
\setbeamercolor{palette tertiary}{bg=mllavender,fg=white}
\setbeamercolor{palette quaternary}{bg=mlpurple,fg=white}

\setbeamercolor{structure}{fg=mlpurple}
\setbeamercolor{section in toc}{fg=mlpurple}
\setbeamercolor{subsection in toc}{fg=mlblue}
\setbeamercolor{title}{fg=mlpurple}
\setbeamercolor{frametitle}{fg=mlpurple,bg=mllavender3}
\setbeamercolor{block title}{bg=mllavender2,fg=mlpurple}
\setbeamercolor{block body}{bg=mllavender4,fg=black}

\setbeamertemplate{navigation symbols}{}
\setbeamertemplate{itemize items}[circle]
\setbeamertemplate{enumerate items}[default]
\setbeamersize{text margin left=5mm,text margin right=5mm}

\newcommand{\bottomnote}[1]{%
\vfill
\vspace{-2mm}
\textcolor{mllavender2}{\rule{\textwidth}{0.4pt}}
\vspace{1mm}
\footnotesize
\textbf{#1}
}

\title{Monetary Economics of Digital Currencies}
\subtitle{L02: Money Theory Meets Cryptocurrency\\[0.3em]\normalsize Can Bitcoin replace the dollar? What monetary theory tells us.}
\author{Economics of Digital Finance}
\institute{BSc Course}
\date{}

\begin{document}

% Title slide
\begin{frame}[plain]
\titlepage
\end{frame}

% Outline
\begin{frame}[t]{Lesson Overview}
\begin{columns}[T]
\column{0.48\textwidth}
\textbf{Today's Topics}
\begin{enumerate}
\item Functions of money revisited
\item Quantity theory in digital age
\item Cryptocurrencies as money
\item Stablecoin economics
\item Currency substitution
\end{enumerate}

\column{0.48\textwidth}
\textbf{Learning Objectives}
\begin{itemize}
\item Apply monetary theory to digital currencies
\item Assess crypto against money functions
\item Analyze stablecoin stability mechanisms
\item Understand Gresham's Law (``bad money drives out good'') and its implications
\end{itemize}
\end{columns}

\bottomnote{Monetary economics provides rigorous framework for evaluating digital currencies}
\end{frame}

% Functions of Money Deep Dive
\begin{frame}[t]{Functions of Money: Economic Analysis}
\begin{columns}[T]
\column{0.48\textwidth}
\textbf{Medium of Exchange}

Economic rationale:
\begin{itemize}
\item Eliminates barter inefficiency
\item Reduces search and matching costs (time and effort to find trading partners)
\item Transaction cost = $c_b - c_m$ where $c_m \ll c_b$ (where $\ll$ means ``much less than'')

\textit{(Transaction cost with barter minus cost with money; money makes trading much cheaper)}

\smallskip\textbf{Example:} If barter costs \$5 per trade ($c_b$) but using money costs \$0.50 ($c_m$), the saving is $\$5 - \$0.50 = \$4.50$ per trade.
\end{itemize}

\vspace{0.3em}
Requirements:
\begin{itemize}
\item Acceptability (network effect)
\item Divisibility (can be split into small amounts for any transaction size)
\item Portability (easy to carry or transfer)
\end{itemize}

\column{0.48\textwidth}
\textbf{Unit of Account}

Economic rationale:
\begin{itemize}
\item Reduces cognitive costs (mental effort to compare prices)
\item With $n$ goods: $\frac{n(n-1)}{2} \rightarrow n-1$ prices

\textit{(With n goods, barter needs n(n-1)/2 exchange rates; money needs only n-1 prices)}

\smallskip\textbf{Example:} With 4 goods, barter needs $4 \times 3 / 2 = 6$ exchange rates. With money, you need only $4 - 1 = 3$ prices.
\item Enables economic calculation
\end{itemize}

\vspace{0.3em}
\textbf{Store of Value}

Requirements:
\begin{itemize}
\item Stable purchasing power
\item Low volatility: the volatility of money's purchasing power should be low relative to the goods it buys

\textit{(If money's value fluctuates more than the things you buy, it fails as a reliable store of value)}
\item Inflation protection
\end{itemize}
\end{columns}

\bottomnote{These functions matter because digital currencies must satisfy all three to replace traditional money effectively}
\end{frame}

% Quantity Theory
\begin{frame}[t]{Quantity Theory of Money in Digital Age}
\begin{columns}[T]
\column{0.48\textwidth}
\textbf{Classical Equation of Exchange}
$$MV = PY$$

\textit{(Money supply times velocity equals price level times real output---how money flows through economy)}

\begin{itemize}
\item $M$ = Money supply
\item $V$ = Velocity of circulation
\item $P$ = Price level
\item $Y$ = Real output
\end{itemize}

\vspace{0.3em}
\textbf{Implications}
\begin{itemize}
\item If $V$ stable: $\Delta M \rightarrow \Delta P$

\textit{(If velocity is stable, increasing money supply leads to higher prices---more money chasing same goods)}

\smallskip\textbf{Example:} If $M{=}\$100$, $V{=}2$, $Y{=}100$ units, then $P = MV/Y = \$2$. If $M$ doubles to \$200, $P$ doubles to \$4.
\item Seigniorage = $\frac{\dot{M}}{P}$, where $\dot{M}$ denotes the rate of change of $M$ over time

\textit{(Rate of money creation divided by price level---profit from printing money)}

\smallskip\textbf{Example:} If a central bank prints \$10B in new money and $P{=}1$, seigniorage $= \$10\text{B}/1 = \$10\text{B}$ in real purchasing power.
\item Inflation tax on money holders (when new money is printed, your existing money buys less---a hidden tax)
\end{itemize}

\column{0.48\textwidth}
\textbf{Digital Currency Complications}

Bitcoin example:
\begin{itemize}
\item $M$ fixed at 21 million (deflationary---since supply cannot grow, prices in Bitcoin would fall over time, discouraging spending)
\item $V$ highly volatile and hard to measure
\item Which $P$? (Crypto priced in fiat)
\end{itemize}

\vspace{0.3em}
\textbf{Velocity Puzzle}
\begin{itemize}
\item Traditional money: M2 (cash + savings + money market) $V \approx 1.5$; M1 (cash + checking accounts) velocity is higher at 5--7
\item Bitcoin: $V$ varies 2-20+---this unpredictable velocity makes quantity theory unreliable for crypto
\item Stablecoins: Very high turnover (velocity), meaning each stablecoin changes hands many times per year
\end{itemize}
\end{columns}

\bottomnote{Quantity theory matters because it explains how Bitcoin's fixed supply creates deflationary pressure, making it impractical as everyday money}
\end{frame}

% Quantity Theory Velocity Chart
\begin{frame}[t]{Quantity Theory: Fiat vs.\ Digital Velocity}
\begin{center}
\includegraphics[width=0.55\textwidth]{04_quantity_theory_velocity/chart.pdf}
\end{center}
\vspace{-2mm}
\begin{itemize}\footnotesize
\item Using $P = MV/Y$: higher velocity (orange, $V{=}10$) means the same money supply produces much higher prices than fiat velocity (blue, $V{=}1.5$).
\item Digital currencies circulate faster, so fewer coins are needed to support the same level of economic activity.
\item The shaded gap shows the ``inflation pressure'' created by higher velocity at every money supply level.
\end{itemize}

\bottomnote{Chart uses the Fisher equation $MV = PY$ with fixed real output $Y{=}100$; higher velocity amplifies money's impact on prices}
\end{frame}

% Bitcoin Volatility
\begin{frame}[t]{Bitcoin as Money: The Volatility Problem}
\begin{center}
\includegraphics[width=0.55\textwidth]{01_bitcoin_volatility/chart.pdf}
\end{center}
\vspace{-2mm}
\begin{itemize}\footnotesize
\item \textbf{Note:} This chart is a \textbf{GARCH(1,1) statistical simulation} mimicking Bitcoin-like volatility patterns---not actual Bitcoin price data.
\item \textbf{Volatility clustering:} Big price swings tend to cluster together (shaded regions), then calm periods follow.
\item \textbf{Annualized volatility} of 50--85\% means Bitcoin's value can roughly double or halve within a single year.
\item Bitcoin-like returns (orange) swing far wider than stock-market-like returns (blue).
\end{itemize}

\bottomnote{High volatility (50-85\% annually) makes Bitcoin impractical as a unit of account---imagine if the dollar's value changed 50\% per year}
\end{frame}

% Cryptocurrency Assessment
\begin{frame}[t]{Cryptocurrencies as Money: Economic Assessment}
\begin{columns}[T]
\column{0.48\textwidth}
\textbf{Medium of Exchange: Grade C-}
\begin{itemize}
\item Limited merchant acceptance
\item High transaction costs (at times)
\item 10-60 min confirmation times
\item Scalability trilemma (must sacrifice one of: decentralization (no single controller), security (resistant to attacks), or speed (fast transactions))
\end{itemize}

\vspace{0.3em}
\textbf{Unit of Account: Grade F}
\begin{itemize}
\item Extreme volatility
\item ``Menu cost'' (cost of constantly repricing goods) of repricing
\item No contracts denominated in BTC (Bitcoin's ticker symbol)
\end{itemize}

\column{0.48\textwidth}
\textbf{Store of Value: Grade C}
\begin{itemize}
\item Long-term appreciation (but volatile)
\item Digital gold narrative (the claim that Bitcoin, like physical gold, is a scarce store of value)
\item Moves in sync with risk assets (tends to fall when stocks fall)---this undermines the ``digital gold'' claim since it doesn't protect portfolios during crashes
\end{itemize}

\vspace{0.3em}
\textbf{Yermack (2015) Conclusion}

``Bitcoin behaves more like a speculative investment than a currency''
\begin{itemize}
\item Low correlation with major currencies
\item High correlation with tech stocks
\item Driven by speculation, not trade
\end{itemize}
\end{columns}

\bottomnote{These poor grades matter because they explain why Bitcoin hasn't replaced traditional money despite 15+ years of existence}
\end{frame}

% Money Functions Matrix
\begin{frame}[t]{Money Functions: Comparative Assessment}
\begin{center}
\includegraphics[width=0.50\textwidth]{03_money_functions_matrix/chart.pdf}
\end{center}
\vspace{-2mm}
\begin{itemize}\footnotesize
\item Scores are \textbf{illustrative assessments} based on current properties, not measured data.
\item \textbf{CBDC scores are projections} based on design goals---no major retail CBDC has been fully deployed yet.
\item Traditional fiat excels as unit of account; Bitcoin scores poorly due to volatility; stablecoins trade off decentralization for stability.
\end{itemize}

\bottomnote{CBDCs designed to achieve high scores across all functions; stablecoins compromise on decentralization}
\end{frame}

% Stablecoin Economics
\begin{frame}[t]{Stablecoin Economics: Design and Mechanisms}
\begin{columns}[T]
\column{0.48\textwidth}
\textbf{Types by Collateral}

\vspace{0.3em}
\textbf{1. Fiat-backed (USDT/Tether, USDC/USD Coin)}
\begin{itemize}
\item 1:1 reserve in bank accounts
\item Trust in issuer and audits
\item Redemption guarantee
\end{itemize}

\vspace{0.3em}
\textbf{2. Crypto-backed (DAI/MakerDAO)}
\begin{itemize}
\item Over-collateralized (150\%+)

\textit{(e.g., deposit \$150 in ETH (Ether) to borrow \$100 in DAI)}
\item Smart contract enforcement (automated rules coded into the blockchain that execute without human intervention)
\item Liquidation mechanisms (forced sale when collateral value drops)
\end{itemize}

\column{0.48\textwidth}
\textbf{3. Algorithmic (failed: UST/TerraUSD)}
\begin{itemize}
\item No collateral backing
\item Arbitrage-based (relying on profit-seekers to keep prices aligned) stability
\item Prone to death spirals (falling price triggers more selling, which pushes price lower still---a self-reinforcing collapse)
\end{itemize}

\vspace{0.5em}
\textbf{Economic Trade-offs}
\begin{itemize}
\item Capital efficiency (getting maximum lending from minimum reserves) vs.\ safety
\item Centralization vs. transparency
\item Scalability vs. collateral needs
\end{itemize}
\end{columns}

\bottomnote{Terra/UST collapse (2022) showed algorithmic designs are inherently fragile}
\end{frame}

% Stablecoin Market
\begin{frame}[t]{Stablecoin Reserve Dynamics Under Stress}
\begin{center}
\includegraphics[width=0.55\textwidth]{02_stablecoin_market/chart.pdf}
\end{center}
\vspace{-2mm}
\begin{itemize}\footnotesize
\item \textbf{Reserve ratio} = how much real money backs each stablecoin (100\% = fully backed; below 100\% = fractional reserve).
\item Based on the \textbf{Diamond \& Dybvig (1983)} bank-run model, which explains how rational depositors can cause collapses by all trying to withdraw at once.
\item When reserves drop below a critical threshold, de-peg probability spikes---just like a traditional bank run.
\end{itemize}

\bottomnote{The Terra/UST collapse in May 2022 showed algorithmic stablecoins can fail catastrophically when confidence breaks (\$40B+ lost)}
\end{frame}

% Gresham's Law
\begin{frame}[t]{Gresham's Law and Currency Substitution}
\begin{columns}[T]
\column{0.48\textwidth}
\textbf{Gresham's Law}

``Bad money drives out good''
\begin{itemize}
\item When two currencies circulate at a fixed exchange rate
\item Undervalued (``good'') currency is hoarded
\item Overvalued (``bad'') currency is spent

\smallskip\textit{Note: Gresham's Law strictly requires a fixed exchange rate. In crypto, there is no fixed rate, but the \textbf{analog} is similar: users spend stablecoins (whose value doesn't rise) and hoard Bitcoin (hoping it appreciates).}
\end{itemize}

\vspace{0.3em}
\textbf{Digital Application}
\begin{itemize}
\item Bitcoin hoarded (``HODL''---crypto slang for ``hold,'' originally a famous typo)
\item Stablecoins used for transactions
\item Self-fulfilling (outcomes that occur because people expect them): reduces velocity
\end{itemize}

\column{0.48\textwidth}
\textbf{Currency Substitution}

Dollarization analogy:
\begin{itemize}
\item Weak local currency replaced
\item ``Crypto-ization'' in high-inflation countries
\item Argentina, Venezuela, Turkey cases

\textit{(countries where high inflation drove citizens to hold USD or crypto instead of local currency; e.g., Argentina had 5+ million crypto users (${\sim}10\%$ of population) with inflation exceeding 100\% in 2023)}
\end{itemize}

\vspace{0.3em}
\textbf{Economic Consequences}
\begin{itemize}
\item Loss of monetary policy autonomy---countries cannot fight recessions or control inflation independently
\item Seigniorage transfer abroad---foreign entities profit from money creation that should benefit the local economy
\item Financial stability risks
\end{itemize}
\end{columns}

\bottomnote{Currency competition creates both opportunities and risks for monetary systems}
\end{frame}

% Gresham's Law Simulation Chart
\begin{frame}[t]{Gresham's Law: Simulation of Currency Substitution}
\begin{center}
\includegraphics[width=0.55\textwidth]{05_greshams_law_simulation/chart.pdf}
\end{center}
\vspace{-2mm}
\begin{itemize}\footnotesize
\item \textbf{Agent-based simulation:} 1,000 agents choose which currency to spend each period. Currency~A depreciates faster (``bad money'').
\item As more people spend Currency~A, others follow (positive feedback), creating an S-curve tipping point around period 40--50.
\item After the tipping point, Currency~B (``good money'') almost disappears from circulation---it is hoarded, not spent.
\end{itemize}

\bottomnote{Simulation uses a logit feedback model (Selgin, 1996); parameters calibrated to show tipping dynamics, not real currency data}
\end{frame}

% Money Demand
\begin{frame}[t]{Money Demand for Digital Currencies}
\begin{columns}[T]
\column{0.48\textwidth}
\textbf{Traditional Money Demand}
$$M^d/P = L(Y, i)$$

\textit{(Real money demand depends on income Y (positive) and interest rate i (negative))}

\begin{itemize}
\item $L_Y > 0$: higher income $\rightarrow$ hold more money (transaction motive)
\item $L_i < 0$: higher interest rate $\rightarrow$ hold less money (opportunity cost of keeping cash instead of earning interest)
\item Baumol-Tobin (a model of optimal cash holding) inventory model
\end{itemize}

\smallskip\textbf{Example:} If your monthly income is \$3{,}000 and rates are 5\%, you might hold \$500 in cash. If rates rise to 10\%, you reduce cash to \$300 (since the cost of not earning interest is higher).

\vspace{0.3em}
\textbf{Portfolio Approach}
$$M^d = f(W, r_m, r_b, \pi^e, \sigma)$$
\begin{itemize}
\item $W$ = total wealth; $r_m$ = return on money; $r_b$ = return on bonds; $\pi^e$ = expected inflation; $\sigma$ = risk (uncertainty)
\item Higher wealth $\rightarrow$ hold more money; higher bond returns $\rightarrow$ hold less money; higher expected inflation $\rightarrow$ hold less money
\end{itemize}

\column{0.48\textwidth}
\textbf{Crypto Money Demand}

Additional factors:
\begin{itemize}
\item Speculative motive dominates (people hold crypto mainly hoping the price will rise, not for everyday purchases)
\item Network effects matter
\item Regulatory risk premium (extra return demanded due to regulatory uncertainty)
\end{itemize}

\vspace{0.3em}
\textbf{Empirical Challenges}
\begin{itemize}
\item What is ``crypto money supply''?
\item How to measure crypto velocity?
\item Multiple exchanges, prices
\end{itemize}
\end{columns}

\bottomnote{Traditional money demand models require significant adaptation for crypto analysis}
\end{frame}

% Seigniorage
\begin{frame}[t]{Seigniorage in Digital Currency Systems}
\begin{columns}[T]
\column{0.48\textwidth}
\textbf{Traditional Seigniorage}
$$S = \frac{\dot{M}}{P} = \frac{\Delta M}{M} \cdot \frac{M}{P}$$

\textit{(Seigniorage = growth rate of money times real money balances---how much value the money issuer extracts)}

\smallskip\textbf{Example:} If money supply grows 5\% ($\Delta M/M = 0.05$) and real balances are $M/P = \$100\text{B}$, seigniorage $= 0.05 \times \$100\text{B} = \$5\text{B}$.

\begin{itemize}
\item Revenue from money creation
\item Accrues to central bank/government
\item Inflation tax on money holders
\end{itemize}

\vspace{0.3em}
\textbf{Bitcoin ``Seigniorage''}
\begin{itemize}
\item Block rewards to miners (computers that validate transactions and earn new coins)
\item Declining over time (halvings: events that cut mining rewards in half every ${\sim}4$ years)
\item Dissipated in mining costs (electricity and hardware expenses consume most of the reward)
\end{itemize}

\column{0.48\textwidth}
\textbf{Stablecoin Seigniorage}
\begin{itemize}
\item Interest on reserves kept by issuer
\item Tether earns billions annually

\textit{(Tether holds user deposits in Treasury bonds (government debt securities) earning 4--5\% interest, keeping the yield for itself)}
\item Users bear opportunity cost
\end{itemize}

\vspace{0.3em}
\textbf{Policy Implications}
\begin{itemize}
\item Who captures monetary rents (profits earned simply from controlling money creation)?
\item Private vs. public money trade-offs
\item CBDC: Returns seigniorage to public
\end{itemize}
\end{columns}

\bottomnote{Stablecoin issuers capture seigniorage that would otherwise accrue to governments}
\end{frame}

% Monetary Policy Implications
\begin{frame}[t]{Monetary Policy Implications}
\begin{columns}[T]
\column{0.48\textwidth}
\textbf{Transmission Mechanism Risks}

\textit{How it works: Central banks raise/lower interest rates $\rightarrow$ banks adjust lending rates $\rightarrow$ businesses and consumers borrow more or less $\rightarrow$ economy speeds up or slows down. If people hold crypto instead of bank deposits, this chain weakens:}
\begin{itemize}
\item Crypto reduces the money multiplier (the process by which bank lending amplifies deposits into a larger money supply---e.g., \$100 deposited can support \$1{,}000 in loans)
\item Interest rate channel (how central bank rate changes affect borrowing) weakened
\item Bank reserves less relevant
\end{itemize}

\vspace{0.3em}
\textbf{Financial Stability}
\begin{itemize}
\item Pro-cyclical (amplifying booms and busts) crypto prices
\item Contagion from crypto crashes
\item Interconnection with TradFi (traditional finance)
\end{itemize}

\column{0.48\textwidth}
\textbf{Central Bank Responses}
\begin{itemize}
\item CBDC development (defensive)

\textit{(central banks developing CBDCs to prevent private stablecoins from undermining monetary control)}
\item Stablecoin regulation
\item Reserve requirements for crypto banks
\end{itemize}

\vspace{0.3em}
\textbf{Long-term Questions}
\begin{itemize}
\item Can crypto coexist with fiat?
\item Optimal regulatory perimeter?
\item International coordination needs?
\end{itemize}
\end{columns}

\bottomnote{Central banks view crypto growth as potential challenge to monetary sovereignty}
\end{frame}

% Key Takeaways
\begin{frame}[t]{Key Takeaways}
\begin{columns}[T]
\column{0.48\textwidth}
\textbf{Main Conclusions}
\begin{enumerate}
\item Bitcoin fails core money functions due to volatility
\item Stablecoins are ``money-like'' but carry risks
\item Quantity theory applies but needs adaptation
\item Seigniorage distribution is policy issue
\end{enumerate}

\column{0.48\textwidth}
\textbf{Economic Framework}
\begin{itemize}
\item Money functions: Medium of Exchange (MoE), Unit of Account (UoA), Store of Value (SoV)
\item Quantity theory: $MV = PY$
\item Gresham's Law and hoarding
\item Currency substitution dynamics
\end{itemize}
\end{columns}

\vspace{0.5em}
\textbf{Core Insight}

Monetary economics reveals why cryptocurrencies struggle as money: they optimize for speculation, not monetary functions. Stablecoins address some issues but create new ones.

\bottomnote{Next lesson: Central Bank Digital Currencies (CBDCs)}
\end{frame}

% Key Terms -- Slide 1 of 3: Money Fundamentals
\begin{frame}[t]{Key Terms (1/3): Money Fundamentals}
\begin{columns}[T]
\column{0.48\textwidth}
\textbf{Medium of Exchange}
Money's function as accepted payment for transactions.

\vspace{0.3em}
\textbf{Unit of Account}
Money's function as standard measure for pricing.

\vspace{0.3em}
\textbf{Store of Value}
Money's function preserving purchasing power over time.

\vspace{0.3em}
\textbf{Barter}
Direct goods exchange without money; requires double coincidence of wants.

\vspace{0.3em}
\textbf{Velocity of Money}
Rate at which money circulates in economy.

\vspace{0.3em}
\textbf{Quantity Theory of Money}
MV = PY relationship linking money, velocity, prices, output.

\column{0.48\textwidth}
\textbf{Seigniorage}
Profit from issuing money; face value minus production cost.

\vspace{0.3em}
\textbf{Inflation Tax}
Hidden tax reducing purchasing power when government prints money.

\vspace{0.3em}
\textbf{Deflationary}
Prices falling over time; fixed-supply currencies trend deflationary.

\vspace{0.3em}
\textbf{Opportunity Cost}
Value of next best alternative foregone when choosing.

\vspace{0.3em}
\textbf{Menu Cost}
Cost of changing prices (like reprinting menus); frequent repricing is expensive.
\end{columns}

\bottomnote{Master these terms before proceeding to subsequent lessons}
\end{frame}

% Key Terms -- Slide 2 of 3: Stablecoins & Market Mechanics
\begin{frame}[t]{Key Terms (2/3): Stablecoins \& Market Mechanics}
\begin{columns}[T]
\column{0.48\textwidth}
\textbf{Stablecoin}
Cryptocurrency maintaining stable value, typically pegged to fiat.

\vspace{0.3em}
\textbf{Fiat-Backed Stablecoin}
Stablecoin backed 1:1 by fiat reserves.

\vspace{0.3em}
\textbf{Algorithmic Stablecoin}
Stablecoin using supply adjustments without full collateral.

\vspace{0.3em}
\textbf{Collateral}
Assets pledged as security for loan or stablecoin.

\vspace{0.3em}
\textbf{Over-collateralization}
Pledging more collateral than loan value for safety.

\vspace{0.3em}
\textbf{Capital Efficiency}
Getting maximum output from minimum capital; over-collateralization is capital-inefficient.

\column{0.48\textwidth}
\textbf{Liquidation}
Forced sale of collateral when value drops below threshold.

\vspace{0.3em}
\textbf{Arbitrage}
Profiting from price differences; maintains stablecoin pegs.

\vspace{0.3em}
\textbf{Death Spiral}
Self-reinforcing collapse where falling prices trigger more selling.

\vspace{0.3em}
\textbf{Risk Assets}
Investments that can lose value (stocks, crypto); opposite of safe assets like government bonds.

\vspace{0.3em}
\textbf{Scalability Trilemma}
Trade-off where blockchains can achieve only two of three: decentralization, security, or speed.
\end{columns}

\bottomnote{Master these terms before proceeding to subsequent lessons}
\end{frame}

% Key Terms -- Slide 3 of 3: Monetary Policy & Crypto
\begin{frame}[t]{Key Terms (3/3): Monetary Policy \& Crypto}
\begin{columns}[T]
\column{0.48\textwidth}
\textbf{Gresham's Law}
``Bad money drives out good''; overvalued currency circulates, undervalued hoarded.

\vspace{0.3em}
\textbf{HODL}
Hold despite price drops; crypto slang resisting sales.

\vspace{0.3em}
\textbf{Dollarization}
Country adopting foreign currency instead of own.

\vspace{0.3em}
\textbf{Money Multiplier}
Bank lending amplifies deposits into larger money supply.

\vspace{0.3em}
\textbf{TradFi}
Traditional Finance; conventional banks versus DeFi.

\column{0.48\textwidth}
\textbf{Monetary Policy Autonomy}
A country's ability to set its own interest rates and control money supply.

\vspace{0.3em}
\textbf{Transmission Mechanism}
How central bank decisions (like rate changes) affect the real economy.

\vspace{0.3em}
\textbf{Pro-cyclical}
Moving with economic cycles---booms get bigger, busts get worse.

\vspace{0.3em}
\textbf{Contagion}
When problems in one market spread to others, like a disease.
\end{columns}

\bottomnote{Master these terms before proceeding to subsequent lessons}
\end{frame}

% References
\begin{frame}[t]{Further Reading}
\begin{columns}[T]
\column{0.48\textwidth}
\textbf{Academic Papers}
\begin{itemize}
\item Yermack (2015): ``Is Bitcoin a Real Currency?''
\item Gorton \& Zhang (2023): ``Taming Wildcat Stablecoins''
\item Brunnermeier et al. (2019): ``The Digitalization of Money''
\end{itemize}

\column{0.48\textwidth}
\textbf{Policy Analysis}
\begin{itemize}
\item BIS (2022): ``The Future Monetary System''
\item IMF (2023): ``Elements of Effective Crypto Policies''
\item ECB (2022): ``Stablecoin Assessment''
\end{itemize}
\end{columns}

\bottomnote{All readings available on course platform}
\end{frame}

\end{document}
