\documentclass[8pt,aspectratio=169]{beamer}
\usetheme{Madrid}
\usepackage{graphicx}
\usepackage{booktabs}
\usepackage{adjustbox}
\usepackage{multicol}
\usepackage{amsmath}

\definecolor{mlblue}{RGB}{0,102,204}
\definecolor{mlpurple}{RGB}{51,51,178}
\definecolor{mllavender}{RGB}{173,173,224}
\definecolor{mllavender2}{RGB}{193,193,232}
\definecolor{mllavender3}{RGB}{204,204,235}
\definecolor{mllavender4}{RGB}{214,214,239}
\definecolor{mlorange}{RGB}{255, 127, 14}
\definecolor{mlgreen}{RGB}{44, 160, 44}
\definecolor{mlred}{RGB}{214, 39, 40}
\definecolor{mlgray}{RGB}{127, 127, 127}
\definecolor{lightgray}{RGB}{240, 240, 240}
\definecolor{midgray}{RGB}{180, 180, 180}

\setbeamercolor{palette primary}{bg=mllavender3,fg=mlpurple}
\setbeamercolor{palette secondary}{bg=mllavender2,fg=mlpurple}
\setbeamercolor{palette tertiary}{bg=mllavender,fg=white}
\setbeamercolor{palette quaternary}{bg=mlpurple,fg=white}
\setbeamercolor{structure}{fg=mlpurple}
\setbeamercolor{section in toc}{fg=mlpurple}
\setbeamercolor{subsection in toc}{fg=mlblue}
\setbeamercolor{title}{fg=mlpurple}
\setbeamercolor{frametitle}{fg=mlpurple,bg=mllavender3}
\setbeamercolor{block title}{bg=mllavender2,fg=mlpurple}
\setbeamercolor{block body}{bg=mllavender4,fg=black}
\setbeamertemplate{navigation symbols}{}
\setbeamertemplate{itemize items}[circle]
\setbeamertemplate{enumerate items}[default]
\setbeamersize{text margin left=5mm,text margin right=5mm}

\newcommand{\bottomnote}[1]{%
\vfill
\vspace{-2mm}
\textcolor{mllavender2}{\rule{\textwidth}{0.4pt}}
\vspace{1mm}
\footnotesize
\textbf{#1}
}

\title{Platform and Token Economics: Mathematical Models and Mechanism Design}
\subtitle{L05 Extended: Formalizing Network Effects, Token Valuation, and Governance\\[0.3em]\normalsize From Katz-Shapiro equilibria to quadratic voting welfare}
\author{Economics of Digital Finance}
\institute{BSc Course}
\date{}

\begin{document}

%% ============================================================
%% SECTION 1: Bridge from Basic Lecture (1 title + 3 content = 4 frames)
%% ============================================================

% --- Frame 1: Title ---
\begin{frame}[plain]
\titlepage
\end{frame}

\section{Bridge from Basic Lecture}

% --- Frame 2: From Concepts to Models ---
\begin{frame}[t]{From Concepts to Models}
\begin{columns}[T]
\column{0.48\textwidth}
\textbf{What We Know from L05}
\begin{enumerate}
\item Platforms exhibit network effects (when a product becomes more valuable as more people use it)
\item Token velocity (how fast tokens change hands) drives token value
\item Winner-take-all (one platform captures most of the market) is common in platform markets
\item Governance design affects fairness and efficiency
\end{enumerate}

\column{0.48\textwidth}
\textbf{What We'll Formalize}
\begin{enumerate}
\item Katz-Shapiro (1985): adoption equilibria with network externalities
\item Samani (2017): token velocity model and staking effects
\item Weyl \& Lalley (2018): quadratic voting welfare analysis
\item Gibrat (1931): stochastic market concentration
\item Hotelling (1929): multi-platform competition with transport costs
\end{enumerate}
\end{columns}

\bottomnote{This lecture builds mathematical foundations for the platform and token concepts introduced in the basic L05 lecture.}
\end{frame}

% --- Frame 3: Mathematical Toolkit (NO Greek) ---
\begin{frame}[t]{Mathematical Toolkit}
\begin{columns}[T]
\column{0.48\textwidth}
\textbf{Key Mathematical Concepts}
\begin{itemize}
\item Equilibrium analysis (finding where supply equals demand, or where no player wants to change strategy)
\item Comparative statics (how the equilibrium shifts when a parameter changes)
\item Welfare measurement (quantifying how well off different groups are under a policy)
\end{itemize}

\column{0.48\textwidth}
\textbf{Notation Preview}

\vspace{0.3em}
\begin{adjustbox}{max width=\textwidth}
\begin{tabular}{ll}
\toprule
\textbf{Symbol} & \textbf{Meaning} \\
\midrule
$V$ & Network value or token velocity \\
$n$ & Number or fraction of adopters \\
$c$ & Adoption cost \\
$s$ & Market share of a platform \\
$M$ & Token supply (number of tokens) \\
$V_{tok}$ & Token velocity (turnover rate) \\
$P$ & Token price in dollars \\
$Q$ & Transaction volume \\
$w$ & Voting weight \\
$x$ & Voting allocation \\
$HHI$ & Herfindahl-Hirschman Index \\
\bottomrule
\end{tabular}
\end{adjustbox}
\end{columns}

\bottomnote{All derivations use BSc-level calculus. Greek letters will be introduced one at a time starting next slide. Full notation in Appendix~A1.}
\end{frame}

% --- Frame 4: Introducing sigma ---
\begin{frame}[t]{Network Effects: The Strength Parameter}
\begin{columns}[T]
\column{0.48\textwidth}
\textbf{Introducing $\sigma$ = network effect strength}

The Katz-Shapiro (1985) model captures how network size creates value. We define:
$$b(n) = \frac{\sigma \cdot n}{1 + \sigma \cdot n}$$

where:
\begin{itemize}
\item $b(n)$ = benefit of joining when fraction $n$ of the market has adopted
\item $\sigma$ = network effect strength (how much each additional user adds to value)
\item $n \in [0, 1]$ = fraction of potential users who have adopted
\end{itemize}

\vspace{0.3em}
\textbf{Key property:} $b(n)$ is concave (rising but at a decreasing rate)---the first users add the most value, later users add less.

\column{0.48\textwidth}
\textbf{Equilibrium Condition}

A user adopts if benefit exceeds cost: $b(n) \geq c$.

\vspace{0.3em}
At equilibrium: $b(n^*) = c$, giving:
$$n^* = \frac{c}{\sigma(1 - c)}$$

\textbf{Three outcomes:}
\begin{enumerate}
\item $n^* = 0$ (nobody adopts---always exists)
\item $n^* = \frac{c}{\sigma(1-c)}$ (tipping point---unstable)
\item Full adoption (stable, if past tipping point)
\end{enumerate}

\vspace{0.3em}
\textbf{Existence condition:} interior equilibrium exists only if $c < \frac{\sigma}{1 + \sigma}$. Otherwise the cost is too high for any adoption.
\end{columns}

\bottomnote{Katz \& Shapiro (1985): the benefit function $b(n) = \sigma n / (1 + \sigma n)$ captures diminishing marginal network effects.}
\end{frame}

%% ============================================================
%% SECTION 2: Katz-Shapiro Network Adoption Equilibria (4 frames)
%% ============================================================
\section{Katz-Shapiro Network Adoption Equilibria}

% --- Frame 5: Worked Example ---
\begin{frame}[t]{Worked Example: Katz-Shapiro Equilibria}
\begin{columns}[T]
\column{0.48\textwidth}
\textbf{Parameters:} $\sigma = 2.0$, $c = 0.25$

\vspace{0.3em}
\textbf{Step 1: Critical boundary}
$$c^* = \frac{\sigma}{1 + \sigma} = \frac{2}{3} = 0.667$$
Since $c = 0.25 < 0.667$: interior equilibrium exists.

\vspace{0.3em}
\textbf{Step 2: Find tipping point}
$$n^* = \frac{c}{\sigma(1-c)} = \frac{0.25}{2.0 \times 0.75} = \frac{0.25}{1.5} = 0.167$$

\vspace{0.3em}
\textbf{Step 3: Verify with $b(n^*)$}
$$b(0.167) = \frac{2 \times 0.167}{1 + 2 \times 0.167} = \frac{0.333}{1.333} = 0.25 = c \;\checkmark$$

\column{0.48\textwidth}
\textbf{Interpretation}

\begin{itemize}
\item $n^* = 0$: no-adoption equilibrium (stable---if nobody uses it, nobody wants to join)
\item $n^* = 0.17$: tipping point (unstable---the ``critical mass'' threshold)
\item Above $n^* = 0.17$: platform tips to full adoption (stable)
\end{itemize}

\vspace{0.3em}
\textbf{Policy implication:} to launch successfully, a platform must subsidize past $n = 0.17$ (17\% of the market). Below that, it collapses to zero.

\vspace{0.3em}
\textbf{Compare across costs:}

\begin{tabular}{lll}
\toprule
$c$ & Tipping $n^*$ & Adoptable? \\
\midrule
0.15 & 0.088 & Yes (easy) \\
0.25 & 0.167 & Yes (moderate) \\
0.35 & 0.269 & Yes (hard) \\
0.70 & $> 1$ & No \\
\bottomrule
\end{tabular}
\end{columns}

\bottomnote{The tipping point $n^* = c/(\sigma(1-c))$ rises with cost and falls with network effect strength---cheaper or more valuable networks tip more easily.}
\end{frame}

% --- Frame 6: Katz-Shapiro Equilibrium Map (Chart 06) ---
\begin{frame}[t]{Katz-Shapiro Equilibrium Map}
\begin{center}
\includegraphics[width=0.60\textwidth]{06_katz_shapiro_equilibrium/chart.pdf}
\end{center}

\begin{itemize}
\item Panel~(a): The purple benefit curve $b(n) = 2n/(1+2n)$ intersects each cost line at the tipping point. Below that point the platform collapses; above it, the platform tips to full adoption.
\item Panel~(b): The phase diagram shows the critical boundary $c^* = \sigma/(1+\sigma)$. Below the curve, adoption is possible; above it, cost is prohibitive regardless of network size.
\end{itemize}

\bottomnote{Katz \& Shapiro (1985): the phase boundary separates the adoption-possible region from the no-adoption region in $(\sigma, c)$ space.}
\end{frame}

% --- Frame 7: Cross-reference and Extensions ---
\begin{frame}[t]{Network Effects: Empirical Evidence and Extensions}
\begin{columns}[T]
\column{0.48\textwidth}
\textbf{Empirical Calibration}

How large is $\sigma$ in practice?

\vspace{0.3em}
\begin{adjustbox}{max width=\textwidth}
\begin{tabular}{lll}
\toprule
\textbf{Platform} & \textbf{Est.\ $\sigma$} & \textbf{Source} \\
\midrule
Bitcoin & 1.5--2.5 & Metcalfe regression \\
Ethereum & 2.0--3.5 & TVL correlations \\
Visa/MC & 3.0--5.0 & Merchant adoption data \\
Telephone (1900s) & 1.0--1.5 & Historical data \\
\bottomrule
\end{tabular}
\end{adjustbox}

\vspace{0.3em}
Higher $\sigma$ means easier tipping. Ethereum's DeFi ecosystem created $\sigma > 2$ through composability (the ability of different DeFi protocols to interact, creating compound network effects).

\column{0.48\textwidth}
\textbf{Extensions}
\begin{itemize}
\item \textbf{Heterogeneous users:} different users have different costs $c_i$, so adoption is gradual rather than all-or-nothing
\item \textbf{Multi-platform:} users can multi-home (use several platforms), weakening winner-take-all
\item \textbf{Dynamic adoption:} users adopt sequentially, not simultaneously---early adopters face higher cost
\end{itemize}

\vspace{0.3em}
\textbf{Cross-reference:} For empirical network valuation using Metcalfe's Law ($V \propto n^2$), see L01 Extended.
\end{columns}

\bottomnote{The Katz-Shapiro model is a building block. Real platforms have heterogeneous users, multi-homing, and dynamic adoption paths.}
\end{frame}

% --- Frame 8: Introducing theta ---
\begin{frame}[t]{Token Velocity: Introducing the Stake Ratio}
\begin{columns}[T]
\column{0.48\textwidth}
\textbf{Introducing $\theta$ = stake ratio}

The stake ratio is the fraction of tokens locked in staking (locked to earn rewards and secure the network) rather than circulating freely.

\vspace{0.3em}
\textbf{Effective velocity:}
$$V_{eff} = V_{base} \times (1 - \theta)$$

\begin{itemize}
\item $V_{base}$ = velocity if all tokens circulated freely
\item $\theta \in [0, 1)$ = fraction locked (staked)
\item $V_{eff}$ = actual velocity of circulating tokens
\end{itemize}

\vspace{0.3em}
\textbf{Token price (from Fisher's equation):}
$$P_{token} = \frac{GDP_{network}}{M \times V_{eff}} = \frac{GDP_{network}}{M \times V_{base} \times (1 - \theta)}$$

\column{0.48\textwidth}
\textbf{Why $\theta$ Matters}

\vspace{0.3em}
\begin{adjustbox}{max width=\textwidth}
\begin{tabular}{llll}
\toprule
\textbf{Token} & $\theta$ & $V_{base}$ & $V_{eff}$ \\
\midrule
BTC & 0 & 5 & 5.00 \\
ETH & 0.25 & 15 & 11.25 \\
SOL & 0.70 & 25 & 7.50 \\
ADA & 0.65 & 20 & 7.00 \\
\bottomrule
\end{tabular}
\end{adjustbox}

\vspace{0.3em}
Despite SOL's high $V_{base} = 25$, heavy staking ($\theta = 0.70$) reduces effective velocity to 7.50---comparable to BTC's 5.0.

\vspace{0.3em}
\textbf{Design insight:} staking is a ``velocity sink'' (mechanism that slows how fast tokens change hands)---it reduces $V_{eff}$ and supports token price.
\end{columns}

\bottomnote{Samani (2017): token velocity is the central challenge in cryptoeconomics. Staking is the primary velocity sink mechanism.}
\end{frame}

%% ============================================================
%% SECTION 3: Token Valuation and Velocity Economics (4 frames)
%% ============================================================
\section{Token Valuation and Velocity Economics}

% --- Frame 9: Worked Example with theta ---
\begin{frame}[t]{Worked Example: Staking and Token Value}
\begin{columns}[T]
\column{0.48\textwidth}
\textbf{Setup:} ETH with $\theta = 0.25$, $V_{base} = 15$

\vspace{0.3em}
\textbf{Step 1: Effective velocity}
$$V_{eff} = 15 \times (1 - 0.25) = 15 \times 0.75 = 11.25$$

\textbf{Step 2: Token price (M = 120M, GDP = \$100B)}
$$P = \frac{100 \times 10^9}{120 \times 10^6 \times 11.25} = \frac{100B}{1.35B} = \$74.07$$

\textbf{Step 3: Compare without staking ($\theta = 0$)}
$$P_{\theta=0} = \frac{100B}{120M \times 15} = \frac{100B}{1.8B} = \$55.56$$

\vspace{0.3em}
\textbf{Value uplift from staking:}
$$\Delta P = \$74.07 - \$55.56 = \$18.52 \;(+33\%)$$

\column{0.48\textwidth}
\textbf{Sensitivity: How $\theta$ Affects Price}

\vspace{0.3em}
\begin{tabular}{lll}
\toprule
$\theta$ & $V_{eff}$ & $P_{token}$ \\
\midrule
0.00 & 15.00 & \$55.56 \\
0.10 & 13.50 & \$61.73 \\
0.25 & 11.25 & \$74.07 \\
0.50 & 7.50 & \$111.11 \\
0.70 & 4.50 & \$185.19 \\
\bottomrule
\end{tabular}

\vspace{0.3em}
\textbf{Warning:} very high $\theta$ (say $> 0.80$) creates illiquidity risk---too few tokens circulating for normal transactions. The protocol freezes.

\vspace{0.3em}
\textbf{Optimal $\theta$:} balances price support against liquidity needs. Most PoS chains target $\theta \in [0.30, 0.70]$.
\end{columns}

\bottomnote{Staking lifts token price by $1/(1-\theta)$. At $\theta = 0.25$: $+33\%$. At $\theta = 0.50$: $+100\%$. Diminishing liquidity is the trade-off.}
\end{frame}

% --- Frame 10: Token Value Heatmap (Chart 07) ---
\begin{frame}[t]{Token Value Heatmap}
\begin{center}
\includegraphics[width=0.60\textwidth]{07_token_value_heatmap/chart.pdf}
\end{center}

\begin{itemize}
\item Panel~(a): Heatmap of token price across velocity and supply. Stars mark BTC, ETH, and SOL at their approximate coordinates. Moving left (lower V) or down (lower M) increases price.
\item Panel~(b): The staking effect---as stake ratio $\theta$ increases, effective velocity drops and price rises. The red zone marks where excessive staking creates dangerous illiquidity.
\end{itemize}

\bottomnote{Samani (2017): $P = GDP/(M \times V)$. Staking reduces effective $V$, acting as a velocity sink that supports token price.}
\end{frame}

% --- Frame 11: Multi-Token Valuation Comparison ---
\begin{frame}[t]{Multi-Token Valuation Comparison}
\begin{columns}[T]
\column{0.48\textwidth}
\textbf{The Equation of Exchange Applied}

\vspace{0.3em}
\begin{adjustbox}{max width=\textwidth}
\begin{tabular}{lrrrrr}
\toprule
& $GDP$ & $M$ & $V_{base}$ & $\theta$ & $P_{model}$ \\
\midrule
BTC & \$4.0T & 19M & 5 & 0 & \$42,105 \\
ETH & \$1.8T & 120M & 15 & 0.25 & \$13,333 \\
SOL & \$1.3T & 400M & 25 & 0.70 & \$433 \\
\bottomrule
\end{tabular}
\end{adjustbox}

\vspace{0.3em}
\textbf{Key insight:} BTC's low velocity ($V = 5$, no staking needed) and tiny supply ($M = 19$M) drive its high per-unit price. SOL has 20$\times$ the supply and 5$\times$ the velocity---but heavy staking partially compensates.

\column{0.48\textwidth}
\textbf{Limitations of the MV = PQ Model}

\begin{itemize}
\item Assumes constant GDP (in reality, GDP grows with adoption)
\item Ignores speculative demand (most crypto volume is speculation, not utility)
\item $V$ is hard to measure (on-chain vs.\ off-chain turnover)
\item Does not capture fee revenue to token holders
\end{itemize}

\vspace{0.3em}
\textbf{Alternative: Discounted Cash Flow}
\begin{itemize}
\item Treat token like equity
\item Value = PV of future fee revenue
\item Works better for fee-earning tokens (ETH, UNI)
\end{itemize}
\end{columns}

\bottomnote{The equation of exchange is a useful first approximation. For precision, combine with DCF models or use network-specific metrics.}
\end{frame}

% --- Frame 12: Introducing phi (voting power) ---
\begin{frame}[t]{Governance Mechanisms: Voting Power Design}
\begin{columns}[T]
\column{0.48\textwidth}
\textbf{Introducing $\phi$ = voting power}

In token governance, $\phi_i$ is the voting power of participant $i$. Different rules assign $\phi$ differently:

\vspace{0.3em}
\textbf{One-Token-One-Vote (1T1V):}
$$\phi_i^{1T1V} = t_i$$
where $t_i$ = tokens held by voter $i$.

\vspace{0.3em}
\textbf{Quadratic Voting (QV):}

Cost of $k$ votes = $k^2$ tokens. So optimal allocation:
$$\phi_i^{QV} = \sqrt{t_i}$$
(a voter with 100 tokens gets 10 votes; with 10{,}000 tokens gets 100 votes---not 100$\times$ more, only 10$\times$ more)

\column{0.48\textwidth}
\textbf{Social Welfare Comparison}

Define social welfare as the sum of voter preferences weighted by voting power:

\vspace{0.3em}
\textbf{1T1V welfare:}
$$W_{1T1V} = \sum_i v_i \cdot \frac{t_i}{\sum_j t_j}$$

\textbf{QV welfare:}
$$W_{QV} = \sum_i v_i \cdot \frac{\sqrt{t_i}}{\sum_j \sqrt{t_j}}$$

where $v_i$ = voter $i$'s preference intensity (how strongly they care about the outcome).

\vspace{0.3em}
\textbf{Key result (Weyl \& Lalley, 2018):} QV is the unique mechanism that maximizes utilitarian welfare when voters have private information about their preference intensity.
\end{columns}

\bottomnote{Weyl \& Lalley (2018): quadratic voting is optimal because it makes the marginal cost of influence proportional to preference intensity.}
\end{frame}

%% ============================================================
%% SECTION 4: Mechanism Design for Token Governance (5 frames)
%% ============================================================
\section{Mechanism Design for Token Governance}

% --- Frame 13: QV Worked Example ---
\begin{frame}[t]{Worked Example: Quadratic Voting in a DAO}
\begin{columns}[T]
\column{0.48\textwidth}
\textbf{Setup:} DAO proposal to change fee from 0.3\% to 0.5\%

\vspace{0.3em}
\textbf{Voters:}

\begin{tabular}{lrlr}
\toprule
\textbf{Voter} & \textbf{Tokens} & \textbf{Preference} & $v_i$ \\
\midrule
Whale & 100{,}000 & Against & $-1$ \\
Fund A & 10{,}000 & For & $+1$ \\
50 users & 100 each & For & $+1$ \\
\bottomrule
\end{tabular}

\vspace{0.3em}
\textbf{Under 1T1V:}
\begin{itemize}
\item Whale: 100{,}000 votes against
\item Fund A: 10{,}000 votes for
\item 50 users: $50 \times 100 = 5{,}000$ votes for
\item Net: $-$85{,}000 $\Rightarrow$ \textbf{Proposal rejected}
\end{itemize}

\column{0.48\textwidth}
\textbf{Under QV:}
\begin{itemize}
\item Whale: $\sqrt{100{,}000} = 316$ votes against
\item Fund A: $\sqrt{10{,}000} = 100$ votes for
\item 50 users: $50 \times \sqrt{100} = 500$ votes for
\item Net: $+284$ $\Rightarrow$ \textbf{Proposal passes}
\end{itemize}

\vspace{0.3em}
\textbf{Welfare analysis:}
\begin{itemize}
\item The 51 ``for'' voters represent broad support
\item The 1 ``against'' voter has concentrated tokens but not more intense preference
\item QV reveals that broad support should win
\item 1T1V masks this by weighting wealth, not preference
\end{itemize}
\end{columns}

\bottomnote{QV gives voice to broad coalitions over concentrated wealth. The whale's 316 votes reflect $\sqrt{\text{tokens}}$, not raw holdings.}
\end{frame}

% --- Frame 14: Governance Welfare Chart (Chart 08) ---
\begin{frame}[t]{Governance Welfare Comparison}
\begin{center}
\includegraphics[width=0.60\textwidth]{08_governance_welfare_comparison/chart.pdf}
\end{center}

\begin{itemize}
\item Panel~(a): As token inequality (Gini coefficient) rises, 1T1V welfare diverges downward from the equal-weight benchmark, while QV stays much closer. The green shaded area shows the QV welfare gain.
\item Panel~(b): At Gini = 0.8, the top whale holds massive 1T1V weight but only moderate QV weight. The ``rest'' (990 voters) collectively gain far more influence under QV.
\end{itemize}

\bottomnote{Buterin, Hitzig \& Weyl (2019): quadratic voting is particularly valuable when token inequality is high---exactly the case in most DAOs.}
\end{frame}

% --- Frame 15: Advanced Governance Mechanisms ---
\begin{frame}[t]{Advanced Governance Mechanisms}
\begin{columns}[T]
\column{0.48\textwidth}
\textbf{Beyond QV: Other Innovations}

\vspace{0.3em}
\textbf{1. Conviction Voting}
\begin{itemize}
\item Voting power accumulates over time while staked on a proposal
\item Rewards long-term conviction over flash votes
\item Used by: 1Hive, Gardens
\end{itemize}

\vspace{0.3em}
\textbf{2. Holographic Consensus}
\begin{itemize}
\item Predictors stake tokens to boost proposals they believe will pass
\item Reduces quorum requirements for ``pre-approved'' proposals
\item Used by: DAOstack
\end{itemize}

\column{0.48\textwidth}
\textbf{3. Futarchy}
\begin{itemize}
\item ``Vote on values, bet on beliefs'' (Hanson, 2013)
\item Two prediction markets per proposal: one conditional on passing, one on rejection
\item Implement whichever market predicts higher welfare metric
\item Used by: Gnosis experiments
\end{itemize}

\vspace{0.3em}
\textbf{Mechanism Design Trade-offs}

\vspace{0.3em}
\begin{adjustbox}{max width=\textwidth}
\begin{tabular}{lccc}
\toprule
& \textbf{Sybil-proof} & \textbf{Fair} & \textbf{Simple} \\
\midrule
1T1V & Yes & No & Yes \\
QV & Partial & Yes & Moderate \\
Conviction & Yes & Moderate & No \\
Futarchy & Yes & Yes & No \\
\bottomrule
\end{tabular}
\end{adjustbox}
\end{columns}

\bottomnote{No mechanism is perfect. Sybil resistance (preventing fake identities from gaining unfair votes) remains the hardest unsolved problem in token governance.}
\end{frame}

% --- Frame 16: Governance Failure Case Studies ---
\begin{frame}[t]{Governance Failure: Case Studies}
\begin{columns}[T]
\column{0.48\textwidth}
\textbf{Case 1: The DAO Hack (2016)}
\begin{itemize}
\item \$60M stolen via reentrancy exploit
\item Governance dilemma: hard fork (reverse theft) or preserve immutability?
\item Ethereum community voted to fork $\Rightarrow$ ETH (forked) vs.\ ETC (original)
\item Lesson: governance must handle unforeseen crises
\end{itemize}

\vspace{0.3em}
\textbf{Case 2: Beanstalk Flash Loan Attack (2022)}
\begin{itemize}
\item Attacker borrowed \$1B in flash loan (borrow and repay in single transaction)
\item Used borrowed tokens to pass governance proposal
\item Drained \$182M from protocol treasury
\item Lesson: 1T1V is vulnerable to temporary token accumulation
\end{itemize}

\column{0.48\textwidth}
\textbf{Case 3: MakerDAO Concentration (ongoing)}
\begin{itemize}
\item Top 20 addresses hold $>$50\% of MKR voting power
\item Most votes pass with $<$5\% participation (rational apathy---when the cost of voting exceeds the expected benefit)
\item Effectively oligarchic despite decentralization narrative
\item Lesson: even well-designed governance concentrates over time
\end{itemize}

\vspace{0.3em}
\textbf{Design Implications}
\begin{itemize}
\item Time locks prevent flash-loan governance attacks
\item Delegation reduces apathy but creates new power centers
\item QV mitigates plutocracy but requires Sybil resistance
\end{itemize}
\end{columns}

\bottomnote{Real-world governance failures reveal limitations of theoretical mechanism design. Implementation details matter as much as the mechanism.}
\end{frame}

% --- Frame 17: Governance Design Scorecard ---
\begin{frame}[t]{Governance Design Scorecard}
\begin{columns}[T]
\column{0.48\textwidth}
\textbf{Evaluating Governance Mechanisms}

\vspace{0.3em}
\begin{adjustbox}{max width=\textwidth}
\begin{tabular}{lccccc}
\toprule
\textbf{Criterion} & \textbf{1T1V} & \textbf{QV} & \textbf{Deleg.} & \textbf{Conv.} & \textbf{Futarchy} \\
\midrule
Sybil-proof & A & C & B & A & A \\
Whale resist. & F & B & D & B & B \\
Participation & B & C & A & B & D \\
Simplicity & A & B & A & D & F \\
Welfare opt. & D & A & C & B & A \\
Flash-loan safe & F & D & B & A & B \\
\bottomrule
\end{tabular}
\end{adjustbox}

\vspace{0.3em}
Grades: A (excellent) to F (failing). No mechanism scores A across all criteria.

\column{0.48\textwidth}
\textbf{Choosing the Right Mechanism}

\begin{itemize}
\item \textbf{High-stakes protocol changes:} time-locked QV with delegation (prevents flash-loan attacks while limiting plutocracy)
\item \textbf{Treasury grants:} conviction voting (rewards deep investigation over snap decisions)
\item \textbf{Parameter tuning:} futarchy (market-based prediction outperforms voting for measurable outcomes)
\item \textbf{Emergency response:} multisig with time delay (speed + security, sacrifices decentralization)
\end{itemize}

\vspace{0.3em}
\textbf{Key insight:} the best governance is context-dependent. Most mature DAOs use hybrid systems combining multiple mechanisms.
\end{columns}

\bottomnote{Practical governance design combines mechanisms. No single voting rule solves all problems---DAOs increasingly adopt context-specific hybrids.}
\end{frame}

%% ============================================================
%% SECTION 5: Multi-Platform Competition (4 frames)
%% ============================================================
\section{Multi-Platform Competition}

% --- Frame 17: Introducing lambda (Hotelling) ---
\begin{frame}[t]{Multi-Platform Competition: The Hotelling Model}
\begin{columns}[T]
\column{0.48\textwidth}
\textbf{Introducing $\lambda$ = transport cost}

In Hotelling's (1929) model, platforms are located on a line and users incur ``transport costs'' (switching costs, learning costs, ecosystem lock-in) proportional to distance.

\vspace{0.3em}
\textbf{User utility from platform $j$:}
$$U_{ij} = V_j + \sigma \cdot n_j - \lambda \cdot |x_i - l_j| - p_j$$

\begin{itemize}
\item $V_j$ = platform intrinsic value
\item $\sigma \cdot n_j$ = network effect (more users = more value)
\item $\lambda \cdot |x_i - l_j|$ = transport cost (mismatch between user preference $x_i$ and platform location $l_j$)
\item $p_j$ = platform fee
\end{itemize}

\column{0.48\textwidth}
\textbf{Blockchain Application}

``Transport cost'' in crypto = cost of switching ecosystems:
\begin{itemize}
\item Learning a new programming language (Solidity vs.\ Move vs.\ Rust)
\item Migrating smart contracts
\item Rebuilding liquidity
\item Losing composability with existing dApps
\end{itemize}

\vspace{0.3em}
\textbf{Key insight:} when $\lambda$ is high (large transport costs), multiple platforms can coexist even with network effects. When $\lambda$ is low (easy switching), winner-take-all dominates.

\vspace{0.3em}
\textbf{Example:} Ethereum and Solana coexist because $\lambda$ is high (different ecosystems, languages, tooling). Ethereum L2s compete more fiercely because $\lambda$ is lower (shared security, similar tooling).
\end{columns}

\bottomnote{Hotelling (1929): transport costs create differentiation that can sustain multiple competing platforms despite network effects.}
\end{frame}

% --- Frame 18: Gibrat with Network Effects ---
\begin{frame}[t]{Stochastic Concentration: Gibrat with Network Effects}
\begin{columns}[T]
\column{0.48\textwidth}
\textbf{Extended Gibrat Model}

The Gibrat (1931) law of proportionate growth, augmented with network effect advantage:
$$S_{i,t+1} = S_{i,t}^{1+\gamma} \times (1 + \mu + \sigma \cdot \varepsilon_{i,t})$$

\begin{itemize}
\item $S_{i,t}$ = market share of platform $i$ at time $t$
\item $\gamma > 0$ = network effect advantage (larger platforms grow disproportionately)
\item $\mu = 0$ = no systematic drift
\item $\sigma = 0.15$ = random shock volatility
\item $\varepsilon_{i,t} \sim N(0, 1)$ = random growth shock
\end{itemize}

After each period, shares are renormalized to sum to 1.

\column{0.48\textwidth}
\textbf{Key results (from simulation)}

\begin{itemize}
\item \textbf{Pure Gibrat ($\gamma = 0$):} random growth alone produces some concentration, but slowly
\item \textbf{With network effects ($\gamma = 0.1$):} concentration is faster and more extreme---one platform typically dominates by $T = 100$
\item \textbf{HHI trajectory:} starts at 0.20 (equal shares among 5), rises to 0.3--0.5 with network effects
\end{itemize}

\vspace{0.3em}
\textbf{Policy implication:} in platform markets with $\gamma > 0$, laissez-faire leads to monopoly. Interoperability mandates (reducing effective $\gamma$) can preserve competition.
\end{columns}

\bottomnote{Gibrat (1931) + Arthur (1989): network effects turn random advantages into persistent dominance through positive feedback loops.}
\end{frame}

% --- Frame 19: Multi-Platform Market Share Chart (Chart 09) ---
\begin{frame}[t]{Multi-Platform Market Share Evolution}
\begin{center}
\includegraphics[width=0.60\textwidth]{09_multiplatform_market_share/chart.pdf}
\end{center}

\begin{itemize}
\item Panel~(a): Five platforms start with equal 20\% share. Random shocks plus network effects ($\gamma = 0.1$) quickly tip the market---one platform emerges dominant while others shrink or stagnate.
\item Panel~(b): HHI comparison shows network effects (red) accelerate concentration far beyond pure random growth (blue dashed). The ``highly concentrated'' threshold (0.25) is reached much earlier.
\end{itemize}

\bottomnote{Shapiro \& Varian (1999): even small network effects transform random variation into persistent winner-take-all concentration.}
\end{frame}

% --- Frame 20: Competition Policy ---
\begin{frame}[t]{Competition Policy for Platform Markets}
\begin{columns}[T]
\column{0.48\textwidth}
\textbf{The Antitrust Challenge}

Traditional antitrust tools (breakups, price regulation) don't translate well to platforms:
\begin{itemize}
\item Breakups destroy network effects (the value IS the network)
\item Price regulation is hard when the ``product'' is a protocol
\item Market definition is unclear (is ETH competing with SOL or with banks?)
\end{itemize}

\vspace{0.3em}
\textbf{New Tools for Platform Markets}
\begin{itemize}
\item \textbf{Interoperability mandates:} require bridges and common standards (reduce $\lambda$)
\item \textbf{Data portability:} users can move their history and assets (reduce switching costs)
\item \textbf{Non-discrimination:} platforms cannot disadvantage rival services
\end{itemize}

\column{0.48\textwidth}
\textbf{The EU Digital Markets Act (DMA)}

\vspace{0.3em}
\begin{adjustbox}{max width=\textwidth}
\begin{tabular}{ll}
\toprule
\textbf{DMA Provision} & \textbf{Crypto Parallel} \\
\midrule
Interoperability & Cross-chain bridges \\
Data portability & Wallet export \\
Anti-self-preferencing & No oracle manipulation \\
Transparency & Open-source code \\
\bottomrule
\end{tabular}
\end{adjustbox}

\vspace{0.3em}
\textbf{Open question:} Can decentralized governance substitute for antitrust regulation? Or does token concentration create the same power asymmetries that antitrust addresses?
\end{columns}

\bottomnote{Competition policy must evolve for platform markets. Decentralization promises self-regulation but token concentration may undermine it.}
\end{frame}

%% ============================================================
%% SECTION 6: Platform Dominance Measurement (4 frames)
%% ============================================================
\section{Platform Dominance Measurement}

% --- Frame 21: HHI and Concentration Metrics ---
\begin{frame}[t]{Measuring Platform Dominance: HHI and Beyond}
\begin{columns}[T]
\column{0.48\textwidth}
\textbf{Herfindahl-Hirschman Index}
$$HHI = \sum_{i=1}^{N} s_i^2$$

where $s_i$ is the market share of firm $i$ (as a fraction).

\vspace{0.3em}
\textbf{Interpretation (DOJ/FTC guidelines):}

\begin{tabular}{ll}
\toprule
HHI & Market Type \\
\midrule
$< 0.15$ & Competitive \\
$0.15$--$0.25$ & Moderately concentrated \\
$> 0.25$ & Highly concentrated \\
$1.0$ & Perfect monopoly \\
\bottomrule
\end{tabular}

\vspace{0.3em}
\textbf{Worked example:} 3 L1 chains with shares 55\%, 30\%, 15\%:
$$HHI = 0.55^2 + 0.30^2 + 0.15^2 = 0.3025 + 0.09 + 0.0225 = 0.415$$
Highly concentrated.

\column{0.48\textwidth}
\textbf{Alternative Metrics}

\vspace{0.3em}
\textbf{Concentration Ratio ($CR_k$):}
$$CR_4 = \sum_{i=1}^{4} s_i$$
(sum of top 4 shares)

\vspace{0.3em}
\textbf{Nakamoto Coefficient:}

Minimum number of entities needed to control $>$50\% of a resource (block production, staking power, governance votes). Lower = more centralized.

\vspace{0.3em}
\begin{tabular}{ll}
\toprule
Chain & Nakamoto Coeff. \\
\midrule
Bitcoin & $\sim$4 (mining pools) \\
Ethereum & $\sim$3 (staking services) \\
Solana & $\sim$19 (validators) \\
\bottomrule
\end{tabular}
\end{columns}

\bottomnote{HHI measures market-level concentration. The Nakamoto coefficient measures within-chain centralization---both matter for platform economics.}
\end{frame}

% --- Frame 22: Cross-Reference and DEX Concentration ---
\begin{frame}[t]{DEX Market Concentration}
\begin{columns}[T]
\column{0.48\textwidth}
\textbf{DEX Market Shares (2024 est.)}

\vspace{0.3em}
\begin{adjustbox}{max width=\textwidth}
\begin{tabular}{lrr}
\toprule
\textbf{DEX} & \textbf{Share} & $s_i^2$ \\
\midrule
Uniswap & 45\% & 0.2025 \\
PancakeSwap & 15\% & 0.0225 \\
Curve & 10\% & 0.0100 \\
SushiSwap & 8\% & 0.0064 \\
Others & 22\% & 0.0484 \\
\midrule
\textbf{HHI} & & \textbf{0.290} \\
\bottomrule
\end{tabular}
\end{adjustbox}

\vspace{0.3em}
Highly concentrated ($HHI > 0.25$), but declining from $\sim$0.95 in 2018 when Uniswap was essentially a monopoly.

\column{0.48\textwidth}
\textbf{Why DEX Concentration Differs from CEX}

\begin{itemize}
\item DEXs are forkable (anyone can copy the code)---low barriers to entry
\item Liquidity is more portable (LPs can move easily between pools)
\item But: brand trust and smart contract audits create switching costs
\item Network effects in liquidity: deeper pools attract more traders
\end{itemize}

\vspace{0.3em}
\textbf{Cross-reference:} For detailed analysis of AMM pricing and liquidity provider dynamics, see L06 Extended (Market Microstructure).
\end{columns}

\bottomnote{DEX markets are concentrating but less than traditional finance. Open-source code reduces barriers but liquidity network effects persist.}
\end{frame}

% --- Frame 23: Token Design Space Chart (Chart 10) ---
\begin{frame}[t]{Token Design Space}
\begin{center}
\includegraphics[width=0.60\textwidth]{10_token_design_space/chart.pdf}
\end{center}

\begin{itemize}
\item Panel~(a): Iso-GDP curves show the velocity-market cap trade-off. BTC sits on a high-GDP curve with low velocity; SOL is on a lower curve with high velocity. Moving left (adding velocity sinks) increases market cap for given GDP.
\item Panel~(b): Token classification by design---payment tokens cluster at high V / low GDP, while store-of-value tokens sit at low V / high GDP. Bubble size reflects market cap.
\end{itemize}

\bottomnote{Samani (2017), Burniske (2017): token design choices (velocity sinks, supply caps) determine where a token sits in the design space.}
\end{frame}

% --- Frame 24: HHI Evolution Chart (Chart 11) ---
\begin{frame}[t]{Platform HHI Evolution}
\begin{center}
\includegraphics[width=0.60\textwidth]{11_platform_hhi_evolution/chart.pdf}
\end{center}

\begin{itemize}
\item Panel~(a): L1 blockchain HHI has declined from 0.65 (Bitcoin dominance) in 2015 to 0.35 in 2024 as Ethereum, Solana, and others gained share. DEX HHI shows similar deconcentration from Uniswap's early monopoly.
\item Panel~(b): Cross-industry comparison reveals crypto markets are less concentrated than Big Tech (web search HHI = 0.92, social media = 0.70) but still above the ``competitive'' threshold.
\end{itemize}

\bottomnote{DOJ/FTC guidelines: HHI above 0.25 is ``highly concentrated.'' Crypto is deconcentrating but remains above this threshold in most segments.}
\end{frame}

%% ============================================================
%% SECTION 7: Synthesis and Policy Implications (3 frames)
%% ============================================================
\section{Synthesis and Policy Implications}

% --- Frame 25: Model Synthesis ---
\begin{frame}[t]{Model Synthesis}
\begin{columns}[T]
\column{0.48\textwidth}
\textbf{Converging Insights}
\begin{itemize}
\item \textbf{Katz-Shapiro:} platforms tip past critical mass ($n^* = c/(\sigma(1-c))$); higher $\sigma$ means easier tipping
\item \textbf{Samani/Fisher:} token price = $GDP/(M \times V_{eff})$; staking reduces $V_{eff}$ and supports price
\item \textbf{Weyl-Lalley:} QV maximizes welfare when token inequality is high (Gini $> 0.5$)
\item \textbf{Gibrat + network effects:} random growth with $\gamma > 0$ produces winner-take-all concentration
\end{itemize}

\column{0.48\textwidth}
\textbf{Design Principles for Token Platforms}

\vspace{0.3em}
\begin{adjustbox}{max width=\textwidth}
\begin{tabular}{ll}
\toprule
\textbf{Principle} & \textbf{Model Basis} \\
\midrule
Subsidize past tipping point & Katz-Shapiro: $n^* > 0$ \\
Add velocity sinks (staking) & Samani: reduce $V_{eff}$ \\
Use QV for governance & Weyl-Lalley: welfare optimal \\
Monitor HHI trajectory & Gibrat: concentration risk \\
Reduce transport costs & Hotelling: enable competition \\
\bottomrule
\end{tabular}
\end{adjustbox}
\end{columns}

\bottomnote{Five distinct models converge on a unified toolkit for platform token design. Each model addresses a different dimension of the problem.}
\end{frame}

% --- Frame 26: Open Questions ---
\begin{frame}[t]{Open Questions and Research Frontiers}
\begin{columns}[T]
\column{0.48\textwidth}
\textbf{Unresolved Issues}
\begin{itemize}
\item \textbf{Optimal staking ratio:} what $\theta$ maximizes long-run protocol health? (Price vs.\ liquidity trade-off)
\item \textbf{Sybil-resistant QV:} how to prevent identity splitting to game $\sqrt{t_i}$?
\item \textbf{Cross-chain network effects:} do bridges create or destroy value?
\item \textbf{Regulatory arbitrage:} does competition policy work when platforms are decentralized?
\end{itemize}

\column{0.48\textwidth}
\textbf{Connections to Other Lectures}
\begin{itemize}
\item \textbf{L01 Extended:} Metcalfe's Law empirical estimation for network valuation
\item \textbf{L03 Extended:} CBDC as a competing platform to DeFi
\item \textbf{L04 Extended:} Payment system network effects and interoperability
\item \textbf{L06 Extended:} AMM market microstructure and liquidity provision
\end{itemize}

\vspace{0.3em}
\textbf{Assignment Connection}
\begin{itemize}
\item Exercises use the Katz-Shapiro tipping point formula
\item Quiz questions test velocity model calculations
\end{itemize}
\end{columns}

\bottomnote{Platform token economics is an active research frontier. These models provide the analytical toolkit for evaluating new proposals and protocols.}
\end{frame}

% --- Frame 27: References ---
\begin{frame}[t]{References}
\small
\begin{itemize}
\setlength{\itemsep}{2pt}
\item Katz, M.L.\ \& Shapiro, C.\ (1985). Network Externalities, Competition, and Compatibility. \textit{American Economic Review}, 75(3), 424--440.
\item Samani, K.\ (2017). Understanding Token Velocity. \textit{Multicoin Capital}.
\item Burniske, C.\ (2017). Cryptoasset Valuations. \textit{Placeholder VC}.
\item Weyl, E.G.\ \& Lalley, S.P.\ (2018). Quadratic Voting: How Mechanism Design Can Radicalize Democracy. \textit{AEA Papers \& Proceedings}, 108, 33--37.
\item Buterin, V., Hitzig, Z.\ \& Weyl, E.G.\ (2019). A Flexible Design for Funding Public Goods. \textit{Management Science}, 65(11), 5171--5187.
\item Gibrat, R.\ (1931). \textit{Les In\'egalit\'es \'Economiques}. Sirey.
\item Shapiro, C.\ \& Varian, H.R.\ (1999). \textit{Information Rules: A Strategic Guide to the Network Economy}. Harvard Business Press.
\item Hotelling, H.\ (1929). Stability in Competition. \textit{Economic Journal}, 39(153), 41--57.
\end{itemize}

\bottomnote{Full references for all models and data sources cited in this lecture.}
\end{frame}

%% ============================================================
%% SECTION 8: Appendix (3 frames)
%% ============================================================
\appendix
\section{Appendix}

% --- Frame 28 (A1): Complete Notation Table ---
\begin{frame}[t]{Appendix A1: Complete Notation Table}
\begin{center}
\begin{adjustbox}{max width=0.95\textwidth, max totalheight=0.78\textheight}
\begin{tabular}{lll}
\toprule
\textbf{Symbol} & \textbf{Meaning} & \textbf{Section} \\
\midrule
$n$ & Adoption fraction $\in [0,1]$ & Katz-Shapiro \\
$\sigma$ & Network effect strength & Katz-Shapiro \\
$c$ & Adoption cost & Katz-Shapiro \\
$b(n)$ & Benefit function: $\sigma n/(1+\sigma n)$ & Katz-Shapiro \\
$n^*$ & Equilibrium adoption: $c/(\sigma(1-c))$ & Katz-Shapiro \\
$V$ & Network value (Metcalfe) or velocity & Various \\
$M$ & Token supply & Token Valuation \\
$V_{tok}$ / $V_{base}$ & Base token velocity & Token Valuation \\
$V_{eff}$ & Effective velocity: $V_{base}(1-\theta)$ & Token Valuation \\
$\theta$ & Stake ratio (fraction locked) & Token Valuation \\
$P$ & Token price: $GDP/(M \times V_{eff})$ & Token Valuation \\
$Q$ & Transaction volume & Token Valuation \\
$\phi_i$ & Voting power of voter $i$ & Governance \\
$t_i$ & Token holdings of voter $i$ & Governance \\
$v_i$ & Preference intensity of voter $i$ & Governance \\
$W$ & Social welfare & Governance \\
$S_{i,t}$ & Market share of platform $i$ at time $t$ & Gibrat \\
$\gamma$ & Network effect growth advantage & Gibrat \\
$\mu$ & Mean growth rate & Gibrat \\
$\lambda$ & Transport cost (Hotelling) & Competition \\
$HHI$ & Herfindahl-Hirschman Index: $\sum s_i^2$ & Dominance \\
$CR_k$ & Concentration ratio (top $k$ shares) & Dominance \\
\bottomrule
\end{tabular}
\end{adjustbox}
\end{center}

\bottomnote{Reference page for all mathematical notation used in this lecture.}
\end{frame}

% --- Frame 29 (A2): Katz-Shapiro Full Derivation ---
\begin{frame}[t]{Appendix A2: Katz-Shapiro Equilibrium Derivation}
\begin{columns}[T]
\column{0.48\textwidth}
\textbf{Benefit Function Derivation}

Start from user $i$'s utility:
$$u_i = v(n) - c = \frac{\sigma n}{1 + \sigma n} - c$$

The benefit $b(n) = \sigma n / (1 + \sigma n)$ is:
\begin{itemize}
\item $b(0) = 0$ (no network, no benefit)
\item $b(n) \to 1$ as $n \to \infty$ (saturating)
\item $b'(n) = \sigma / (1 + \sigma n)^2 > 0$ (increasing)
\item $b''(n) < 0$ (concave---diminishing returns)
\end{itemize}

\textbf{Equilibrium:} $b(n^*) = c$
\begin{align*}
\frac{\sigma n^*}{1 + \sigma n^*} &= c \\
\sigma n^* &= c(1 + \sigma n^*) \\
\sigma n^* - c\sigma n^* &= c \\
n^*(\sigma - c\sigma) &= c \\
n^* &= \frac{c}{\sigma(1 - c)}
\end{align*}

\column{0.48\textwidth}
\textbf{Stability Analysis}

At the interior equilibrium $n^*$:
\begin{itemize}
\item For $n < n^*$: $b(n) < c \Rightarrow$ users leave $\Rightarrow n$ falls to 0
\item For $n > n^*$: $b(n) > c \Rightarrow$ users join $\Rightarrow n$ rises to 1
\end{itemize}

Therefore $n^*$ is \textbf{unstable}---it is the tipping point.

\vspace{0.3em}
\textbf{Existence condition:}
$$n^* \leq 1 \iff \frac{c}{\sigma(1-c)} \leq 1 \iff c \leq \frac{\sigma}{1+\sigma}$$

This defines the critical boundary in the phase diagram (Panel~b of Chart~06).

\vspace{0.3em}
\textbf{Comparative statics:}
$$\frac{\partial n^*}{\partial \sigma} = -\frac{c}{\sigma^2(1-c)} < 0$$

Stronger network effects ($\uparrow \sigma$) lower the tipping point ($\downarrow n^*$), making adoption easier.
\end{columns}

\bottomnote{Full derivation of the Katz-Shapiro fulfilled-expectations equilibrium. The tipping point $n^*$ is unstable by construction.}
\end{frame}

% --- Frame 30 (A3): QV Optimality Proof Sketch ---
\begin{frame}[t]{Appendix A3: Quadratic Voting Optimality}
\begin{columns}[T]
\column{0.48\textwidth}
\textbf{Why $k^2$? The Marginal Cost Argument}

Under QV, the cost of $k$ votes is $k^2$ tokens. The marginal cost of the $k$-th vote:
$$MC(k) = k^2 - (k-1)^2 = 2k - 1$$

This means marginal cost is linear in votes cast. A voter with preference intensity $v_i$ optimizes:
$$\max_k \; v_i \cdot k - k^2$$
FOC: $v_i = 2k^* \Rightarrow k^* = v_i / 2$

\vspace{0.3em}
\textbf{Key property:} votes cast are proportional to preference intensity ($k^* \propto v_i$), regardless of wealth.

\column{0.48\textwidth}
\textbf{Welfare Optimality (Sketch)}

Social welfare:
$$W = \sum_i v_i \cdot d$$
where $d \in \{0, 1\}$ is the policy decision. Optimal $d = 1$ iff $\sum_i v_i > 0$.

Under QV, decision passes iff:
$$\sum_i k_i^* = \sum_i \frac{v_i}{2} > 0 \iff \sum_i v_i > 0$$

So QV implements the welfare-optimal decision in the limit (Weyl \& Lalley, 2018).

\vspace{0.3em}
\textbf{1T1V fails because:}
$$\sum_i t_i \cdot \text{sign}(v_i) \neq \sum_i v_i$$

Wealth ($t_i$) contaminates the aggregation of preferences ($v_i$).
\end{columns}

\bottomnote{Weyl \& Lalley (2018): QV is the unique voting rule where marginal cost equals marginal benefit for truthful preference revelation.}
\end{frame}

\end{document}
