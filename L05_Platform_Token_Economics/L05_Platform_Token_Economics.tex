\documentclass[8pt,aspectratio=169]{beamer}
\usetheme{Madrid}
\usepackage{graphicx}
\usepackage{booktabs}
\usepackage{adjustbox}
\usepackage{multicol}
\usepackage{amsmath}

% Color definitions
\definecolor{mlblue}{RGB}{0,102,204}
\definecolor{mlpurple}{RGB}{51,51,178}
\definecolor{mllavender}{RGB}{173,173,224}
\definecolor{mllavender2}{RGB}{193,193,232}
\definecolor{mllavender3}{RGB}{204,204,235}
\definecolor{mllavender4}{RGB}{214,214,239}
\definecolor{mlorange}{RGB}{255, 127, 14}
\definecolor{mlgreen}{RGB}{44, 160, 44}
\definecolor{mlred}{RGB}{214, 39, 40}
\definecolor{mlgray}{RGB}{127, 127, 127}

\definecolor{lightgray}{RGB}{240, 240, 240}
\definecolor{midgray}{RGB}{180, 180, 180}

% Apply custom colors to Madrid theme
\setbeamercolor{palette primary}{bg=mllavender3,fg=mlpurple}
\setbeamercolor{palette secondary}{bg=mllavender2,fg=mlpurple}
\setbeamercolor{palette tertiary}{bg=mllavender,fg=white}
\setbeamercolor{palette quaternary}{bg=mlpurple,fg=white}

\setbeamercolor{structure}{fg=mlpurple}
\setbeamercolor{section in toc}{fg=mlpurple}
\setbeamercolor{subsection in toc}{fg=mlblue}
\setbeamercolor{title}{fg=mlpurple}
\setbeamercolor{frametitle}{fg=mlpurple,bg=mllavender3}
\setbeamercolor{block title}{bg=mllavender2,fg=mlpurple}
\setbeamercolor{block body}{bg=mllavender4,fg=black}

\setbeamertemplate{navigation symbols}{}
\setbeamertemplate{itemize items}[circle]
\setbeamertemplate{enumerate items}[default]
\setbeamersize{text margin left=5mm,text margin right=5mm}

\newcommand{\bottomnote}[1]{%
\vfill
\vspace{-2mm}
\textcolor{mllavender2}{\rule{\textwidth}{0.4pt}}
\vspace{1mm}
\footnotesize
\textbf{#1}
}

\title{Platform and Token Economics}
\subtitle{L05: Network Effects and Tokenomics}
\author{Economics of Digital Finance}
\institute{BSc Course}
\date{}

\begin{document}

% Title slide
\begin{frame}[plain]
\titlepage
\end{frame}

% Outline
\begin{frame}[t]{Lesson Overview}
\begin{columns}[T]
\column{0.48\textwidth}
\textbf{Today's Topics}
\begin{enumerate}
\item Platform economics foundations
\item Network effects and externalities
\item Critical mass and tipping points
\item Token velocity and value mechanisms
\item Tokenomics supply schedules
\item Winner-take-all market dynamics
\item Two-sided markets in digital finance
\item Governance and voting mechanisms
\end{enumerate}

\column{0.48\textwidth}
\textbf{Learning Objectives}
\begin{itemize}
\item Understand network effects in digital platforms
\item Analyze token velocity and its impact on value
\item Apply platform economics to blockchain adoption
\item Evaluate tokenomics design trade-offs
\item Assess governance mechanisms
\end{itemize}
\end{columns}

\bottomnote{In L04 we studied how payment networks work. Now we ask: \textit{why} do some networks grow to dominate while others fail? Platform economics provides the answer.}
\end{frame}

% Platform Economics Introduction
\begin{frame}[t]{Platform Economics: Introduction}
\begin{columns}[T]
\column{0.48\textwidth}
\textbf{What is a Platform?}

A platform is a two-sided market (platform connecting two user groups who need each other) or multi-sided market that:
\begin{itemize}
\item Facilitates interactions between distinct user groups
\item Creates value through network effects (when a product becomes more valuable as more people use it)
\item Exhibits cross-side externalities (when adding users on one side of the platform benefits users on the other side---e.g., more merchants attract more consumers)
\item Often displays winner-take-all dynamics \textit{(one platform captures most of the market)}
\end{itemize}

\vspace{0.3em}
\textbf{Examples in Digital Finance}
\begin{itemize}
\item Ethereum: developers and users
\item Exchanges: buyers and sellers
\item Payment networks: merchants and consumers
\end{itemize}

\vspace{0.3em}
\textbf{What is a Token?}

A digital asset recorded on a blockchain. Tokens can serve as currency, grant voting rights, provide access to a service, or represent ownership.

\column{0.48\textwidth}
\textbf{Key Economic Features}
\begin{itemize}
\item Network externalities (direct and indirect)
\item Multi-homing costs \textit{(cost of using multiple platforms at once)} and switching costs \textit{(cost of changing platforms)}
\item Platform competition vs. cooperation
\item Governance and control
\end{itemize}

\vspace{0.3em}
\textbf{Why Platform Economics Matters}

Blockchain systems are inherently platforms:
\begin{itemize}
\item Validators (computers that verify blockchain transactions, earning rewards), developers, users interact
\item Token value depends on network size
\item Adoption follows platform dynamics
\end{itemize}
\end{columns}

\bottomnote{Platform economics explains why some cryptocurrencies succeed while others fail}
\end{frame}

% Network Effects and Externalities
\begin{frame}[t]{Network Effects: Theory and Dynamics}
\begin{columns}[T]
\column{0.48\textwidth}
\textbf{Types of Network Effects}

\vspace{0.3em}
\textbf{1. Direct Network Effects}
\begin{itemize}
\item Value increases with same-side users
\item Example: More Bitcoin holders $\rightarrow$ more liquidity \textit{(ease of buying or selling without moving the price)}
\item Katz-Shapiro (1985): If each new user adds \$10 of value to every existing user, then each user's value $V_i = 10n$, and total network value is $10n^2$ (this is Metcalfe's Law).
\end{itemize}

\vspace{0.3em}
\textbf{2. Indirect Network Effects}
\begin{itemize}
\item Value increases with other-side users
\item Example: More Ethereum users $\rightarrow$ more dApps (decentralized applications)
\item Cross-side externalities drive adoption
\end{itemize}

\vspace{0.3em}
\textbf{Real Example: Ethereum}

2020: $\sim$0.4M daily active addresses $\rightarrow$ 200 dApps

2023: $\sim$0.5M daily active addresses $\rightarrow$ 3000+ dApps

More users attract developers; more dApps attract users.

\column{0.48\textwidth}
\textbf{Economic Implications}
\begin{itemize}
\item Positive feedback loops \textit{(self-reinforcing cycles: more users attract more users---snowball effect)}
\item Multiple equilibria possible
\item Coordination problems in early stages \textit{(everyone waits for others to join first)}
\item Path dependence \textit{(today's state depends on history, not just current conditions)} and lock-in \textit{(high switching costs trap users)}
\end{itemize}

\vspace{0.3em}
\textbf{Measurement Challenge}

How to quantify network effects?
\begin{itemize}
\item Active addresses (users)
\item Transaction volume (activity)
\item Developer activity (ecosystem)
\item Metcalfe's Law: $V \propto n^2$

\textit{(Network value grows with the square of users: doubling users quadruples value)}

\textit{Why $n^2$? With $n$ users, each connects to $n-1$ others, giving $n(n-1)/2$ connections---approximately $n^2$. Caveat: not all connections are equally valuable.}
\end{itemize}
\end{columns}

\bottomnote{Katz \& Shapiro (1985): Network externalities create strategic complementarities in adoption}
\end{frame}

% Chart: Platform Adoption Dynamics
\begin{frame}[t]{Platform Adoption Dynamics}
\begin{center}
\includegraphics[width=0.65\textwidth]{01_platform_adoption_dynamics/chart.pdf}
\end{center}

\bottomnote{This chart shows the Katz-Shapiro (1985) fulfilled-expectations model. Where the blue curve crosses the dashed line, expectations match reality (equilibrium). Green dots are stable; red dots are unstable (critical mass). The `Penguin Effect' describes how nobody wants to adopt first.}
\end{frame}

% Critical Mass and Tipping Points
\begin{frame}[t]{Critical Mass and Tipping Points}
\begin{columns}[T]
\column{0.48\textwidth}
\textbf{The Critical Mass Problem}

\vspace{0.3em}
A platform needs critical mass (minimum users needed for network to become self-sustaining) to succeed:
\begin{itemize}
\item Below critical mass: adoption stalls
\item Above critical mass: positive feedback drives growth
\item Tipping point: inflection in adoption curve
\end{itemize}

\vspace{0.3em}
\textbf{Strategic Implications}
\begin{itemize}
\item Early subsidies to reach critical mass
\item Free services to attract one side
\item Cross-subsidization strategies
\end{itemize}

\column{0.48\textwidth}
\textbf{Theoretical Foundation}

Schelling (1978): Coordination games \textit{(situations where everyone benefits from choosing the same option, but agreeing which is the hard part)} with multiple equilibria.

\textit{Equilibrium: a state where no participant wants to change behavior. Stable = the system returns to it after a push. Unstable = any small push sends the system away (the knife-edge tipping point).}

\textit{(Schelling studied tipping points in social settings; Katz \& Shapiro (1985) adapted the concept to technology adoption)}
\begin{itemize}
\item High adoption: stable equilibrium
\item Low adoption: stable equilibrium
\item Critical mass: unstable equilibrium
\end{itemize}

\vspace{0.3em}
\textbf{Application to Tokens}
\begin{itemize}
\item ICOs (Initial Coin Offerings---token fundraising) reduce bootstrapping costs \textit{(the expense of getting from zero users to a viable community)}
\item Token airdrops create initial user base
\item DeFi (Decentralized Finance) liquidity mining \textit{(providing capital---`liquidity'---to a protocol's trading pool in exchange for token rewards)} accelerates adoption
\end{itemize}
\end{columns}

\bottomnote{Schelling (1978): Critical mass creates tipping dynamics in social and economic systems}
\end{frame}

% Chart: Network Effects and Critical Mass
\begin{frame}[t]{Network Effects and Critical Mass}
\begin{center}
\includegraphics[width=0.65\textwidth]{02_network_effects_critical_mass/chart.pdf}
\end{center}

\bottomnote{Value per user increases with network size; critical mass is where adoption becomes self-reinforcing}
\end{frame}

% Token Velocity and Value
\begin{frame}[t]{Token Velocity and Value}
\begin{columns}[T]
\column{0.48\textwidth}
\textbf{The Quantity Equation for Tokens}

Fisher's equation applied to tokens:
\[
MV = PQ
\]

\textit{(Money supply times velocity equals price level times transactions---if tokens circulate faster, each is worth less)}

\begin{itemize}
\item $M$: Token supply (total number of tokens in existence)
\item $V$: Velocity (transactions per period)
\item $P$: Price level in token terms
\item $Q$: Transaction volume (total goods and services bought with the token per period)
\end{itemize}

\vspace{0.3em}
Rearranging for the price level:
\[
P = \frac{MV}{Q}
\]

In dollar terms, token price:
\[
P_{\text{token}} = \frac{\text{Total dollar transaction volume}}{M \times V}
\]

\textit{(Token price = total economic activity divided by supply times velocity)}

\column{0.48\textwidth}
\textbf{The Velocity Problem} \textit{(Why is high velocity bad for token holders?)}

High velocity reduces token value:
\begin{itemize}
\item Tokens used only for transactions
\item No incentive to hold
\item Limited value capture
\end{itemize}

\vspace{0.3em}
\textbf{Velocity Sinks}

Mechanisms to reduce velocity (mechanism that slows how fast tokens change hands):
\begin{itemize}
\item Staking (locking tokens to earn rewards and secure the network) requirements (lock-up periods)
\item Governance (how decisions are made about the protocol) rights (voting power)
\item Fee discounts for holders
\item Burn mechanisms \textit{(permanently destroying tokens to reduce supply---deflationary means total supply shrinks over time, making remaining tokens scarcer)}
\end{itemize}

\vspace{0.3em}
\textbf{Worked Example: MV = PQ}

Suppose: PQ = \$1B/year (network activity), M = 10M tokens.

Token A (payment only, V=50): $P_{\text{token}} = \frac{\$1\text{B}}{10\text{M} \times 50} = \$2.00$

Token B (staking rewards, V=5): $P_{\text{token}} = \frac{\$1\text{B}}{10\text{M} \times 5} = \$20.00$

Token B is 10$\times$ more valuable because it circulates 10$\times$ slower.
\end{columns}

\bottomnote{Samani (2017): Token velocity is central challenge in cryptoeconomics design}
\end{frame}

% Chart: Token Velocity Sink
\begin{frame}[t]{Token Velocity Sinks}
\begin{center}
\includegraphics[width=0.65\textwidth]{03_token_velocity_sink/chart.pdf}
\end{center}

\bottomnote{Staking and governance create holding incentives, reducing velocity and supporting token value}
\end{frame}

% Tokenomics Supply Schedules
\begin{frame}[t]{Tokenomics: Supply Schedules}
\begin{columns}[T]
\column{0.48\textwidth}
\textbf{Supply Design Choices}

\vspace{0.3em}
\textbf{1. Fixed Supply}
\begin{itemize}
\item Example: Bitcoin (21M cap)
\item Deflationary if adoption grows
\item Reduces inflation risk
\item May limit flexibility
\end{itemize}

\vspace{0.3em}
\textbf{2. Inflationary Supply}
\begin{itemize}
\item Example: Ethereum (no hard cap)
\item Rewards validators/miners
\item Funds ecosystem development
\item Dilutes existing holders
\end{itemize}

\column{0.48\textwidth}
\textbf{3. Algorithmic Adjustment}
\begin{itemize}
\item Supply responds to demand
\item Example: Stablecoins---DAI (decentralized, still operating) and Terra/UST (algorithmic, collapsed in 2022 losing \~\$40B---a cautionary tale)
\item Attempts price stability
\item Complex mechanism design
\end{itemize}

\vspace{0.3em}
\textbf{Economic Trade-offs}
\begin{itemize}
\item Credibility vs. flexibility
\item Early adopters vs. late adopters
\item Short-term incentives vs. long-term value
\end{itemize}
\end{columns}

\bottomnote{Token supply schedules balance incentive alignment with long-term sustainability}
\end{frame}

% Chart: Tokenomics Supply Schedule
\begin{frame}[t]{Tokenomics Supply Schedules}
\begin{center}
\includegraphics[width=0.65\textwidth]{04_tokenomics_supply_schedule/chart.pdf}
\end{center}

\bottomnote{Token supply design involves both high-level monetary policy (fixed vs.\ inflationary vs.\ algorithmic) and micro-level vesting schedules (how tokens unlock over time). This chart illustrates vesting mechanics.}
\end{frame}

% Winner-Take-All Dynamics
\begin{frame}[t]{Winner-Take-All Market Dynamics}
\begin{columns}[T]
\column{0.48\textwidth}
\textbf{Why Winner-Take-All?}

Platform markets often concentrate:
\begin{itemize}
\item Network effects favor largest player
\item Multi-homing costs create lock-in \textit{(using multiple platforms is expensive, so users stay with one)}
\item Liquidity begets liquidity \textit{(traders go where other traders are, creating a self-reinforcing cycle)}
\item Switching costs preserve dominance
\end{itemize}

\vspace{0.3em}
\textbf{Evidence in Crypto Markets}
\begin{itemize}
\item Bitcoin dominance in store of value
\item Ethereum dominance in smart contracts (self-executing programs on a blockchain)
\item Exchange concentration (Binance, Coinbase)
\end{itemize}

\column{0.48\textwidth}
\textbf{Theoretical Foundation}

If Platform A has 80\% share and Platform B has 20\%, and both grow at 10\%/year, A gains 8 points while B gains only 2---the big get bigger. This is Gibrat's Law:
\[
\frac{dS_i}{dt} = \alpha S_i + \epsilon_i
\]

\textit{(Larger platforms grow faster on average, leading to market concentration over time)}

\vspace{0.3em}
A simple model (Pólya urn) illustrates this: imagine drawing balls from an urn and adding one more of the same color---early random draws determine the final mix, just as early adopter choices determine which platform dominates.

\vspace{0.3em}
\textbf{Policy Implications}
\begin{itemize}
\item Antitrust (laws preventing excessive market dominance) concerns in platform markets
\item Interoperability (ability of different blockchains to communicate and transfer assets) requirements
\item Challenges to decentralization narrative
\end{itemize}
\end{columns}

\bottomnote{Shapiro \& Varian (1999): Winner-take-all is common in information economies}
\end{frame}

% Chart: Winner-Take-All Market Share
\begin{frame}[t]{Winner-Take-All Market Concentration}
\begin{center}
\includegraphics[width=0.65\textwidth]{05_winner_take_all_market_share/chart.pdf}
\end{center}

\bottomnote{Network effects and switching costs lead to market concentration around dominant platform. Next: dominant platforms are two-sided markets---how do they set prices across both sides?}
\end{frame}

% Two-Sided Markets
\begin{frame}[t]{Two-Sided Markets in Digital Finance}
\begin{columns}[T]
\column{0.48\textwidth}
\textbf{Defining Two-Sided Markets}

Rochet \& Tirole (2003): A market is two-sided if:
\begin{itemize}
\item Platform serves two distinct groups
\item Cross-side externalities exist
\item Price structure matters, not just level \textit{(e.g., charging merchants 3\% and consumers 0\% differs from 1.5\% each)}
\end{itemize}

\vspace{0.3em}
\textbf{Examples}
\begin{itemize}
\item Exchanges: traders and liquidity providers
\item Payment networks: merchants and consumers
\item DeFi protocols: borrowers and lenders
\end{itemize}

\column{0.48\textwidth}
\textbf{Pricing Strategies}

This explains why crypto exchanges offer zero-fee trading to retail (subsidized side) while charging market makers.

Platform can subsidize one side:
\begin{itemize}
\item Loss leaders \textit{(offering below cost to attract one group)}
\item Charge the other side more to recoup losses \textit{(extract the value they gain from the platform)}
\item Example: Free wallets, fee-paying traders
\end{itemize}

\vspace{0.3em}
\textbf{Implications for Tokenomics}
\begin{itemize}
\item Fee structures affect both sides
\item Subsidy design critical for bootstrapping
\item Governance must balance interests
\end{itemize}
\end{columns}

\bottomnote{Rochet \& Tirole (2003, 2006): Two-sided market theory explains platform pricing. Next: How platforms make decisions---governance.}
\end{frame}

% Governance and Voting
\begin{frame}[t]{Governance and Voting Mechanisms}
\begin{columns}[T]
\column{0.48\textwidth}
\textbf{Why Governance Matters}

Token holders often have governance rights:
\begin{itemize}
\item Protocol parameter changes
\item Treasury fund allocation
\item Upgrade decisions
\end{itemize}

\vspace{0.3em}
\textbf{Voting Mechanisms}
\begin{itemize}
\item One-token-one-vote (plutocracy (rule by the wealthy---those with most tokens have most votes) risk)
\item Quadratic voting (voting power increases with square root of tokens, not linearly; 100 tokens $\rightarrow$ 100 linear votes but only $\sqrt{100} = 10$ quadratic votes) (Weyl \& Lalley)
\item Delegated voting \textit{(assign your voting power to a representative who votes on your behalf)}
\item Futarchy \textit{(governance by prediction market---bet on which policy will work best, then implement the winner. Example: Two prediction markets bet on token price under different fee proposals; implement whichever predicts higher value)}
\end{itemize}

\column{0.48\textwidth}
\textbf{Challenges}
\begin{itemize}
\item Low voter turnout (rational apathy---\textit{when the cost of voting exceeds the expected benefit, so voters abstain. If you hold \$100 of tokens and the vote concerns a 0.2\% fee change, the impact on you is \$0.20---not worth 10 minutes of research})
\item Plutocracy: wealth concentration
\item Governance attacks (hostile takeovers)
\item Short-term vs. long-term interests
\end{itemize}

\vspace{0.3em}
\textbf{Mechanism Design} \textit{(creating rules and incentives so self-interested actors produce desired outcomes)}

How to align incentives?
\begin{itemize}
\item Time-weighted voting \textit{(voting power increases the longer you hold tokens, rewarding long-term commitment)}
\item Skin-in-the-game requirements \textit{(voters must lock tokens at risk if their vote harms the protocol)}
\item Reputation systems \textit{(tracking voter history to weight votes by past accuracy)}
\end{itemize}

\vspace{0.3em}
\textbf{Real Example: DAO (Decentralized Autonomous Organization---run by smart contracts and token-holder votes instead of a board) Voting}

Proposal: Change protocol fee from 0.3\% to 0.5\%

Whale with 100K tokens: 100K votes

100 users with 100 tokens: 10K votes

Whale wins despite majority opposition.
\end{columns}

\bottomnote{Weyl \& Lalley (2018): Quadratic voting mitigates plutocracy in token governance}
\end{frame}

% Platform Competition
\begin{frame}[t]{Platform Competition Dynamics}
\begin{columns}[T]
\column{0.48\textwidth}
\textbf{Compete or Cooperate?}

Platforms face strategic choices:
\begin{itemize}
\item Proprietary standards \textit{(owned by one company)} vs. open protocols \textit{(freely usable by anyone)}
\item Exclusivity vs. interoperability
\item Walled gardens \textit{(closed systems that restrict users from leaving)} vs. open ecosystems
\end{itemize}

\vspace{0.3em}
\textbf{Blockchain Context}
\begin{itemize}
\item L1 (Layer 1---base blockchains like Ethereum) blockchains compete for users
\item L2 (Layer 2---solutions built on top of L1) solutions cooperate with L1s
\item Cross-chain bridges \textit{(protocols that transfer assets between different blockchains)} enable multi-homing
\end{itemize}

\column{0.48\textwidth}
\textbf{Strategic Trade-offs}

\vspace{0.3em}
\textbf{Exclusivity Benefits}
\begin{itemize}
\item Capture full value from users
\item Differentiation and branding
\item Control over user experience
\end{itemize}

\vspace{0.3em}
\textbf{Interoperability Benefits}
\begin{itemize}
\item Larger network effects
\item Reduced user friction
\item Ecosystem growth
\end{itemize}
\end{columns}

\bottomnote{Farrell \& Saloner (1985): Standardization and compatibility in platform markets}
\end{frame}

% Tokenomics Design Principles
\begin{frame}[t]{Tokenomics Design: Key Principles}
\begin{columns}[T]
\column{0.48\textwidth}
\textbf{1. Incentive Alignment}

Token design must align stakeholders:
\begin{itemize}
\item Users: utility and low fees
\item Developers: rewards and funding
\item Validators: security incentives
\item Investors: value appreciation
\end{itemize}

\vspace{0.3em}
\textbf{2. Velocity Management}
\begin{itemize}
\item Create holding incentives (staking)
\item Avoid pure transaction tokens
\item Add non-financial utility
\end{itemize}

\column{0.48\textwidth}
\textbf{3. Supply Credibility}
\begin{itemize}
\item Clear, predictable issuance
\item Algorithmic enforcement
\item Resist arbitrary changes
\end{itemize}

\vspace{0.3em}
\textbf{4. Governance Design}
\begin{itemize}
\item Avoid plutocracy
\item Ensure long-term focus
\item Balance efficiency and inclusiveness
\end{itemize}
\end{columns}

\vspace{0.5em}
\textbf{Core Message}

Good tokenomics balances short-term adoption incentives with long-term value sustainability.

\bottomnote{Effective tokenomics requires careful mechanism design informed by platform economics}
\end{frame}

% Key Takeaways
\begin{frame}[t]{Key Takeaways}
\begin{columns}[T]
\column{0.48\textwidth}
\textbf{What We Covered}
\begin{enumerate}
\item Platform economics foundations
\item Network effects and critical mass
\item Token velocity and value mechanisms
\item Supply schedule design
\item Winner-take-all dynamics
\item Two-sided markets
\item Governance mechanisms
\end{enumerate}

\column{0.48\textwidth}
\textbf{Core Insights}
\begin{itemize}
\item Network effects drive crypto adoption
\item Token value depends on velocity management
\item Supply schedules balance incentives
\item Governance is a mechanism design problem
\item Winner-take-all is common but not inevitable
\end{itemize}
\end{columns}

\vspace{0.5em}
\textbf{Looking Ahead}

Next lesson (L06): Market microstructure---how crypto markets discover prices and provide liquidity.

\bottomnote{Platform economics explains the success and failure patterns in cryptocurrency markets}
\end{frame}

% Key Terms (Page 1 of 2)
\begin{frame}[t]{Key Terms (1/2)}
\begin{columns}[T]
\column{0.48\textwidth}
\textbf{Network Externality}
Benefit or cost imposed on users when another user joins the network.

\vspace{0.3em}
\textbf{Critical Mass}
Minimum adoption level needed for network effects to become self-sustaining.

\vspace{0.3em}
\textbf{Token Velocity}
Rate at which tokens change hands; high velocity can depress token value.

\vspace{0.3em}
\textbf{Tokenomics}
Economic design of token systems including supply schedules, incentives, and governance rights.

\vspace{0.3em}
\textbf{Winner-Take-All}
Market dynamic where network effects lead to one dominant platform capturing most market share.

\vspace{0.3em}
\textbf{Platform Governance}
Rules and mechanisms for decision-making about platform evolution and resource allocation.

\vspace{0.3em}
\textbf{Two-Sided Market}
Platform connecting two user groups who need each other (e.g., merchants and consumers).

\vspace{0.3em}
\textbf{Cross-Side Externalities}
Effects on one user group when the other group grows (e.g., more users attract more developers).

\column{0.48\textwidth}
\textbf{Multi-Homing Costs}
Costs of using multiple competing platforms simultaneously.

\vspace{0.3em}
\textbf{Switching Costs}
Costs (financial or behavioral) of moving from one platform to another.

\vspace{0.3em}
\textbf{Path Dependence}
Current state depends on historical choices, not just current conditions.

\vspace{0.3em}
\textbf{Lock-In}
Situation where switching costs make users captive to existing platform.

\vspace{0.3em}
\textbf{Velocity Sink}
Mechanism that slows how fast tokens change hands, supporting value.

\vspace{0.3em}
\textbf{Staking}
Locking tokens to earn rewards and secure the network.

\vspace{0.3em}
\textbf{Burn Mechanism}
Process of permanently removing tokens from circulation to create scarcity.

\vspace{0.3em}
\textbf{Metcalfe's Law}
Network value grows with the square of users: $V \propto n^2$.
\end{columns}

\bottomnote{Platform economics explains why some crypto networks thrive while others fail}
\end{frame}

% Key Terms (Page 2 of 2)
\begin{frame}[t]{Key Terms (2/2)}
\begin{columns}[T]
\column{0.48\textwidth}
\textbf{Plutocracy}
Rule by the wealthy---those with most tokens have most votes.

\vspace{0.3em}
\textbf{Quadratic Voting}
Voting power increases with square root of tokens, not linearly, to mitigate plutocracy.

\vspace{0.3em}
\textbf{Liquidity Mining}
Distributing tokens to users who provide liquidity to protocols, accelerating adoption.

\vspace{0.3em}
\textbf{Airdrops}
Free distribution of tokens to wallet addresses to bootstrap network or reward loyalty.

\column{0.48\textwidth}
\textbf{dApps}
Decentralized applications built on blockchain platforms.

\vspace{0.3em}
\textbf{ICO}
Initial Coin Offering---token fundraising mechanism for blockchain projects.

\vspace{0.3em}
\textbf{Layer 1 (L1)}
Base blockchain protocols (e.g., Ethereum, Bitcoin).

\vspace{0.3em}
\textbf{Layer 2 (L2)}
Scaling solutions built on top of Layer 1 blockchains.

\vspace{0.3em}
\textbf{Gini Coefficient}
Measure of inequality from 0 (perfect equality) to 1 (one entity has everything). Used to measure market concentration.

\vspace{0.3em}
\textbf{Validators}
Computers that verify blockchain transactions, earning rewards.

\vspace{0.3em}
\textbf{Protocol}
The rules governing how a blockchain network operates.

\vspace{0.3em}
\textbf{Interoperability}
Ability of different blockchains to communicate and transfer assets.
\end{columns}

\bottomnote{Understanding these terms is essential for analyzing platform and token economics}
\end{frame}

% Further Reading
\begin{frame}[t]{Further Reading}
\begin{columns}[T]
\column{0.48\textwidth}
\textbf{Foundational Papers}
\begin{itemize}
\item Katz \& Shapiro (1985): ``Network Externalities, Competition, and Compatibility''
\item Rochet \& Tirole (2003): ``Platform Competition in Two-Sided Markets''
\item Catalini \& Gans (2020): ``Some Simple Economics of the Blockchain''
\end{itemize}

\column{0.48\textwidth}
\textbf{Tokenomics}
\begin{itemize}
\item Samani (2017): ``Velocity of Tokens'' (Medium)
\item Buterin (2017): ``On Medium-of-Exchange Token Valuations''
\item Weyl \& Lalley (2018): ``Quadratic Voting''
\end{itemize}
\end{columns}

\bottomnote{All readings available on course platform}
\end{frame}

\end{document}
