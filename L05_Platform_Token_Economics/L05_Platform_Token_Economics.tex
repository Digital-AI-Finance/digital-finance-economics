\documentclass[8pt,aspectratio=169]{beamer}
\usetheme{Madrid}
\usepackage{graphicx}
\usepackage{booktabs}
\usepackage{adjustbox}
\usepackage{multicol}
\usepackage{amsmath}

% Color definitions
\definecolor{mlblue}{RGB}{0,102,204}
\definecolor{mlpurple}{RGB}{51,51,178}
\definecolor{mllavender}{RGB}{173,173,224}
\definecolor{mllavender2}{RGB}{193,193,232}
\definecolor{mllavender3}{RGB}{204,204,235}
\definecolor{mllavender4}{RGB}{214,214,239}
\definecolor{mlorange}{RGB}{255, 127, 14}
\definecolor{mlgreen}{RGB}{44, 160, 44}
\definecolor{mlred}{RGB}{214, 39, 40}
\definecolor{mlgray}{RGB}{127, 127, 127}

\definecolor{lightgray}{RGB}{240, 240, 240}
\definecolor{midgray}{RGB}{180, 180, 180}

% Apply custom colors to Madrid theme
\setbeamercolor{palette primary}{bg=mllavender3,fg=mlpurple}
\setbeamercolor{palette secondary}{bg=mllavender2,fg=mlpurple}
\setbeamercolor{palette tertiary}{bg=mllavender,fg=white}
\setbeamercolor{palette quaternary}{bg=mlpurple,fg=white}

\setbeamercolor{structure}{fg=mlpurple}
\setbeamercolor{section in toc}{fg=mlpurple}
\setbeamercolor{subsection in toc}{fg=mlblue}
\setbeamercolor{title}{fg=mlpurple}
\setbeamercolor{frametitle}{fg=mlpurple,bg=mllavender3}
\setbeamercolor{block title}{bg=mllavender2,fg=mlpurple}
\setbeamercolor{block body}{bg=mllavender4,fg=black}

\setbeamertemplate{navigation symbols}{}
\setbeamertemplate{itemize items}[circle]
\setbeamertemplate{enumerate items}[default]
\setbeamersize{text margin left=5mm,text margin right=5mm}

\newcommand{\bottomnote}[1]{%
\vfill
\vspace{-2mm}
\textcolor{mllavender2}{\rule{\textwidth}{0.4pt}}
\vspace{1mm}
\footnotesize
\textbf{#1}
}

\title{Platform and Token Economics}
\subtitle{L05: Network Effects and Tokenomics}
\author{Economics of Digital Finance}
\institute{BSc Course}
\date{}

\begin{document}

% Title slide
\begin{frame}[plain]
\titlepage
\end{frame}

% Outline
\begin{frame}[t]{Lesson Overview}
\begin{columns}[T]
\column{0.48\textwidth}
\textbf{Today's Topics}
\begin{enumerate}
\item Platform economics foundations
\item Network effects and externalities
\item Critical mass and tipping points
\item Token velocity and value mechanisms
\item Tokenomics supply schedules
\item Winner-take-all market dynamics
\item Two-sided markets in digital finance
\item Governance and voting mechanisms
\end{enumerate}

\column{0.48\textwidth}
\textbf{Learning Objectives}
\begin{itemize}
\item Understand network effects in digital platforms
\item Analyze token velocity and its impact on value
\item Apply platform economics to blockchain adoption
\item Evaluate tokenomics design trade-offs
\item Assess governance mechanisms
\end{itemize}
\end{columns}

\bottomnote{This lesson applies platform economics to understand cryptocurrency and token systems}
\end{frame}

% Platform Economics Introduction
\begin{frame}[t]{Platform Economics: Introduction}
\begin{columns}[T]
\column{0.48\textwidth}
\textbf{What is a Platform?}

A platform is a two-sided or multi-sided market that:
\begin{itemize}
\item Facilitates interactions between distinct user groups
\item Creates value through network effects
\item Exhibits cross-side externalities
\item Often displays winner-take-all dynamics
\end{itemize}

\vspace{0.3em}
\textbf{Examples in Digital Finance}
\begin{itemize}
\item Ethereum: developers and users
\item Exchanges: buyers and sellers
\item Payment networks: merchants and consumers
\end{itemize}

\column{0.48\textwidth}
\textbf{Key Economic Features}
\begin{itemize}
\item Network externalities (direct and indirect)
\item Multi-homing costs and switching costs
\item Platform competition vs. cooperation
\item Governance and control
\end{itemize}

\vspace{0.3em}
\textbf{Why Platform Economics Matters}

Blockchain systems are inherently platforms:
\begin{itemize}
\item Validators, developers, users interact
\item Token value depends on network size
\item Adoption follows platform dynamics
\end{itemize}
\end{columns}

\bottomnote{Platform economics explains why some cryptocurrencies succeed while others fail}
\end{frame}

% Network Effects and Externalities
\begin{frame}[t]{Network Effects: Theory and Dynamics}
\begin{columns}[T]
\column{0.48\textwidth}
\textbf{Types of Network Effects}

\vspace{0.3em}
\textbf{1. Direct Network Effects}
\begin{itemize}
\item Value increases with same-side users
\item Example: More Bitcoin holders $\rightarrow$ more liquidity
\item Katz-Shapiro (1985): $V_i = v(n)$ where $v'(n) > 0$
\end{itemize}

\vspace{0.3em}
\textbf{2. Indirect Network Effects}
\begin{itemize}
\item Value increases with other-side users
\item Example: More Ethereum users $\rightarrow$ more dApps
\item Cross-side externalities drive adoption
\end{itemize}

\column{0.48\textwidth}
\textbf{Economic Implications}
\begin{itemize}
\item Positive feedback loops
\item Multiple equilibria possible
\item Coordination problems in early stages
\item Path dependence and lock-in
\end{itemize}

\vspace{0.3em}
\textbf{Measurement Challenge}

How to quantify network effects?
\begin{itemize}
\item Active addresses (users)
\item Transaction volume (activity)
\item Developer activity (ecosystem)
\item Metcalfe's Law: $V \propto n^2$
\end{itemize}
\end{columns}

\bottomnote{Katz \& Shapiro (1985): Network externalities create strategic complementarities in adoption}
\end{frame}

% Chart: Platform Adoption Dynamics
\begin{frame}[t]{Platform Adoption Dynamics}
\begin{center}
\includegraphics[width=0.65\textwidth]{01_platform_adoption_dynamics/chart.pdf}
\end{center}

\bottomnote{S-curve adoption patterns reflect network effects: slow start, rapid growth, eventual saturation}
\end{frame}

% Critical Mass and Tipping Points
\begin{frame}[t]{Critical Mass and Tipping Points}
\begin{columns}[T]
\column{0.48\textwidth}
\textbf{The Critical Mass Problem}

\vspace{0.3em}
A platform needs critical mass to become self-sustaining:
\begin{itemize}
\item Below critical mass: adoption stalls
\item Above critical mass: positive feedback drives growth
\item Tipping point: inflection in adoption curve
\end{itemize}

\vspace{0.3em}
\textbf{Strategic Implications}
\begin{itemize}
\item Early subsidies to reach critical mass
\item Free services to attract one side
\item Cross-subsidization strategies
\end{itemize}

\column{0.48\textwidth}
\textbf{Theoretical Foundation}

Schelling (1978): Coordination games with multiple equilibria
\begin{itemize}
\item High adoption: stable equilibrium
\item Low adoption: stable equilibrium
\item Critical mass: unstable equilibrium
\end{itemize}

\vspace{0.3em}
\textbf{Application to Tokens}
\begin{itemize}
\item Initial coin offerings (ICOs) reduce bootstrapping costs
\item Token airdrops create initial user base
\item Liquidity mining accelerates adoption
\end{itemize}
\end{columns}

\bottomnote{Schelling (1978): Critical mass creates tipping dynamics in social and economic systems}
\end{frame}

% Chart: Network Effects and Critical Mass
\begin{frame}[t]{Network Effects and Critical Mass}
\begin{center}
\includegraphics[width=0.65\textwidth]{02_network_effects_critical_mass/chart.pdf}
\end{center}

\bottomnote{Value per user increases with network size; critical mass is where adoption becomes self-reinforcing}
\end{frame}

% Token Velocity and Value
\begin{frame}[t]{Token Velocity and Value}
\begin{columns}[T]
\column{0.48\textwidth}
\textbf{The Quantity Equation for Tokens}

Fisher's equation applied to tokens:
\[
MV = PQ
\]
\begin{itemize}
\item $M$: Token supply (monetary base)
\item $V$: Velocity (transactions per period)
\item $P$: Price level in token terms
\item $Q$: Real output (network activity)
\end{itemize}

\vspace{0.3em}
Solving for token value:
\[
P_{\text{token}} = \frac{PQ}{MV}
\]

\column{0.48\textwidth}
\textbf{The Velocity Problem}

High velocity reduces token value:
\begin{itemize}
\item Tokens used only for transactions
\item No incentive to hold
\item Limited value capture
\end{itemize}

\vspace{0.3em}
\textbf{Velocity Sinks}

Mechanisms to reduce velocity:
\begin{itemize}
\item Staking requirements (lock-up periods)
\item Governance rights (voting power)
\item Fee discounts for holders
\item Burn mechanisms (deflationary)
\end{itemize}
\end{columns}

\bottomnote{Samuelson (2017): Token velocity is central challenge in cryptoeconomics design}
\end{frame}

% Chart: Token Velocity Sink
\begin{frame}[t]{Token Velocity Sinks}
\begin{center}
\includegraphics[width=0.65\textwidth]{03_token_velocity_sink/chart.pdf}
\end{center}

\bottomnote{Staking and governance create holding incentives, reducing velocity and supporting token value}
\end{frame}

% Tokenomics Supply Schedules
\begin{frame}[t]{Tokenomics: Supply Schedules}
\begin{columns}[T]
\column{0.48\textwidth}
\textbf{Supply Design Choices}

\vspace{0.3em}
\textbf{1. Fixed Supply}
\begin{itemize}
\item Example: Bitcoin (21M cap)
\item Deflationary if adoption grows
\item Reduces inflation risk
\item May limit flexibility
\end{itemize}

\vspace{0.3em}
\textbf{2. Inflationary Supply}
\begin{itemize}
\item Example: Ethereum (no hard cap)
\item Rewards validators/miners
\item Funds ecosystem development
\item Dilutes existing holders
\end{itemize}

\column{0.48\textwidth}
\textbf{3. Algorithmic Adjustment}
\begin{itemize}
\item Supply responds to demand
\item Example: Stablecoins (DAI, Terra)
\item Attempts price stability
\item Complex mechanism design
\end{itemize}

\vspace{0.3em}
\textbf{Economic Trade-offs}
\begin{itemize}
\item Credibility vs. flexibility
\item Early adopters vs. late adopters
\item Short-term incentives vs. long-term value
\end{itemize}
\end{columns}

\bottomnote{Token supply schedules balance incentive alignment with long-term sustainability}
\end{frame}

% Chart: Tokenomics Supply Schedule
\begin{frame}[t]{Tokenomics Supply Schedules}
\begin{center}
\includegraphics[width=0.65\textwidth]{04_tokenomics_supply_schedule/chart.pdf}
\end{center}

\bottomnote{Different supply schedules create distinct inflation dynamics and holder incentives}
\end{frame}

% Winner-Take-All Dynamics
\begin{frame}[t]{Winner-Take-All Market Dynamics}
\begin{columns}[T]
\column{0.48\textwidth}
\textbf{Why Winner-Take-All?}

Platform markets often concentrate:
\begin{itemize}
\item Network effects favor largest player
\item Multi-homing costs create lock-in
\item Liquidity begets liquidity
\item Switching costs preserve dominance
\end{itemize}

\vspace{0.3em}
\textbf{Evidence in Crypto Markets}
\begin{itemize}
\item Bitcoin dominance in store of value
\item Ethereum dominance in smart contracts
\item Exchange concentration (Binance, Coinbase)
\end{itemize}

\column{0.48\textwidth}
\textbf{Theoretical Foundation}

Gibrat's Law: proportional growth leads to concentration
\[
\frac{dS_i}{dt} = \alpha S_i + \epsilon_i
\]
Where $S_i$ is firm $i$'s market share.

\vspace{0.3em}
\textbf{Policy Implications}
\begin{itemize}
\item Antitrust concerns in platform markets
\item Interoperability requirements
\item Challenges to decentralization narrative
\end{itemize}
\end{columns}

\bottomnote{Shapiro \& Varian (1999): Winner-take-all is common in information economies}
\end{frame}

% Chart: Winner-Take-All Market Share
\begin{frame}[t]{Winner-Take-All Market Concentration}
\begin{center}
\includegraphics[width=0.65\textwidth]{05_winner_take_all_market_share/chart.pdf}
\end{center}

\bottomnote{Network effects and switching costs lead to market concentration around dominant platform}
\end{frame}

% Two-Sided Markets
\begin{frame}[t]{Two-Sided Markets in Digital Finance}
\begin{columns}[T]
\column{0.48\textwidth}
\textbf{Defining Two-Sided Markets}

Rochet \& Tirole (2003): A market is two-sided if:
\begin{itemize}
\item Platform serves two distinct groups
\item Cross-side externalities exist
\item Price structure matters (not just level)
\end{itemize}

\vspace{0.3em}
\textbf{Examples}
\begin{itemize}
\item Exchanges: traders and liquidity providers
\item Payment networks: merchants and consumers
\item DeFi protocols: borrowers and lenders
\end{itemize}

\column{0.48\textwidth}
\textbf{Pricing Strategies}

Platform can subsidize one side:
\begin{itemize}
\item Loss leaders to attract one group
\item Extract surplus from other side
\item Example: Free wallets, fee-paying traders
\end{itemize}

\vspace{0.3em}
\textbf{Implications for Tokenomics}
\begin{itemize}
\item Fee structures affect both sides
\item Subsidy design critical for bootstrapping
\item Governance must balance interests
\end{itemize}
\end{columns}

\bottomnote{Rochet \& Tirole (2003, 2006): Two-sided market theory explains platform pricing}
\end{frame}

% Governance and Voting
\begin{frame}[t]{Governance and Voting Mechanisms}
\begin{columns}[T]
\column{0.48\textwidth}
\textbf{Why Governance Matters}

Token holders often have governance rights:
\begin{itemize}
\item Protocol parameter changes
\item Treasury fund allocation
\item Upgrade decisions
\end{itemize}

\vspace{0.3em}
\textbf{Voting Mechanisms}
\begin{itemize}
\item One-token-one-vote (plutocracy risk)
\item Quadratic voting (Weyl \& Lalley)
\item Delegated voting (representation)
\item Futarchy (prediction markets)
\end{itemize}

\column{0.48\textwidth}
\textbf{Challenges}
\begin{itemize}
\item Low voter turnout (rational apathy)
\item Plutocracy: wealth concentration
\item Governance attacks (hostile takeovers)
\item Short-term vs. long-term interests
\end{itemize}

\vspace{0.3em}
\textbf{Mechanism Design}

How to align incentives?
\begin{itemize}
\item Time-weighted voting (long-term holders)
\item Skin-in-the-game requirements
\item Reputation systems
\end{itemize}
\end{columns}

\bottomnote{Weyl \& Lalley (2018): Quadratic voting mitigates plutocracy in token governance}
\end{frame}

% Platform Competition
\begin{frame}[t]{Platform Competition Dynamics}
\begin{columns}[T]
\column{0.48\textwidth}
\textbf{Compete or Cooperate?}

Platforms face strategic choices:
\begin{itemize}
\item Proprietary standards vs. open protocols
\item Exclusivity vs. interoperability
\item Walled gardens vs. ecosystems
\end{itemize}

\vspace{0.3em}
\textbf{Blockchain Context}
\begin{itemize}
\item L1 blockchains compete for users
\item L2 solutions cooperate with L1s
\item Cross-chain bridges enable multi-homing
\end{itemize}

\column{0.48\textwidth}
\textbf{Strategic Trade-offs}

\vspace{0.3em}
\textbf{Exclusivity Benefits}
\begin{itemize}
\item Capture full value from users
\item Differentiation and branding
\item Control over user experience
\end{itemize}

\vspace{0.3em}
\textbf{Interoperability Benefits}
\begin{itemize}
\item Larger network effects
\item Reduced user friction
\item Ecosystem growth
\end{itemize}
\end{columns}

\bottomnote{Farrell \& Saloner (1985): Standardization and compatibility in platform markets}
\end{frame}

% Tokenomics Design Principles
\begin{frame}[t]{Tokenomics Design: Key Principles}
\begin{columns}[T]
\column{0.48\textwidth}
\textbf{1. Incentive Alignment}

Token design must align stakeholders:
\begin{itemize}
\item Users: utility and low fees
\item Developers: rewards and funding
\item Validators: security incentives
\item Investors: value appreciation
\end{itemize}

\vspace{0.3em}
\textbf{2. Velocity Management}
\begin{itemize}
\item Create holding incentives (staking)
\item Avoid pure transaction tokens
\item Add non-financial utility
\end{itemize}

\column{0.48\textwidth}
\textbf{3. Supply Credibility}
\begin{itemize}
\item Clear, predictable issuance
\item Algorithmic enforcement
\item Resist arbitrary changes
\end{itemize}

\vspace{0.3em}
\textbf{4. Governance Design}
\begin{itemize}
\item Avoid plutocracy
\item Ensure long-term focus
\item Balance efficiency and inclusiveness
\end{itemize}
\end{columns}

\vspace{0.5em}
\textbf{Core Message}

Good tokenomics balances short-term adoption incentives with long-term value sustainability.

\bottomnote{Effective tokenomics requires careful mechanism design informed by platform economics}
\end{frame}

% Key Takeaways
\begin{frame}[t]{Key Takeaways}
\begin{columns}[T]
\column{0.48\textwidth}
\textbf{What We Covered}
\begin{enumerate}
\item Platform economics foundations
\item Network effects and critical mass
\item Token velocity and value mechanisms
\item Supply schedule design
\item Winner-take-all dynamics
\item Two-sided markets
\item Governance mechanisms
\end{enumerate}

\column{0.48\textwidth}
\textbf{Core Insights}
\begin{itemize}
\item Network effects drive crypto adoption
\item Token value depends on velocity management
\item Supply schedules balance incentives
\item Governance is a mechanism design problem
\item Winner-take-all is common but not inevitable
\end{itemize}
\end{columns}

\vspace{0.5em}
\textbf{Looking Ahead}

Next lesson (L06): Market microstructure---how crypto markets discover prices and provide liquidity.

\bottomnote{Platform economics explains the success and failure patterns in cryptocurrency markets}
\end{frame}

% Further Reading
\begin{frame}[t]{Further Reading}
\begin{columns}[T]
\column{0.48\textwidth}
\textbf{Foundational Papers}
\begin{itemize}
\item Katz \& Shapiro (1985): ``Network Externalities, Competition, and Compatibility''
\item Rochet \& Tirole (2003): ``Platform Competition in Two-Sided Markets''
\item Catalini \& Gans (2020): ``Some Simple Economics of the Blockchain''
\end{itemize}

\column{0.48\textwidth}
\textbf{Tokenomics}
\begin{itemize}
\item Samuelson (2017): ``Velocity of Tokens'' (Medium)
\item Buterin (2017): ``On Medium-of-Exchange Token Valuations''
\item Weyl \& Lalley (2018): ``Quadratic Voting''
\end{itemize}
\end{columns}

\bottomnote{All readings available on course platform}
\end{frame}

\end{document}
